%html2tex: Version  2.7 of June 17, 2008.
%Written by  F.J. Faase.  http://www.iwriteiam.nl/

\documentclass[10pt]{article}%
\usepackage{amssymb}
\usepackage{geometry}
\usepackage{indentfirst}
\usepackage{amsmath}
\usepackage{amsfonts}
\usepackage{graphicx}%
\setcounter{MaxMatrixCols}{30}
%TCIDATA{OutputFilter=latex2.dll}
%TCIDATA{Version=5.00.0.2606}
%TCIDATA{CSTFile=40 LaTeX article.cst}
%TCIDATA{Created=Friday, March 30, 2007 00:21:27}
%TCIDATA{LastRevised=Wednesday, June 10, 2009 11:42:33}
%TCIDATA{<META NAME="GraphicsSave" CONTENT="32">}
%TCIDATA{<META NAME="SaveForMode" CONTENT="1">}
%TCIDATA{BibliographyScheme=Manual}
%TCIDATA{<META NAME="DocumentShell" CONTENT="Standard LaTeX\Blank - Standard LaTeX Article">}
%TCIDATA{Language=American English}
\newtheorem{theorem}{Theorem}
\newtheorem{acknowledgement}[theorem]{Acknowledgement}
\newtheorem{algorithm}[theorem]{Algorithm}
\newtheorem{axiom}[theorem]{Axiom}
\newtheorem{case}[theorem]{Case}
\newtheorem{claim}[theorem]{Claim}
\newtheorem{conclusion}[theorem]{Conclusion}
\newtheorem{condition}[theorem]{Condition}
\newtheorem{conjecture}[theorem]{Conjecture}
\newtheorem{corollary}[theorem]{Corollary}
\newtheorem{criterion}[theorem]{Criterion}
\newtheorem{definition}[theorem]{Definition}
\newtheorem{example}[theorem]{Example}
\newtheorem{exercise}[theorem]{Exercise}
\newtheorem{lemma}[theorem]{Lemma}
\newtheorem{notation}[theorem]{Notation}
\newtheorem{problem}[theorem]{Problem}
\newtheorem{proposition}[theorem]{Proposition}
\newtheorem{remark}[theorem]{Remark}
\newtheorem{solution}[theorem]{Solution}
\newtheorem{summary}[theorem]{Summary}
\newenvironment{proof}[1][Proof]{\noindent\textbf{#1.} }{\ \rule{0.5em}{0.5em}}
\geometry{left=1in,right=1in,top=1in,bottom=1in}

\begin{document}

%%%%% BEGINNING OF DOCUMENT BODY %%%%%
% html: Beginning of file: `clean.html'
% DOCTYPE HTML PUBLIC "-//W3C//DTD HTML 4.01//EN"
%  This is a (PRE) block.  Make sure it's left aligned or your toc title will be off. 

\section*{Working with a ``memoized-pipeline'' data structure (WORK IN PROGRESS)}

\label{f0}
\subsection*{Key Words}

geometry, memoized-pipeline, extending, modifying, data structure, geoquant, quantities, singleton, observer, observable

\subsection*{Authors}

\begin{itemize}\item  Alex Henniges
\item  Joseph Thomas
\end{itemize}

\subsection*{Introduction}

The memoized-pipeline is a data structure we developed for investigating geometries defined on triangulations. It is particularly suited to the situation in which we need to specify the values of some geometric quantities (independent variables) and then need to rapidly calculate the values of some other quantities (the dependent variables). Basically, we achieve this speedup by trading space for time. Usually, the definitions of the dependent variables have many intermediate values in common. By saving these values the first time we compute them, and then reusing them later, we can avoid a lot of useless recalculation. This strategy of saving calculated values, which can be found in most algorithms textbooks, is called ``memoization.'' 

In implementing various geometries, we have already developed code and techniques for making memoization an automatic part of encoding a geometry. In this tutorial, we describe how to take advantage of this existing code.

\subsection*{Implementation Details}

The underlying implementation of the pipeline is designed to solve two problems in a fairly user-friendly way:
\begin{enumerate}
\item  We would like to be able to identify geometric quantities with positions on the triangulation. For example, we can speak of the dihedral angle associated with a particular edge on a tetrahedron. We would like to be able to write code in the same way.
\item  We would like memoization to be nearly automatic. In other words, when writing a particular quantity, the programmer shouldn't have to think much about what happens to memoize that quantity's value.
\end{enumerate}
\begin{quotation} \end{quotation}Taking the programmer's perspective, we can view quantities as being specified by 3 pieces of information:
\begin{enumerate}
\item  A position on the triangulation.
\item  A definition of the other quantities (if any) needed to calculate the value of the current quantity, and where those quantities can be found on the triangulation.
\item  A formula for calculating a quantity's value, given the values of the other quantities it depends on.
\end{enumerate}
Usually, specifying just these 3 pieces of information is enough to create a new type of quantity. To help speed the development of quantities, we have developed a Ruby script, \texttt{makeQuantity.rb}, that generates much of the source code. This can be invoked at the command line as follows:
{\small{\begin{verbatim} 
> ruby makeQuantity.rb [quantity]
\end{verbatim}
}}
This produces two files, \texttt{\mbox{$[$}quantity\mbox{$]$}.h} and \texttt{\mbox{$[$}quantity\mbox{$]$}.cpp}.

\subsection{The ``anatomy'' of \texttt{quantity.h}}

In \texttt{C++}, header files serve several purposes. Among other uses, a header file can:
\begin{itemize}
\item  Specify dependencies on other parts of the project.   
\item  Define an interface for other parts of your project to use. This includes:
\begin{itemize}
\item  Definitions for new data-types (like classes).
\item  Definitions for procedure calls (what arguments a procedure takes, and what it returns).
\end{itemize}
\end{itemize}
By default, \texttt{makeQuantity.rb} gives you the following header file to use (here, we chose \texttt{quantity/QUANTITY} as the quantity name, in practice, this is filled out by the script). 
{\small{\begin{verbatim} 
#ifndef QUANTITY_H_
#define QUANTITY_H_

#include "geoquant.h"

/******************REGION 1*******************
 * This is where you load the headers of the *
 * quantities you require.                   *
 *********************************************/

class quantity : public virtual GeoQuant {
protected:
  quantity( SIMPLICES );
  void recalculate();
  /****************REGION 2*********************
   * The quantity references you need go here. *
   *********************************************/

public:
  ~quantity();
  static quantity* At( SIMPLICES );
  static void CleanUp();
};
#endif /* QUANTITY_H_ */
\end{verbatim}
}}
The two important areas of the header are labeled \texttt{REGION 1} and \texttt{REGION 2}. In region 1, you specify the header files for the quantities and utilities you use in the rest of your quantity. These \texttt{\#include} statements can be thought of as providing definitions for the data and procedures you want to use in building your quantity. In region 2, you specify the data associated with a given instance of the quantity; typically this amounts to several references to other quantities, or a data structure that manages references to other quantities. Lastly, you will need to modify the region tagged \texttt{SIMPLICES} so that it reflects a collection of simplices that describe your quantity's position on the triangulation.

\subsection{The ``anatomy'' of \texttt{quantity.cpp}}

In general, a \texttt{.cpp} file provides the internal implementation to support the operations described in the corresponding header file. Editing this file will be a little more complicated. By default, \texttt{makeQuantity.rb} will produce the following \texttt{.cpp} file: 
{\small{\begin{verbatim} 
#include "quantity.h"

#include <map>
#include <new>
using namespace std;

#define map<TriPosition, quantity*, TriPositionCompare> quantityIndex 
static quantityIndex* Index = NULL;

quantity::quantity( SIMPLICES ){
  /* REGION 1 */
}

quantity::~quantity(){
  /* REGION 2 */
}

void quantity::recalculate(){
  /* REGION 3 */
}

quantity* quantity::At( SIMPLICES ){
  TriPosition T( NUMSIMPLICES, SIMPLICES );
  if( Index == NULL ) Index = new quantityIndex();
  quantityIndex::iterator iter = Index->find( T );

  if( iter == Index->end() ){
    quantity* val = new quantity( SIMPLICES );
    Index->insert( make_pair( T, val ) );
    return val;
  } else {
    return iter->second;
  }
}

void quantity::CleanUp(){
  if( Index == NULL ) return;
  quantityIndex::iterator iter;
  for(iter = Index->begin(); iter != Index->end(); iter++)
    delete iter->second;
  delete Index;
}
\end{verbatim}
}}
There are a few smaller areas to fill out, but in general defining the quantity requires the following three definitions:
\begin{itemize}
\item  Region 1 specifies how to obtain references on the quantities your quantity depends on. Typically, this will involve using the input simplex information and some utilities for inspecting the triangulation to look up the quantities needed for later calculations.
\item  Region 2 specifies how to release any data structures built up using dynamic memory. In many cases, this field will be left blank.
\item  Region 3 specifies how to calculate the value of an instance of the quantity. Typically, this will occur in two steps:
\begin{enumerate}
\item  Using the quantity references obtained in region 1, we acquire the current values of the quantities used in the calculation.
\item  Using a formula and the values found in step 1, we calculate the value of the current quantity.
\end{enumerate}
\end{itemize}

\subsection*{An Extended Example}

Perhaps the easiest way to understand the system is by examining a few working quantities. In this example, we consider the code written to represent the ``Dual Area Segment\"{} quantity discussed in  \"{}Discrete conformal variations and scalar curvature on piecewise flat two and three dimensional manifolds''\footnote{See URL http://arxiv.org/abs/0906.1560} (in the paper, this quantity is also called A\(_{ij,kl}\)).

PICTURES AND A PROSE DESCRIPTION GO HERE

{\small{\begin{verbatim} 
#ifndef DUALAREASEGMENT_H_
#define DUALAREASEGMENT_H_

#include "geoquant.h"
#include "triposition.h"

#include "edge_height.h"
#include "face_height.h"

class DualAreaSegment : public virtual GeoQuant {
private:
  EdgeHeight* hij_k;
  EdgeHeight* hij_l;
  FaceHeight* hijk_l;
  FaceHeight* hijl_k;

protected:
  DualAreaSegment( Edge& e, Tetra& t );
  void recalculate();

public:
  ~DualAreaSegment();
  static DualAreaSegment* At( Edge& e, Tetra& t );
  static void CleanUp();
  static void Record( char* filename );
};

#endif /* DUALAREASEGMENT_H_ */
\end{verbatim}
}}
{\small{\begin{verbatim} 
#include "dualareasegment.h"
#include "miscmath.h"

#include <stdio.h>

typedef map<TriPosition, DualAreaSegment*, TriPositionCompare> DualAreaSegmentIndex;
static DualAreaSegmentIndex* Index = NULL;

DualAreaSegment::DualAreaSegment( Edge& e, Tetra& t ){
  StdTetra st = labelTetra( t, e );  // We use the topological tools in miscmath to
                                     // label the tetrahedron with respect to edge e.

  Face& fa123 = Triangulation::faceTable[ st.f123 ];
  Face& fa124 = Triangulation::faceTable[ st.f124 ];

  hij_k = EdgeHeight::At( e, fa123 );  // Here we use the calculated topological values
  hij_l = EdgeHeight::At( e, fa124 );  // to look up the edge and face heights required
  hijk_l = FaceHeight::At( fa123, t ); // to calculate the dual area.
  hijl_k = FaceHeight::At( fa124, t );

  hij_k->addDependent(this);  // Here we notify the quantities we reference
  hij_l->addDependent(this);  // that we wish to observe them (this is the            
  hijk_l->addDependent(this); // where an important part of our observer-
  hijl_k->addDependent(this); // observable design is implemented).
}

DualAreaSegment::~DualAreaSegment(){} // We didn't allocate any memory to store
                                      // this quantity's data, so the destructor 
                                      // can be left blank.

void DualAreaSegment::recalculate(){
  double Hij_k = hij_k->getValue();   // Step 1: We use the quantity references
  double Hijk_l = hijk_l->getValue(); // to obtain correct values for the referenced
  double Hij_l = hij_l->getValue();   // quantities.
  double Hijl_k = hijl_k->getValue();

  // Step 2: We use a formula to calculate the value of the dual-area segment.
  value = 0.5*(Hij_k * Hijk_l + Hij_l * Hijl_k); 
}

DualAreaSegment* DualAreaSegment::At( Edge& e, Tetra& t ){
  TriPosition T( 2, e.getSerialNumber(), t.getSerialNumber() );
  if( Index == NULL ) Index = new DualAreaSegmentIndex();
  DualAreaSegmentIndex::iterator iter = Index->find( T );

  if( iter == Index->end() ){
    DualAreaSegment* val = new DualAreaSegment( e, t );
    Index->insert( make_pair( T, val ) );
    return val;
  } else {
    return iter->second;
  }
}

void DualAreaSegment::CleanUp(){
  if( Index == NULL ) return;
  DualAreaSegmentIndex::iterator iter;
  for(iter = Index->begin(); iter != Index->end(); iter++)
    delete iter->second;
  delete Index;
}

\end{verbatim}
}}

\subsection*{Common Mistakes}

A panoply of debugging hints should go here.

\subsection*{Fancier Tricks}

Techniques for less common quantities go here.

\subsection*{Limitations, Areas to Improve}

Discussion of the current topological assumptions our code makes.
    
% html: End of file: `clean.html'

%%%%% END OF DOCUMENT BODY %%%%%
% In the future, we might want to put some additional data here, such
% as when the documentation was converted from wiki to TeX.
%

\end{document}
