%html2tex: Version  2.7 of June 17, 2008.
%Written by  F.J. Faase.  http://www.iwriteiam.nl/

\documentclass[10pt]{article}%
\usepackage{amssymb}
\usepackage{geometry}
\usepackage{indentfirst}
\usepackage{amsmath}
\usepackage{amsfonts}
\usepackage{graphicx}%
\setcounter{MaxMatrixCols}{30}
%TCIDATA{OutputFilter=latex2.dll}
%TCIDATA{Version=5.00.0.2606}
%TCIDATA{CSTFile=40 LaTeX article.cst}
%TCIDATA{Created=Friday, March 30, 2007 00:21:27}
%TCIDATA{LastRevised=Wednesday, June 10, 2009 11:42:33}
%TCIDATA{<META NAME="GraphicsSave" CONTENT="32">}
%TCIDATA{<META NAME="SaveForMode" CONTENT="1">}
%TCIDATA{BibliographyScheme=Manual}
%TCIDATA{<META NAME="DocumentShell" CONTENT="Standard LaTeX\Blank - Standard LaTeX Article">}
%TCIDATA{Language=American English}
\newtheorem{theorem}{Theorem}
\newtheorem{acknowledgement}[theorem]{Acknowledgement}
\newtheorem{algorithm}[theorem]{Algorithm}
\newtheorem{axiom}[theorem]{Axiom}
\newtheorem{case}[theorem]{Case}
\newtheorem{claim}[theorem]{Claim}
\newtheorem{conclusion}[theorem]{Conclusion}
\newtheorem{condition}[theorem]{Condition}
\newtheorem{conjecture}[theorem]{Conjecture}
\newtheorem{corollary}[theorem]{Corollary}
\newtheorem{criterion}[theorem]{Criterion}
\newtheorem{definition}[theorem]{Definition}
\newtheorem{example}[theorem]{Example}
\newtheorem{exercise}[theorem]{Exercise}
\newtheorem{lemma}[theorem]{Lemma}
\newtheorem{notation}[theorem]{Notation}
\newtheorem{problem}[theorem]{Problem}
\newtheorem{proposition}[theorem]{Proposition}
\newtheorem{remark}[theorem]{Remark}
\newtheorem{solution}[theorem]{Solution}
\newtheorem{summary}[theorem]{Summary}
\newenvironment{proof}[1][Proof]{\noindent\textbf{#1.} }{\ \rule{0.5em}{0.5em}}
\geometry{left=1in,right=1in,top=1in,bottom=1in}

\begin{document}

%%%%% BEGINNING OF DOCUMENT BODY %%%%%
% html: Beginning of file: `clean.html'
% DOCTYPE HTML PUBLIC "-//W3C//DTD HTML 4.01//EN"
%  This is a (PRE) block.  Make sure it's left aligned or your toc title will be off. 

\section*{Introduction}

\label{f0}This page contains basic information on the documentation of the functions and classes for the geocam project.

\section*{Details}

Add classes and functions to the following lists as follows:
\begin{quotation} 1.  locate the appropriate function type (double, int, void, etc.), 2.  Place the function in alphabetical order with exactly the following format (in edit mode):\end{quotation}{\small{\begin{verbatim} 
||ExampleFunction1||double||Kathrine||draft||
||ExampleFunction2||int||Jules||done||
||ExampleFunction3||vector||Lord Vader||partial||
||ExampleFunction4||void||Zoltar|| ||   <- leave "Progress" blank if no progress
||ExampleFunction5||bool|| || ||        <- leave "Owner" blank if it is unknown
\end{verbatim}
}}
 \textbf{Class}  \textbf{Owner}  \textbf{Progress} ApproximatorAlexdraftArea    CircleAlex Curvature2D  Curvature3D  DihedralAngle  Edge  EdgeCurvature  Eta  EuclideanAngle  EulerApprox  Face  GeoQuant  Length  Line  PartialEdgeDan  PointAlex Radius  RungaApproxJoe  SimplexAlex SphericalAngle  Tetra  TriangulationAlex TriangulationCoordinateSystemAlex TriPosition  Vertex  Volume  PairAlex TripleAlex 
 \textbf{Function}  \textbf{Type}  \textbf{Owner}  \textbf{Progress}      \textbf{boolean functions}  \textbf{boolean functions}   isAccuratebool  isDelaunayboolAlexobsoleteisDoubleTriangleboolAlexobsoleteisPrecisebool  isWeightedDelaunayboolAlexobsoleteread3DTriangulationFileboolAlex      \textbf{double functions}  \textbf{double functions}   adjDiffEQdouble  Aij\_kldoubleDan LaTex angledoubleAlex calcNormalizationdoubleDan  CayleyvolumeSqdoubleDan  CayleyvolumeSqDerivativedoubleDan  curvaturedouble  Curvature\_PartialdoubleDan LaTex dijdoubleDan LaTex distancePointdouble  dualAreadouble  dualLengthdouble  EHRdoubleDan LaTex EHR\_PartialdoubleDan LaTex EHR\_Second\_PartialdoubleDan LaTex FdoubleDan  FEdoubleDan  FEEdoubleDan  FRdoubleDan  getDualdoubleAlex getHeightdoubleAlex hij\_kdoubleDan LaTex hijk\_ldoubleDan LaTex inRadiusdoubleAlexobsoleteLij\_stardoubleDan LaTex plotdoubleDan  SecondPartialdoubleDan  Total\_CurvaturedoubleDan LaTex Total\_VolumedoubleDan LaTex Volume\_PartialdoubleDan LaTex Volume\_Second\_PartialdoubleDan LaTex volumeSqdoubleDan       \textbf{int functions}  \textbf{int functions}   addEdgeToEdgeintAlex addVertexToEdgeintAlex addVertexToFaceintAlex addVertexToVertexintAlex LinearEquationsSolvingint???  makeFaceintAlex makeTetraintAlex      \textbf{point functions}  \textbf{point functions}   findPointpointAlex rotateVectorpointAlex      \textbf{vector functions}  \textbf{vector functions}   circleIntersectionvectorAlex getTrianglesvector\mbox{$<$}triangle\_parts\mbox{$>$}Kurt listDifferencevectorAlex listIntersectionvectorAlex quadraticvectorAlex      \textbf{void functions}  \textbf{void functions}   addvoidAlex addCrossCapvoidTom addHandlevoidTom addLeafvoidTom addNewVertexvoidTom addTrianglevoidTom AdjHypRiccvoid  AdjSpherRiccivoid  calcDeltaFEvoidDan  calcDeltaFRvoidDan  errorMessagevoid  flipvoidKurt fourOneMovevoidTom HessianvoidDan  HypRiccivoid  loadRadiivoid  make3DTriangulationFilevoidAlexcompletemakeTriangulationFilevoidAlex MinMaxvoidDan  Newtons\_MethodvoidDan  oneFourMovevoidTom oneThreeMovevoidTom print3DResultsStepvoidAlexcompleteprintDatavoid  printResultsNumvoidAlexcompleteprintResultsNumStepsvoidAlex printResultsStepvoidAlexcompleteprintResultsVertexvoidAlexcompleteprintResultsVolumesvoidAlex readEtasvoidAlex  recalculatevoid  removevoidAlex removeVertexvoidAlex SpherRiccivoid  StdRiccivoid  threeOneMovevoidTom threeTwoMovevoidTom updateEtasvoidDan  updateRadiivoidDan  validateArgsvoid  write3DTriangulationFilevoidAlex writeTriangulationFilevoidAlex writeTriangulationFileWithDatavoidAlex Yamabevoid       \textbf{constructor functions}  \textbf{constructor functions}   Areaconstructor  Curvature2Dconstructor  Curvature3Dconstructor  DihedralAngleconstructor  EdgeCurvatureconstructor  Etaconstructor  EuclideanAngleconstructor  HyperbolicAngleconstructor  IndLengthconstructor  Lengthconstructor  PartialEdgeconstructor  Radiusconstructor  SphericalAngleconstructor  TriPositionconstructor  Volumeconstructor  
    
% html: End of file: `clean.html'

%%%%% END OF DOCUMENT BODY %%%%%
% In the future, we might want to put some additional data here, such
% as when the documentation was converted from wiki to TeX.
%

\end{document}
