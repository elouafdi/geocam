%TCIDATA{Version=5.00.0.2606}
%TCIDATA{LaTeXparent=0,0,classes.tex}
                      

%%%%% BEGINNING OF DOCUMENT BODY %%%%%
% html: Beginning of file: `clean.html'
% DOCTYPE HTML PUBLIC "-//W3C//DTD HTML 4.01//EN"
%  This is a (PRE) block.  Make sure it's left aligned or your toc title will be off. 

\section*{\texttt{NewtonsMethod}}

\label{f0}

\subsection*{Key Words}

\begin{quotation}
gradient, hessian, extrema
\end{quotation}

\subsection*{Authors}

\begin{itemize}
\item Alex Henniges

\item Dan Champion
\end{itemize}

\subsection*{Introduction}

\begin{quotation}
The \texttt{NewtonsMethod} class is used to find an extremum of a given
functional. In addition to the functional given, the user can provide the
gradient or hessian function. If not, these are approximated during run-time.
\end{quotation}

\subsection*{Subsidiaries}

\begin{quotation}
Functions:
\end{quotation}

\begin{itemize}
\item NewtonsMethod::maximize

\item NewtonsMethod::step

\item NewtonsMethod::setPrintFunc

\item NewtonsMethod::printInfo
\end{itemize}

\subsection*{Description}

\begin{quotation}
As a class, \texttt{NewtonsMethod} is used as a general way to perform
Newton's method on a function in order to find its extrema. Newton's method
will find an extrema much faster than Euler's method, but is also more
complicated. In order for Newton's method to work, it requires the first and
second-order partial derivatives in addition to the original function. If
the user knows these explicitly, they can be passed in the \mbox{$[$}%
\#Constructor constructor\mbox{$]$} and should lead to more accurate and
possible quicker calculations. If the first and secord-partial derivatives
are not provided, then they will be approximated using quotients.

For our purposes within the \texttt{NewtonsMethod} class, the original
function is defined to take as a parameter an array of \texttt{double}s that
represent the values of each variable. The function should return a double.
The gradient function is defined to take an array of doubles that also
represent the point at which the partial derivatives should be calculated.
In addition, it takes another array of doubles for the partial derivatives
to be placed in. Lastly, a hessian function is also defined to take an array
of doubles for the point at which the calculation is being done. It also
takes a two-dimensional array of double for the second-order partial
derivatives to be placed in. Both the gradient and hessian function do not
return a value.
\end{quotation}

\subsection*{Constructor}

\begin{quotation}
There are three constructors for the \texttt{NewtonsMethod} class, allowing
for a combination of potential functions that can be given explicitly. In
addition, every constructor must be given an integer that indicates the
number of variables given to the functional.{\small }
\end{quotation}

\begin{verbatim}
{\small     typedef double (*orig_function)(double vars[])
}
{\small     typedef void   (*gradient)(double vars[], double sol[])
}
{\small     typedef void   (*hessian)(double vars[], double *sol[])
}
 
{\small     NewtonsMethod(orig_function func, int numVars)
}
{\small     NewtonsMethod(orig_function func, gradient df, int numVars)
}
{\small 
    NewtonsMethod(orig_function func, gradient df, hessian d2f, int numVars)
}
{\small   
}
\end{verbatim}

\subsection*{Practicum}

\begin{quotation}
Below is a full example of how to use the \texttt{NewtonsMethod} class to
find the minimum of an ellipse. In this case, the gradient and hessian are
not given. The maximum found is at (0,0).{\small }
\end{quotation}

\begin{verbatim}
{\small    // This function takes two variables.
}
{\small    // f(x, y) = (1 - x^/4 - y^2/9) ^(1/2)
}
{\small    double ellipse(double vars[]) {
}
{\small        double val = 1 - pow(vars[0], 2) / 4 - pow(vars[1], 2) / 9;
}
{\small        return sqrt(val);
}
{\small    }
}
 
{\small    int main(int arg, char** argv) {
}
{\small     // Create the NewtonsMethod object, 2 variables
}
{\small     NewtonsMethod *nm = new NewtonsMethod(ellipse, 2);
}
{\small     // Build the array that holds the initial values.
}
{\small     double initial[] = {0.1, 2.5};
}
{\small     // Build the array that will hold the final solution.
}
{\small     double soln[2];
}
{\small     // Run the maximize function
}
{\small     nm->maximize(initial, soln);
}
 
{\small     // Display the results
}
{\small     printf("\nSolution: %.10f, %.10f\n", soln[0], soln[1]);
}
 
{\small     return 0;
}
{\small    }
}
{\small   
}
\end{verbatim}

\begin{quotation}
Using the same ellipse, one can use the \texttt{step} function instead of 
\texttt{maximize} to gain greater flexibility over the procedure. In this
case, we also print out useful information after each step.{\small }
\end{quotation}

\begin{verbatim}
{\small      // This function takes two variables.
}
{\small    double ellipse(double vars[]) {
}
{\small        double val = 1 - pow(vars[0], 2) / 4 - pow(vars[1], 2) / 9;
}
{\small        return sqrt(val);
}
{\small    }
}
 
{\small    int main(int arg, char** argv) {
}
{\small     NewtonsMethod *nm = new NewtonsMethod(ellipse, 2);
}
{\small     double x_n[] = {1, 1};
}
{\small     int i = 1;
}
{\small     fprintf(stdout, "Initial\n-----------------\n");
}
{\small     for(int j = 0; j < 2; j++) {
}
{\small       fprintf(stdout, "x_n_%d[%d] = %f\n", i, j, x_n[j]);
}
{\small     }
}
{\small 
    // Continue with the procedure until the length of the gradient is 
}
{\small     // less than 0.000001.
}
{\small     while(nm->step(x_n) > 0.000001) {
}
{\small       fprintf(stdout, "\n***** Step %d *****\n", i++);
}
{\small       nm->printInfo(stdout);
}
{\small       for(int j = 0; j < 2; j++) {
}
{\small         fprintf(stdout, "x_n_%d[%d] = %f\n", i, j, x_n[j]);
}
{\small       }
}
{\small     }
}
{\small     printf("\nSolution: %.10f\n", x_n[0]);
}
 
{\small     return 0;
}
{\small    }
}
{\small   
}
\end{verbatim}

\begin{quotation}
In this example, we use a one variable Gaussian function, but provide a
gradient and hessian, as well. The maximum is found at \texttt{x = 0}.%
{\small }
\end{quotation}

\begin{verbatim}
{\small    // The function, e^(-x^2).
}
{\small    double gaussian(double vars[]) {
}
{\small        return exp(-pow(vars[0], 2));
}
{\small    }
}
 
{\small    // The gradient function, -2x * e^(-x^2).
}
{\small    // Note that the solution is placed in the array.
}
{\small    void gradFunc(double vars[], double sol[]) {
}
{\small      sol[0] = -2 * vars[0] * func(vars);
}
{\small    }
}
 
{\small    // The hessian function, e^(-x^2)(4x^2 - 2).
}
{\small    // Note that the solution is placed in a matrix.
}
{\small    void hessFunc(double vars[], double *sol[]) {
}
{\small      sol[0][0] = func(vars) * (4 * pow(vars[0], 2) - 2);
}
{\small    }
}
 
{\small    int main(int arg, char** argv) {
}
{\small     // Create the NewtonsMethod object
}
{\small 
    NewtonsMethod *nm = new NewtonsMethod(gaussian, gradFunc, hessFunc, 1);
}
{\small     // Build the array that holds the initial value.
}
{\small     double initial[] = {0.1};
}
{\small     // Build the array that will hold the final solution.
}
{\small     double soln[1];
}
{\small     // Run the maximize function
}
{\small     nm->maximize(initial, soln);
}
 
{\small     // Display the results
}
{\small     printf("\nSolution: %.10f\n", soln[0]);
}
 
{\small     return 0;
}
{\small    }
}
{\small   
}
\end{verbatim}

\subsection*{Limitations}

\begin{quotation}
There are limitations with the \texttt{NewtonsMethod} class with regards to
the approximation of the gradient and hessian. In both cases, a delta value
for the quotients is hard-coded as 10$^{-5}$. This could lead to accuracy
issues when the point where the derivative is being calculated is less than
this value. It can also be too accurate at times and lead to unnecessarily
slowing down the procedure. One solution could be to provide a function
where a delta value is set by the user.
\end{quotation}

\subsection*{Revisions}

\begin{itemize}
\item subversion 876, 7/16/09: Added a NewtonsMethod class for general
maximizing.
\end{itemize}

\subsection*{Future Work}

\begin{itemize}
\item 7/16 - Provide greater flexibility to the user for approximating the
gradient and hessian.
\end{itemize}

% html: End of file: `clean.html'

%%%%% END OF DOCUMENT BODY %%%%%
% In the future, we might want to put some additional data here, such
% as when the documentation was converted from wiki to TeX.
%
