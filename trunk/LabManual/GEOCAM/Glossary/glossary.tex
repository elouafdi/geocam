%TCIDATA{Version=5.00.0.2606}
%TCIDATA{LaTeXparent=0,0,geocam.tex}
                      
%TCIDATA{ChildDefaults=chapter:1,page:1}


\chapter{Glossary}

\begin{description}
\item[circle power] Given a circle $C$ and a point $P$, let $L$ be the line
through the point $P$ that passes through the center of the circle. \ Let $A$
and $B$ be the intersection points of $L$ with the circle.\ Define a signed
distance $\left\Vert \overline{PX}\right\Vert $ for a line segment $%
\overline{PX}$ to be negative if the line segment lies entirely within the
circle, and positive otherwise. \ The \textit{circle power} of $P$ relative
to $C$, denoted by $pow_{C}\left( P\right) $, is given by:%
\[
pow_{C}\left( P\right) =\left\Vert \overline{PA}\right\Vert \left\Vert 
\overline{PB}\right\Vert .
\]%
Alternatively, if $C$ is defined implicitly by $\left( x-x_{c}\right)
^{2}+\left( y-y_{c}\right) ^{2}=r_{c}^{2}$, then the circle power can be
expressed as:%
\[
pow_{C}\left( P\right) =\left( x-x_{c}\right) ^{2}+\left( y-y_{c}\right)
^{2}-r_{c}^{2}.
\]

\item[common power point] The point of the plane (or $%
%TCIMACRO{\U{211d} }%
%BeginExpansion
\mathbb{R}
%EndExpansion
^{3}$) containing a decorated triangle (or tetrahedron) that has the same
circle power with respect to each of the weight circles. \ 

\item[decorated simplex (edge, triangle, tetrahedron,...)] A simplex is
called \textit{decorated} when weights are assigned to the vertices of the
simplex and then actualized by embedding the simplex into Euclidean space
together with spheres centered at the vertices with radii determined by the
weights. \ The orthocircle (if one exists) is sometimes considered part of
the decorated simplex when appropriate.

\item[edge curvature] Given a three-dimensional piecewise flat manifold $%
\left( M,\mathcal{T},d\right) $, the \textit{edge curvature} along an edge $%
\left\{ i,j\right\} $, measures how much that edge differs from Euclidean
space. \ Specifically, the edge curvature $K_{ij}$ is given by 
\[
K_{ij}=\left( 2\pi -\sum\limits_{\substack{ k,l\text{, such that}  \\ %
\left\{ i,j,k,l\right\} \in \mathcal{T}}}\beta _{ij,kl}\right) l_{ij},
\]%
where $l_{ij}$ is the edge length, and $\beta _{ij,kl}$ is the dihedral
angle of the edge $\left\{ i,j\right\} $ of the tetrahedron $\left\{
i,j,k,l\right\} $. \ In a triangulation of three-dimensional Euclidean space 
$K_{ij}=0$ for all edges. \ 

\item[Einstein constant] For a 3-dimensional piecewise flat manifold $\left(
M,\mathcal{T},d\right) $, the \textit{Einstein constant} $\lambda $ is given
by%
\[
\lambda =\frac{\mathcal{EHR}\left( M,\mathcal{T},d\right) }{3\mathcal{V}},
\]%
where $\mathcal{EHR}\left( M,\mathcal{T},d\right) $ is the
Einstein-Hilbert-Regge functional and $\mathcal{V}$ is the total volume.

\item[Einstein metric] Given a 3-dimensional piecewise flat manifold $\left(
M,\mathcal{T},d\right) $, we say that $d$ is an \textit{Einstein metric}
provided there exists $\lambda \in \mathbb{R}$ such that for all edges $%
\left\{ i,j\right\} $ in the triangulation we have:%
\[
K_{ij}=\lambda l_{ij}\frac{\partial \mathcal{V}}{\partial l_{ij}},
\]%
where $K_{ij}$ is the edge curvature, $l_{ij}$ is the edge length, and $%
\mathcal{V}$ is the total volume. By summing both sides, we see that $%
\lambda $ is the Einstein constant.

\item[hinge] A hinge consists of a pair of triangles that share a single
edge. \ Often, two adjacent triangles in a simplicial surface are identified
as a hinge while studying the edge they share. \ Any hinge can be
isometrically embedding $%
%TCIMACRO{\U{211d} }%
%BeginExpansion
\mathbb{R}
%EndExpansion
^{2}$. \ There is a natural generalization to three (and higher) dimensions
for two tetrahedra sharing a face. \ Similarly, this generalized hinge can
be isometrically embedded in $%
%TCIMACRO{\U{211d} }%
%BeginExpansion
\mathbb{R}
%EndExpansion
^{3}$. \ 

\item[inversive distance] Start with a decorated edge $e_{ij}$, that is, an
edge of length $l_{ij}$ with weight circles of radius $r_{i},r_{j}$ centered
on its vertices. \ The inversive distance $\eta _{ij}$ of the edge $e_{ij}$
can be calculated with the formula:%
\[
\eta _{ij}=\frac{l_{ij}^{2}-r_{i}^{2}-r_{j}^{2}}{2r_{i}r_{j}}.
\]%
When the two weight circles intersect with angle $\theta _{ij}$, we have the
simple formula:%
\[
\eta _{ij}=-\cos \left( \theta _{ij}\right) .
\]%
The former formula was obtained by using the law of cosines and solving for
the $\cos \left( \theta _{ij}\right) $ term. \ When the weight circles do
not intersect and do not contain one or the other $\eta _{ij}>1$. \ When the
weight circles intersect at some angle then $-1\leq \eta _{ij}\leq 1$. \ If
one weight circle contains the other we have $\eta _{ij}<1$. \ 

\item[manifold] A second countable, Hausdorff topological space $M$ is a 
\textit{manifold} provided there is an integer $n>0$ such that for each $%
x\in M$ there is an open set $U_{x}$ containing $x$ and a homeomorphism $%
h_{x}:U_{x}\rightarrow B\left( 1,0\right) \subset 
%TCIMACRO{\U{211d} }%
%BeginExpansion
\mathbb{R}
%EndExpansion
^{n}$. \ A discrete (or piecewise flat) space is a manifold provided the
sub-simplices satisfy the following conditions:

\item Dimension 2:

\begin{itemize}
\item All edges have exactly two adjacent faces.

\item For a vertex $v$, the faces incident upon $v$ can be arranged
cyclically as $f_{1},f_{2},...,f_{N},f_{1},...$ so that there is an edge
(containing $v$ as an endpoint) between each pair of consecutive faces $%
f_{i},f_{i+1}$, where $N+1$ is understood to be $1$. \ 
\end{itemize}

\item Dimension 3:

\begin{itemize}
\item All faces have exactly two adjacent tetrahedra.

\item For each edge $e$, the tetrahedra incident upon $e$ can be arranged
cyclically as $\sigma _{1},\sigma _{2},...,\sigma _{M},\sigma _{1},...$ so
that there is a face (containing $e$ as an edge) between each pair of
consecutive tetrahedra $\sigma _{i},\sigma _{i+1}$ where $M+1$ is understood
to be $1$. \ 

\item For each vertex $v$, the number of incident edges, faces and
tetrahedra, $E,F,T$, respectively (including multiple occurrences) satisfy:%
\[
E-F+T=2.
\]
\end{itemize}

\item More generally, given a simplicial manifold $M$ of dimension $n$, and
a sub-simplex $\sigma $ of dimension $m<n$, the sub-simplices of $M$ of
dimension greater than $m$ have the structure of $S^{n-m-1}$.

\item[orthocircle] Given a decorated triangle, provided the common power
point is outside all of the weight circles, there exists a circle that is
orthogonal to each of the weight circles. \ That is, the \textit{orthocircle}
is the circle that intersects each of the weight circles orthogonally. \ The
orthocircle does not exist when the common power point is on or inside all
three circles, however if the common power point is at infinity, there is a
line that is orthogonal to the three weight circles which will also be
identified as the orthocircle.

\item[piecewise flat manifold] A triple $\left( M,\mathcal{T},d\right) $
where $\left( M,\mathcal{T}\right) $ is a compact triangulated manifold with
triangulation $\mathcal{T}$, $d$ is a metric on $M$ so that the restriction
of $d$ to each simplex of $\mathcal{T}$ is isometric to a Euclidean simplex
of the same dimension. \ 

\item[triangulation] A collection of $n$-dimensional simplices $\mathcal{T}$
together with pairwise identifications for the $\left( n-1\right) $%
-dimensional faces of the simplices. \ More restrictions are needed to
ensure that the resultant space is a manifold. \ Alternatively, given a
space $M$ of dimension $n$, a \textit{triangulation} of $M$ is a subdivision
of $M$ into components $\left\{ \sigma _{i}\right\} $ by (hyper)surfaces (of
dimension $n-1$) so that each component is homeomorphic to an $n$%
-dimensional ball, and each component is combinatorially (as determined by
the subdivisions) equivalent to an $n$-simplex. \ 

\item[pseudo manifold] A discrete \textit{pseudo manifold} is a relaxation
of the manifold conditions for a discrete space. \ For dimensions two and
three, the bullet conditions given in the entry on manifold may no longer
hold. \ However, the manifold condition is still (trivially) satisfied in
the interior of all top dimensional simplices.

\item[weighted triangulation (or hinge)] A triangulation together with a map 
$w:V\rightarrow 
%TCIMACRO{\U{211d} }%
%BeginExpansion
\mathbb{R}
%EndExpansion
$, where $V$ is the set of vertices of the triangulation. \ Each simplex
becomes a decorated simplex. \ 

\item[dimension] 

\item[metric] 

\item[curvature] 

\item[scalar curvature] 

\item[constant scalar curvature] 

\item[geometric flow] A geometric flow is a differential equation on a
geometry (loosely defined) which depends only on geometric quantities
(usually curvatures). Usual examples are Ricci flow, mean curvature flow,
and Yamabe flow on Riemannian manifolds. On discrete geometries, there are
discrete Ricci and Yamabe flows.

\item[curvature flow] 

\item[Yamabe flow] 

\item[normalized total scalar curvature functional] 

\item[Einstein-Hilbert functional] Given a Riemannian manifold $\left(
M,g\right) ,$ the Einstein-Hilbert functional $\mathcal{EH}\left( M,g\right) 
$ is defined by 
\[
\mathcal{EH}\left( M,g\right) =\int_{M}RdV
\]%
where $R$ is the scalar curvature of $g$ and $dV$ is the volume form of $g.$
The functional is constant on two-dimensional manifolds, giving $4\pi $
times the Euler characteristic. In higher dimensions, its critical points
are Ricci flat (i.e., the Ricci tensor is zero) and critical points
restricted to metrics of a fixed volume are Einstein manifolds.

\item[Einstein-Hilbert-Regge functional] 

\item[conformal class] On a smooth Riemannian manifold $\left( M,g\right) ,$
the conformal class $\left[ g\right] $ is the equivalence class of all
Riemannian manifolds $\left( M,e^{f}g\right) ,$ where $f$ is a smooth
function. On a metrized piecewise flat manifold $\left( M,T,d\right) $, the
conformal class $\left[ d\right] $ is the equivalence class .....

\item[conformal deformation] 

\item[min-max procedure] 

\item[Yamabe constant] On a closed manifold $M,$ the Yamabe constant (also
called the $\sigma $-constant) is the number 
\[
\sigma \left( M^{n}\right) =\sup_{g}\inf_{g^{\prime }\in \left[ g\right] }%
\frac{\mathcal{EH}\left( M,g^{\prime }\right) }{V\left( M,g^{\prime }\right)
^{2/n}},
\]%
where the $\sup $ is over all metrics and the $\inf $ is over all metrics in
the same conformal class $\left[ g\right] $ as $g.$

\item[flip] 

\item[Pachner move] 

\item[flip algorithm] 

\item[convex hinge] 

\item[nonconvex hinge] 

\item[bone] 

\item[Delaunay triangulation (or hinge)] 

\item[weighted Delaunay triangulation] 

\item[Voronoi diagram (or cell)] 

\item[weighted Voronoi diagram (or cell)] 

\item[power diagram (or cell)] 

\item[negative triangles] 

\item[dual, Poincare dual] 

\item[dual length] 

\item[dual volume] 
\end{description}
