
\documentclass{article}
\usepackage{amsmath}
\usepackage{amssymb}

%%%%%%%%%%%%%%%%%%%%%%%%%%%%%%%%%%%%%%%%%%%%%%%%%%%%%%%%%%%%%%%%%%%%%%%%%%%%%%%%%%%%%%%%%%%%%%%%%%%%%%%%%%%%%%%%%%%%%%%%%%%%%%%%%%%%%%%%%%%%%%%%%%%%%%%%%%%%%%%%%%%%%%%%%%%%%%%%%%%%%%%%%%%%%%%%%%%%%%%%%%%%%%%%%%%%%%%%%%%%%%%%%%%%
%TCIDATA{OutputFilter=LATEX.DLL}
%TCIDATA{Version=5.00.0.2606}
%TCIDATA{<META NAME="SaveForMode" CONTENT="1">}
%TCIDATA{BibliographyScheme=Manual}
%TCIDATA{Created=Monday, February 15, 2010 10:13:06}
%TCIDATA{LastRevised=Monday, February 15, 2010 11:14:02}
%TCIDATA{<META NAME="GraphicsSave" CONTENT="32">}
%TCIDATA{<META NAME="DocumentShell" CONTENT="Standard LaTeX\Blank - Standard LaTeX Article">}
%TCIDATA{Language=American English}
%TCIDATA{CSTFile=40 LaTeX article.cst}

\newtheorem{theorem}{Theorem}
\newtheorem{acknowledgement}[theorem]{Acknowledgement}
\newtheorem{algorithm}[theorem]{Algorithm}
\newtheorem{axiom}[theorem]{Axiom}
\newtheorem{case}[theorem]{Case}
\newtheorem{claim}[theorem]{Claim}
\newtheorem{conclusion}[theorem]{Conclusion}
\newtheorem{condition}[theorem]{Condition}
\newtheorem{conjecture}[theorem]{Conjecture}
\newtheorem{corollary}[theorem]{Corollary}
\newtheorem{criterion}[theorem]{Criterion}
\newtheorem{definition}[theorem]{Definition}
\newtheorem{example}[theorem]{Example}
\newtheorem{exercise}[theorem]{Exercise}
\newtheorem{lemma}[theorem]{Lemma}
\newtheorem{notation}[theorem]{Notation}
\newtheorem{problem}[theorem]{Problem}
\newtheorem{proposition}[theorem]{Proposition}
\newtheorem{remark}[theorem]{Remark}
\newtheorem{solution}[theorem]{Solution}
\newtheorem{summary}[theorem]{Summary}
\newenvironment{proof}[1][Proof]{\noindent\textbf{#1.} }{\ \rule{0.5em}{0.5em}}
%\input{tcilatex}

\begin{document}


bone

Center of a simplex

circle and sphere packing \newline

\textbf{combinatorial Ricci-flow:}
	A partial differential equation on the curvatures and radii of a two-dimensional manifold. In the Euclidean case, the system is $\frac{\partial r_i }{\partial t} = -K_i * r_i$. However, this can lead to every radius shrinking to zero. We thus use a normalized system, $\frac{\partial r_i }{\partial t} = (\bar{K} - K_i) * r_i$ where $\bar{K}$ is the average curvature. \newline

conformal deformation

convex hinge

Curvature

Curvature flow

Delaunay triangulation (or hinge)

Dirichlet Energy

Discrete Curvature

dual volume

dual length

dual, Poincare dual

Euclidean Law of Cosines

flip

flip algorithm

min-max procedure

negative triangles

\textbf{Newton's Method}: A method for finding the roots of a function. For real-valued functions on $\mathbb{R}$, the n-th step of Newton's Method approximates a root from the previous step as $x_n = x_{n-1} - \frac{f(x_{n-1})}{f'(x_{n-1})}$. That is, we determine the equation of the tangent line at $x_{n-1}$, then set $x_n$ to be the x-intercept of this line. \newline

\noindent We are trying to optimize real-valued functions on $\mathbb{R}^m$, so we want to use Newton's Method to find the points at which $Df(\mathbf{x})$, the gradient of a function $f$, is the zero-vector. We therefore need the second derivative, $D^2f(\mathbf{x})$, called the hessian which is an $m$ by $m$ matrix. So a step of the method yields $\mathbf{x_n} = \mathbf{x_{n-1}} - [D^2f(\mathbf{x_{n-1}})]^{-1} * Df(\mathbf{x_{n-1}})$. \newline

nonconvex hinge

Normalized total scalar curvature functional

Pachner move

power diagram (or cell)

Runga-Kutta

Schl\"{a}fli formula

solid angle

Spherical Law of Cosines

Voronoi diagram (or cell)

weighted Delaunay triangulation

weighted Voronoi diagram (or cell)

Yamabe Energy

Yamabe Constant

Yamabe Flow

\bigskip 

\subsection{Quantities}

$r_{i}$ (radii, vertex weights)

$\eta _{ij}$ (etas, inversive distance, edge weights)

$\alpha _{i}$ (alphas, vertex weight coefficient)

$d_{ij}$ partial edge

$l_{ij}$ length%
\begin{eqnarray*}
l_{ij} &=&\sqrt{\alpha _{i}r_{i}^{2}+\alpha _{j}r_{j}^{2}+2r_{i}r_{j}\eta
_{ij}} \\
&=&d_{ij}+d_{ji}
\end{eqnarray*}

$A_{ijk}$ face area

$V_{ijkl}$ tetrahedron volume

$\gamma _{i,jk}$ face angle

$\beta _{ij,kl}$ dihedral angle (at an edge in a tetrahedron)

$\alpha _{i,jkl}$ solid angle

$h_{ij,k}$ edge height%
\[
h_{ij,k}=\frac{d_{ik}-d_{ij}\cos \gamma _{i,jk}}{\sin \gamma _{i,jk}}
\]

$h_{ijk,l}$ face height%
\[
h_{ijk,l}=\frac{h_{ij,l}-h_{ij,k}\cos \beta _{ij,kl}}{\sin \beta _{ij,kl}}
\]

$A_{ij,kl}$ dual area (of an edge in a tetrahedron)%
\[
A_{ij,kl}=\frac{1}{2}\left( h_{ij,k}h_{ijk,l}+h_{ij,l}h_{ijl,k}\right) 
\]

$l_{ij}^{\ast }$ dual length (2D) or dual area (3D) 
\begin{eqnarray*}
2D &:&l_{ij}^{\ast }=h_{ij,k}+h_{ij,l} \\
3D &:&l_{ij}^{\ast }=\sum_{\substack{ k,l\text{ s.t.} \\ \left\{
i,j,k,l\right\} \in \mathcal{T}}}A_{ij,kl}
\end{eqnarray*}

$K_{i}$ vertex curvature (scalar curvature)%
\begin{eqnarray*}
2D &:&K_{i}=2\pi -\sum_{\substack{ j,k\text{ s.t.} \\ \left\{ i,j,k\right\}
\in \mathcal{T}}}\gamma _{i,jk} \\
3D &:&K_{i}=\sum_{\substack{ j\text{ s.t.} \\ \left\{ i,j\right\} \in E}}%
\left( 2\pi -\sum_{\substack{ k,l\text{ s.t.} \\ \left\{ i,j,k,l\right\} \in 
\mathcal{T}}}\beta _{ij,kl}\right) d_{ij}
\end{eqnarray*}

$K_{ij}$ edge curvature (3D)%
\[
K_{ij}=\left( 2\pi -\sum_{\substack{ k,l\text{ s.t.} \\ \left\{
i,j,k,l\right\} \in \mathcal{T}}}\beta _{ij,kl}\right) l_{ij}
\]

$\lambda $ Einstein Constant%
\[
\lambda =\frac{EHR\left( M,\mathcal{T},l\right) }{3\mathcal{V}\left( M,%
\mathcal{T},l\right) }
\]

$EHR\left( M,\mathcal{T},l\right) $ (3D)%
\begin{eqnarray*}
EHR\left( M,\mathcal{T},l\right)  &=&\sum_{i}K_{i} \\
&=&\sum_{\substack{ i,j\text{ s.t.} \\ \left\{ i,j\right\} \in E}}K_{ij}
\end{eqnarray*}

$V_{i}$ dual volume of a vertex%
\[
V_{i}=\sum_{\substack{ j,k,l\text{ s.t.} \\ \left\{ i,j,k,l\right\} \in 
\mathcal{T}}}h_{ijk,l}A_{ijk}
\]

$\mathcal{V}$ total volume

$c_{ijk}$ center of a weighted triangle (embedded)

$c_{ijkl}$ center of a weighted tetrahedron (embedded)

\end{document}
