%TCIDATA{Version=5.00.0.2606}
%TCIDATA{LaTeXparent=0,0,cplusplus.tex}
                      

%%%%% BEGINNING OF DOCUMENT BODY %%%%%
% html: Beginning of file: `clean.html'
% DOCTYPE HTML PUBLIC "-//W3C//DTD HTML 4.01//EN"
%  This is a (PRE) block.  Make sure it's left aligned or your toc title will be off. 

\section*{Working with a ``memoized-pipeline'' data structure (WORK IN
PROGRESS)}

\label{f0}

\subsection*{Key Words}

geometry, memoized-pipeline, extending, modifying, data structure, geoquant,
quantities, singleton, observer, observable

\subsection*{Authors}

\begin{itemize}
\item Alex Henniges

\item Joseph Thomas
\end{itemize}

\subsection*{Introduction}

The memoized-pipeline is a data structure we developed for investigating
geometries defined on triangulations. It is particularly suited to the
situation in which we need to specify the values of some geometric
quantities (independent variables) and then need to rapidly calculate the
values of some other quantities (the dependent variables). Basically, we
achieve this speedup by trading space for time. Usually, the definitions of
the dependent variables have many intermediate values in common. By saving
these values the first time we compute them, and then reusing them later, we
can avoid a lot of useless recalculation. This strategy of saving calculated
values, which can be found in most algorithms textbooks, is called
``memoization.''

In implementing various geometries, we have already developed code and
techniques for making memoization an automatic part of encoding a geometry.
In this tutorial, we describe how to take advantage of this existing code.

\subsection*{Implementation Details}

The underlying implementation of the pipeline is designed to solve two
problems in a fairly user-friendly way:

\begin{enumerate}
\item We would like to be able to identify geometric quantities with
positions on the triangulation. For example, we can speak of the dihedral
angle associated with a particular edge on a tetrahedron. We would like to
be able to write code in the same way.

\item We would like memoization to be nearly automatic. In other words, when
writing a particular quantity, the programmer shouldn't have to think much
about what happens to memoize that quantity's value.
\end{enumerate}

Taking the programmer's perspective, we can view quantities as being
specified by 3 pieces of information:

\begin{enumerate}
\item A position on the triangulation.

\item A definition of the other quantities (if any) needed to calculate the
value of the current quantity, and where those quantities can be found on
the triangulation.

\item A formula for calculating a quantity's value, given the values of the
other quantities it depends on.
\end{enumerate}

Usually, specifying just these 3 pieces of information is enough to create a
new type of quantity. To help speed the development of quantities, we have
developed a Ruby script, \texttt{makeQuantity.rb}, that generates much of
the source code. This can be invoked at the command line as follows: {\small 
}
\begin{verbatim}
{\small > ruby makeQuantity.rb [quantity]
}
\end{verbatim}

This produces two files, \texttt{\mbox{$[$}quantity\mbox{$]$}.h} and \texttt{%
\mbox{$[$}quantity\mbox{$]$}.cpp}.

\subsection{The ``anatomy'' of \texttt{quantity.h}}

In \texttt{C++}, header files serve several purposes. Among other uses, a
header file can:

\begin{itemize}
\item Specify dependencies on other parts of the project.

\item Define an interface for other parts of your project to use. This
includes:

\begin{itemize}
\item Definitions for new data-types (like classes).

\item Definitions for procedure calls (what arguments a procedure takes, and
what it returns).
\end{itemize}
\end{itemize}

By default, \texttt{makeQuantity.rb} gives you the following header file to
use (here, we chose \texttt{quantity/QUANTITY} as the quantity name, in
practice, this is filled out by the script). {\small }
\begin{verbatim}
{\small #ifndef QUANTITY_H_
}
{\small #define QUANTITY_H_
}
 
{\small #include "geoquant.h"
}
 
{\small /******************REGION 1*******************
}
{\small  * This is where you load the headers of the *
}
{\small  * quantities you require.                   *
}
{\small  *********************************************/
}
 
{\small class quantity : public virtual GeoQuant {
}
{\small protected:
}
{\small   quantity( SIMPLICES );
}
{\small   void recalculate();
}
{\small   /****************REGION 2*********************
}
{\small    * The quantity references you need go here. *
}
{\small    *********************************************/
}
 
{\small public:
}
{\small   ~quantity();
}
{\small   static quantity* At( SIMPLICES );
}
{\small   static void CleanUp();
}
{\small };
}
{\small #endif /* QUANTITY_H_ */
}
\end{verbatim}

The two important areas of the header are labeled \texttt{REGION 1} and 
\texttt{REGION 2}. In region 1, you specify the header files for the
quantities and utilities you use in the rest of your quantity. These \texttt{%
\#include} statements can be thought of as providing definitions for the
data and procedures you want to use in building your quantity. In region 2,
you specify the data associated with a given instance of the quantity;
typically this amounts to several references to other quantities, or a data
structure that manages references to other quantities. Lastly, you will need
to modify the region tagged \texttt{SIMPLICES} so that it reflects a
collection of simplices that describe your quantity's position on the
triangulation.

\subsection{The ``anatomy'' of \texttt{quantity.cpp}}

In general, a \texttt{.cpp} file provides the internal implementation to
support the operations described in the corresponding header file. Editing
this file will be a little more complicated. By default, \texttt{%
makeQuantity.rb} will produce the following \texttt{.cpp} file: {\small }
\begin{verbatim}
{\small #include "quantity.h"
}
 
{\small #include <map>
}
{\small #include <new>
}
{\small using namespace std;
}
 
{\small 
#define map<TriPosition, quantity*, TriPositionCompare> quantityIndex 
}
{\small static quantityIndex* Index = NULL;
}
 
{\small quantity::quantity( SIMPLICES ){
}
{\small   /* REGION 1 */
}
{\small }
}
 
{\small quantity::~quantity(){
}
{\small   /* REGION 2 */
}
{\small }
}
 
{\small void quantity::recalculate(){
}
{\small   /* REGION 3 */
}
{\small }
}
 
{\small quantity* quantity::At( SIMPLICES ){
}
{\small   TriPosition T( NUMSIMPLICES, SIMPLICES );
}
{\small   if( Index == NULL ) Index = new quantityIndex();
}
{\small   quantityIndex::iterator iter = Index->find( T );
}
 
{\small   if( iter == Index->end() ){
}
{\small     quantity* val = new quantity( SIMPLICES );
}
{\small     Index->insert( make_pair( T, val ) );
}
{\small     return val;
}
{\small   } else {
}
{\small     return iter->second;
}
{\small   }
}
{\small }
}
 
{\small void quantity::CleanUp(){
}
{\small   if( Index == NULL ) return;
}
{\small   quantityIndex::iterator iter;
}
{\small   for(iter = Index->begin(); iter != Index->end(); iter++)
}
{\small     delete iter->second;
}
{\small   delete Index;
}
{\small }
}
\end{verbatim}

There are a few smaller areas to fill out, but in general defining the
quantity requires the following three definitions:

\begin{itemize}
\item Region 1 specifies how to obtain references on the quantities your
quantity depends on. Typically, this will involve using the input simplex
information and some utilities for inspecting the triangulation to look up
the quantities needed for later calculations.

\item Region 2 specifies how to release any data structures built up using
dynamic memory. In many cases, this field will be left blank.

\item Region 3 specifies how to calculate the value of an instance of the
quantity. Typically, this will occur in two steps:

\begin{enumerate}
\item Using the quantity references obtained in region 1, we acquire the
current values of the quantities used in the calculation.

\item Using a formula and the values found in step 1, we calculate the value
of the current quantity.
\end{enumerate}
\end{itemize}

\subsection*{An Extended Example}

Perhaps the easiest way to understand the system is by examining a few
working quantities. In this example, we consider the code written to
represent the ``Dual Area Segment\"{} quantity discussed in \"{}Discrete
conformal variations and scalar curvature on piecewise flat two and three
dimensional manifolds''\footnote{%
See URL http://arxiv.org/abs/0906.1560} (in the paper, this quantity is also
called A$_{ij,kl}$).

PICTURES AND A PROSE DESCRIPTION GO HERE

{\small }
\begin{verbatim}
{\small #ifndef DUALAREASEGMENT_H_
}
{\small #define DUALAREASEGMENT_H_
}
 
{\small #include "geoquant.h"
}
{\small #include "triposition.h"
}
 
{\small #include "edge_height.h"
}
{\small #include "face_height.h"
}
 
{\small class DualAreaSegment : public virtual GeoQuant {
}
{\small private:
}
{\small   EdgeHeight* hij_k;
}
{\small   EdgeHeight* hij_l;
}
{\small   FaceHeight* hijk_l;
}
{\small   FaceHeight* hijl_k;
}
 
{\small protected:
}
{\small   DualAreaSegment( Edge& e, Tetra& t );
}
{\small   void recalculate();
}
 
{\small public:
}
{\small   ~DualAreaSegment();
}
{\small   static DualAreaSegment* At( Edge& e, Tetra& t );
}
{\small   static void CleanUp();
}
{\small   static void Record( char* filename );
}
{\small };
}
 
{\small #endif /* DUALAREASEGMENT_H_ */
}
\end{verbatim}

{\small }
\begin{verbatim}
{\small #include "dualareasegment.h"
}
{\small #include "miscmath.h"
}
 
{\small #include <stdio.h>
}
 
{\small 
typedef map<TriPosition, DualAreaSegment*, TriPositionCompare> DualAreaSegmentIndex;
}
{\small static DualAreaSegmentIndex* Index = NULL;
}
 
{\small DualAreaSegment::DualAreaSegment( Edge& e, Tetra& t ){
}
{\small 
  StdTetra st = labelTetra( t, e );  // We use the topological tools in miscmath to
}
{\small 
                                     // label the tetrahedron with respect to edge e.
}
 
{\small   Face& fa123 = Triangulation::faceTable[ st.f123 ];
}
{\small   Face& fa124 = Triangulation::faceTable[ st.f124 ];
}
 
{\small 
  hij_k = EdgeHeight::At( e, fa123 );  // Here we use the calculated topological values
}
{\small 
  hij_l = EdgeHeight::At( e, fa124 );  // to look up the edge and face heights required
}
{\small 
  hijk_l = FaceHeight::At( fa123, t ); // to calculate the dual area.
}
{\small   hijl_k = FaceHeight::At( fa124, t );
}
 
{\small 
  hij_k->addDependent(this);  // Here we notify the quantities we reference
}
{\small 
  hij_l->addDependent(this);  // that we wish to observe them (this is the            
}
{\small 
  hijk_l->addDependent(this); // where an important part of our observer-
}
{\small   hijl_k->addDependent(this); // observable design is implemented).
}
{\small }
}
 
{\small 
DualAreaSegment::~DualAreaSegment(){} // We didn't allocate any memory to store
}
{\small 
                                      // this quantity's data, so the destructor 
}
{\small                                       // can be left blank.
}
 
{\small void DualAreaSegment::recalculate(){
}
{\small 
  double Hij_k = hij_k->getValue();   // Step 1: We use the quantity references
}
{\small 
  double Hijk_l = hijk_l->getValue(); // to obtain correct values for the referenced
}
{\small   double Hij_l = hij_l->getValue();   // quantities.
}
{\small   double Hijl_k = hijl_k->getValue();
}
 
{\small 
  // Step 2: We use a formula to calculate the value of the dual-area segment.
}
{\small   value = 0.5*(Hij_k * Hijk_l + Hij_l * Hijl_k); 
}
{\small }
}
 
{\small DualAreaSegment* DualAreaSegment::At( Edge& e, Tetra& t ){
}
{\small   TriPosition T( 2, e.getSerialNumber(), t.getSerialNumber() );
}
{\small   if( Index == NULL ) Index = new DualAreaSegmentIndex();
}
{\small   DualAreaSegmentIndex::iterator iter = Index->find( T );
}
 
{\small   if( iter == Index->end() ){
}
{\small     DualAreaSegment* val = new DualAreaSegment( e, t );
}
{\small     Index->insert( make_pair( T, val ) );
}
{\small     return val;
}
{\small   } else {
}
{\small     return iter->second;
}
{\small   }
}
{\small }
}
 
{\small void DualAreaSegment::CleanUp(){
}
{\small   if( Index == NULL ) return;
}
{\small   DualAreaSegmentIndex::iterator iter;
}
{\small   for(iter = Index->begin(); iter != Index->end(); iter++)
}
{\small     delete iter->second;
}
{\small   delete Index;
}
{\small }
}
 
\end{verbatim}

\subsection*{Common Mistakes}

A panoply of debugging hints should go here.

\subsection*{Fancier Tricks}

Techniques for less common quantities go here.

\subsection*{Limitations, Areas to Improve}

Discussion of the current topological assumptions our code makes.

% html: End of file: `clean.html'

%%%%% END OF DOCUMENT BODY %%%%%
% In the future, we might want to put some additional data here, such
% as when the documentation was converted from wiki to TeX.
%
