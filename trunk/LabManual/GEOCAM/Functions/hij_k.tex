%TCIDATA{Version=5.00.0.2606}
%TCIDATA{LaTeXparent=1,1,functions.tex}
                      

\section*{\texttt{EdgeHeight::EdgeHeight\label{Edge Height FUNCTION}}}

\subsection*{Function Prototype}

\texttt{double EdgeHeight( Vertex vi, Vertex vj, Vertex vk)}

\subsection*{Key Words}

Edge height, partial edge, geoquant.

\subsection*{Authors}

Daniel Champion

\subsection*{Introduction}

The function \texttt{EdgeHeight} calculates the edge height (to the center
of a triangular face) of an edge in the triangulation. \ 

\subsection*{Subsidiaries}

\textbf{Functions:}

\qquad \texttt{PartialEdge}

\qquad\texttt{Geometry::angle}

\qquad\texttt{listIntersection}

\textbf{Global Variables:} radii, etas.

\textbf{Local Variables:} Vertex vi, vj, vk.

\subsection*{Description}

The calculation of \texttt{EdgeHeight} involves the simple formula:%
\begin{equation*}
\text{\texttt{EdgeHeight(vi, vj, vk)}}=
\end{equation*}%
\begin{equation*}
\frac{\left( \text{\texttt{PartialEdge(vi,vk)}}-\text{\texttt{%
PartialEdge(vi,vj)}}\cos \left( \alpha _{i,jk}\right) \right) }{\sin \left(
\alpha _{i,jk}\right) }
\end{equation*}%
where $\alpha _{i,jk}$ is the angle at vertex $vi$ of triangle $\left\{
vi,vj,vk\right\} $. \ A geometric interpretation of this quantity is a
follows. \ Given a decorated triangle (triangle with radii and eta values
assigned to the vertices and edges respectively), the center of this
triangle can be calculated as the common power point of its embedding into
two-dimensional Euclidean space. \ The perpendicular distance from this
center point to the edge $\left\{ vi,vj\right\} $ is exactly \texttt{%
EdgeHeight(vi, vj, vk)}. \ Take note that the first two vertices in the
function call correspond to the preferred edge, and the third vertex in the
function call identifies the triangle. \ 

A primary use of this function is in the calculation of several quantities
needed for the \texttt{CurvaturePartial} function used in the optimization
of the normalized Einstein-Hilbert-Regge functional.

\subsection*{Practicum}

An example of the use of this function is in the calculation of the dual
areas, \texttt{DualAreaSegment}, to an edge of a three dimensional
triangulation.

\qquad \texttt{double DualAreaSegment( Vertex vi, Vertex vj, Vertex vk,
Vertex vl)}

\qquad\qquad\texttt{\{}

\qquad \qquad \texttt{double result =
0.5*(EdgeHeight(vi,vj,vk)*FaceHeight(vi,vj,vk,vl)}

\qquad \qquad \qquad \texttt{+EdgeHeight(vi,vj,vl)*FaceHeight(vi,vj,vl,vk));}

\qquad\qquad\texttt{return result;}

\qquad\qquad\texttt{\}}

\subsection*{Limitations}

\texttt{EdgeHeight} must receive as input three vertices of a face of the
triangulation. \ Moreover, the first two vertices in the function call
identify an edge, and can be in any order, however the third vertex in the
function call identifies the face and can not be permuted with the other two
vertices. \ 

\subsection*{Revisions}

subversion 757, 6/8/09, \texttt{EdgeHeight} created.

subversion 1055, 3/12/10, \texttt{EdgeHeight}\ converted to a geoquant.

\subsection*{Testing}

This function was not tested.

\subsection*{Future Work}

This function has been incorporated into the Geometry class geoquants, and
thus this entry needs to be updated. \ 
