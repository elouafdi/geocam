%TCIDATA{Version=5.00.0.2606}
%TCIDATA{LaTeXparent=0,0,functions.tex}
                      

%%%%% BEGINNING OF DOCUMENT BODY %%%%%
% html: Beginning of file: `clean.html'
% DOCTYPE HTML PUBLIC "-//W3C//DTD HTML 4.01//EN"
%  This is a (PRE) block.  Make sure it's left aligned or your toc title will be off. 

\section*{\texttt{flip}}

\label{f0}{\small }
\begin{verbatim}
{\small Edge flip(Edge e)
}
\end{verbatim}

\subsection*{Keywords}

\begin{quotation}
flip, delaunay
\end{quotation}

\subsection*{Authors}

\begin{quotation}
Kurt Norwood
\end{quotation}

\subsection*{Introduction}

\begin{quotation}
The \texttt{flip} function takes a single Edge as a parameter and performs a
flip on it. This involves determining the new length of the edge after the
flip and changing the topological information in the edge being flipped as
well as all of the edge's adjacent simplices. This can be thought of as
taking two triangles which share an edge (the parameter to flip) and making
two new triangles which share an edge between the two vertices which were
previously non-adjacent.
\end{quotation}

\subsection*{Subsidiaries}

\begin{quotation}
Functions:
\end{quotation}

{\small }
\begin{verbatim}
{\small     void flipPP(struct simps b)
}
 
{\small     void flipPN(struct simps b)
}
 
{\small     void flipNN(struct simps b)
}
 
{\small     void topo_flip(Edge, struct simps)
}
 
{\small     bool prep_for_flip(Edge, struct simps*)
}
\end{verbatim}

\begin{quotation}
Global Variables:Local Variables:
\end{quotation}

{\small }
\begin{verbatim}
{\small     Edge e
}
\end{verbatim}

\subsection*{Description}

\begin{quotation}
flip begins by calling the prep\_for\_flip function, that will setup the
struct given to it to contain all the important information necessary for
the flip to occur, such as indices for the different simplices and the
lengths of the triangles' edges, and the two angles which are not incident
on the edge being flipped. The struct looks like:
\end{quotation}

{\small }
\begin{verbatim}
{\small struct simps {
}
{\small        int v0, v1, v2, v3, e0, e1, e2, e3, e4, f0, f1;
}
{\small        double e0_len, e1_len, e2_len, e3_len, e4_len;
}
{\small        double a0, a2;
}
{\small };
}
\end{verbatim}

\begin{quotation}
With all this information known, the next step is to determine the type of
flip that is to occur. The possibilities are broken up three ways: positive
positive (PP), positive negative (PN), negative negative (NN); based on the
initial condition of the two triangles. This will determine which of flipPP,
flipPN, flipNN is called. Within these function is logic which should
compute the new edge length and assign it to the edge e, and determine the
positive/negative configuration of the two triangles and assign the
appropriate boolean value to each face.

With the new edge length computed and assigned, the topo\_flip function is
called which performs the rearrangement of all the adjacencies of the
different simplices which are adjacent to edge e.

The edge is returned.
\end{quotation}

\subsection*{Practicum}

{\small }
\begin{verbatim}
{\small   Edge e;
}
{\small   e = Triangulation::edgeTable[indexOfE];
}
{\small   e = flip(e);
}
\end{verbatim}

\begin{quotation}
one thing to note is that in future implementations the edge being given as
the parameter may be different than the one returned
\end{quotation}

\subsection*{Limitations}

\begin{quotation}
The biggest limitation of the flip function is that it currently only works
for bistellar flips. If higher dimensional flips are required this function
will need to be modified heavily.
\end{quotation}

\subsection*{Revisions}

\begin{quotation}
------------------------------------------------------------------------r816 %
\mbox{$|$} kortox \mbox{$|$} 2009-06-29 12:41:33 -0700 (Mon, 29 Jun 2009) %
\mbox{$|$} 1 line

have all the new\_flip stuff up to date and working with the new geometry
classes

------------------------------------------------------------------------r795 %
\mbox{$|$} kortox \mbox{$|$} 2009-06-18 17:58:30 -0700 (Thu, 18 Jun 2009) %
\mbox{$|$} 5 lines

anyway, this is a project for devopment of the flip algorithm, so far it
contains a new flip function which is intended to replace the flip function
that was previously in Triangulation/triangulationmorphs.cpp

main currently contains some test functions that can be called one at a time
manually and should produce output that can indicate how the flip function
is performing, this testing really needs to be improved
\end{quotation}

\subsection*{Testing}

\begin{quotation}
Initially testing was done inefficiently by manually analyzing what was
written by the writeTriangulationFile function. Now that we have a way to
display the triangulation, we can select an edge and flip it in the display
and see that the flip occurred correctly. Granted this should at sometime in
the future be automated, but for now if there is an issue we can try to
debug it with the display.
\end{quotation}

\subsection*{Future Work}

\begin{itemize}
\item Adding the ability to flip in higher dimensions. This would involve
altering the function to take a Simplex object instead of an edge so that it
is more general.

\item We'll most likely want to have the function add an edge to the
triangulation instead of just reposition the edge given, since this will
lend itself better to the possible addition of 3-1 flips. Related to this
would also be changing the return type to be a vector of Simplex objects for
generality's sake.

\item Moving the whole thing to a different file with an appropriate name
other than new\_flip
\end{itemize}

% html: End of file: `clean.html'

%%%%% END OF DOCUMENT BODY %%%%%
% In the future, we might want to put some additional data here, such
% as when the documentation was converted from wiki to TeX.
%
