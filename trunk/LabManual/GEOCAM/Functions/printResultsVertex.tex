%TCIDATA{Version=5.00.0.2606}
%TCIDATA{LaTeXparent=0,0,functions.tex}
                      

%%%%% BEGINNING OF DOCUMENT BODY %%%%%
% html: Beginning of file: `clean.html'
% DOCTYPE HTML PUBLIC "-//W3C//DTD HTML 4.01//EN"
%  This is a (PRE) block.  Make sure it's left aligned or your toc title will be off. 

\section*{\texttt{printResultsVertex}}

\label{f0}

\begin{quotation}
{\small }
\end{quotation}

\begin{verbatim}
{\small 
   void printResultsVertex(char* fileName, vector<double>* radii, vector<double>* curvs)
}
{\small    
}
\end{verbatim}

\subsection*{Key Words}

\begin{quotation}
radii, curvatures, file, flow, vertex, print
\end{quotation}

\subsection*{Authors}

\begin{quotation}
Alex Henniges
\end{quotation}

\subsection*{Introduction}

\begin{quotation}
The \texttt{printResultsVertex} function prints out the results of a
curvature flow, with the results grouped by each vertex of the
triangulation. These results will be written to the file given by \texttt{%
filename}.
\end{quotation}

\subsection*{Subsidiaries}

\begin{quotation}
Functions:

Global Variables:

Local Variables: \texttt{int vertSize}, \texttt{int numSteps}
\end{quotation}

\subsection*{Description}

\begin{quotation}
Prints the results of a curvature flow into the file given by \texttt{%
filename}. The results, that is, the radii and curvature values, are given
by vectors of doubles. Most commonly, these vectors are taken from the
Approximator class after the flow is run. The \texttt{printResultsVertex}
function determines the number of vertices of the current triangulation and
the total number of steps are then derived from this and the size of the
vectors.

There are several ways to display the results. The \texttt{printResultsVertex%
} function groups by vertex. This means that for each vertex of the
triangulation, the radii and curvature values for each step is given. An
example is shown below. Other formats are given by printResultsStep,
printResultsNum, print3DResultsStep.
\end{quotation}

\subsection*{Practicum}

\begin{quotation}
Example:{\small }
\end{quotation}

\begin{verbatim}
{\small 
  // Print the results of a curvature flow with Approximator app into file "ODEResult.txt"
}
{\small 
  printResultsVertex("./ODEResults.txt", app->radiiHistory, app->curvHistory);
}
{\small   
}
\end{verbatim}

\begin{quotation}
The output of such an example may then be{\small }
\end{quotation}

\begin{verbatim}
{\small          : 
}
{\small          :
}
{\small        Step  298        0.8272161        3.1425516
}
{\small        Step  299        0.8272081        3.1425294
}
{\small        Step  300        0.8272004        3.1425078
}
 
{\small        Vertex:   2        Radius                Curv
}
 
{\small        ---------------------------------
}
{\small        Step    0        1.0000000        3.1415927
}
{\small        Step    1        1.0000000        3.5987926
}
{\small        Step    2        0.9954280        3.5877017
}
{\small        Step    3        0.9909873        3.5768692
}
{\small        Step    4        0.9866738        3.5662905
}
{\small          :
}
{\small          :
}
{\small   
}
\end{verbatim}

\subsection*{Limitations}

\begin{quotation}
Currently the \texttt{printResultsVertex} function is limited in the
information it prints. As our curvature flow has evolved to record
additional information such as volumes, it may be time to explore a more
robust form for displaying results. As there is considerable dependence on
the Approximator for the data vectors, it may be wise to place this and
similar functions in the Approximator class.
\end{quotation}

\subsection*{Revisions}

\begin{itemize}
\item subversion 545, 9/29/08: Moved the printing of results out of calcFlow
and into a new function.

\item subversion 783, 6/18/09: Small modifications in response to changes in
the Approximator class.
\end{itemize}

\subsection*{Testing}

\begin{quotation}
The \texttt{printResultsVertex} function was tested by running multiple
curvature flows and printing the results. It was considered working when the
format of the data was as desired.
\end{quotation}

\subsection*{Future Work}

\begin{itemize}
\item 6/29 - Recreate the print functions to print more data and be more
flexible.

\item 6/29 - Move the print functions into the Approximator class.
\end{itemize}

% html: End of file: `clean.html'

%%%%% END OF DOCUMENT BODY %%%%%
% In the future, we might want to put some additional data here, such
% as when the documentation was converted from wiki to TeX.
%
