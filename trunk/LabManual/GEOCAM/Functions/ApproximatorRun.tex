%TCIDATA{Version=5.00.0.2606}
%TCIDATA{LaTeXparent=0,0,functions.tex}
                      

%%%%% BEGINNING OF DOCUMENT BODY %%%%%
% html: Beginning of file: `clean.html'
% DOCTYPE HTML PUBLIC "-//W3C//DTD HTML 4.01//EN"
%  This is a (PRE) block.  Make sure it's left aligned or your toc title will be off. 

\section*{\texttt{Approximator::run}}

\label{f0}

\begin{quotation}
{\small }
\end{quotation}

\begin{verbatim}
{\small int run(int numsteps, double stepsize)        
}
{\small int run(double precision, double stepsize)
}
{\small int run(double precision, int maxNumSteps, double stepsize)
}
{\small   
}
\end{verbatim}

\subsection*{Keywords}

\begin{quotation}
flow, curvature, stepsize, precision, accuracy, approximator
\end{quotation}

\subsection*{Authors}

\begin{itemize}
\item Joseph Thomas

\item Alex Henniges
\end{itemize}

\subsection*{Introduction}

\begin{quotation}
The \texttt{run} function of the Approximator class runs a system of
differential equations representing a curvature flow for either a number of
steps or until the values are within a desired precision. The system to use
and how steps are performed is given in the constructor of the approximator.
The type of run is based on the parameters given. The \texttt{run} function
returns 0 on success or any other number if an error is encountered.
\end{quotation}

\subsection*{Subsidiaries}

\begin{quotation}
Functions:
\end{quotation}

\begin{itemize}
\item \texttt{Approximator::step}

\item \texttt{Approximator::isPrecise}

\item \texttt{Approximator::isAccurate}

\item \texttt{Approximator::getLatest}

\begin{enumerate}
\item \texttt{Approximator::recordState}
\end{enumerate}
\end{itemize}

\begin{quotation}
Global Variables: \texttt{radii}, \texttt{curvatures}

Local Variables: \texttt{int numsteps}, \texttt{double stepsize}, \texttt{%
double precision}, \texttt{double accuracy}
\end{quotation}

\subsection*{Description}

\begin{quotation}
If the \texttt{run} function is given a number of steps, it will call its
step function that number of times. In between steps, the \texttt{run}
function will record the current state of any values that have been
requested to be recorded (this is specified in the constructor).

If the \texttt{run} function is given a precision, it will continue to call
its step function until the desired quantities (curvature in two dimensions
and curvature divide by radius in three dimensions) have converged within
the precision bounds. Precision is defined to be the difference between
subsequent values of a quantity. Therefore, precision is a measure of how
much a value is changing. In between steps, the \texttt{run} function will
record the current state of any values that have been requested to be
recorded (this is specified in the constructor).

A flow can also be run with a precision and a max number of steps that will
stop once one of the conditions is reached. The last parameter of any run
indicates the step size of the flow. A lower step size will lead to more
accurate steps, but a longer time to convergence.

The \texttt{run} function and the overarching Approximator class exists as
an improvement over the curvature flows of earlier versions of the Geocam
project. The \texttt{run} function provides the skeleton that is similar for
all types of curvature flows. Beyond the constructor, this should be the
only thing a user calls from the Approximator class.
\end{quotation}

\subsection*{Practicum}

\begin{quotation}
Example:
\end{quotation}

{\small }
\begin{verbatim}
{\small 
// Create an approximator that uses the Euler method on a Yamabe flow.
}
{\small Approximator *app = new EulerApprox(Yamabe);
}
 
{\small // Run a Yamabe flow for 300 steps with a stepsize of 0.01.
}
{\small app->run(300, 0.01);
}
{\small // Run with a precision bound of 0.000001 and a stepsize of 0.01
}
{\small app->run(0.000001, 0.01);
}
\end{verbatim}

\subsection*{Limitations}

\begin{quotation}
The \texttt{run} function is limited in the systems of differential
equations that it can run. It is designed to run with curvature flows and,
when precision is used, expects the values to converge. If a precision run
is performed on a flow that does not converge, the \texttt{run} function
will not stop. If a new curvature flow is created whose convergence is not
the usual (as in curvature divided by radius in Yamabe flow) then the 
\texttt{run} function will have to be modified to accommodate for this.
\end{quotation}

\subsection*{Revisions}

\begin{itemize}
\item subversion 659, 5/1/09: Initial \texttt{run} function uploaded to the
code.

\item subversion 679, 6/3/09: \texttt{run} function modified to work with
new Geometry structure.

\item subversion 761, 6/12/09: \texttt{run} function modified to work with
new quantity structure.

\item subversion 787, 6/18/09: Added new \texttt{run} options to
approximator. Removed accuracy. Checks for bad numbers.
\end{itemize}

\subsection*{Testing}

\begin{quotation}
The function was tested by performing two and three dimensional flows on
familiar triangulations. The start and end values for radii and curvature
was then compared with our expected values. The expected values were
obtained from the earlier curvature flows we had (see \mbox{$[$}%
\#Description Description\mbox{$]$} above). We also checked that the end
values were within the precision and accuracy bounds when they were in
effect.
\end{quotation}

\subsection*{Future Work}

\begin{itemize}
\item 6/17 - % <strike name="Future Work">
Add more run options (ex. precision and maxNumSteps). 
% </strike name="Future Work">
\textbf{Complete (6/18)}

\item 6/17 - % <strike name="Future Work">
Have a run stop the moment an undefined number appears. 
% </strike name="Future Work">
\textbf{Complete (6/18)}
\end{itemize}

\begin{quotation}
No future work is planned at this time.
\end{quotation}

% html: End of file: `clean.html'

%%%%% END OF DOCUMENT BODY %%%%%
% In the future, we might want to put some additional data here, such
% as when the documentation was converted from wiki to TeX.
%
