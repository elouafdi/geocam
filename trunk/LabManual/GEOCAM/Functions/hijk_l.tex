%TCIDATA{Version=5.00.0.2606}
%TCIDATA{LaTeXparent=1,1,functions.tex}
                      

\section*{\texttt{FaceHeight::FaceHeight\label{Face Height FUNCTION}}}

\subsection*{Function Prototype}

\texttt{double FaceHeight( Vertex vi, Vertex vj, Vertex vk, Vertex vl)}

\subsection*{Key Words}

Face height, edge height, geoquant.

\subsection*{Authors}

Daniel Champion

\subsection*{Introduction}

The function \texttt{FaceHeight}\ calculates the face height to the center
of a tetrahedron. \ 

\subsection*{Subsidiaries}

\textbf{Functions:}

\qquad\texttt{Geometry::dihedralAngle}

\qquad \texttt{EdgeHeight}

\qquad\texttt{listIntersection}

\textbf{Global Variables: }\ radii, etas.

\textbf{Local Variables:} \ Vertex vi, vj, vk, vl.

\subsection*{Description}

The calculation of \texttt{FaceHeight} involves the simple formula:%
\begin{equation*}
\text{\texttt{FaceHeight(vi, vj, vk, vl)}}=
\end{equation*}

\begin{equation*}
\frac{\left( \text{\texttt{EdgeHeight(vi,vj,vl)}}-\text{\texttt{%
EdgeHeight(vi,vj,vk)}}\cos (\beta _{ij,kl})\right) }{\sin \left( \beta
_{ij,kl}\right) }
\end{equation*}%
where $\beta _{ij,kl}$ is the dihedral angle along edge $\left\{
vi,vj\right\} $ of tetrahedron $\left\{ vi,vj,vk,vl\right\} $. \ A geometric
interpretation of this quantity is a follows. \ Given a decorated
tetrahedron (tetrahedron with radii and eta values assigned to the vertices
and edges respectively), the center of this tetrahedron can be calculated as
the common power point of its embedding into three-dimensional Euclidean
space. \ The perpendicular distance from this center point to the face $%
\left\{ vi,vj,vk\right\} $ is exactly \texttt{FaceHeight (vi, vj, vk, vl)}.
\ Take note that the first three vertices in the function call correspond to
the preferred face, and the fourth vertex in the function call identifies
the tetrahedron. \ 

A primary use of this function is in the calculation of several quantities
needed for the \texttt{CurvaturePartial} function used in the optimization
of the normalized Einstein-Hilbert-Regge functional.

\subsection*{Practicum}

An example of the use of this function is in the calculation of the dual
areas, \texttt{DualAreaSegment}, to an edge of a three dimensional
triangulation.

\qquad \texttt{double DualAreaSegment( Vertex vi, Vertex vj, Vertex vk,
Vertex vl)}

\qquad\qquad\texttt{\{}

\qquad \qquad \texttt{double result =
0.5*(EdgeHeight(vi,vj,vk)*FaceHeight(vi,vj,vk,vl)}

\qquad \qquad \qquad \texttt{+EdgeHeight(vi,vj,vl)*FaceHeight(vi,vj,vl,vk));}

\qquad\qquad\texttt{return result;}

\qquad\qquad\texttt{\}}

\subsection*{Limitations}

\texttt{faceHeight} must receive as input four vertices of a tetrahedron of
the triangulation. \ Moreover, the first three vertices in the function call
identify a face and can be in any order, however the fourth vertex in the
function call identifies the tetrahedron and can not be permuted with the
other three vertices. \ 

\subsection*{Revisions}

subversion 757, 6/8/09, \texttt{FaceHeight} created.

subversion 1055, 3/12/10, \texttt{FaceHeight}\ converted to a geoquant.

\subsection*{Testing}

This function was not tested.

\subsection*{Future Work}

This function has been incorporated into the Geometry class geoquants, and
thus this entry needs to be updated. \ 
