%TCIDATA{Version=5.00.0.2606}
%TCIDATA{LaTeXparent=0,0,functions.tex}
                      

%%%%% BEGINNING OF DOCUMENT BODY %%%%%
% html: Beginning of file: `clean.html'
% DOCTYPE HTML PUBLIC "-//W3C//DTD HTML 4.01//EN"
%  This is a (PRE) block.  Make sure it's left aligned or your toc title will be off. 

\section*{\texttt{NewtonsMethod::optimize}}

\label{f0}{\small }
\begin{verbatim}
{\small void optimize(double initial[], double soln[])
}
\end{verbatim}

\subsection*{Keywords}

\begin{quotation}
Newtons Method, optimize, extremum, gradient, hessian
\end{quotation}

\subsection*{Authors}

\begin{itemize}
\item Alex Henniges
\end{itemize}

\subsection*{Introduction}

\begin{quotation}
The \texttt{optimize} function of the NewtonsMethod class is designed to
find either a maximum or minimum of a functional near a given point.
\end{quotation}

\subsection*{Subsidiaries}

\begin{quotation}
Functions:
\end{quotation}

\begin{itemize}
\item \texttt{NewtonsMethod::step}
\end{itemize}

\subsection*{Description}

\begin{quotation}
The \texttt{optimize} function is called once by the user and it will
continue to loop until an extremum of the functional is found. The
functional is given in the constructor for NewtonsMethod. The initial point
is the first parameter of the \texttt{optimize} function and the solution
point is placed in the second parameter. This means that if one cannot be
found, the function will loop without end. This is unlike the \texttt{step}
function used within \texttt{optimize} that can also be used by a client
program to gain much greater flexibility in the optimization process, such
as more leeway on when to stop and allowing for data collection in between.
See the \texttt{step} function for a description of how the optimization is
performed.
\end{quotation}

\subsection*{Practicum}

\begin{quotation}
Example:{\small }
\end{quotation}

\begin{verbatim}
{\small     double func(double vars[]) {
}
{\small        double val = 1 - pow(vars[0], 2) / 4 - pow(vars[1], 2) / 9;
}
{\small        return sqrt(val);
}
{\small     }
}
{\small     
}
{\small     NewtonsMethod *nm = new NewtonsMethod(func, 2);
}
{\small     double initial[] = {1, 1};
}
{\small     double soln[2];
}
 
{\small     nm->optimize(initial, soln);
}
{\small   
}
\end{verbatim}

\subsection*{Limitations}

\begin{quotation}
The \texttt{optimize} function is limited in its termination condition. This
must be a constant over any use of the \texttt{optimize} function. It is
also limited in that it may not terminate at all and the user will be forced
to quit the program. Instead of modifying the function, these limitations
are addressed by the \texttt{step} function which trades simplicity in terms
of number of lines for greater flexibility.
\end{quotation}

\subsection*{Revisions}

\begin{itemize}
\item subversion 876 7/16/09: Added a NewtonsMethod class for general
maximizing.

\item subversion 906 8/3/09: Changed the name of the function maximize to
optimize in the NewtonsMethod class.
\end{itemize}

\subsection*{Testing}

\begin{quotation}
Newtons Method has been tested using several functions of 1 or 2 variables
including the Gaussian function. It has been tested with both approximating
the gradient and hessian and when both are given explicitly.
\end{quotation}

\subsection*{Future Work}

\begin{itemize}
\item 8/4 - Add the ability to only move partially in the direction of the
gradient.
\end{itemize}

% html: End of file: `clean.html'

%%%%% END OF DOCUMENT BODY %%%%%
% In the future, we might want to put some additional data here, such
% as when the documentation was converted from wiki to TeX.
%
