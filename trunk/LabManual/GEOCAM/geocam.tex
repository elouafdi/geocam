
\documentclass{book}
%%%%%%%%%%%%%%%%%%%%%%%%%%%%%%%%%%%%%%%%%%%%%%%%%%%%%%%%%%%%%%%%%%%%%%%%%%%%%%%%%%%%%%%%%%%%%%%%%%%%%%%%%%%%%%%%%%%%%%%%%%%%%%%%%%%%%%%%%%%%%%%%%%%%%%%%%%%%%%%%%%%%%%%%%%%%%%%%%%%%%%%%%%%%%%%%%%%%%%%%%%%%%%%%%%%%%%%%%%%%%%%%%%%%%%%%%%%%%%%%%%%%%%%%%%%%
\usepackage{makeidx}
\usepackage{amssymb}
\usepackage{eurosym}
\usepackage{amsfonts}
\usepackage{amsmath}

\setcounter{MaxMatrixCols}{10}
%TCIDATA{OutputFilter=LATEX.DLL}
%TCIDATA{Version=5.00.0.2606}
%TCIDATA{<META NAME="SaveForMode" CONTENT="1">}
%TCIDATA{BibliographyScheme=Manual}
%TCIDATA{Created=Monday, July 27, 2009 13:07:40}
%TCIDATA{LastRevised=Friday, August 14, 2009 14:58:02}
%TCIDATA{<META NAME="GraphicsSave" CONTENT="32">}
%TCIDATA{<META NAME="DocumentShell" CONTENT="Standard LaTeX\Standard LaTeX Book">}
%TCIDATA{CSTFile=40 LaTeX Book.cst}

\newtheorem{theorem}{Theorem}
\newtheorem{acknowledgement}[theorem]{Acknowledgement}
\newtheorem{algorithm}[theorem]{Algorithm}
\newtheorem{axiom}[theorem]{Axiom}
\newtheorem{case}[theorem]{Case}
\newtheorem{claim}[theorem]{Claim}
\newtheorem{conclusion}[theorem]{Conclusion}
\newtheorem{condition}[theorem]{Condition}
\newtheorem{conjecture}[theorem]{Conjecture}
\newtheorem{corollary}[theorem]{Corollary}
\newtheorem{criterion}[theorem]{Criterion}
\newtheorem{definition}[theorem]{Definition}
\newtheorem{example}[theorem]{Example}
\newtheorem{exercise}[theorem]{Exercise}
\newtheorem{lemma}[theorem]{Lemma}
\newtheorem{notation}[theorem]{Notation}
\newtheorem{problem}[theorem]{Problem}
\newtheorem{proposition}[theorem]{Proposition}
\newtheorem{remark}[theorem]{Remark}
\newtheorem{solution}[theorem]{Solution}
\newtheorem{summary}[theorem]{Summary}
\newenvironment{proof}[1][Proof]{\noindent\textbf{#1.} }{\ \rule{0.5em}{0.5em}}
\input{tcilatex}

\begin{document}

\frontmatter
\title{GEOCAM\\
Geometric Evolutions on Computational Abstract Manifolds}
\author{Daniel Champion, David Glickenstein, Yuliya Gorlina, \and Alex
Henniges, Kurtis Norwood, Jeff Taft, \and Joseph Thomas, Andrea Young}
\date{Summer 2009}
\maketitle
\tableofcontents

\chapter*{Preface}

\markboth{PREFACE}{PREFACE}This is is the preface. It is an unnumbered
chapter. The \verb|markboth| TeX field at the beginning of this paragraph
sets the correct page heading for the Preface portion of the document. The
preface does not appear in the table of contents.

\mainmatter

\part{Introductory Information}

%TCIMACRO{\QSubDoc{Include introduction}{%TCIDATA{Version=5.00.0.2606}
%TCIDATA{LaTeXparent=0,0,geocam.tex}
                      

\chapter{Project Introduction}
}}%
%BeginExpansion
%TCIDATA{Version=5.00.0.2606}
%TCIDATA{LaTeXparent=0,0,geocam.tex}
                      

\chapter{Project Introduction}
%
%EndExpansion

%TCIMACRO{\QSubDoc{Include cplusplus}{%TCIDATA{Version=5.00.0.2606}
%TCIDATA{LaTeXparent=0,0,geocam.tex}
                      

\chapter{The C++ Language}

%TCIMACRO{%
%\QSubDoc{Include Building_and_modifying_a_memoized_geometry}{%TCIDATA{Version=5.00.0.2606}
%TCIDATA{LaTeXparent=0,0,cplusplus.tex}
                      

%%%%% BEGINNING OF DOCUMENT BODY %%%%%
% html: Beginning of file: `clean.html'
% DOCTYPE HTML PUBLIC "-//W3C//DTD HTML 4.01//EN"
%  This is a (PRE) block.  Make sure it's left aligned or your toc title will be off. 

\section*{Working with a ``memoized-pipeline'' data structure (WORK IN
PROGRESS)}

\label{f0}

\subsection*{Key Words}

geometry, memoized-pipeline, extending, modifying, data structure, geoquant,
quantities, singleton, observer, observable

\subsection*{Authors}

\begin{itemize}
\item Alex Henniges

\item Joseph Thomas
\end{itemize}

\subsection*{Introduction}

The memoized-pipeline is a data structure we developed for investigating
geometries defined on triangulations. It is particularly suited to the
situation in which we need to specify the values of some geometric
quantities (independent variables) and then need to rapidly calculate the
values of some other quantities (the dependent variables). Basically, we
achieve this speedup by trading space for time. Usually, the definitions of
the dependent variables have many intermediate values in common. By saving
these values the first time we compute them, and then reusing them later, we
can avoid a lot of useless recalculation. This strategy of saving calculated
values, which can be found in most algorithms textbooks, is called
``memoization.''

In implementing various geometries, we have already developed code and
techniques for making memoization an automatic part of encoding a geometry.
In this tutorial, we describe how to take advantage of this existing code.

\subsection*{Implementation Details}

The underlying implementation of the pipeline is designed to solve two
problems in a fairly user-friendly way:

\begin{enumerate}
\item We would like to be able to identify geometric quantities with
positions on the triangulation. For example, we can speak of the dihedral
angle associated with a particular edge on a tetrahedron. We would like to
be able to write code in the same way.

\item We would like memoization to be nearly automatic. In other words, when
writing a particular quantity, the programmer shouldn't have to think much
about what happens to memoize that quantity's value.
\end{enumerate}

Taking the programmer's perspective, we can view quantities as being
specified by 3 pieces of information:

\begin{enumerate}
\item A position on the triangulation.

\item A definition of the other quantities (if any) needed to calculate the
value of the current quantity, and where those quantities can be found on
the triangulation.

\item A formula for calculating a quantity's value, given the values of the
other quantities it depends on.
\end{enumerate}

Usually, specifying just these 3 pieces of information is enough to create a
new type of quantity. To help speed the development of quantities, we have
developed a Ruby script, \texttt{makeQuantity.rb}, that generates much of
the source code. This can be invoked at the command line as follows: {\small 
}
\begin{verbatim}
{\small > ruby makeQuantity.rb [quantity]
}
\end{verbatim}

This produces two files, \texttt{\mbox{$[$}quantity\mbox{$]$}.h} and \texttt{%
\mbox{$[$}quantity\mbox{$]$}.cpp}.

\subsection{The ``anatomy'' of \texttt{quantity.h}}

In \texttt{C++}, header files serve several purposes. Among other uses, a
header file can:

\begin{itemize}
\item Specify dependencies on other parts of the project.

\item Define an interface for other parts of your project to use. This
includes:

\begin{itemize}
\item Definitions for new data-types (like classes).

\item Definitions for procedure calls (what arguments a procedure takes, and
what it returns).
\end{itemize}
\end{itemize}

By default, \texttt{makeQuantity.rb} gives you the following header file to
use (here, we chose \texttt{quantity/QUANTITY} as the quantity name, in
practice, this is filled out by the script). {\small }
\begin{verbatim}
{\small #ifndef QUANTITY_H_
}
{\small #define QUANTITY_H_
}
 
{\small #include "geoquant.h"
}
 
{\small /******************REGION 1*******************
}
{\small  * This is where you load the headers of the *
}
{\small  * quantities you require.                   *
}
{\small  *********************************************/
}
 
{\small class quantity : public virtual GeoQuant {
}
{\small protected:
}
{\small   quantity( SIMPLICES );
}
{\small   void recalculate();
}
{\small   /****************REGION 2*********************
}
{\small    * The quantity references you need go here. *
}
{\small    *********************************************/
}
 
{\small public:
}
{\small   ~quantity();
}
{\small   static quantity* At( SIMPLICES );
}
{\small   static void CleanUp();
}
{\small };
}
{\small #endif /* QUANTITY_H_ */
}
\end{verbatim}

The two important areas of the header are labeled \texttt{REGION 1} and 
\texttt{REGION 2}. In region 1, you specify the header files for the
quantities and utilities you use in the rest of your quantity. These \texttt{%
\#include} statements can be thought of as providing definitions for the
data and procedures you want to use in building your quantity. In region 2,
you specify the data associated with a given instance of the quantity;
typically this amounts to several references to other quantities, or a data
structure that manages references to other quantities. Lastly, you will need
to modify the region tagged \texttt{SIMPLICES} so that it reflects a
collection of simplices that describe your quantity's position on the
triangulation.

\subsection{The ``anatomy'' of \texttt{quantity.cpp}}

In general, a \texttt{.cpp} file provides the internal implementation to
support the operations described in the corresponding header file. Editing
this file will be a little more complicated. By default, \texttt{%
makeQuantity.rb} will produce the following \texttt{.cpp} file: {\small }
\begin{verbatim}
{\small #include "quantity.h"
}
 
{\small #include <map>
}
{\small #include <new>
}
{\small using namespace std;
}
 
{\small 
#define map<TriPosition, quantity*, TriPositionCompare> quantityIndex 
}
{\small static quantityIndex* Index = NULL;
}
 
{\small quantity::quantity( SIMPLICES ){
}
{\small   /* REGION 1 */
}
{\small }
}
 
{\small quantity::~quantity(){
}
{\small   /* REGION 2 */
}
{\small }
}
 
{\small void quantity::recalculate(){
}
{\small   /* REGION 3 */
}
{\small }
}
 
{\small quantity* quantity::At( SIMPLICES ){
}
{\small   TriPosition T( NUMSIMPLICES, SIMPLICES );
}
{\small   if( Index == NULL ) Index = new quantityIndex();
}
{\small   quantityIndex::iterator iter = Index->find( T );
}
 
{\small   if( iter == Index->end() ){
}
{\small     quantity* val = new quantity( SIMPLICES );
}
{\small     Index->insert( make_pair( T, val ) );
}
{\small     return val;
}
{\small   } else {
}
{\small     return iter->second;
}
{\small   }
}
{\small }
}
 
{\small void quantity::CleanUp(){
}
{\small   if( Index == NULL ) return;
}
{\small   quantityIndex::iterator iter;
}
{\small   for(iter = Index->begin(); iter != Index->end(); iter++)
}
{\small     delete iter->second;
}
{\small   delete Index;
}
{\small }
}
\end{verbatim}

There are a few smaller areas to fill out, but in general defining the
quantity requires the following three definitions:

\begin{itemize}
\item Region 1 specifies how to obtain references on the quantities your
quantity depends on. Typically, this will involve using the input simplex
information and some utilities for inspecting the triangulation to look up
the quantities needed for later calculations.

\item Region 2 specifies how to release any data structures built up using
dynamic memory. In many cases, this field will be left blank.

\item Region 3 specifies how to calculate the value of an instance of the
quantity. Typically, this will occur in two steps:

\begin{enumerate}
\item Using the quantity references obtained in region 1, we acquire the
current values of the quantities used in the calculation.

\item Using a formula and the values found in step 1, we calculate the value
of the current quantity.
\end{enumerate}
\end{itemize}

\subsection*{An Extended Example}

Perhaps the easiest way to understand the system is by examining a few
working quantities. In this example, we consider the code written to
represent the ``Dual Area Segment\"{} quantity discussed in \"{}Discrete
conformal variations and scalar curvature on piecewise flat two and three
dimensional manifolds''\footnote{%
See URL http://arxiv.org/abs/0906.1560} (in the paper, this quantity is also
called A$_{ij,kl}$).

PICTURES AND A PROSE DESCRIPTION GO HERE

{\small }
\begin{verbatim}
{\small #ifndef DUALAREASEGMENT_H_
}
{\small #define DUALAREASEGMENT_H_
}
 
{\small #include "geoquant.h"
}
{\small #include "triposition.h"
}
 
{\small #include "edge_height.h"
}
{\small #include "face_height.h"
}
 
{\small class DualAreaSegment : public virtual GeoQuant {
}
{\small private:
}
{\small   EdgeHeight* hij_k;
}
{\small   EdgeHeight* hij_l;
}
{\small   FaceHeight* hijk_l;
}
{\small   FaceHeight* hijl_k;
}
 
{\small protected:
}
{\small   DualAreaSegment( Edge& e, Tetra& t );
}
{\small   void recalculate();
}
 
{\small public:
}
{\small   ~DualAreaSegment();
}
{\small   static DualAreaSegment* At( Edge& e, Tetra& t );
}
{\small   static void CleanUp();
}
{\small   static void Record( char* filename );
}
{\small };
}
 
{\small #endif /* DUALAREASEGMENT_H_ */
}
\end{verbatim}

{\small }
\begin{verbatim}
{\small #include "dualareasegment.h"
}
{\small #include "miscmath.h"
}
 
{\small #include <stdio.h>
}
 
{\small 
typedef map<TriPosition, DualAreaSegment*, TriPositionCompare> DualAreaSegmentIndex;
}
{\small static DualAreaSegmentIndex* Index = NULL;
}
 
{\small DualAreaSegment::DualAreaSegment( Edge& e, Tetra& t ){
}
{\small 
  StdTetra st = labelTetra( t, e );  // We use the topological tools in miscmath to
}
{\small 
                                     // label the tetrahedron with respect to edge e.
}
 
{\small   Face& fa123 = Triangulation::faceTable[ st.f123 ];
}
{\small   Face& fa124 = Triangulation::faceTable[ st.f124 ];
}
 
{\small 
  hij_k = EdgeHeight::At( e, fa123 );  // Here we use the calculated topological values
}
{\small 
  hij_l = EdgeHeight::At( e, fa124 );  // to look up the edge and face heights required
}
{\small 
  hijk_l = FaceHeight::At( fa123, t ); // to calculate the dual area.
}
{\small   hijl_k = FaceHeight::At( fa124, t );
}
 
{\small 
  hij_k->addDependent(this);  // Here we notify the quantities we reference
}
{\small 
  hij_l->addDependent(this);  // that we wish to observe them (this is the            
}
{\small 
  hijk_l->addDependent(this); // where an important part of our observer-
}
{\small   hijl_k->addDependent(this); // observable design is implemented).
}
{\small }
}
 
{\small 
DualAreaSegment::~DualAreaSegment(){} // We didn't allocate any memory to store
}
{\small 
                                      // this quantity's data, so the destructor 
}
{\small                                       // can be left blank.
}
 
{\small void DualAreaSegment::recalculate(){
}
{\small 
  double Hij_k = hij_k->getValue();   // Step 1: We use the quantity references
}
{\small 
  double Hijk_l = hijk_l->getValue(); // to obtain correct values for the referenced
}
{\small   double Hij_l = hij_l->getValue();   // quantities.
}
{\small   double Hijl_k = hijl_k->getValue();
}
 
{\small 
  // Step 2: We use a formula to calculate the value of the dual-area segment.
}
{\small   value = 0.5*(Hij_k * Hijk_l + Hij_l * Hijl_k); 
}
{\small }
}
 
{\small DualAreaSegment* DualAreaSegment::At( Edge& e, Tetra& t ){
}
{\small   TriPosition T( 2, e.getSerialNumber(), t.getSerialNumber() );
}
{\small   if( Index == NULL ) Index = new DualAreaSegmentIndex();
}
{\small   DualAreaSegmentIndex::iterator iter = Index->find( T );
}
 
{\small   if( iter == Index->end() ){
}
{\small     DualAreaSegment* val = new DualAreaSegment( e, t );
}
{\small     Index->insert( make_pair( T, val ) );
}
{\small     return val;
}
{\small   } else {
}
{\small     return iter->second;
}
{\small   }
}
{\small }
}
 
{\small void DualAreaSegment::CleanUp(){
}
{\small   if( Index == NULL ) return;
}
{\small   DualAreaSegmentIndex::iterator iter;
}
{\small   for(iter = Index->begin(); iter != Index->end(); iter++)
}
{\small     delete iter->second;
}
{\small   delete Index;
}
{\small }
}
 
\end{verbatim}

\subsection*{Common Mistakes}

A panoply of debugging hints should go here.

\subsection*{Fancier Tricks}

Techniques for less common quantities go here.

\subsection*{Limitations, Areas to Improve}

Discussion of the current topological assumptions our code makes.

% html: End of file: `clean.html'

%%%%% END OF DOCUMENT BODY %%%%%
% In the future, we might want to put some additional data here, such
% as when the documentation was converted from wiki to TeX.
%
}}%
%BeginExpansion
%TCIDATA{Version=5.00.0.2606}
%TCIDATA{LaTeXparent=0,0,cplusplus.tex}
                      

%%%%% BEGINNING OF DOCUMENT BODY %%%%%
% html: Beginning of file: `clean.html'
% DOCTYPE HTML PUBLIC "-//W3C//DTD HTML 4.01//EN"
%  This is a (PRE) block.  Make sure it's left aligned or your toc title will be off. 

\section*{Working with a ``memoized-pipeline'' data structure (WORK IN
PROGRESS)}

\label{f0}

\subsection*{Key Words}

geometry, memoized-pipeline, extending, modifying, data structure, geoquant,
quantities, singleton, observer, observable

\subsection*{Authors}

\begin{itemize}
\item Alex Henniges

\item Joseph Thomas
\end{itemize}

\subsection*{Introduction}

The memoized-pipeline is a data structure we developed for investigating
geometries defined on triangulations. It is particularly suited to the
situation in which we need to specify the values of some geometric
quantities (independent variables) and then need to rapidly calculate the
values of some other quantities (the dependent variables). Basically, we
achieve this speedup by trading space for time. Usually, the definitions of
the dependent variables have many intermediate values in common. By saving
these values the first time we compute them, and then reusing them later, we
can avoid a lot of useless recalculation. This strategy of saving calculated
values, which can be found in most algorithms textbooks, is called
``memoization.''

In implementing various geometries, we have already developed code and
techniques for making memoization an automatic part of encoding a geometry.
In this tutorial, we describe how to take advantage of this existing code.

\subsection*{Implementation Details}

The underlying implementation of the pipeline is designed to solve two
problems in a fairly user-friendly way:

\begin{enumerate}
\item We would like to be able to identify geometric quantities with
positions on the triangulation. For example, we can speak of the dihedral
angle associated with a particular edge on a tetrahedron. We would like to
be able to write code in the same way.

\item We would like memoization to be nearly automatic. In other words, when
writing a particular quantity, the programmer shouldn't have to think much
about what happens to memoize that quantity's value.
\end{enumerate}

Taking the programmer's perspective, we can view quantities as being
specified by 3 pieces of information:

\begin{enumerate}
\item A position on the triangulation.

\item A definition of the other quantities (if any) needed to calculate the
value of the current quantity, and where those quantities can be found on
the triangulation.

\item A formula for calculating a quantity's value, given the values of the
other quantities it depends on.
\end{enumerate}

Usually, specifying just these 3 pieces of information is enough to create a
new type of quantity. To help speed the development of quantities, we have
developed a Ruby script, \texttt{makeQuantity.rb}, that generates much of
the source code. This can be invoked at the command line as follows: {\small 
}
\begin{verbatim}
{\small > ruby makeQuantity.rb [quantity]
}
\end{verbatim}

This produces two files, \texttt{\mbox{$[$}quantity\mbox{$]$}.h} and \texttt{%
\mbox{$[$}quantity\mbox{$]$}.cpp}.

\subsection{The ``anatomy'' of \texttt{quantity.h}}

In \texttt{C++}, header files serve several purposes. Among other uses, a
header file can:

\begin{itemize}
\item Specify dependencies on other parts of the project.

\item Define an interface for other parts of your project to use. This
includes:

\begin{itemize}
\item Definitions for new data-types (like classes).

\item Definitions for procedure calls (what arguments a procedure takes, and
what it returns).
\end{itemize}
\end{itemize}

By default, \texttt{makeQuantity.rb} gives you the following header file to
use (here, we chose \texttt{quantity/QUANTITY} as the quantity name, in
practice, this is filled out by the script). {\small }
\begin{verbatim}
{\small #ifndef QUANTITY_H_
}
{\small #define QUANTITY_H_
}
 
{\small #include "geoquant.h"
}
 
{\small /******************REGION 1*******************
}
{\small  * This is where you load the headers of the *
}
{\small  * quantities you require.                   *
}
{\small  *********************************************/
}
 
{\small class quantity : public virtual GeoQuant {
}
{\small protected:
}
{\small   quantity( SIMPLICES );
}
{\small   void recalculate();
}
{\small   /****************REGION 2*********************
}
{\small    * The quantity references you need go here. *
}
{\small    *********************************************/
}
 
{\small public:
}
{\small   ~quantity();
}
{\small   static quantity* At( SIMPLICES );
}
{\small   static void CleanUp();
}
{\small };
}
{\small #endif /* QUANTITY_H_ */
}
\end{verbatim}

The two important areas of the header are labeled \texttt{REGION 1} and 
\texttt{REGION 2}. In region 1, you specify the header files for the
quantities and utilities you use in the rest of your quantity. These \texttt{%
\#include} statements can be thought of as providing definitions for the
data and procedures you want to use in building your quantity. In region 2,
you specify the data associated with a given instance of the quantity;
typically this amounts to several references to other quantities, or a data
structure that manages references to other quantities. Lastly, you will need
to modify the region tagged \texttt{SIMPLICES} so that it reflects a
collection of simplices that describe your quantity's position on the
triangulation.

\subsection{The ``anatomy'' of \texttt{quantity.cpp}}

In general, a \texttt{.cpp} file provides the internal implementation to
support the operations described in the corresponding header file. Editing
this file will be a little more complicated. By default, \texttt{%
makeQuantity.rb} will produce the following \texttt{.cpp} file: {\small }
\begin{verbatim}
{\small #include "quantity.h"
}
 
{\small #include <map>
}
{\small #include <new>
}
{\small using namespace std;
}
 
{\small 
#define map<TriPosition, quantity*, TriPositionCompare> quantityIndex 
}
{\small static quantityIndex* Index = NULL;
}
 
{\small quantity::quantity( SIMPLICES ){
}
{\small   /* REGION 1 */
}
{\small }
}
 
{\small quantity::~quantity(){
}
{\small   /* REGION 2 */
}
{\small }
}
 
{\small void quantity::recalculate(){
}
{\small   /* REGION 3 */
}
{\small }
}
 
{\small quantity* quantity::At( SIMPLICES ){
}
{\small   TriPosition T( NUMSIMPLICES, SIMPLICES );
}
{\small   if( Index == NULL ) Index = new quantityIndex();
}
{\small   quantityIndex::iterator iter = Index->find( T );
}
 
{\small   if( iter == Index->end() ){
}
{\small     quantity* val = new quantity( SIMPLICES );
}
{\small     Index->insert( make_pair( T, val ) );
}
{\small     return val;
}
{\small   } else {
}
{\small     return iter->second;
}
{\small   }
}
{\small }
}
 
{\small void quantity::CleanUp(){
}
{\small   if( Index == NULL ) return;
}
{\small   quantityIndex::iterator iter;
}
{\small   for(iter = Index->begin(); iter != Index->end(); iter++)
}
{\small     delete iter->second;
}
{\small   delete Index;
}
{\small }
}
\end{verbatim}

There are a few smaller areas to fill out, but in general defining the
quantity requires the following three definitions:

\begin{itemize}
\item Region 1 specifies how to obtain references on the quantities your
quantity depends on. Typically, this will involve using the input simplex
information and some utilities for inspecting the triangulation to look up
the quantities needed for later calculations.

\item Region 2 specifies how to release any data structures built up using
dynamic memory. In many cases, this field will be left blank.

\item Region 3 specifies how to calculate the value of an instance of the
quantity. Typically, this will occur in two steps:

\begin{enumerate}
\item Using the quantity references obtained in region 1, we acquire the
current values of the quantities used in the calculation.

\item Using a formula and the values found in step 1, we calculate the value
of the current quantity.
\end{enumerate}
\end{itemize}

\subsection*{An Extended Example}

Perhaps the easiest way to understand the system is by examining a few
working quantities. In this example, we consider the code written to
represent the ``Dual Area Segment\"{} quantity discussed in \"{}Discrete
conformal variations and scalar curvature on piecewise flat two and three
dimensional manifolds''\footnote{%
See URL http://arxiv.org/abs/0906.1560} (in the paper, this quantity is also
called A$_{ij,kl}$).

PICTURES AND A PROSE DESCRIPTION GO HERE

{\small }
\begin{verbatim}
{\small #ifndef DUALAREASEGMENT_H_
}
{\small #define DUALAREASEGMENT_H_
}
 
{\small #include "geoquant.h"
}
{\small #include "triposition.h"
}
 
{\small #include "edge_height.h"
}
{\small #include "face_height.h"
}
 
{\small class DualAreaSegment : public virtual GeoQuant {
}
{\small private:
}
{\small   EdgeHeight* hij_k;
}
{\small   EdgeHeight* hij_l;
}
{\small   FaceHeight* hijk_l;
}
{\small   FaceHeight* hijl_k;
}
 
{\small protected:
}
{\small   DualAreaSegment( Edge& e, Tetra& t );
}
{\small   void recalculate();
}
 
{\small public:
}
{\small   ~DualAreaSegment();
}
{\small   static DualAreaSegment* At( Edge& e, Tetra& t );
}
{\small   static void CleanUp();
}
{\small   static void Record( char* filename );
}
{\small };
}
 
{\small #endif /* DUALAREASEGMENT_H_ */
}
\end{verbatim}

{\small }
\begin{verbatim}
{\small #include "dualareasegment.h"
}
{\small #include "miscmath.h"
}
 
{\small #include <stdio.h>
}
 
{\small 
typedef map<TriPosition, DualAreaSegment*, TriPositionCompare> DualAreaSegmentIndex;
}
{\small static DualAreaSegmentIndex* Index = NULL;
}
 
{\small DualAreaSegment::DualAreaSegment( Edge& e, Tetra& t ){
}
{\small 
  StdTetra st = labelTetra( t, e );  // We use the topological tools in miscmath to
}
{\small 
                                     // label the tetrahedron with respect to edge e.
}
 
{\small   Face& fa123 = Triangulation::faceTable[ st.f123 ];
}
{\small   Face& fa124 = Triangulation::faceTable[ st.f124 ];
}
 
{\small 
  hij_k = EdgeHeight::At( e, fa123 );  // Here we use the calculated topological values
}
{\small 
  hij_l = EdgeHeight::At( e, fa124 );  // to look up the edge and face heights required
}
{\small 
  hijk_l = FaceHeight::At( fa123, t ); // to calculate the dual area.
}
{\small   hijl_k = FaceHeight::At( fa124, t );
}
 
{\small 
  hij_k->addDependent(this);  // Here we notify the quantities we reference
}
{\small 
  hij_l->addDependent(this);  // that we wish to observe them (this is the            
}
{\small 
  hijk_l->addDependent(this); // where an important part of our observer-
}
{\small   hijl_k->addDependent(this); // observable design is implemented).
}
{\small }
}
 
{\small 
DualAreaSegment::~DualAreaSegment(){} // We didn't allocate any memory to store
}
{\small 
                                      // this quantity's data, so the destructor 
}
{\small                                       // can be left blank.
}
 
{\small void DualAreaSegment::recalculate(){
}
{\small 
  double Hij_k = hij_k->getValue();   // Step 1: We use the quantity references
}
{\small 
  double Hijk_l = hijk_l->getValue(); // to obtain correct values for the referenced
}
{\small   double Hij_l = hij_l->getValue();   // quantities.
}
{\small   double Hijl_k = hijl_k->getValue();
}
 
{\small 
  // Step 2: We use a formula to calculate the value of the dual-area segment.
}
{\small   value = 0.5*(Hij_k * Hijk_l + Hij_l * Hijl_k); 
}
{\small }
}
 
{\small DualAreaSegment* DualAreaSegment::At( Edge& e, Tetra& t ){
}
{\small   TriPosition T( 2, e.getSerialNumber(), t.getSerialNumber() );
}
{\small   if( Index == NULL ) Index = new DualAreaSegmentIndex();
}
{\small   DualAreaSegmentIndex::iterator iter = Index->find( T );
}
 
{\small   if( iter == Index->end() ){
}
{\small     DualAreaSegment* val = new DualAreaSegment( e, t );
}
{\small     Index->insert( make_pair( T, val ) );
}
{\small     return val;
}
{\small   } else {
}
{\small     return iter->second;
}
{\small   }
}
{\small }
}
 
{\small void DualAreaSegment::CleanUp(){
}
{\small   if( Index == NULL ) return;
}
{\small   DualAreaSegmentIndex::iterator iter;
}
{\small   for(iter = Index->begin(); iter != Index->end(); iter++)
}
{\small     delete iter->second;
}
{\small   delete Index;
}
{\small }
}
 
\end{verbatim}

\subsection*{Common Mistakes}

A panoply of debugging hints should go here.

\subsection*{Fancier Tricks}

Techniques for less common quantities go here.

\subsection*{Limitations, Areas to Improve}

Discussion of the current topological assumptions our code makes.

% html: End of file: `clean.html'

%%%%% END OF DOCUMENT BODY %%%%%
% In the future, we might want to put some additional data here, such
% as when the documentation was converted from wiki to TeX.
%
%
%EndExpansion

\bigskip 

\bigskip 
}}%
%BeginExpansion
%TCIDATA{Version=5.00.0.2606}
%TCIDATA{LaTeXparent=0,0,geocam.tex}
                      

\chapter{The C++ Language}

%TCIMACRO{%
%\QSubDoc{Include Building_and_modifying_a_memoized_geometry}{%TCIDATA{Version=5.00.0.2606}
%TCIDATA{LaTeXparent=0,0,cplusplus.tex}
                      

%%%%% BEGINNING OF DOCUMENT BODY %%%%%
% html: Beginning of file: `clean.html'
% DOCTYPE HTML PUBLIC "-//W3C//DTD HTML 4.01//EN"
%  This is a (PRE) block.  Make sure it's left aligned or your toc title will be off. 

\section*{Working with a ``memoized-pipeline'' data structure (WORK IN
PROGRESS)}

\label{f0}

\subsection*{Key Words}

geometry, memoized-pipeline, extending, modifying, data structure, geoquant,
quantities, singleton, observer, observable

\subsection*{Authors}

\begin{itemize}
\item Alex Henniges

\item Joseph Thomas
\end{itemize}

\subsection*{Introduction}

The memoized-pipeline is a data structure we developed for investigating
geometries defined on triangulations. It is particularly suited to the
situation in which we need to specify the values of some geometric
quantities (independent variables) and then need to rapidly calculate the
values of some other quantities (the dependent variables). Basically, we
achieve this speedup by trading space for time. Usually, the definitions of
the dependent variables have many intermediate values in common. By saving
these values the first time we compute them, and then reusing them later, we
can avoid a lot of useless recalculation. This strategy of saving calculated
values, which can be found in most algorithms textbooks, is called
``memoization.''

In implementing various geometries, we have already developed code and
techniques for making memoization an automatic part of encoding a geometry.
In this tutorial, we describe how to take advantage of this existing code.

\subsection*{Implementation Details}

The underlying implementation of the pipeline is designed to solve two
problems in a fairly user-friendly way:

\begin{enumerate}
\item We would like to be able to identify geometric quantities with
positions on the triangulation. For example, we can speak of the dihedral
angle associated with a particular edge on a tetrahedron. We would like to
be able to write code in the same way.

\item We would like memoization to be nearly automatic. In other words, when
writing a particular quantity, the programmer shouldn't have to think much
about what happens to memoize that quantity's value.
\end{enumerate}

Taking the programmer's perspective, we can view quantities as being
specified by 3 pieces of information:

\begin{enumerate}
\item A position on the triangulation.

\item A definition of the other quantities (if any) needed to calculate the
value of the current quantity, and where those quantities can be found on
the triangulation.

\item A formula for calculating a quantity's value, given the values of the
other quantities it depends on.
\end{enumerate}

Usually, specifying just these 3 pieces of information is enough to create a
new type of quantity. To help speed the development of quantities, we have
developed a Ruby script, \texttt{makeQuantity.rb}, that generates much of
the source code. This can be invoked at the command line as follows: {\small 
}
\begin{verbatim}
{\small > ruby makeQuantity.rb [quantity]
}
\end{verbatim}

This produces two files, \texttt{\mbox{$[$}quantity\mbox{$]$}.h} and \texttt{%
\mbox{$[$}quantity\mbox{$]$}.cpp}.

\subsection{The ``anatomy'' of \texttt{quantity.h}}

In \texttt{C++}, header files serve several purposes. Among other uses, a
header file can:

\begin{itemize}
\item Specify dependencies on other parts of the project.

\item Define an interface for other parts of your project to use. This
includes:

\begin{itemize}
\item Definitions for new data-types (like classes).

\item Definitions for procedure calls (what arguments a procedure takes, and
what it returns).
\end{itemize}
\end{itemize}

By default, \texttt{makeQuantity.rb} gives you the following header file to
use (here, we chose \texttt{quantity/QUANTITY} as the quantity name, in
practice, this is filled out by the script). {\small }
\begin{verbatim}
{\small #ifndef QUANTITY_H_
}
{\small #define QUANTITY_H_
}
 
{\small #include "geoquant.h"
}
 
{\small /******************REGION 1*******************
}
{\small  * This is where you load the headers of the *
}
{\small  * quantities you require.                   *
}
{\small  *********************************************/
}
 
{\small class quantity : public virtual GeoQuant {
}
{\small protected:
}
{\small   quantity( SIMPLICES );
}
{\small   void recalculate();
}
{\small   /****************REGION 2*********************
}
{\small    * The quantity references you need go here. *
}
{\small    *********************************************/
}
 
{\small public:
}
{\small   ~quantity();
}
{\small   static quantity* At( SIMPLICES );
}
{\small   static void CleanUp();
}
{\small };
}
{\small #endif /* QUANTITY_H_ */
}
\end{verbatim}

The two important areas of the header are labeled \texttt{REGION 1} and 
\texttt{REGION 2}. In region 1, you specify the header files for the
quantities and utilities you use in the rest of your quantity. These \texttt{%
\#include} statements can be thought of as providing definitions for the
data and procedures you want to use in building your quantity. In region 2,
you specify the data associated with a given instance of the quantity;
typically this amounts to several references to other quantities, or a data
structure that manages references to other quantities. Lastly, you will need
to modify the region tagged \texttt{SIMPLICES} so that it reflects a
collection of simplices that describe your quantity's position on the
triangulation.

\subsection{The ``anatomy'' of \texttt{quantity.cpp}}

In general, a \texttt{.cpp} file provides the internal implementation to
support the operations described in the corresponding header file. Editing
this file will be a little more complicated. By default, \texttt{%
makeQuantity.rb} will produce the following \texttt{.cpp} file: {\small }
\begin{verbatim}
{\small #include "quantity.h"
}
 
{\small #include <map>
}
{\small #include <new>
}
{\small using namespace std;
}
 
{\small 
#define map<TriPosition, quantity*, TriPositionCompare> quantityIndex 
}
{\small static quantityIndex* Index = NULL;
}
 
{\small quantity::quantity( SIMPLICES ){
}
{\small   /* REGION 1 */
}
{\small }
}
 
{\small quantity::~quantity(){
}
{\small   /* REGION 2 */
}
{\small }
}
 
{\small void quantity::recalculate(){
}
{\small   /* REGION 3 */
}
{\small }
}
 
{\small quantity* quantity::At( SIMPLICES ){
}
{\small   TriPosition T( NUMSIMPLICES, SIMPLICES );
}
{\small   if( Index == NULL ) Index = new quantityIndex();
}
{\small   quantityIndex::iterator iter = Index->find( T );
}
 
{\small   if( iter == Index->end() ){
}
{\small     quantity* val = new quantity( SIMPLICES );
}
{\small     Index->insert( make_pair( T, val ) );
}
{\small     return val;
}
{\small   } else {
}
{\small     return iter->second;
}
{\small   }
}
{\small }
}
 
{\small void quantity::CleanUp(){
}
{\small   if( Index == NULL ) return;
}
{\small   quantityIndex::iterator iter;
}
{\small   for(iter = Index->begin(); iter != Index->end(); iter++)
}
{\small     delete iter->second;
}
{\small   delete Index;
}
{\small }
}
\end{verbatim}

There are a few smaller areas to fill out, but in general defining the
quantity requires the following three definitions:

\begin{itemize}
\item Region 1 specifies how to obtain references on the quantities your
quantity depends on. Typically, this will involve using the input simplex
information and some utilities for inspecting the triangulation to look up
the quantities needed for later calculations.

\item Region 2 specifies how to release any data structures built up using
dynamic memory. In many cases, this field will be left blank.

\item Region 3 specifies how to calculate the value of an instance of the
quantity. Typically, this will occur in two steps:

\begin{enumerate}
\item Using the quantity references obtained in region 1, we acquire the
current values of the quantities used in the calculation.

\item Using a formula and the values found in step 1, we calculate the value
of the current quantity.
\end{enumerate}
\end{itemize}

\subsection*{An Extended Example}

Perhaps the easiest way to understand the system is by examining a few
working quantities. In this example, we consider the code written to
represent the ``Dual Area Segment\"{} quantity discussed in \"{}Discrete
conformal variations and scalar curvature on piecewise flat two and three
dimensional manifolds''\footnote{%
See URL http://arxiv.org/abs/0906.1560} (in the paper, this quantity is also
called A$_{ij,kl}$).

PICTURES AND A PROSE DESCRIPTION GO HERE

{\small }
\begin{verbatim}
{\small #ifndef DUALAREASEGMENT_H_
}
{\small #define DUALAREASEGMENT_H_
}
 
{\small #include "geoquant.h"
}
{\small #include "triposition.h"
}
 
{\small #include "edge_height.h"
}
{\small #include "face_height.h"
}
 
{\small class DualAreaSegment : public virtual GeoQuant {
}
{\small private:
}
{\small   EdgeHeight* hij_k;
}
{\small   EdgeHeight* hij_l;
}
{\small   FaceHeight* hijk_l;
}
{\small   FaceHeight* hijl_k;
}
 
{\small protected:
}
{\small   DualAreaSegment( Edge& e, Tetra& t );
}
{\small   void recalculate();
}
 
{\small public:
}
{\small   ~DualAreaSegment();
}
{\small   static DualAreaSegment* At( Edge& e, Tetra& t );
}
{\small   static void CleanUp();
}
{\small   static void Record( char* filename );
}
{\small };
}
 
{\small #endif /* DUALAREASEGMENT_H_ */
}
\end{verbatim}

{\small }
\begin{verbatim}
{\small #include "dualareasegment.h"
}
{\small #include "miscmath.h"
}
 
{\small #include <stdio.h>
}
 
{\small 
typedef map<TriPosition, DualAreaSegment*, TriPositionCompare> DualAreaSegmentIndex;
}
{\small static DualAreaSegmentIndex* Index = NULL;
}
 
{\small DualAreaSegment::DualAreaSegment( Edge& e, Tetra& t ){
}
{\small 
  StdTetra st = labelTetra( t, e );  // We use the topological tools in miscmath to
}
{\small 
                                     // label the tetrahedron with respect to edge e.
}
 
{\small   Face& fa123 = Triangulation::faceTable[ st.f123 ];
}
{\small   Face& fa124 = Triangulation::faceTable[ st.f124 ];
}
 
{\small 
  hij_k = EdgeHeight::At( e, fa123 );  // Here we use the calculated topological values
}
{\small 
  hij_l = EdgeHeight::At( e, fa124 );  // to look up the edge and face heights required
}
{\small 
  hijk_l = FaceHeight::At( fa123, t ); // to calculate the dual area.
}
{\small   hijl_k = FaceHeight::At( fa124, t );
}
 
{\small 
  hij_k->addDependent(this);  // Here we notify the quantities we reference
}
{\small 
  hij_l->addDependent(this);  // that we wish to observe them (this is the            
}
{\small 
  hijk_l->addDependent(this); // where an important part of our observer-
}
{\small   hijl_k->addDependent(this); // observable design is implemented).
}
{\small }
}
 
{\small 
DualAreaSegment::~DualAreaSegment(){} // We didn't allocate any memory to store
}
{\small 
                                      // this quantity's data, so the destructor 
}
{\small                                       // can be left blank.
}
 
{\small void DualAreaSegment::recalculate(){
}
{\small 
  double Hij_k = hij_k->getValue();   // Step 1: We use the quantity references
}
{\small 
  double Hijk_l = hijk_l->getValue(); // to obtain correct values for the referenced
}
{\small   double Hij_l = hij_l->getValue();   // quantities.
}
{\small   double Hijl_k = hijl_k->getValue();
}
 
{\small 
  // Step 2: We use a formula to calculate the value of the dual-area segment.
}
{\small   value = 0.5*(Hij_k * Hijk_l + Hij_l * Hijl_k); 
}
{\small }
}
 
{\small DualAreaSegment* DualAreaSegment::At( Edge& e, Tetra& t ){
}
{\small   TriPosition T( 2, e.getSerialNumber(), t.getSerialNumber() );
}
{\small   if( Index == NULL ) Index = new DualAreaSegmentIndex();
}
{\small   DualAreaSegmentIndex::iterator iter = Index->find( T );
}
 
{\small   if( iter == Index->end() ){
}
{\small     DualAreaSegment* val = new DualAreaSegment( e, t );
}
{\small     Index->insert( make_pair( T, val ) );
}
{\small     return val;
}
{\small   } else {
}
{\small     return iter->second;
}
{\small   }
}
{\small }
}
 
{\small void DualAreaSegment::CleanUp(){
}
{\small   if( Index == NULL ) return;
}
{\small   DualAreaSegmentIndex::iterator iter;
}
{\small   for(iter = Index->begin(); iter != Index->end(); iter++)
}
{\small     delete iter->second;
}
{\small   delete Index;
}
{\small }
}
 
\end{verbatim}

\subsection*{Common Mistakes}

A panoply of debugging hints should go here.

\subsection*{Fancier Tricks}

Techniques for less common quantities go here.

\subsection*{Limitations, Areas to Improve}

Discussion of the current topological assumptions our code makes.

% html: End of file: `clean.html'

%%%%% END OF DOCUMENT BODY %%%%%
% In the future, we might want to put some additional data here, such
% as when the documentation was converted from wiki to TeX.
%
}}%
%BeginExpansion
%TCIDATA{Version=5.00.0.2606}
%TCIDATA{LaTeXparent=0,0,cplusplus.tex}
                      

%%%%% BEGINNING OF DOCUMENT BODY %%%%%
% html: Beginning of file: `clean.html'
% DOCTYPE HTML PUBLIC "-//W3C//DTD HTML 4.01//EN"
%  This is a (PRE) block.  Make sure it's left aligned or your toc title will be off. 

\section*{Working with a ``memoized-pipeline'' data structure (WORK IN
PROGRESS)}

\label{f0}

\subsection*{Key Words}

geometry, memoized-pipeline, extending, modifying, data structure, geoquant,
quantities, singleton, observer, observable

\subsection*{Authors}

\begin{itemize}
\item Alex Henniges

\item Joseph Thomas
\end{itemize}

\subsection*{Introduction}

The memoized-pipeline is a data structure we developed for investigating
geometries defined on triangulations. It is particularly suited to the
situation in which we need to specify the values of some geometric
quantities (independent variables) and then need to rapidly calculate the
values of some other quantities (the dependent variables). Basically, we
achieve this speedup by trading space for time. Usually, the definitions of
the dependent variables have many intermediate values in common. By saving
these values the first time we compute them, and then reusing them later, we
can avoid a lot of useless recalculation. This strategy of saving calculated
values, which can be found in most algorithms textbooks, is called
``memoization.''

In implementing various geometries, we have already developed code and
techniques for making memoization an automatic part of encoding a geometry.
In this tutorial, we describe how to take advantage of this existing code.

\subsection*{Implementation Details}

The underlying implementation of the pipeline is designed to solve two
problems in a fairly user-friendly way:

\begin{enumerate}
\item We would like to be able to identify geometric quantities with
positions on the triangulation. For example, we can speak of the dihedral
angle associated with a particular edge on a tetrahedron. We would like to
be able to write code in the same way.

\item We would like memoization to be nearly automatic. In other words, when
writing a particular quantity, the programmer shouldn't have to think much
about what happens to memoize that quantity's value.
\end{enumerate}

Taking the programmer's perspective, we can view quantities as being
specified by 3 pieces of information:

\begin{enumerate}
\item A position on the triangulation.

\item A definition of the other quantities (if any) needed to calculate the
value of the current quantity, and where those quantities can be found on
the triangulation.

\item A formula for calculating a quantity's value, given the values of the
other quantities it depends on.
\end{enumerate}

Usually, specifying just these 3 pieces of information is enough to create a
new type of quantity. To help speed the development of quantities, we have
developed a Ruby script, \texttt{makeQuantity.rb}, that generates much of
the source code. This can be invoked at the command line as follows: {\small 
}
\begin{verbatim}
{\small > ruby makeQuantity.rb [quantity]
}
\end{verbatim}

This produces two files, \texttt{\mbox{$[$}quantity\mbox{$]$}.h} and \texttt{%
\mbox{$[$}quantity\mbox{$]$}.cpp}.

\subsection{The ``anatomy'' of \texttt{quantity.h}}

In \texttt{C++}, header files serve several purposes. Among other uses, a
header file can:

\begin{itemize}
\item Specify dependencies on other parts of the project.

\item Define an interface for other parts of your project to use. This
includes:

\begin{itemize}
\item Definitions for new data-types (like classes).

\item Definitions for procedure calls (what arguments a procedure takes, and
what it returns).
\end{itemize}
\end{itemize}

By default, \texttt{makeQuantity.rb} gives you the following header file to
use (here, we chose \texttt{quantity/QUANTITY} as the quantity name, in
practice, this is filled out by the script). {\small }
\begin{verbatim}
{\small #ifndef QUANTITY_H_
}
{\small #define QUANTITY_H_
}
 
{\small #include "geoquant.h"
}
 
{\small /******************REGION 1*******************
}
{\small  * This is where you load the headers of the *
}
{\small  * quantities you require.                   *
}
{\small  *********************************************/
}
 
{\small class quantity : public virtual GeoQuant {
}
{\small protected:
}
{\small   quantity( SIMPLICES );
}
{\small   void recalculate();
}
{\small   /****************REGION 2*********************
}
{\small    * The quantity references you need go here. *
}
{\small    *********************************************/
}
 
{\small public:
}
{\small   ~quantity();
}
{\small   static quantity* At( SIMPLICES );
}
{\small   static void CleanUp();
}
{\small };
}
{\small #endif /* QUANTITY_H_ */
}
\end{verbatim}

The two important areas of the header are labeled \texttt{REGION 1} and 
\texttt{REGION 2}. In region 1, you specify the header files for the
quantities and utilities you use in the rest of your quantity. These \texttt{%
\#include} statements can be thought of as providing definitions for the
data and procedures you want to use in building your quantity. In region 2,
you specify the data associated with a given instance of the quantity;
typically this amounts to several references to other quantities, or a data
structure that manages references to other quantities. Lastly, you will need
to modify the region tagged \texttt{SIMPLICES} so that it reflects a
collection of simplices that describe your quantity's position on the
triangulation.

\subsection{The ``anatomy'' of \texttt{quantity.cpp}}

In general, a \texttt{.cpp} file provides the internal implementation to
support the operations described in the corresponding header file. Editing
this file will be a little more complicated. By default, \texttt{%
makeQuantity.rb} will produce the following \texttt{.cpp} file: {\small }
\begin{verbatim}
{\small #include "quantity.h"
}
 
{\small #include <map>
}
{\small #include <new>
}
{\small using namespace std;
}
 
{\small 
#define map<TriPosition, quantity*, TriPositionCompare> quantityIndex 
}
{\small static quantityIndex* Index = NULL;
}
 
{\small quantity::quantity( SIMPLICES ){
}
{\small   /* REGION 1 */
}
{\small }
}
 
{\small quantity::~quantity(){
}
{\small   /* REGION 2 */
}
{\small }
}
 
{\small void quantity::recalculate(){
}
{\small   /* REGION 3 */
}
{\small }
}
 
{\small quantity* quantity::At( SIMPLICES ){
}
{\small   TriPosition T( NUMSIMPLICES, SIMPLICES );
}
{\small   if( Index == NULL ) Index = new quantityIndex();
}
{\small   quantityIndex::iterator iter = Index->find( T );
}
 
{\small   if( iter == Index->end() ){
}
{\small     quantity* val = new quantity( SIMPLICES );
}
{\small     Index->insert( make_pair( T, val ) );
}
{\small     return val;
}
{\small   } else {
}
{\small     return iter->second;
}
{\small   }
}
{\small }
}
 
{\small void quantity::CleanUp(){
}
{\small   if( Index == NULL ) return;
}
{\small   quantityIndex::iterator iter;
}
{\small   for(iter = Index->begin(); iter != Index->end(); iter++)
}
{\small     delete iter->second;
}
{\small   delete Index;
}
{\small }
}
\end{verbatim}

There are a few smaller areas to fill out, but in general defining the
quantity requires the following three definitions:

\begin{itemize}
\item Region 1 specifies how to obtain references on the quantities your
quantity depends on. Typically, this will involve using the input simplex
information and some utilities for inspecting the triangulation to look up
the quantities needed for later calculations.

\item Region 2 specifies how to release any data structures built up using
dynamic memory. In many cases, this field will be left blank.

\item Region 3 specifies how to calculate the value of an instance of the
quantity. Typically, this will occur in two steps:

\begin{enumerate}
\item Using the quantity references obtained in region 1, we acquire the
current values of the quantities used in the calculation.

\item Using a formula and the values found in step 1, we calculate the value
of the current quantity.
\end{enumerate}
\end{itemize}

\subsection*{An Extended Example}

Perhaps the easiest way to understand the system is by examining a few
working quantities. In this example, we consider the code written to
represent the ``Dual Area Segment\"{} quantity discussed in \"{}Discrete
conformal variations and scalar curvature on piecewise flat two and three
dimensional manifolds''\footnote{%
See URL http://arxiv.org/abs/0906.1560} (in the paper, this quantity is also
called A$_{ij,kl}$).

PICTURES AND A PROSE DESCRIPTION GO HERE

{\small }
\begin{verbatim}
{\small #ifndef DUALAREASEGMENT_H_
}
{\small #define DUALAREASEGMENT_H_
}
 
{\small #include "geoquant.h"
}
{\small #include "triposition.h"
}
 
{\small #include "edge_height.h"
}
{\small #include "face_height.h"
}
 
{\small class DualAreaSegment : public virtual GeoQuant {
}
{\small private:
}
{\small   EdgeHeight* hij_k;
}
{\small   EdgeHeight* hij_l;
}
{\small   FaceHeight* hijk_l;
}
{\small   FaceHeight* hijl_k;
}
 
{\small protected:
}
{\small   DualAreaSegment( Edge& e, Tetra& t );
}
{\small   void recalculate();
}
 
{\small public:
}
{\small   ~DualAreaSegment();
}
{\small   static DualAreaSegment* At( Edge& e, Tetra& t );
}
{\small   static void CleanUp();
}
{\small   static void Record( char* filename );
}
{\small };
}
 
{\small #endif /* DUALAREASEGMENT_H_ */
}
\end{verbatim}

{\small }
\begin{verbatim}
{\small #include "dualareasegment.h"
}
{\small #include "miscmath.h"
}
 
{\small #include <stdio.h>
}
 
{\small 
typedef map<TriPosition, DualAreaSegment*, TriPositionCompare> DualAreaSegmentIndex;
}
{\small static DualAreaSegmentIndex* Index = NULL;
}
 
{\small DualAreaSegment::DualAreaSegment( Edge& e, Tetra& t ){
}
{\small 
  StdTetra st = labelTetra( t, e );  // We use the topological tools in miscmath to
}
{\small 
                                     // label the tetrahedron with respect to edge e.
}
 
{\small   Face& fa123 = Triangulation::faceTable[ st.f123 ];
}
{\small   Face& fa124 = Triangulation::faceTable[ st.f124 ];
}
 
{\small 
  hij_k = EdgeHeight::At( e, fa123 );  // Here we use the calculated topological values
}
{\small 
  hij_l = EdgeHeight::At( e, fa124 );  // to look up the edge and face heights required
}
{\small 
  hijk_l = FaceHeight::At( fa123, t ); // to calculate the dual area.
}
{\small   hijl_k = FaceHeight::At( fa124, t );
}
 
{\small 
  hij_k->addDependent(this);  // Here we notify the quantities we reference
}
{\small 
  hij_l->addDependent(this);  // that we wish to observe them (this is the            
}
{\small 
  hijk_l->addDependent(this); // where an important part of our observer-
}
{\small   hijl_k->addDependent(this); // observable design is implemented).
}
{\small }
}
 
{\small 
DualAreaSegment::~DualAreaSegment(){} // We didn't allocate any memory to store
}
{\small 
                                      // this quantity's data, so the destructor 
}
{\small                                       // can be left blank.
}
 
{\small void DualAreaSegment::recalculate(){
}
{\small 
  double Hij_k = hij_k->getValue();   // Step 1: We use the quantity references
}
{\small 
  double Hijk_l = hijk_l->getValue(); // to obtain correct values for the referenced
}
{\small   double Hij_l = hij_l->getValue();   // quantities.
}
{\small   double Hijl_k = hijl_k->getValue();
}
 
{\small 
  // Step 2: We use a formula to calculate the value of the dual-area segment.
}
{\small   value = 0.5*(Hij_k * Hijk_l + Hij_l * Hijl_k); 
}
{\small }
}
 
{\small DualAreaSegment* DualAreaSegment::At( Edge& e, Tetra& t ){
}
{\small   TriPosition T( 2, e.getSerialNumber(), t.getSerialNumber() );
}
{\small   if( Index == NULL ) Index = new DualAreaSegmentIndex();
}
{\small   DualAreaSegmentIndex::iterator iter = Index->find( T );
}
 
{\small   if( iter == Index->end() ){
}
{\small     DualAreaSegment* val = new DualAreaSegment( e, t );
}
{\small     Index->insert( make_pair( T, val ) );
}
{\small     return val;
}
{\small   } else {
}
{\small     return iter->second;
}
{\small   }
}
{\small }
}
 
{\small void DualAreaSegment::CleanUp(){
}
{\small   if( Index == NULL ) return;
}
{\small   DualAreaSegmentIndex::iterator iter;
}
{\small   for(iter = Index->begin(); iter != Index->end(); iter++)
}
{\small     delete iter->second;
}
{\small   delete Index;
}
{\small }
}
 
\end{verbatim}

\subsection*{Common Mistakes}

A panoply of debugging hints should go here.

\subsection*{Fancier Tricks}

Techniques for less common quantities go here.

\subsection*{Limitations, Areas to Improve}

Discussion of the current topological assumptions our code makes.

% html: End of file: `clean.html'

%%%%% END OF DOCUMENT BODY %%%%%
% In the future, we might want to put some additional data here, such
% as when the documentation was converted from wiki to TeX.
%
%
%EndExpansion

\bigskip 

\bigskip 
%
%EndExpansion

\part{System Documentation}

%TCIMACRO{\QSubDoc{Include system}{%TCIDATA{Version=5.00.0.2606}
%TCIDATA{LaTeXparent=0,0,geocam.tex}
                      

\chapter{System}
}}%
%BeginExpansion
%TCIDATA{Version=5.00.0.2606}
%TCIDATA{LaTeXparent=0,0,geocam.tex}
                      

\chapter{System}
%
%EndExpansion

%TCIMACRO{\QSubDoc{Include subsystems}{%TCIDATA{Version=5.00.0.2606}
%TCIDATA{LaTeXparent=0,0,geocam.tex}
                      

\chapter{Subsystems}
}}%
%BeginExpansion
%TCIDATA{Version=5.00.0.2606}
%TCIDATA{LaTeXparent=0,0,geocam.tex}
                      

\chapter{Subsystems}
%
%EndExpansion

%TCIMACRO{\QSubDoc{Include functions}{%TCIDATA{Version=5.00.0.2606}
%TCIDATA{LaTeXparent=0,0,geocam.tex}
                      

\chapter{Functions}

%TCIMACRO{\QSubDoc{Include Aij_kl}{%TCIDATA{Version=5.00.0.2606}
%TCIDATA{LaTeXparent=1,1,functions.tex}
                      

\section*{\texttt{DualAreaSegment::DualAreaSegment}}

\subsection*{Function Prototype}

\texttt{double DualAreaSegment( Vertex vi, Vertex vj, Vertex vk, Vertex vl)}

\subsection*{Key Words}

Dual area, curvature, partial derivative, Einstein-Hilbert-Regge, geoquant.

\subsection*{Authors}

Daniel Champion

\subsection*{Introduction}

\texttt{DualAreaSegment} calculates the dual area to an edge of a
tetrahedron. \ 

\subsection*{Subsidiaries}

\textbf{Functions:}

\qquad \texttt{EdgeHeight}

\qquad \qquad \texttt{PartialEdge}

\qquad\qquad\texttt{Geometry::angle}

\qquad \texttt{FaceHeight}

\qquad\qquad\texttt{Geometry::dihedralAngle}

\textbf{Global Variables: \ }radii, etas

\textbf{Local Variables:} \ none.

\subsection*{Description}

\texttt{DualAreaSegment} is calculated with the formula:%
\begin{equation*}
\text{\texttt{DualAreaSegment(vi, vj, vk, vl)}}=
\end{equation*}%
\begin{equation*}
\frac{1}{2}\left( 
\begin{array}{c}
\text{\texttt{EdgeHeight(vi,vj,vk)}}\cdot \text{\texttt{%
FaceHeight(vi,vj,vk,vl)}} \\ 
+\text{\texttt{EdgeHeight(vi,vj,vl)}}\cdot \text{\texttt{%
FaceHeight(vi,vj,vl,vk)}}%
\end{array}%
\right) 
\end{equation*}%
\texttt{EdgeHeight} and \texttt{FaceHeight} are calculated with the
following formulae:%
\begin{align*}
\text{\texttt{EdgeHeight(vi, vj, vk)}}& =\frac{\left( \text{\texttt{%
PartialEdge(vi,vk)}}-\text{\texttt{PartialEdge(vi,vj)}}\cos \left( \alpha
_{i,jk}\right) \right) }{\sin \left( \alpha _{i,jk}\right) } \\
\text{\texttt{FaceHeight(vi, vj, vk, vl)}}& =\frac{\left( \text{\texttt{%
EdgeHeight(vi,vj,vl)}}-\text{\texttt{EdgeHeight(vi,vj,vk)}}\cos (\beta
_{ij,kl})\right) }{\sin \left( \beta _{ij,kl}\right) }
\end{align*}%
where $\alpha _{i,jk}$ is the angle at vertex $vi$ of triangle $\left\{
vi,vj,vk\right\} $, and $\beta _{ij,kl}$ is the dihedral angle along edge $%
\left\{ vi,vj\right\} $ of tetrahedron $\left\{ vi,vj,vk,vl\right\} $
(implemented with the functions \texttt{Geometry::angle} and \texttt{%
Geometry::dihedralAngle} respectively).

\texttt{DualAreaSegment} was created for the calculation performed in the
function \texttt{DualArea}, which is used in the computation of the partial
derivatives of curvature. \ These partial derivatives of curvature are used
in the calculation of the second order partial derivatives of the
Einstein-Hilbert-Regge functional for use in the optimization of said
functional using Newton's method. \ 

\subsection*{Practicum}

As an example of the usage of this function, we will calculate the dual area
to the edge $eij=\left\{ vi,vj\right\} $ (see entry: \texttt{DualArea}). \
To do this, we will sum the dual areas to each tetrahedron containing the
edge $eij$. \ 

\bigskip

\texttt{vector\TEXTsymbol{<}int\TEXTsymbol{>} sum\_over =
*(eij.getLocalTetras());}

\texttt{double sum = 0.0;}

\texttt{vector\TEXTsymbol{<}int\TEXTsymbol{>} T\_vertices, e\_vertices;}

\texttt{Tetra T;}

\texttt{Vertex vi,vj,vk,vl;}

\texttt{for(i=0; i\TEXTsymbol{<}sum\_over.size(); ++i) \{}

\qquad\texttt{T = Triangulation::tetraTable[sum\_over[i]];}

\qquad\texttt{T\_vertices = *(T.getLocalVertices());}

\qquad\texttt{e\_vertices = *(eij.getLocalVertices());}

\qquad\texttt{vi = Triangulation::vertexTable[e\_vertices[0]];}

\qquad\texttt{vj = Triangulation::vertexTable[e\_vertices[1]];}

\qquad\texttt{vk = Triangulation::vertexTable[listDifference(\&T\_vertices,
\&e\_vertices)[0]];}

\qquad\texttt{vl = Triangulation::vertexTable[listDifference(\&T\_vertices,
\&e\_vertices)[1]];}

\qquad \texttt{sum += DualAreaSegment(vi, vj, vk, vl);}

\qquad\texttt{\}}

\texttt{return sum;}

\subsection*{Limitations}

\texttt{DualAreaSegment} if fully operational and has no known limitations.
\ The function will output appropriate values provided it receives as input
four distinct vertices that define a tetrahedron.

\subsection*{Revisions}

subversion 757, 7/8/09, \texttt{DualAreaSegment} created.

subversion 1055, 3/12/10, \texttt{DualAreaSegment}\ converted to a geoquant.

\subsection*{Testing}

Trials were run and the calculations returned were verified by hand.

\subsection*{Future Work}

No future work planned.
}}%
%BeginExpansion
%TCIDATA{Version=5.00.0.2606}
%TCIDATA{LaTeXparent=1,1,functions.tex}
                      

\section*{\texttt{DualAreaSegment::DualAreaSegment}}

\subsection*{Function Prototype}

\texttt{double DualAreaSegment( Vertex vi, Vertex vj, Vertex vk, Vertex vl)}

\subsection*{Key Words}

Dual area, curvature, partial derivative, Einstein-Hilbert-Regge, geoquant.

\subsection*{Authors}

Daniel Champion

\subsection*{Introduction}

\texttt{DualAreaSegment} calculates the dual area to an edge of a
tetrahedron. \ 

\subsection*{Subsidiaries}

\textbf{Functions:}

\qquad \texttt{EdgeHeight}

\qquad \qquad \texttt{PartialEdge}

\qquad\qquad\texttt{Geometry::angle}

\qquad \texttt{FaceHeight}

\qquad\qquad\texttt{Geometry::dihedralAngle}

\textbf{Global Variables: \ }radii, etas

\textbf{Local Variables:} \ none.

\subsection*{Description}

\texttt{DualAreaSegment} is calculated with the formula:%
\begin{equation*}
\text{\texttt{DualAreaSegment(vi, vj, vk, vl)}}=
\end{equation*}%
\begin{equation*}
\frac{1}{2}\left( 
\begin{array}{c}
\text{\texttt{EdgeHeight(vi,vj,vk)}}\cdot \text{\texttt{%
FaceHeight(vi,vj,vk,vl)}} \\ 
+\text{\texttt{EdgeHeight(vi,vj,vl)}}\cdot \text{\texttt{%
FaceHeight(vi,vj,vl,vk)}}%
\end{array}%
\right) 
\end{equation*}%
\texttt{EdgeHeight} and \texttt{FaceHeight} are calculated with the
following formulae:%
\begin{align*}
\text{\texttt{EdgeHeight(vi, vj, vk)}}& =\frac{\left( \text{\texttt{%
PartialEdge(vi,vk)}}-\text{\texttt{PartialEdge(vi,vj)}}\cos \left( \alpha
_{i,jk}\right) \right) }{\sin \left( \alpha _{i,jk}\right) } \\
\text{\texttt{FaceHeight(vi, vj, vk, vl)}}& =\frac{\left( \text{\texttt{%
EdgeHeight(vi,vj,vl)}}-\text{\texttt{EdgeHeight(vi,vj,vk)}}\cos (\beta
_{ij,kl})\right) }{\sin \left( \beta _{ij,kl}\right) }
\end{align*}%
where $\alpha _{i,jk}$ is the angle at vertex $vi$ of triangle $\left\{
vi,vj,vk\right\} $, and $\beta _{ij,kl}$ is the dihedral angle along edge $%
\left\{ vi,vj\right\} $ of tetrahedron $\left\{ vi,vj,vk,vl\right\} $
(implemented with the functions \texttt{Geometry::angle} and \texttt{%
Geometry::dihedralAngle} respectively).

\texttt{DualAreaSegment} was created for the calculation performed in the
function \texttt{DualArea}, which is used in the computation of the partial
derivatives of curvature. \ These partial derivatives of curvature are used
in the calculation of the second order partial derivatives of the
Einstein-Hilbert-Regge functional for use in the optimization of said
functional using Newton's method. \ 

\subsection*{Practicum}

As an example of the usage of this function, we will calculate the dual area
to the edge $eij=\left\{ vi,vj\right\} $ (see entry: \texttt{DualArea}). \
To do this, we will sum the dual areas to each tetrahedron containing the
edge $eij$. \ 

\bigskip

\texttt{vector\TEXTsymbol{<}int\TEXTsymbol{>} sum\_over =
*(eij.getLocalTetras());}

\texttt{double sum = 0.0;}

\texttt{vector\TEXTsymbol{<}int\TEXTsymbol{>} T\_vertices, e\_vertices;}

\texttt{Tetra T;}

\texttt{Vertex vi,vj,vk,vl;}

\texttt{for(i=0; i\TEXTsymbol{<}sum\_over.size(); ++i) \{}

\qquad\texttt{T = Triangulation::tetraTable[sum\_over[i]];}

\qquad\texttt{T\_vertices = *(T.getLocalVertices());}

\qquad\texttt{e\_vertices = *(eij.getLocalVertices());}

\qquad\texttt{vi = Triangulation::vertexTable[e\_vertices[0]];}

\qquad\texttt{vj = Triangulation::vertexTable[e\_vertices[1]];}

\qquad\texttt{vk = Triangulation::vertexTable[listDifference(\&T\_vertices,
\&e\_vertices)[0]];}

\qquad\texttt{vl = Triangulation::vertexTable[listDifference(\&T\_vertices,
\&e\_vertices)[1]];}

\qquad \texttt{sum += DualAreaSegment(vi, vj, vk, vl);}

\qquad\texttt{\}}

\texttt{return sum;}

\subsection*{Limitations}

\texttt{DualAreaSegment} if fully operational and has no known limitations.
\ The function will output appropriate values provided it receives as input
four distinct vertices that define a tetrahedron.

\subsection*{Revisions}

subversion 757, 7/8/09, \texttt{DualAreaSegment} created.

subversion 1055, 3/12/10, \texttt{DualAreaSegment}\ converted to a geoquant.

\subsection*{Testing}

Trials were run and the calculations returned were verified by hand.

\subsection*{Future Work}

No future work planned.
%
%EndExpansion

\bigskip

\bigskip 

%TCIMACRO{\QSubDoc{Include ApproximatorRun}{%html2tex: Version  2.7 of June 17, 2008.
%Written by  F.J. Faase.  http://www.iwriteiam.nl/

clean.html (42) : unknown <strike>.
clean.html (42) : unknown </strike>.
clean.html (42) : unknown <strike>.
clean.html (42) : unknown </strike>.
\documentclass[10pt]{article}%
\usepackage{amssymb}
\usepackage{geometry}
\usepackage{indentfirst}
\usepackage{amsmath}
\usepackage{amsfonts}
\usepackage{graphicx}%
\setcounter{MaxMatrixCols}{30}
%TCIDATA{OutputFilter=latex2.dll}
%TCIDATA{Version=5.00.0.2606}
%TCIDATA{CSTFile=40 LaTeX article.cst}
%TCIDATA{Created=Friday, March 30, 2007 00:21:27}
%TCIDATA{LastRevised=Wednesday, June 10, 2009 11:42:33}
%TCIDATA{<META NAME="GraphicsSave" CONTENT="32">}
%TCIDATA{<META NAME="SaveForMode" CONTENT="1">}
%TCIDATA{BibliographyScheme=Manual}
%TCIDATA{<META NAME="DocumentShell" CONTENT="Standard LaTeX\Blank - Standard LaTeX Article">}
%TCIDATA{Language=American English}
\newtheorem{theorem}{Theorem}
\newtheorem{acknowledgement}[theorem]{Acknowledgement}
\newtheorem{algorithm}[theorem]{Algorithm}
\newtheorem{axiom}[theorem]{Axiom}
\newtheorem{case}[theorem]{Case}
\newtheorem{claim}[theorem]{Claim}
\newtheorem{conclusion}[theorem]{Conclusion}
\newtheorem{condition}[theorem]{Condition}
\newtheorem{conjecture}[theorem]{Conjecture}
\newtheorem{corollary}[theorem]{Corollary}
\newtheorem{criterion}[theorem]{Criterion}
\newtheorem{definition}[theorem]{Definition}
\newtheorem{example}[theorem]{Example}
\newtheorem{exercise}[theorem]{Exercise}
\newtheorem{lemma}[theorem]{Lemma}
\newtheorem{notation}[theorem]{Notation}
\newtheorem{problem}[theorem]{Problem}
\newtheorem{proposition}[theorem]{Proposition}
\newtheorem{remark}[theorem]{Remark}
\newtheorem{solution}[theorem]{Solution}
\newtheorem{summary}[theorem]{Summary}
\newenvironment{proof}[1][Proof]{\noindent\textbf{#1.} }{\ \rule{0.5em}{0.5em}}
\geometry{left=1in,right=1in,top=1in,bottom=1in}

\begin{document}

%%%%% BEGINNING OF DOCUMENT BODY %%%%%
% html: Beginning of file: `clean.html'
% DOCTYPE HTML PUBLIC "-//W3C//DTD HTML 4.01//EN"
%  This is a (PRE) block.  Make sure it's left aligned or your toc title will be off. 

\section*{\texttt{Approximator::run}}

\label{f0}\begin{quotation} {\small{\begin{verbatim} 
int run(int numsteps, double stepsize)        
int run(double precision, double stepsize)
int run(double precision, int maxNumSteps, double stepsize)
  \end{verbatim}
}}
\end{quotation}
\subsection*{Keywords}

\begin{quotation} flow, curvature, stepsize, precision, accuracy, approximator\end{quotation}

\subsection*{Authors}

\begin{itemize}\item  Joseph Thomas
\item  Alex Henniges
\end{itemize}

\subsection*{Introduction}

\begin{quotation} The \texttt{run} function of the Approximator class runs a system of differential equations representing a curvature flow for either a number of steps or until the values are within a desired precision. The system to use and how steps are performed is given in the constructor of the approximator. The type of run is based on the parameters given. The \texttt{run} function returns 0 on success or any other number if an error is encountered.\end{quotation}

\subsection*{Subsidiaries}

\begin{quotation} Functions: \end{quotation}
\begin{itemize}
\item  \texttt{Approximator::step}
\item  \texttt{Approximator::isPrecise}
\item  \texttt{Approximator::isAccurate} 
\item  \texttt{Approximator::getLatest}
\begin{enumerate}
\item  \texttt{Approximator::recordState}
\end{enumerate}
\end{itemize}
\begin{quotation} Global Variables: \texttt{radii}, \texttt{curvatures}\end{quotation}
\begin{quotation} Local Variables: \texttt{int numsteps}, \texttt{double stepsize}, \texttt{double precision}, \texttt{double accuracy}\end{quotation}

\subsection*{Description}

\begin{quotation} If the \texttt{run} function is given a number of steps, it will call its step function that number of times. In between steps, the \texttt{run} function will record the current state of any values that have been requested to be recorded (this is specified in the constructor).\end{quotation}
\begin{quotation} If the \texttt{run} function is given a precision, it will continue to call its step function until the desired quantities (curvature in two dimensions and curvature divide by radius in three dimensions) have converged within the precision bounds. Precision is defined to be the difference between subsequent values of a quantity. Therefore, precision is a measure of how much a value is changing. In between steps, the \texttt{run} function will record the current state of any values that have been requested to be recorded (this is specified in the constructor).        \end{quotation}
\begin{quotation} A flow can also be run with a precision and a max number of steps that will stop once one of the conditions is reached. The last parameter of any run indicates the step size of the flow. A lower step size will lead to more accurate steps, but a longer time to convergence.\end{quotation}
\begin{quotation} The \texttt{run} function and the overarching Approximator class exists as an improvement over the curvature flows of earlier versions of the Geocam project. The \texttt{run} function provides the skeleton that is similar for all types of curvature flows. Beyond the constructor, this should be the only thing a user calls from the Approximator class.\end{quotation}

\subsection*{Practicum}

\begin{quotation} Example:\end{quotation}{\small{\begin{verbatim} 
// Create an approximator that uses the Euler method on a Yamabe flow.
Approximator *app = new EulerApprox(Yamabe);

// Run a Yamabe flow for 300 steps with a stepsize of 0.01.
app->run(300, 0.01);
// Run with a precision bound of 0.000001 and a stepsize of 0.01
app->run(0.000001, 0.01);
\end{verbatim}
}}

\subsection*{Limitations}

\begin{quotation} The \texttt{run} function is limited in the systems of differential equations that it can run. It is designed to run with curvature flows and, when precision is used, expects the values to converge. If a precision run is performed on a flow that does not converge, the \texttt{run} function will not stop. If a new curvature flow is created whose convergence is not the usual (as in curvature divided by radius in Yamabe flow) then the \texttt{run} function will have to be modified to accommodate for this.\end{quotation}

\subsection*{Revisions}

\begin{itemize}\item  subversion 659, 5/1/09: Initial \texttt{run} function uploaded to the code.
\item  subversion 679, 6/3/09: \texttt{run} function modified to work with new Geometry structure.
\item  subversion 761, 6/12/09: \texttt{run} function modified to work with new quantity structure.
\item  subversion 787, 6/18/09: Added new \texttt{run} options to approximator. Removed accuracy. Checks for bad numbers.
\end{itemize}

\subsection*{Testing}

\begin{quotation} The function was tested by performing two and three dimensional flows on familiar triangulations. The start and end values for radii and curvature was then compared with our expected values. The expected values were obtained from the earlier curvature flows we had (see \mbox{$[$}\#Description Description\mbox{$]$} above). We also checked that the end values were within the precision and accuracy bounds when they were in effect. \end{quotation}

\subsection*{Future Work}

\begin{itemize}\item  6/17 - 
% <strike name="Future Work">
Add more run options (ex. precision and maxNumSteps).
% </strike name="Future Work">
 \textbf{Complete (6/18)}
\item  6/17 - 
% <strike name="Future Work">
Have a run stop the moment an undefined number appears.
% </strike name="Future Work">
 \textbf{Complete (6/18)}
\end{itemize}
\begin{quotation} No future work is planned at this time.\end{quotation}
    
% html: End of file: `clean.html'

%%%%% END OF DOCUMENT BODY %%%%%
% In the future, we might want to put some additional data here, such
% as when the documentation was converted from wiki to TeX.
%

\end{document}
}}%
%BeginExpansion
%html2tex: Version  2.7 of June 17, 2008.
%Written by  F.J. Faase.  http://www.iwriteiam.nl/

clean.html (42) : unknown <strike>.
clean.html (42) : unknown </strike>.
clean.html (42) : unknown <strike>.
clean.html (42) : unknown </strike>.
\documentclass[10pt]{article}%
\usepackage{amssymb}
\usepackage{geometry}
\usepackage{indentfirst}
\usepackage{amsmath}
\usepackage{amsfonts}
\usepackage{graphicx}%
\setcounter{MaxMatrixCols}{30}
%TCIDATA{OutputFilter=latex2.dll}
%TCIDATA{Version=5.00.0.2606}
%TCIDATA{CSTFile=40 LaTeX article.cst}
%TCIDATA{Created=Friday, March 30, 2007 00:21:27}
%TCIDATA{LastRevised=Wednesday, June 10, 2009 11:42:33}
%TCIDATA{<META NAME="GraphicsSave" CONTENT="32">}
%TCIDATA{<META NAME="SaveForMode" CONTENT="1">}
%TCIDATA{BibliographyScheme=Manual}
%TCIDATA{<META NAME="DocumentShell" CONTENT="Standard LaTeX\Blank - Standard LaTeX Article">}
%TCIDATA{Language=American English}
\newtheorem{theorem}{Theorem}
\newtheorem{acknowledgement}[theorem]{Acknowledgement}
\newtheorem{algorithm}[theorem]{Algorithm}
\newtheorem{axiom}[theorem]{Axiom}
\newtheorem{case}[theorem]{Case}
\newtheorem{claim}[theorem]{Claim}
\newtheorem{conclusion}[theorem]{Conclusion}
\newtheorem{condition}[theorem]{Condition}
\newtheorem{conjecture}[theorem]{Conjecture}
\newtheorem{corollary}[theorem]{Corollary}
\newtheorem{criterion}[theorem]{Criterion}
\newtheorem{definition}[theorem]{Definition}
\newtheorem{example}[theorem]{Example}
\newtheorem{exercise}[theorem]{Exercise}
\newtheorem{lemma}[theorem]{Lemma}
\newtheorem{notation}[theorem]{Notation}
\newtheorem{problem}[theorem]{Problem}
\newtheorem{proposition}[theorem]{Proposition}
\newtheorem{remark}[theorem]{Remark}
\newtheorem{solution}[theorem]{Solution}
\newtheorem{summary}[theorem]{Summary}
\newenvironment{proof}[1][Proof]{\noindent\textbf{#1.} }{\ \rule{0.5em}{0.5em}}
\geometry{left=1in,right=1in,top=1in,bottom=1in}

\begin{document}

%%%%% BEGINNING OF DOCUMENT BODY %%%%%
% html: Beginning of file: `clean.html'
% DOCTYPE HTML PUBLIC "-//W3C//DTD HTML 4.01//EN"
%  This is a (PRE) block.  Make sure it's left aligned or your toc title will be off. 

\section*{\texttt{Approximator::run}}

\label{f0}\begin{quotation} {\small{\begin{verbatim} 
int run(int numsteps, double stepsize)        
int run(double precision, double stepsize)
int run(double precision, int maxNumSteps, double stepsize)
  \end{verbatim}
}}
\end{quotation}
\subsection*{Keywords}

\begin{quotation} flow, curvature, stepsize, precision, accuracy, approximator\end{quotation}

\subsection*{Authors}

\begin{itemize}\item  Joseph Thomas
\item  Alex Henniges
\end{itemize}

\subsection*{Introduction}

\begin{quotation} The \texttt{run} function of the Approximator class runs a system of differential equations representing a curvature flow for either a number of steps or until the values are within a desired precision. The system to use and how steps are performed is given in the constructor of the approximator. The type of run is based on the parameters given. The \texttt{run} function returns 0 on success or any other number if an error is encountered.\end{quotation}

\subsection*{Subsidiaries}

\begin{quotation} Functions: \end{quotation}
\begin{itemize}
\item  \texttt{Approximator::step}
\item  \texttt{Approximator::isPrecise}
\item  \texttt{Approximator::isAccurate} 
\item  \texttt{Approximator::getLatest}
\begin{enumerate}
\item  \texttt{Approximator::recordState}
\end{enumerate}
\end{itemize}
\begin{quotation} Global Variables: \texttt{radii}, \texttt{curvatures}\end{quotation}
\begin{quotation} Local Variables: \texttt{int numsteps}, \texttt{double stepsize}, \texttt{double precision}, \texttt{double accuracy}\end{quotation}

\subsection*{Description}

\begin{quotation} If the \texttt{run} function is given a number of steps, it will call its step function that number of times. In between steps, the \texttt{run} function will record the current state of any values that have been requested to be recorded (this is specified in the constructor).\end{quotation}
\begin{quotation} If the \texttt{run} function is given a precision, it will continue to call its step function until the desired quantities (curvature in two dimensions and curvature divide by radius in three dimensions) have converged within the precision bounds. Precision is defined to be the difference between subsequent values of a quantity. Therefore, precision is a measure of how much a value is changing. In between steps, the \texttt{run} function will record the current state of any values that have been requested to be recorded (this is specified in the constructor).        \end{quotation}
\begin{quotation} A flow can also be run with a precision and a max number of steps that will stop once one of the conditions is reached. The last parameter of any run indicates the step size of the flow. A lower step size will lead to more accurate steps, but a longer time to convergence.\end{quotation}
\begin{quotation} The \texttt{run} function and the overarching Approximator class exists as an improvement over the curvature flows of earlier versions of the Geocam project. The \texttt{run} function provides the skeleton that is similar for all types of curvature flows. Beyond the constructor, this should be the only thing a user calls from the Approximator class.\end{quotation}

\subsection*{Practicum}

\begin{quotation} Example:\end{quotation}{\small{\begin{verbatim} 
// Create an approximator that uses the Euler method on a Yamabe flow.
Approximator *app = new EulerApprox(Yamabe);

// Run a Yamabe flow for 300 steps with a stepsize of 0.01.
app->run(300, 0.01);
// Run with a precision bound of 0.000001 and a stepsize of 0.01
app->run(0.000001, 0.01);
\end{verbatim}
}}

\subsection*{Limitations}

\begin{quotation} The \texttt{run} function is limited in the systems of differential equations that it can run. It is designed to run with curvature flows and, when precision is used, expects the values to converge. If a precision run is performed on a flow that does not converge, the \texttt{run} function will not stop. If a new curvature flow is created whose convergence is not the usual (as in curvature divided by radius in Yamabe flow) then the \texttt{run} function will have to be modified to accommodate for this.\end{quotation}

\subsection*{Revisions}

\begin{itemize}\item  subversion 659, 5/1/09: Initial \texttt{run} function uploaded to the code.
\item  subversion 679, 6/3/09: \texttt{run} function modified to work with new Geometry structure.
\item  subversion 761, 6/12/09: \texttt{run} function modified to work with new quantity structure.
\item  subversion 787, 6/18/09: Added new \texttt{run} options to approximator. Removed accuracy. Checks for bad numbers.
\end{itemize}

\subsection*{Testing}

\begin{quotation} The function was tested by performing two and three dimensional flows on familiar triangulations. The start and end values for radii and curvature was then compared with our expected values. The expected values were obtained from the earlier curvature flows we had (see \mbox{$[$}\#Description Description\mbox{$]$} above). We also checked that the end values were within the precision and accuracy bounds when they were in effect. \end{quotation}

\subsection*{Future Work}

\begin{itemize}\item  6/17 - 
% <strike name="Future Work">
Add more run options (ex. precision and maxNumSteps).
% </strike name="Future Work">
 \textbf{Complete (6/18)}
\item  6/17 - 
% <strike name="Future Work">
Have a run stop the moment an undefined number appears.
% </strike name="Future Work">
 \textbf{Complete (6/18)}
\end{itemize}
\begin{quotation} No future work is planned at this time.\end{quotation}
    
% html: End of file: `clean.html'

%%%%% END OF DOCUMENT BODY %%%%%
% In the future, we might want to put some additional data here, such
% as when the documentation was converted from wiki to TeX.
%

\end{document}
%
%EndExpansion

\bigskip 

\bigskip 

%TCIMACRO{%
%\QSubDoc{Include Curvature_Partial}{%TCIDATA{Version=5.00.0.2606}
%TCIDATA{LaTeXparent=1,1,functions.tex}
                      

\section*{\texttt{CurvaturePartial::CurvaturePartial}}

\subsection*{Function Prototype}

\texttt{double CurvaturePartial( Vertex v\_i, Vertex v\_l )}

\subsection*{Key Words}

Curvature, Einstein-Hilbert-Regge, functional, partial derivative, geoquant.

\subsection*{Authors}

Daniel Champion

\subsection*{Introduction}

CurvaturePartial calculates the partial derivative of the curvature at a
vertex with respect to the log radius of another (possibly the same) vertex.
\ 

\subsection*{Subsidiaries}

Functions:

\qquad\texttt{isAdjVertex}

\qquad \texttt{DualArea}

\qquad \qquad \texttt{DualAreaSegment}

\qquad \qquad \qquad \texttt{FaceHeight}

\qquad \qquad \qquad \qquad \texttt{EdgeHeight}

\qquad \qquad \qquad \qquad \qquad \texttt{PartialEdge}

\qquad\texttt{listDifference}

\qquad\texttt{listIntersection}

Global Variables: \ curvature, dihedralAngle, eta, length, radius

Local Variables: none.

\subsection*{Description}

\texttt{CurvaturePartial} receives as inputs two vertices $v_{i}$ and $v_{l}$%
. \ The first corresponds to the vertex of interest, the second corresponds
to the vertex of differentiation. \ That is,%
\begin{equation*}
\text{\texttt{CurvaturePartial (v\_i, v\_l)}}=\frac{\partial }{\partial \log
r_{l}}K_{i},
\end{equation*}%
where $r_{l}$ is the radius at vertex $v_{l}$, and $K_{i}$ is the curvature
at vertex $v_{i}$. \ 

The function begins implementation by determining the relationship between $i
$ and $l$ via the trichotomy $v_{i}=v_{l}$, $v_{i}$ is adjacent to $v_{l}$,
or $v_{i}$ and $v_{l}$ are not endpoints of any edge. \ Each of the three
cases are calculated differently. \ The general formula for the variation of
curvature w.r.t. log radii was calculated by Prof. David Glickenstein and is
available at arXiv:0906.1560v1:%
\begin{equation*}
\delta K_{i}=-\sum_{edges\text{ }\left\{ i,j\right\} }\left( 2\frac{%
l_{ij}^{\ast }}{l_{ij}}-\frac{r_{i}^{2}r_{j}^{2}\left( 1-\eta
_{ij}^{2}\right) }{l_{ij}^{2}}K_{ij}\right) \left( \delta f_{j}-\delta
f_{i}\right) +K_{i}\delta f_{i},
\end{equation*}%
where $f_{i}=\log r_{i}$, $l_{ij}$ is the length of the edge $\left\{
i,j\right\} $, and $l_{ij}^{\ast }$ is the dual area calculated with the
function \texttt{DualArea}.

When $i=l$, the formula for the partial derivative $\frac{\partial}{%
\partial\log r_{l}}K_{i}$ becomes:%
\begin{equation*}
\frac{\partial}{\partial\log r_{i}}K_{i}=\sum_{edges\text{ }\left\{
i,j\right\} }\left( 2\frac{l_{ij}^{\ast}}{l_{ij}}-\frac{r_{i}^{2}r_{j}^{2}%
\left( 1-\eta_{ij}^{2}\right) }{l_{ij}^{2}}K_{ij}\right) +K_{i}.
\end{equation*}

When $v_{i}$ is adjacent to $v_{l}$ only one term in the sum survives:%
\begin{equation*}
\frac{\partial}{\partial\log r_{l}}K_{i}=-\left( 2\frac{l_{il}^{\ast}}{l_{il}%
}-\frac{r_{i}^{2}r_{l}^{2}\left( 1-\eta_{il}^{2}\right) }{l_{il}^{2}}%
K_{il}\right) .
\end{equation*}

When $v_{i}$ and $v_{j}$ are not endpoints of any edge the partial
derivative is zero. \ 

This function was created to assist in the computation of the second
derivatives of the normalized Einstein-Hilbert-Regge functional:%
\begin{equation*}
EHR=\frac{\sum K_{i}}{\sqrt[3]{\sum\limits_{tetra\text{ }t}Vol(t)}}.
\end{equation*}

Surprisingly the first derivatives of the normalized $EHR$ functional do not
require the formula for the partial derivative of curvature since 
\begin{equation*}
\frac{\partial EHR}{\partial\log r_{i}}=K_{i}.
\end{equation*}

However, the second order partial derivatives of $EHR$ certainly require the
formulas for $\frac{\partial}{\partial\log r_{l}}K_{i}$ given above. \ These
second order partial derivatives are used to construct a Hessian matrix
which is then used the optimization of the $EHR$ functional using Newton's
method (implemented by \texttt{Newtons\_Method}).

\subsection*{Practicum}

When called, \texttt{CurvaturePartial(v\_i, v\_l)} returns the partial
derivative $\frac{\partial }{\partial \log r_{l}}K_{i}$. \ An example of its
usage is in the calculation of the second order partial derivatives of the
normalized $EHR$ functional. \ For this example let 
\begin{align*}
\text{\texttt{VolSumPartial\_i}}& =\sum_{tetra\text{ }t}\frac{\partial V_{t}%
}{\partial \log r_{i}}, \\
\text{\texttt{VolSumPartial\_j}}& =\sum_{tetra\text{ }t}\frac{\partial V_{t}%
}{\partial \log r_{j}}, \\
\text{\texttt{VolSumSecondPartial}}& =\sum_{tetra\text{ }t}\frac{\partial
^{2}V_{t}}{\partial \log r_{i}\partial \log r_{j}} \\
K& =\sum_{i}K_{i}, \\
V& =\sum_{tetra\text{ }t}V_{t}.
\end{align*}%
Then the second order partial derivative $\frac{\partial ^{2}EHR}{\partial
\log r_{i}\partial \log r_{j}}$ is calculated by:

\bigskip

\qquad \texttt{result = pow(V,
(-4.0/3.0))*(1.0/3.0)*(3*V*CurvaturePartial(i,j)}

\qquad\qquad\qquad \texttt{%
-Geometry::curvature(Triangulation::vertexTable[i])*VolSumPartial\_j}

\qquad\qquad\qquad \texttt{%
-Geometry::curvature(Triangulation::vertexTable[j])*VolSumPartial\_i}

\qquad\qquad\qquad\texttt{+(4.0/3.0)*K*pow(V,
-1.0)*VolSumPartial\_i*VolSumPartial\_j}

\qquad\qquad\qquad\texttt{-K*VolSumSecondPartial);}

\bigskip

\subsection*{Limitations}

The function \texttt{CurvaturePartial} is operational for all pairs of input
integers $i$ and $l$ that are in the vertex table. \ If one of the arguments
is not in the vertex table, the function will output zero. \ 

\subsection*{Revisions}

subversion 757, 7/6/09, \texttt{CurvaturePartial} created.

subversion 1055, 3/12/10, \texttt{CurvaturePartial}\ converted to a geoquant.

\subsection*{Testing}

This function has not been tested.

\subsection*{Future Work}

Using the calculation of the partial derivative of curvature in Mathematica,
it should be compared to the output from \texttt{CurvaturePartial}.
}}%
%BeginExpansion
%TCIDATA{Version=5.00.0.2606}
%TCIDATA{LaTeXparent=1,1,functions.tex}
                      

\section*{\texttt{CurvaturePartial::CurvaturePartial}}

\subsection*{Function Prototype}

\texttt{double CurvaturePartial( Vertex v\_i, Vertex v\_l )}

\subsection*{Key Words}

Curvature, Einstein-Hilbert-Regge, functional, partial derivative, geoquant.

\subsection*{Authors}

Daniel Champion

\subsection*{Introduction}

CurvaturePartial calculates the partial derivative of the curvature at a
vertex with respect to the log radius of another (possibly the same) vertex.
\ 

\subsection*{Subsidiaries}

Functions:

\qquad\texttt{isAdjVertex}

\qquad \texttt{DualArea}

\qquad \qquad \texttt{DualAreaSegment}

\qquad \qquad \qquad \texttt{FaceHeight}

\qquad \qquad \qquad \qquad \texttt{EdgeHeight}

\qquad \qquad \qquad \qquad \qquad \texttt{PartialEdge}

\qquad\texttt{listDifference}

\qquad\texttt{listIntersection}

Global Variables: \ curvature, dihedralAngle, eta, length, radius

Local Variables: none.

\subsection*{Description}

\texttt{CurvaturePartial} receives as inputs two vertices $v_{i}$ and $v_{l}$%
. \ The first corresponds to the vertex of interest, the second corresponds
to the vertex of differentiation. \ That is,%
\begin{equation*}
\text{\texttt{CurvaturePartial (v\_i, v\_l)}}=\frac{\partial }{\partial \log
r_{l}}K_{i},
\end{equation*}%
where $r_{l}$ is the radius at vertex $v_{l}$, and $K_{i}$ is the curvature
at vertex $v_{i}$. \ 

The function begins implementation by determining the relationship between $i
$ and $l$ via the trichotomy $v_{i}=v_{l}$, $v_{i}$ is adjacent to $v_{l}$,
or $v_{i}$ and $v_{l}$ are not endpoints of any edge. \ Each of the three
cases are calculated differently. \ The general formula for the variation of
curvature w.r.t. log radii was calculated by Prof. David Glickenstein and is
available at arXiv:0906.1560v1:%
\begin{equation*}
\delta K_{i}=-\sum_{edges\text{ }\left\{ i,j\right\} }\left( 2\frac{%
l_{ij}^{\ast }}{l_{ij}}-\frac{r_{i}^{2}r_{j}^{2}\left( 1-\eta
_{ij}^{2}\right) }{l_{ij}^{2}}K_{ij}\right) \left( \delta f_{j}-\delta
f_{i}\right) +K_{i}\delta f_{i},
\end{equation*}%
where $f_{i}=\log r_{i}$, $l_{ij}$ is the length of the edge $\left\{
i,j\right\} $, and $l_{ij}^{\ast }$ is the dual area calculated with the
function \texttt{DualArea}.

When $i=l$, the formula for the partial derivative $\frac{\partial}{%
\partial\log r_{l}}K_{i}$ becomes:%
\begin{equation*}
\frac{\partial}{\partial\log r_{i}}K_{i}=\sum_{edges\text{ }\left\{
i,j\right\} }\left( 2\frac{l_{ij}^{\ast}}{l_{ij}}-\frac{r_{i}^{2}r_{j}^{2}%
\left( 1-\eta_{ij}^{2}\right) }{l_{ij}^{2}}K_{ij}\right) +K_{i}.
\end{equation*}

When $v_{i}$ is adjacent to $v_{l}$ only one term in the sum survives:%
\begin{equation*}
\frac{\partial}{\partial\log r_{l}}K_{i}=-\left( 2\frac{l_{il}^{\ast}}{l_{il}%
}-\frac{r_{i}^{2}r_{l}^{2}\left( 1-\eta_{il}^{2}\right) }{l_{il}^{2}}%
K_{il}\right) .
\end{equation*}

When $v_{i}$ and $v_{j}$ are not endpoints of any edge the partial
derivative is zero. \ 

This function was created to assist in the computation of the second
derivatives of the normalized Einstein-Hilbert-Regge functional:%
\begin{equation*}
EHR=\frac{\sum K_{i}}{\sqrt[3]{\sum\limits_{tetra\text{ }t}Vol(t)}}.
\end{equation*}

Surprisingly the first derivatives of the normalized $EHR$ functional do not
require the formula for the partial derivative of curvature since 
\begin{equation*}
\frac{\partial EHR}{\partial\log r_{i}}=K_{i}.
\end{equation*}

However, the second order partial derivatives of $EHR$ certainly require the
formulas for $\frac{\partial}{\partial\log r_{l}}K_{i}$ given above. \ These
second order partial derivatives are used to construct a Hessian matrix
which is then used the optimization of the $EHR$ functional using Newton's
method (implemented by \texttt{Newtons\_Method}).

\subsection*{Practicum}

When called, \texttt{CurvaturePartial(v\_i, v\_l)} returns the partial
derivative $\frac{\partial }{\partial \log r_{l}}K_{i}$. \ An example of its
usage is in the calculation of the second order partial derivatives of the
normalized $EHR$ functional. \ For this example let 
\begin{align*}
\text{\texttt{VolSumPartial\_i}}& =\sum_{tetra\text{ }t}\frac{\partial V_{t}%
}{\partial \log r_{i}}, \\
\text{\texttt{VolSumPartial\_j}}& =\sum_{tetra\text{ }t}\frac{\partial V_{t}%
}{\partial \log r_{j}}, \\
\text{\texttt{VolSumSecondPartial}}& =\sum_{tetra\text{ }t}\frac{\partial
^{2}V_{t}}{\partial \log r_{i}\partial \log r_{j}} \\
K& =\sum_{i}K_{i}, \\
V& =\sum_{tetra\text{ }t}V_{t}.
\end{align*}%
Then the second order partial derivative $\frac{\partial ^{2}EHR}{\partial
\log r_{i}\partial \log r_{j}}$ is calculated by:

\bigskip

\qquad \texttt{result = pow(V,
(-4.0/3.0))*(1.0/3.0)*(3*V*CurvaturePartial(i,j)}

\qquad\qquad\qquad \texttt{%
-Geometry::curvature(Triangulation::vertexTable[i])*VolSumPartial\_j}

\qquad\qquad\qquad \texttt{%
-Geometry::curvature(Triangulation::vertexTable[j])*VolSumPartial\_i}

\qquad\qquad\qquad\texttt{+(4.0/3.0)*K*pow(V,
-1.0)*VolSumPartial\_i*VolSumPartial\_j}

\qquad\qquad\qquad\texttt{-K*VolSumSecondPartial);}

\bigskip

\subsection*{Limitations}

The function \texttt{CurvaturePartial} is operational for all pairs of input
integers $i$ and $l$ that are in the vertex table. \ If one of the arguments
is not in the vertex table, the function will output zero. \ 

\subsection*{Revisions}

subversion 757, 7/6/09, \texttt{CurvaturePartial} created.

subversion 1055, 3/12/10, \texttt{CurvaturePartial}\ converted to a geoquant.

\subsection*{Testing}

This function has not been tested.

\subsection*{Future Work}

Using the calculation of the partial derivative of curvature in Mathematica,
it should be compared to the output from \texttt{CurvaturePartial}.
%
%EndExpansion

\bigskip

\bigskip

%TCIMACRO{\QSubDoc{Include dij}{%TCIDATA{Version=5.00.0.2606}
%TCIDATA{LaTeXparent=1,1,functions.tex}
                      

\section*{\texttt{PartialEdge::PartialEdge}}

\subsection*{Function Prototype}

\texttt{double PartialEdge( Vertex vi, Vertex vj)}

\subsection*{Key Words}

Partial length, geoquant.

\subsection*{Authors}

Daniel Champion, ???

\subsection*{Introduction}

The function \texttt{PartialEdge} calculates the distance from a vertex to
the center of an edge (as determined by the center of a decorated triangle).

\subsection*{Subsidiaries}

\textbf{Functions:}

\qquad\texttt{Geometry::length}

\qquad\texttt{listIntersection}

\textbf{Global Variables:} \ radii, etas.

\textbf{Local Variables:} \ Vertex vi, vj.

\subsection*{Description}

The function \texttt{PartialEdge} is calculated with the simple formula:%
\begin{equation*}
\mathtt{PartialEdge}\text{\texttt{(vi,vj)}}=\frac{%
L_{ij}^{2}+r_{i}^{2}-r_{j}^{2}}{2L_{ij}},
\end{equation*}%
where $r_{i},r_{j}$ are the radii at vertices \texttt{vi}, and \texttt{vj}
respectively, and $L_{ij}$ is the length of the edge $\left\{ vi,vj\right\} $%
. \ Notice that this formula is not symmetric in $i$ and $j$. \ 

This function plays an important role in several areas of the project
including curvature, Dirichlet energy, and the optimization of the
Einstein-Hilbert-Regge functional. \ \texttt{PartialEdge} is used in the
calculation of several quantities used in the implementation of the \texttt{%
CurvaturePartial} function, which is used in the optimization of the
normalized Einstein-Hilbert-Regge functional.

\subsection*{Practicum}

An example of the use of this function is in the calculation of the edge
height function \texttt{EdgeHeight}:

\qquad \texttt{double EdgeHeight( Vertex vi, Vertex vj, Vertex vk) \{}

\qquad\qquad\texttt{Face fijk;}

\qquad\qquad\texttt{vector\TEXTsymbol{<}int\TEXTsymbol{>} temp\_ij = }

\qquad\qquad\qquad\texttt{listIntersection(vi.getLocalFaces(),
vj.getLocalFaces());}

\qquad\qquad\texttt{vector\TEXTsymbol{<}int\TEXTsymbol{>} temp = }

\qquad\qquad\qquad\texttt{listIntersection( \&temp\_ij, vk.getLocalFaces());}

\qquad\qquad\texttt{fijk = Triangulation::faceTable[temp[0]];}

\qquad \qquad \texttt{double result = (PartialEdge(vi, vk)-PartialEdge(vi,vj)%
}

\qquad\qquad\qquad\texttt{*cos(Geometry::angle(vi,
fijk)))/sin(Geometry::angle(vi, fijk));}

\qquad\qquad\texttt{return result;}

\qquad\qquad\texttt{\}}

\subsection*{Limitations}

\texttt{PartialEdge} must receive as input two vertices that define an edge
in the triangulation. \ \texttt{PartialEdge} returns distinct values for
each permutation of the input vertices. \ 

\subsection*{Revisions}

subversion 757, 7/13/09, \texttt{PartialEdge} created.

subversion 1055, 3/12/10, \texttt{PartialEdge}\ converted to a geoquant.

\subsection*{Testing}

\texttt{dij} was tested by working out several examples by hand.

\subsection*{Future Work}

This function has been added to the Geometry class geoquants, and thus this
entry needs to be updated.
}}%
%BeginExpansion
%TCIDATA{Version=5.00.0.2606}
%TCIDATA{LaTeXparent=1,1,functions.tex}
                      

\section*{\texttt{PartialEdge::PartialEdge}}

\subsection*{Function Prototype}

\texttt{double PartialEdge( Vertex vi, Vertex vj)}

\subsection*{Key Words}

Partial length, geoquant.

\subsection*{Authors}

Daniel Champion, ???

\subsection*{Introduction}

The function \texttt{PartialEdge} calculates the distance from a vertex to
the center of an edge (as determined by the center of a decorated triangle).

\subsection*{Subsidiaries}

\textbf{Functions:}

\qquad\texttt{Geometry::length}

\qquad\texttt{listIntersection}

\textbf{Global Variables:} \ radii, etas.

\textbf{Local Variables:} \ Vertex vi, vj.

\subsection*{Description}

The function \texttt{PartialEdge} is calculated with the simple formula:%
\begin{equation*}
\mathtt{PartialEdge}\text{\texttt{(vi,vj)}}=\frac{%
L_{ij}^{2}+r_{i}^{2}-r_{j}^{2}}{2L_{ij}},
\end{equation*}%
where $r_{i},r_{j}$ are the radii at vertices \texttt{vi}, and \texttt{vj}
respectively, and $L_{ij}$ is the length of the edge $\left\{ vi,vj\right\} $%
. \ Notice that this formula is not symmetric in $i$ and $j$. \ 

This function plays an important role in several areas of the project
including curvature, Dirichlet energy, and the optimization of the
Einstein-Hilbert-Regge functional. \ \texttt{PartialEdge} is used in the
calculation of several quantities used in the implementation of the \texttt{%
CurvaturePartial} function, which is used in the optimization of the
normalized Einstein-Hilbert-Regge functional.

\subsection*{Practicum}

An example of the use of this function is in the calculation of the edge
height function \texttt{EdgeHeight}:

\qquad \texttt{double EdgeHeight( Vertex vi, Vertex vj, Vertex vk) \{}

\qquad\qquad\texttt{Face fijk;}

\qquad\qquad\texttt{vector\TEXTsymbol{<}int\TEXTsymbol{>} temp\_ij = }

\qquad\qquad\qquad\texttt{listIntersection(vi.getLocalFaces(),
vj.getLocalFaces());}

\qquad\qquad\texttt{vector\TEXTsymbol{<}int\TEXTsymbol{>} temp = }

\qquad\qquad\qquad\texttt{listIntersection( \&temp\_ij, vk.getLocalFaces());}

\qquad\qquad\texttt{fijk = Triangulation::faceTable[temp[0]];}

\qquad \qquad \texttt{double result = (PartialEdge(vi, vk)-PartialEdge(vi,vj)%
}

\qquad\qquad\qquad\texttt{*cos(Geometry::angle(vi,
fijk)))/sin(Geometry::angle(vi, fijk));}

\qquad\qquad\texttt{return result;}

\qquad\qquad\texttt{\}}

\subsection*{Limitations}

\texttt{PartialEdge} must receive as input two vertices that define an edge
in the triangulation. \ \texttt{PartialEdge} returns distinct values for
each permutation of the input vertices. \ 

\subsection*{Revisions}

subversion 757, 7/13/09, \texttt{PartialEdge} created.

subversion 1055, 3/12/10, \texttt{PartialEdge}\ converted to a geoquant.

\subsection*{Testing}

\texttt{dij} was tested by working out several examples by hand.

\subsection*{Future Work}

This function has been added to the Geometry class geoquants, and thus this
entry needs to be updated.
%
%EndExpansion

\bigskip

\bigskip

%TCIMACRO{\QSubDoc{Include EHR}{%TCIDATA{Version=5.00.0.2606}
%TCIDATA{LaTeXparent=1,1,functions.tex}
                      

\section*{\texttt{EHR}}

\subsection*{Function Prototype}

\texttt{double Example\_Function (Vertex, Edge1, Edge2, double, int,...)}

\subsection*{Key Words}

Enter all key words to this function here.

\subsection*{Authors}

Type the primary authors here. Example:

Daniel "Cliff Jumper" Champion

\subsection*{Introduction}

In this space provide a brief introduction to familiarize the reader with
the function listed above. \ 

\subsection*{Subsidiaries}

List all functions used by this function; list all variables (and types)
used by this function. \ Indent to show hierarchy.

Functions:

\qquad Function1

\qquad Function2

\qquad\qquad Function2.1

\qquad\qquad Function2.2

\qquad\qquad\qquad Function2.3

Global Variables:

Local Variables:

\subsection*{Description}

Begin this section with a detailed description of the function. \ For simple
functions provide sufficient theory to define the function, otherwise
outline the theory and cite "calculations were performed in Mathematica..."
etc. if applicable. \ 

Conclude this section with an explanation of why this function exists. \
This would include initial motivation for the creation of the function, as
well as all primary programs (functions) that utilize the function. \ A
brief history of the function can also be given if it serves to explain why
the function exists. \ 

\subsection*{Practicum}

Place any and all practical information about the function in this section.
\ Provide a short example of the use of this function if appropriate. \ This
should be written in code or pseudo-code written in the format below:

\bigskip

\qquad\texttt{begin repeat;}

\qquad\qquad\texttt{result = n+5;}

\qquad\qquad\texttt{end if result \TEXTsymbol{>} 5;}

\qquad\qquad\texttt{n=n+1;}

\qquad\texttt{end repeat;}

\bigskip

\subsection*{Limitations}

Provide a description of the limitations of the function. \ It should be
clear what works and doesn't work about the function from reading this
section. \ 

\subsection*{Revisions}

List the major revisions to the function with dates and a one sentence
comment. \ Example:

subversion 617, 6/8/09, \texttt{Example\_Function} created with severe
limitations.

subversion 618, 6/9/09, \texttt{Example\_Function} was fully commented and
initial testing complete.

subversion 619, 6/10/09, \texttt{Example\_Function} was augmented to utilize
the Geometry class.

\subsection*{Testing}

Describe how the function was tested. \ Include dates and names of test
results if possible.

\subsection*{Future Work}

In this section, describe what changes or increased functionality are
desired for this function. \ It may be helpful to address some of the items
listed in the "Limitations" section.
}}%
%BeginExpansion
%TCIDATA{Version=5.00.0.2606}
%TCIDATA{LaTeXparent=1,1,functions.tex}
                      

\section*{\texttt{EHR}}

\subsection*{Function Prototype}

\texttt{double Example\_Function (Vertex, Edge1, Edge2, double, int,...)}

\subsection*{Key Words}

Enter all key words to this function here.

\subsection*{Authors}

Type the primary authors here. Example:

Daniel "Cliff Jumper" Champion

\subsection*{Introduction}

In this space provide a brief introduction to familiarize the reader with
the function listed above. \ 

\subsection*{Subsidiaries}

List all functions used by this function; list all variables (and types)
used by this function. \ Indent to show hierarchy.

Functions:

\qquad Function1

\qquad Function2

\qquad\qquad Function2.1

\qquad\qquad Function2.2

\qquad\qquad\qquad Function2.3

Global Variables:

Local Variables:

\subsection*{Description}

Begin this section with a detailed description of the function. \ For simple
functions provide sufficient theory to define the function, otherwise
outline the theory and cite "calculations were performed in Mathematica..."
etc. if applicable. \ 

Conclude this section with an explanation of why this function exists. \
This would include initial motivation for the creation of the function, as
well as all primary programs (functions) that utilize the function. \ A
brief history of the function can also be given if it serves to explain why
the function exists. \ 

\subsection*{Practicum}

Place any and all practical information about the function in this section.
\ Provide a short example of the use of this function if appropriate. \ This
should be written in code or pseudo-code written in the format below:

\bigskip

\qquad\texttt{begin repeat;}

\qquad\qquad\texttt{result = n+5;}

\qquad\qquad\texttt{end if result \TEXTsymbol{>} 5;}

\qquad\qquad\texttt{n=n+1;}

\qquad\texttt{end repeat;}

\bigskip

\subsection*{Limitations}

Provide a description of the limitations of the function. \ It should be
clear what works and doesn't work about the function from reading this
section. \ 

\subsection*{Revisions}

List the major revisions to the function with dates and a one sentence
comment. \ Example:

subversion 617, 6/8/09, \texttt{Example\_Function} created with severe
limitations.

subversion 618, 6/9/09, \texttt{Example\_Function} was fully commented and
initial testing complete.

subversion 619, 6/10/09, \texttt{Example\_Function} was augmented to utilize
the Geometry class.

\subsection*{Testing}

Describe how the function was tested. \ Include dates and names of test
results if possible.

\subsection*{Future Work}

In this section, describe what changes or increased functionality are
desired for this function. \ It may be helpful to address some of the items
listed in the "Limitations" section.
%
%EndExpansion

\bigskip

\bigskip

%TCIMACRO{\QSubDoc{Include EHR_Partial}{%TCIDATA{Version=5.00.0.2606}
%TCIDATA{LaTeXparent=1,1,functions.tex}
                      

\section*{\texttt{EHRPartial::EHRPartial}}

\subsection*{Function Prototype}

\texttt{double EHRPartial(int i)}

\subsection*{Key Words}

Einstein-Hilbert-Regge, functional, Newton's method, partial derivative,
geoquant.

\subsection*{Authors}

Daniel Champion

\subsection*{Introduction}

\texttt{EHRPartial} calculates the partial derivative of the normalized
Einstein-Hilbert-Regge functional with respect to log radii. \ 

\subsection*{Subsidiaries}

\textbf{Functions:}

\qquad \texttt{TotalVolume}

\qquad \texttt{TotalCurvature}

\qquad \texttt{VolumePartial}

\qquad\qquad\texttt{listDifference}

\qquad\qquad\texttt{listIntersection}

\textbf{Global Variables:} radii, etas, curvature, volume

\textbf{Local Variables:} \ none

\subsection*{Description}

The normalized Einstein-Hilbert-Regge functional is given by the expression:%
\begin{equation*}
EHR=\frac{\sum\limits_{j}K_{j}}{\sqrt[3]{\sum\limits_{tetra\text{ }t}V_{t}}},
\end{equation*}%
where $K_{i}$ is the curvature at vertex $j$, and $V_{t}$ is the volume of
tetrahedron $t$. \ It can be shown (see arXiv:0906.1560v1) that 
\begin{equation*}
\frac{\partial }{\partial \log r_{i}}\left( \sum\limits_{j}K_{j}\right)
=K_{i},
\end{equation*}%
hence the partial derivative of the normalized EHR functional becomes:%
\begin{align*}
\frac{\partial }{\partial \log r_{i}}EHR& =\frac{K_{i}\sqrt[3]{%
\sum\limits_{tetra\text{ }t}V_{t}}-\frac{1}{3}\left( \sum\limits_{tetra\text{
}t}V_{t}\right) ^{-\frac{2}{3}}\sum\limits_{tetra\text{ }t}\frac{\partial
V_{t}}{\partial \log r_{i}}\sum\limits_{j}K_{j}}{\left( \sum\limits_{tetra%
\text{ }t}V_{t}\right) ^{\frac{2}{3}}} \\
& =V^{-\frac{4}{3}}\left( K_{i}V-\frac{1}{3}K\sum\limits_{tetra\text{ }t}%
\frac{\partial V_{t}}{\partial \log r_{i}}\right) ,
\end{align*}%
where $V$ is the total volume of all tetrahedra in the triangulation and $K$
is the sum of the curvatures over all vertices in the triangulation. \ 
\texttt{EHRPartial (v\_i)} calculates $\frac{\partial }{\partial \log r_{i}}%
EHR$. \ 

The primary use of this function is in the calculation of the negative
gradient of the EHR functional for use in optimization of the functional
using Newton's method. \ The formula for $\frac{\partial }{\partial \log
r_{i}}EHR$ given above was also used in the calculation of the second order
partial derivatives of the EHR functional, implemented in \texttt{%
EHRSecondPartial}.

\subsection*{Practicum}

As an example of the use of this function, the calculation of the gradient
of the EHR functional will be calculated. \ The negative gradient will be
outputted as the array \texttt{EHRneg\_gradient}.

\bigskip

\qquad\texttt{double EHRneg\_gradient[Triangulation::vertexTable.size()];}

\qquad \texttt{for(int i=0; i \TEXTsymbol{<}
Triangulation::vertexTable.size(); ++i) \{}

\qquad \qquad \texttt{Vertex v\_i = Triangulation::vertexTable[i+1];}

\qquad \qquad \texttt{EHRneg\_gradient[i] = -1.0*EHRPartial(v\_i);}

\qquad\qquad\texttt{\}}

\subsection*{Limitations}

The function \texttt{EHRPartial} is fully functional with no known
limitations. \ It will return appropriate values so long as it is called
with an integer in the vertex table. \ 

\subsection*{Revisions}

List the major revisions to the function with dates and a one sentence
comment. \ Example:

subversion 757, 7/7/09, \texttt{EHRPartial} created.

subversion 1055, 3/12/10, \texttt{EHRPartial}\ converted to a geoquant.

\subsection*{Testing}

This function has not been tested.

\subsection*{Future Work}

A testing regime should be instituted for this function. \ 
}}%
%BeginExpansion
%TCIDATA{Version=5.00.0.2606}
%TCIDATA{LaTeXparent=1,1,functions.tex}
                      

\section*{\texttt{EHRPartial::EHRPartial}}

\subsection*{Function Prototype}

\texttt{double EHRPartial(int i)}

\subsection*{Key Words}

Einstein-Hilbert-Regge, functional, Newton's method, partial derivative,
geoquant.

\subsection*{Authors}

Daniel Champion

\subsection*{Introduction}

\texttt{EHRPartial} calculates the partial derivative of the normalized
Einstein-Hilbert-Regge functional with respect to log radii. \ 

\subsection*{Subsidiaries}

\textbf{Functions:}

\qquad \texttt{TotalVolume}

\qquad \texttt{TotalCurvature}

\qquad \texttt{VolumePartial}

\qquad\qquad\texttt{listDifference}

\qquad\qquad\texttt{listIntersection}

\textbf{Global Variables:} radii, etas, curvature, volume

\textbf{Local Variables:} \ none

\subsection*{Description}

The normalized Einstein-Hilbert-Regge functional is given by the expression:%
\begin{equation*}
EHR=\frac{\sum\limits_{j}K_{j}}{\sqrt[3]{\sum\limits_{tetra\text{ }t}V_{t}}},
\end{equation*}%
where $K_{i}$ is the curvature at vertex $j$, and $V_{t}$ is the volume of
tetrahedron $t$. \ It can be shown (see arXiv:0906.1560v1) that 
\begin{equation*}
\frac{\partial }{\partial \log r_{i}}\left( \sum\limits_{j}K_{j}\right)
=K_{i},
\end{equation*}%
hence the partial derivative of the normalized EHR functional becomes:%
\begin{align*}
\frac{\partial }{\partial \log r_{i}}EHR& =\frac{K_{i}\sqrt[3]{%
\sum\limits_{tetra\text{ }t}V_{t}}-\frac{1}{3}\left( \sum\limits_{tetra\text{
}t}V_{t}\right) ^{-\frac{2}{3}}\sum\limits_{tetra\text{ }t}\frac{\partial
V_{t}}{\partial \log r_{i}}\sum\limits_{j}K_{j}}{\left( \sum\limits_{tetra%
\text{ }t}V_{t}\right) ^{\frac{2}{3}}} \\
& =V^{-\frac{4}{3}}\left( K_{i}V-\frac{1}{3}K\sum\limits_{tetra\text{ }t}%
\frac{\partial V_{t}}{\partial \log r_{i}}\right) ,
\end{align*}%
where $V$ is the total volume of all tetrahedra in the triangulation and $K$
is the sum of the curvatures over all vertices in the triangulation. \ 
\texttt{EHRPartial (v\_i)} calculates $\frac{\partial }{\partial \log r_{i}}%
EHR$. \ 

The primary use of this function is in the calculation of the negative
gradient of the EHR functional for use in optimization of the functional
using Newton's method. \ The formula for $\frac{\partial }{\partial \log
r_{i}}EHR$ given above was also used in the calculation of the second order
partial derivatives of the EHR functional, implemented in \texttt{%
EHRSecondPartial}.

\subsection*{Practicum}

As an example of the use of this function, the calculation of the gradient
of the EHR functional will be calculated. \ The negative gradient will be
outputted as the array \texttt{EHRneg\_gradient}.

\bigskip

\qquad\texttt{double EHRneg\_gradient[Triangulation::vertexTable.size()];}

\qquad \texttt{for(int i=0; i \TEXTsymbol{<}
Triangulation::vertexTable.size(); ++i) \{}

\qquad \qquad \texttt{Vertex v\_i = Triangulation::vertexTable[i+1];}

\qquad \qquad \texttt{EHRneg\_gradient[i] = -1.0*EHRPartial(v\_i);}

\qquad\qquad\texttt{\}}

\subsection*{Limitations}

The function \texttt{EHRPartial} is fully functional with no known
limitations. \ It will return appropriate values so long as it is called
with an integer in the vertex table. \ 

\subsection*{Revisions}

List the major revisions to the function with dates and a one sentence
comment. \ Example:

subversion 757, 7/7/09, \texttt{EHRPartial} created.

subversion 1055, 3/12/10, \texttt{EHRPartial}\ converted to a geoquant.

\subsection*{Testing}

This function has not been tested.

\subsection*{Future Work}

A testing regime should be instituted for this function. \ 
%
%EndExpansion

\bigskip

\bigskip

%TCIMACRO{%
%\QSubDoc{Include EHR_Second_Partial}{%TCIDATA{Version=5.00.0.2606}
%TCIDATA{LaTeXparent=1,1,functions.tex}
                      

\section*{\texttt{EHRSecondPartial::EHRSecondPartial}}

\subsection*{Function Prototype}

\texttt{double EHRSecondPartial (Vertex v\_i, Vertex v\_j)}

\subsection*{Key Words}

Einstein-Hilbert-Regge, functional, partial derivative, Hessian, geoquant.

\subsection*{Authors}

Daniel Champion

\subsection*{Introduction}

\texttt{EHRSecondPartial} calculates the second order partial derivatives of
the normalized Einstein-Hilbert-Regge functional with respect to log radii.
\ 

\subsection*{Subsidiaries}

\textbf{Functions:}

\qquad \texttt{CurvaturePartial}

\qquad\qquad\texttt{isAdjVertex}

\qquad \qquad \texttt{DualArea}

\qquad \qquad \qquad \texttt{DualAreaSegment}

\qquad \qquad \qquad \qquad \texttt{FaceHeight}

\qquad \qquad \qquad \qquad \qquad \texttt{EdgeHeight}

\qquad \qquad \qquad \qquad \qquad \qquad \texttt{PartialEdge}

\qquad\qquad\texttt{listDifference}

\qquad\qquad\texttt{listIntersection}

\qquad \texttt{TotalCurvature}

\qquad\qquad\texttt{Geometry::curvature}

\qquad \texttt{TotalVolume}

\qquad\qquad\texttt{Geometry::volume}

\qquad \texttt{VolumePartial}

\qquad\qquad\texttt{listDifference}

\qquad\qquad\texttt{listIntersection.}

\qquad \texttt{VolumeSecondPartial}

\textbf{Global Variables: }\ radii, etas.

\textbf{Local Variables:} \ none.

\subsection*{Description}

The normalized Einstein-Hilbert-Regge functional is given by the expression:%
\begin{equation*}
EHR=\frac{\sum\limits_{j}K_{j}}{\sqrt[3]{\sum\limits_{tetra\text{ }t}V_{t}}},
\end{equation*}
where $K_{i}$ is the curvature at vertex $j$, and $V_{t}$ is the volume of
tetrahedron $t$. \ It can be shown (see arXiv:0906.1560v1) that 
\begin{equation*}
\frac{\partial}{\partial\log r_{i}}\left( \sum\limits_{j}K_{j}\right) =K_{i},
\end{equation*}
hence the partial derivative of the normalized EHR functional simplifies to
become:%
\begin{equation*}
\frac{\partial}{\partial\log r_{i}}EHR=V^{-\frac{4}{3}}\left( K_{i}V-\frac {1%
}{3}K\sum\limits_{tetra\text{ }t}\frac{\partial V_{t}}{\partial\log r_{i}}%
\right) ,
\end{equation*}
where $V$ is the total volume of all tetrahedra in the triangulation and $K$
is the sum of the curvatures over all vertices in the triangulation. \
Differentiating this result with respect to $\log r_{j}$ yields:%
\begin{equation*}
\frac{\partial^{2}}{\partial\log r_{i}\partial\log r_{j}}EHR=V^{-\frac{4}{3}%
}\left( 
\begin{array}{c}
V\frac{\partial K_{i}}{\partial\log r_{j}}-\frac{1}{3}K_{i}\sum\limits_{t}%
\frac{\partial V_{t}}{\partial\log r_{j}}-\frac{1}{3}K_{j}\sum\limits_{t}%
\frac{\partial V_{t}}{\partial\log r_{i}} \\ 
+\frac{4}{9}KV^{-1}\sum\limits_{t}\frac{\partial V_{t}}{\partial\log r_{j}}%
\sum\limits_{t}\frac{\partial V_{t}}{\partial\log r_{i}}-\frac{1}{3}%
K\sum\limits_{t}\frac{\partial^{2}V_{t}}{\partial\log r_{i}\partial\log r_{j}%
}%
\end{array}
\right) .
\end{equation*}

When called, \texttt{EHRSecondPartial} calculates the formula above, that is:%
\begin{equation*}
\text{\texttt{EHR\_Second\_Partial (i,j)}}=\frac{\partial ^{2}}{\partial
\log r_{i}\partial \log r_{j}}EHR.
\end{equation*}

The use of this function is in the population of the Hessian matrix for the
normalized EHR functional. \ This Hessian matrix is used in the optimization
of the EHR functional using Newton's method.

\subsection*{Practicum}

As an example of the usage of \texttt{EHRSecondPartial}, the Hessian matrix
of the normalized EHR functional will be populated. \ In this example, the
Hessian matrix is the array EHRhessian. \ The example reduced computation
time by only calling \texttt{EHRSecondPartial} for the upper triangular
portion of the EHRhessian array, and symmetrically copies the entries above
the diagonal to the corresponding location below the diagonal. \
Furthermore, in C++ arrays begin indexing at zero, however the vertices of
the triangulations begin indexing at 1, requiring a shift of one in the
population step. \ Note that triangulations that do not label the vertices
consecutively will not be compatible with the following code. \ 

\bigskip

\qquad\texttt{double
EHRhessian[Triangulation::vertexTable.size()][Triangulation::vertexTable.size()];%
}

\qquad\texttt{for(int i = 0; i \TEXTsymbol{<}
Triangulation::vertexTable.size(); ++i) \{}

\qquad \qquad \texttt{for(int j = 0; j \TEXTsymbol{<}
Triangulation::vertexTable.size(); ++j) \{}

\qquad \qquad \qquad \texttt{Vertex vi = Triangulation::vertexYable[i+1];}

\qquad \qquad \qquad \texttt{Vertex vj = Triangulation::vertexTable[j+1];}

\qquad \qquad \qquad \texttt{if (i \TEXTsymbol{<}= j) \{}

\qquad \qquad \qquad \qquad \texttt{EHRhessian[i][j]=EHRSecondPartial( vi ,
vj );}

\qquad\qquad\qquad\qquad\texttt{EHRhessian[j][i]=EHRhessian[i][j];}

\qquad\qquad\qquad\qquad\texttt{\}}

\qquad\qquad\qquad\texttt{\}}

\qquad\qquad\texttt{\}}

\subsection*{Limitations}

\texttt{EHRSecondPartial} is fully operational with no known limitations. \
The function will output appropriate values provided it receives as inputs a
pair of integers in the vertex table. \ 

\subsection*{Revisions}

subversion 757, 7/7/09, \texttt{EHRSecondPartial} created.

subversion 1055, 3/12/10, \texttt{EHRSecondPartial}\ converted to a geoquant.

\subsection*{Testing}

This function has not been tested.

\subsection*{Future Work}

A testing regime should be instituted for this function. \ 
}}%
%BeginExpansion
%TCIDATA{Version=5.00.0.2606}
%TCIDATA{LaTeXparent=1,1,functions.tex}
                      

\section*{\texttt{EHRSecondPartial::EHRSecondPartial}}

\subsection*{Function Prototype}

\texttt{double EHRSecondPartial (Vertex v\_i, Vertex v\_j)}

\subsection*{Key Words}

Einstein-Hilbert-Regge, functional, partial derivative, Hessian, geoquant.

\subsection*{Authors}

Daniel Champion

\subsection*{Introduction}

\texttt{EHRSecondPartial} calculates the second order partial derivatives of
the normalized Einstein-Hilbert-Regge functional with respect to log radii.
\ 

\subsection*{Subsidiaries}

\textbf{Functions:}

\qquad \texttt{CurvaturePartial}

\qquad\qquad\texttt{isAdjVertex}

\qquad \qquad \texttt{DualArea}

\qquad \qquad \qquad \texttt{DualAreaSegment}

\qquad \qquad \qquad \qquad \texttt{FaceHeight}

\qquad \qquad \qquad \qquad \qquad \texttt{EdgeHeight}

\qquad \qquad \qquad \qquad \qquad \qquad \texttt{PartialEdge}

\qquad\qquad\texttt{listDifference}

\qquad\qquad\texttt{listIntersection}

\qquad \texttt{TotalCurvature}

\qquad\qquad\texttt{Geometry::curvature}

\qquad \texttt{TotalVolume}

\qquad\qquad\texttt{Geometry::volume}

\qquad \texttt{VolumePartial}

\qquad\qquad\texttt{listDifference}

\qquad\qquad\texttt{listIntersection.}

\qquad \texttt{VolumeSecondPartial}

\textbf{Global Variables: }\ radii, etas.

\textbf{Local Variables:} \ none.

\subsection*{Description}

The normalized Einstein-Hilbert-Regge functional is given by the expression:%
\begin{equation*}
EHR=\frac{\sum\limits_{j}K_{j}}{\sqrt[3]{\sum\limits_{tetra\text{ }t}V_{t}}},
\end{equation*}
where $K_{i}$ is the curvature at vertex $j$, and $V_{t}$ is the volume of
tetrahedron $t$. \ It can be shown (see arXiv:0906.1560v1) that 
\begin{equation*}
\frac{\partial}{\partial\log r_{i}}\left( \sum\limits_{j}K_{j}\right) =K_{i},
\end{equation*}
hence the partial derivative of the normalized EHR functional simplifies to
become:%
\begin{equation*}
\frac{\partial}{\partial\log r_{i}}EHR=V^{-\frac{4}{3}}\left( K_{i}V-\frac {1%
}{3}K\sum\limits_{tetra\text{ }t}\frac{\partial V_{t}}{\partial\log r_{i}}%
\right) ,
\end{equation*}
where $V$ is the total volume of all tetrahedra in the triangulation and $K$
is the sum of the curvatures over all vertices in the triangulation. \
Differentiating this result with respect to $\log r_{j}$ yields:%
\begin{equation*}
\frac{\partial^{2}}{\partial\log r_{i}\partial\log r_{j}}EHR=V^{-\frac{4}{3}%
}\left( 
\begin{array}{c}
V\frac{\partial K_{i}}{\partial\log r_{j}}-\frac{1}{3}K_{i}\sum\limits_{t}%
\frac{\partial V_{t}}{\partial\log r_{j}}-\frac{1}{3}K_{j}\sum\limits_{t}%
\frac{\partial V_{t}}{\partial\log r_{i}} \\ 
+\frac{4}{9}KV^{-1}\sum\limits_{t}\frac{\partial V_{t}}{\partial\log r_{j}}%
\sum\limits_{t}\frac{\partial V_{t}}{\partial\log r_{i}}-\frac{1}{3}%
K\sum\limits_{t}\frac{\partial^{2}V_{t}}{\partial\log r_{i}\partial\log r_{j}%
}%
\end{array}
\right) .
\end{equation*}

When called, \texttt{EHRSecondPartial} calculates the formula above, that is:%
\begin{equation*}
\text{\texttt{EHR\_Second\_Partial (i,j)}}=\frac{\partial ^{2}}{\partial
\log r_{i}\partial \log r_{j}}EHR.
\end{equation*}

The use of this function is in the population of the Hessian matrix for the
normalized EHR functional. \ This Hessian matrix is used in the optimization
of the EHR functional using Newton's method.

\subsection*{Practicum}

As an example of the usage of \texttt{EHRSecondPartial}, the Hessian matrix
of the normalized EHR functional will be populated. \ In this example, the
Hessian matrix is the array EHRhessian. \ The example reduced computation
time by only calling \texttt{EHRSecondPartial} for the upper triangular
portion of the EHRhessian array, and symmetrically copies the entries above
the diagonal to the corresponding location below the diagonal. \
Furthermore, in C++ arrays begin indexing at zero, however the vertices of
the triangulations begin indexing at 1, requiring a shift of one in the
population step. \ Note that triangulations that do not label the vertices
consecutively will not be compatible with the following code. \ 

\bigskip

\qquad\texttt{double
EHRhessian[Triangulation::vertexTable.size()][Triangulation::vertexTable.size()];%
}

\qquad\texttt{for(int i = 0; i \TEXTsymbol{<}
Triangulation::vertexTable.size(); ++i) \{}

\qquad \qquad \texttt{for(int j = 0; j \TEXTsymbol{<}
Triangulation::vertexTable.size(); ++j) \{}

\qquad \qquad \qquad \texttt{Vertex vi = Triangulation::vertexYable[i+1];}

\qquad \qquad \qquad \texttt{Vertex vj = Triangulation::vertexTable[j+1];}

\qquad \qquad \qquad \texttt{if (i \TEXTsymbol{<}= j) \{}

\qquad \qquad \qquad \qquad \texttt{EHRhessian[i][j]=EHRSecondPartial( vi ,
vj );}

\qquad\qquad\qquad\qquad\texttt{EHRhessian[j][i]=EHRhessian[i][j];}

\qquad\qquad\qquad\qquad\texttt{\}}

\qquad\qquad\qquad\texttt{\}}

\qquad\qquad\texttt{\}}

\subsection*{Limitations}

\texttt{EHRSecondPartial} is fully operational with no known limitations. \
The function will output appropriate values provided it receives as inputs a
pair of integers in the vertex table. \ 

\subsection*{Revisions}

subversion 757, 7/7/09, \texttt{EHRSecondPartial} created.

subversion 1055, 3/12/10, \texttt{EHRSecondPartial}\ converted to a geoquant.

\subsection*{Testing}

This function has not been tested.

\subsection*{Future Work}

A testing regime should be instituted for this function. \ 
%
%EndExpansion

\bigskip

\bigskip 

%TCIMACRO{\QSubDoc{Include flip}{%html2tex: Version  2.7 of June 17, 2008.
%Written by  F.J. Faase.  http://www.iwriteiam.nl/

\documentclass[10pt]{article}%
\usepackage{amssymb}
\usepackage{geometry}
\usepackage{indentfirst}
\usepackage{amsmath}
\usepackage{amsfonts}
\usepackage{graphicx}%
\setcounter{MaxMatrixCols}{30}
%TCIDATA{OutputFilter=latex2.dll}
%TCIDATA{Version=5.00.0.2606}
%TCIDATA{CSTFile=40 LaTeX article.cst}
%TCIDATA{Created=Friday, March 30, 2007 00:21:27}
%TCIDATA{LastRevised=Wednesday, June 10, 2009 11:42:33}
%TCIDATA{<META NAME="GraphicsSave" CONTENT="32">}
%TCIDATA{<META NAME="SaveForMode" CONTENT="1">}
%TCIDATA{BibliographyScheme=Manual}
%TCIDATA{<META NAME="DocumentShell" CONTENT="Standard LaTeX\Blank - Standard LaTeX Article">}
%TCIDATA{Language=American English}
\newtheorem{theorem}{Theorem}
\newtheorem{acknowledgement}[theorem]{Acknowledgement}
\newtheorem{algorithm}[theorem]{Algorithm}
\newtheorem{axiom}[theorem]{Axiom}
\newtheorem{case}[theorem]{Case}
\newtheorem{claim}[theorem]{Claim}
\newtheorem{conclusion}[theorem]{Conclusion}
\newtheorem{condition}[theorem]{Condition}
\newtheorem{conjecture}[theorem]{Conjecture}
\newtheorem{corollary}[theorem]{Corollary}
\newtheorem{criterion}[theorem]{Criterion}
\newtheorem{definition}[theorem]{Definition}
\newtheorem{example}[theorem]{Example}
\newtheorem{exercise}[theorem]{Exercise}
\newtheorem{lemma}[theorem]{Lemma}
\newtheorem{notation}[theorem]{Notation}
\newtheorem{problem}[theorem]{Problem}
\newtheorem{proposition}[theorem]{Proposition}
\newtheorem{remark}[theorem]{Remark}
\newtheorem{solution}[theorem]{Solution}
\newtheorem{summary}[theorem]{Summary}
\newenvironment{proof}[1][Proof]{\noindent\textbf{#1.} }{\ \rule{0.5em}{0.5em}}
\geometry{left=1in,right=1in,top=1in,bottom=1in}

\begin{document}

%%%%% BEGINNING OF DOCUMENT BODY %%%%%
% html: Beginning of file: `clean.html'
% DOCTYPE HTML PUBLIC "-//W3C//DTD HTML 4.01//EN"
%  This is a (PRE) block.  Make sure it's left aligned or your toc title will be off. 

\section*{\texttt{flip}}

\label{f0}{\small{\begin{verbatim} 
Edge flip(Edge e)
\end{verbatim}
}}

\subsection*{Keywords}

\begin{quotation} flip, delaunay\end{quotation}

\subsection*{Authors}

\begin{quotation} Kurt Norwood\end{quotation}

\subsection*{Introduction}

\begin{quotation} The \texttt{flip} function takes a single Edge as a parameter and performs a flip on it. This involves determining the new length of the edge after the flip and changing the topological information in the edge being flipped as well as all of the edge's adjacent simplices. This can be thought of as taking two triangles which share an edge (the parameter to flip) and making two new triangles which share an edge between the two vertices which were previously non-adjacent.\end{quotation}

\subsection*{Subsidiaries}

\begin{quotation} Functions:\end{quotation}
{\small{\begin{verbatim} 
    void flipPP(struct simps b)

    void flipPN(struct simps b)

    void flipNN(struct simps b)

    void topo_flip(Edge, struct simps)

    bool prep_for_flip(Edge, struct simps*)
\end{verbatim}
}}
\begin{quotation} Global Variables:Local Variables:\end{quotation}
{\small{\begin{verbatim} 
    Edge e
\end{verbatim}
}}

\subsection*{Description}

\begin{quotation} flip begins by calling the prep\_for\_flip function, that will setup the struct given to it to contain all the important information necessary for the flip to occur, such as indices for the different simplices and the lengths of the triangles' edges, and the two angles which are not incident on the edge being flipped. The struct looks like:\end{quotation}{\small{\begin{verbatim} 
struct simps {
       int v0, v1, v2, v3, e0, e1, e2, e3, e4, f0, f1;
       double e0_len, e1_len, e2_len, e3_len, e4_len;
       double a0, a2;
};
\end{verbatim}
}}
\begin{quotation} With all this information known, the next step is to determine the type of flip that is to occur. The possibilities are broken up three ways: positive positive (PP), positive negative (PN), negative negative (NN); based on the initial condition of the two triangles. This will determine which of flipPP, flipPN, flipNN is called. Within these function is logic which should compute the new edge length and assign it to the edge e, and determine the positive/negative configuration of the two triangles and assign the appropriate boolean value to each face.\end{quotation}
\begin{quotation} With the new edge length computed and assigned, the topo\_flip function is called which performs the rearrangement of all the adjacencies of the different simplices which are adjacent to edge e.\end{quotation}
\begin{quotation} The edge is returned.\end{quotation}

\subsection*{Practicum}

\begin{quotation} \end{quotation}{\small{\begin{verbatim} 
  Edge e;
  e = Triangulation::edgeTable[indexOfE];
  e = flip(e);
\end{verbatim}
}}
\begin{quotation} one thing to note is that in future implementations the edge being given as the parameter may be different than the one returned\end{quotation}

\subsection*{Limitations}

\begin{quotation} The biggest limitation of the flip function is that it currently only works for bistellar flips. If higher dimensional flips are required this function will need to be modified heavily.\end{quotation}

\subsection*{Revisions}

\begin{quotation} ------------------------------------------------------------------------r816 \mbox{$|$} kortox \mbox{$|$} 2009-06-29 12:41:33 -0700 (Mon, 29 Jun 2009) \mbox{$|$} 1 line\end{quotation}
\begin{quotation} have all the new\_flip stuff up to date and working with the new geometry classes\end{quotation}
\begin{quotation} ------------------------------------------------------------------------r795 \mbox{$|$} kortox \mbox{$|$} 2009-06-18 17:58:30 -0700 (Thu, 18 Jun 2009) \mbox{$|$} 5 lines\end{quotation}
\begin{quotation} anyway, this is a project for devopment of the flip algorithm, so far it contains a new flip function which is intended to replace the flip function that was previously in Triangulation/triangulationmorphs.cpp\end{quotation}
\begin{quotation} main currently contains some test functions that can be called one at a time manually and should produce output that can indicate how the flip function is performing, this testing really needs to be improved\end{quotation}

\subsection*{Testing}

\begin{quotation} Initially testing was done inefficiently by manually analyzing what was written by the writeTriangulationFile  function. Now that we have a way to display the triangulation, we can select an edge and flip it in the display and see that the flip occurred correctly. Granted this should at sometime in the future be automated, but for now if there is an issue we can try to debug it with the display.\end{quotation}

\subsection*{Future Work}

\begin{quotation} \end{quotation}\begin{itemize}\item  Adding the ability to flip in higher dimensions. This would involve altering the function to take a Simplex object instead of an edge so that it is more general.
\end{itemize}
\begin{quotation} \end{quotation}\begin{itemize}\item  We'll most likely want to have the function add an edge to the triangulation instead of just reposition the edge given, since this will lend itself better to the possible addition of 3-1 flips. Related to this would also be changing the return type to be a vector of Simplex objects for generality's sake.
\end{itemize}
\begin{quotation} \end{quotation}\begin{itemize}\item  Moving the whole thing to a different file with an appropriate name other than new\_flip
\end{itemize}
    
% html: End of file: `clean.html'

%%%%% END OF DOCUMENT BODY %%%%%
% In the future, we might want to put some additional data here, such
% as when the documentation was converted from wiki to TeX.
%

\end{document}
}}%
%BeginExpansion
%html2tex: Version  2.7 of June 17, 2008.
%Written by  F.J. Faase.  http://www.iwriteiam.nl/

\documentclass[10pt]{article}%
\usepackage{amssymb}
\usepackage{geometry}
\usepackage{indentfirst}
\usepackage{amsmath}
\usepackage{amsfonts}
\usepackage{graphicx}%
\setcounter{MaxMatrixCols}{30}
%TCIDATA{OutputFilter=latex2.dll}
%TCIDATA{Version=5.00.0.2606}
%TCIDATA{CSTFile=40 LaTeX article.cst}
%TCIDATA{Created=Friday, March 30, 2007 00:21:27}
%TCIDATA{LastRevised=Wednesday, June 10, 2009 11:42:33}
%TCIDATA{<META NAME="GraphicsSave" CONTENT="32">}
%TCIDATA{<META NAME="SaveForMode" CONTENT="1">}
%TCIDATA{BibliographyScheme=Manual}
%TCIDATA{<META NAME="DocumentShell" CONTENT="Standard LaTeX\Blank - Standard LaTeX Article">}
%TCIDATA{Language=American English}
\newtheorem{theorem}{Theorem}
\newtheorem{acknowledgement}[theorem]{Acknowledgement}
\newtheorem{algorithm}[theorem]{Algorithm}
\newtheorem{axiom}[theorem]{Axiom}
\newtheorem{case}[theorem]{Case}
\newtheorem{claim}[theorem]{Claim}
\newtheorem{conclusion}[theorem]{Conclusion}
\newtheorem{condition}[theorem]{Condition}
\newtheorem{conjecture}[theorem]{Conjecture}
\newtheorem{corollary}[theorem]{Corollary}
\newtheorem{criterion}[theorem]{Criterion}
\newtheorem{definition}[theorem]{Definition}
\newtheorem{example}[theorem]{Example}
\newtheorem{exercise}[theorem]{Exercise}
\newtheorem{lemma}[theorem]{Lemma}
\newtheorem{notation}[theorem]{Notation}
\newtheorem{problem}[theorem]{Problem}
\newtheorem{proposition}[theorem]{Proposition}
\newtheorem{remark}[theorem]{Remark}
\newtheorem{solution}[theorem]{Solution}
\newtheorem{summary}[theorem]{Summary}
\newenvironment{proof}[1][Proof]{\noindent\textbf{#1.} }{\ \rule{0.5em}{0.5em}}
\geometry{left=1in,right=1in,top=1in,bottom=1in}

\begin{document}

%%%%% BEGINNING OF DOCUMENT BODY %%%%%
% html: Beginning of file: `clean.html'
% DOCTYPE HTML PUBLIC "-//W3C//DTD HTML 4.01//EN"
%  This is a (PRE) block.  Make sure it's left aligned or your toc title will be off. 

\section*{\texttt{flip}}

\label{f0}{\small{\begin{verbatim} 
Edge flip(Edge e)
\end{verbatim}
}}

\subsection*{Keywords}

\begin{quotation} flip, delaunay\end{quotation}

\subsection*{Authors}

\begin{quotation} Kurt Norwood\end{quotation}

\subsection*{Introduction}

\begin{quotation} The \texttt{flip} function takes a single Edge as a parameter and performs a flip on it. This involves determining the new length of the edge after the flip and changing the topological information in the edge being flipped as well as all of the edge's adjacent simplices. This can be thought of as taking two triangles which share an edge (the parameter to flip) and making two new triangles which share an edge between the two vertices which were previously non-adjacent.\end{quotation}

\subsection*{Subsidiaries}

\begin{quotation} Functions:\end{quotation}
{\small{\begin{verbatim} 
    void flipPP(struct simps b)

    void flipPN(struct simps b)

    void flipNN(struct simps b)

    void topo_flip(Edge, struct simps)

    bool prep_for_flip(Edge, struct simps*)
\end{verbatim}
}}
\begin{quotation} Global Variables:Local Variables:\end{quotation}
{\small{\begin{verbatim} 
    Edge e
\end{verbatim}
}}

\subsection*{Description}

\begin{quotation} flip begins by calling the prep\_for\_flip function, that will setup the struct given to it to contain all the important information necessary for the flip to occur, such as indices for the different simplices and the lengths of the triangles' edges, and the two angles which are not incident on the edge being flipped. The struct looks like:\end{quotation}{\small{\begin{verbatim} 
struct simps {
       int v0, v1, v2, v3, e0, e1, e2, e3, e4, f0, f1;
       double e0_len, e1_len, e2_len, e3_len, e4_len;
       double a0, a2;
};
\end{verbatim}
}}
\begin{quotation} With all this information known, the next step is to determine the type of flip that is to occur. The possibilities are broken up three ways: positive positive (PP), positive negative (PN), negative negative (NN); based on the initial condition of the two triangles. This will determine which of flipPP, flipPN, flipNN is called. Within these function is logic which should compute the new edge length and assign it to the edge e, and determine the positive/negative configuration of the two triangles and assign the appropriate boolean value to each face.\end{quotation}
\begin{quotation} With the new edge length computed and assigned, the topo\_flip function is called which performs the rearrangement of all the adjacencies of the different simplices which are adjacent to edge e.\end{quotation}
\begin{quotation} The edge is returned.\end{quotation}

\subsection*{Practicum}

\begin{quotation} \end{quotation}{\small{\begin{verbatim} 
  Edge e;
  e = Triangulation::edgeTable[indexOfE];
  e = flip(e);
\end{verbatim}
}}
\begin{quotation} one thing to note is that in future implementations the edge being given as the parameter may be different than the one returned\end{quotation}

\subsection*{Limitations}

\begin{quotation} The biggest limitation of the flip function is that it currently only works for bistellar flips. If higher dimensional flips are required this function will need to be modified heavily.\end{quotation}

\subsection*{Revisions}

\begin{quotation} ------------------------------------------------------------------------r816 \mbox{$|$} kortox \mbox{$|$} 2009-06-29 12:41:33 -0700 (Mon, 29 Jun 2009) \mbox{$|$} 1 line\end{quotation}
\begin{quotation} have all the new\_flip stuff up to date and working with the new geometry classes\end{quotation}
\begin{quotation} ------------------------------------------------------------------------r795 \mbox{$|$} kortox \mbox{$|$} 2009-06-18 17:58:30 -0700 (Thu, 18 Jun 2009) \mbox{$|$} 5 lines\end{quotation}
\begin{quotation} anyway, this is a project for devopment of the flip algorithm, so far it contains a new flip function which is intended to replace the flip function that was previously in Triangulation/triangulationmorphs.cpp\end{quotation}
\begin{quotation} main currently contains some test functions that can be called one at a time manually and should produce output that can indicate how the flip function is performing, this testing really needs to be improved\end{quotation}

\subsection*{Testing}

\begin{quotation} Initially testing was done inefficiently by manually analyzing what was written by the writeTriangulationFile  function. Now that we have a way to display the triangulation, we can select an edge and flip it in the display and see that the flip occurred correctly. Granted this should at sometime in the future be automated, but for now if there is an issue we can try to debug it with the display.\end{quotation}

\subsection*{Future Work}

\begin{quotation} \end{quotation}\begin{itemize}\item  Adding the ability to flip in higher dimensions. This would involve altering the function to take a Simplex object instead of an edge so that it is more general.
\end{itemize}
\begin{quotation} \end{quotation}\begin{itemize}\item  We'll most likely want to have the function add an edge to the triangulation instead of just reposition the edge given, since this will lend itself better to the possible addition of 3-1 flips. Related to this would also be changing the return type to be a vector of Simplex objects for generality's sake.
\end{itemize}
\begin{quotation} \end{quotation}\begin{itemize}\item  Moving the whole thing to a different file with an appropriate name other than new\_flip
\end{itemize}
    
% html: End of file: `clean.html'

%%%%% END OF DOCUMENT BODY %%%%%
% In the future, we might want to put some additional data here, such
% as when the documentation was converted from wiki to TeX.
%

\end{document}
%
%EndExpansion

\bigskip 

\bigskip 

%TCIMACRO{\QSubDoc{Include GeoquantAt}{%html2tex: Version  2.7 of June 17, 2008.
%Written by  F.J. Faase.  http://www.iwriteiam.nl/

\documentclass[10pt]{article}%
\usepackage{amssymb}
\usepackage{geometry}
\usepackage{indentfirst}
\usepackage{amsmath}
\usepackage{amsfonts}
\usepackage{graphicx}%
\setcounter{MaxMatrixCols}{30}
%TCIDATA{OutputFilter=latex2.dll}
%TCIDATA{Version=5.00.0.2606}
%TCIDATA{CSTFile=40 LaTeX article.cst}
%TCIDATA{Created=Friday, March 30, 2007 00:21:27}
%TCIDATA{LastRevised=Wednesday, June 10, 2009 11:42:33}
%TCIDATA{<META NAME="GraphicsSave" CONTENT="32">}
%TCIDATA{<META NAME="SaveForMode" CONTENT="1">}
%TCIDATA{BibliographyScheme=Manual}
%TCIDATA{<META NAME="DocumentShell" CONTENT="Standard LaTeX\Blank - Standard LaTeX Article">}
%TCIDATA{Language=American English}
\newtheorem{theorem}{Theorem}
\newtheorem{acknowledgement}[theorem]{Acknowledgement}
\newtheorem{algorithm}[theorem]{Algorithm}
\newtheorem{axiom}[theorem]{Axiom}
\newtheorem{case}[theorem]{Case}
\newtheorem{claim}[theorem]{Claim}
\newtheorem{conclusion}[theorem]{Conclusion}
\newtheorem{condition}[theorem]{Condition}
\newtheorem{conjecture}[theorem]{Conjecture}
\newtheorem{corollary}[theorem]{Corollary}
\newtheorem{criterion}[theorem]{Criterion}
\newtheorem{definition}[theorem]{Definition}
\newtheorem{example}[theorem]{Example}
\newtheorem{exercise}[theorem]{Exercise}
\newtheorem{lemma}[theorem]{Lemma}
\newtheorem{notation}[theorem]{Notation}
\newtheorem{problem}[theorem]{Problem}
\newtheorem{proposition}[theorem]{Proposition}
\newtheorem{remark}[theorem]{Remark}
\newtheorem{solution}[theorem]{Solution}
\newtheorem{summary}[theorem]{Summary}
\newenvironment{proof}[1][Proof]{\noindent\textbf{#1.} }{\ \rule{0.5em}{0.5em}}
\geometry{left=1in,right=1in,top=1in,bottom=1in}

\begin{document}

%%%%% BEGINNING OF DOCUMENT BODY %%%%%
% html: Beginning of file: `clean.html'
% DOCTYPE HTML PUBLIC "-//W3C//DTD HTML 4.01//EN"
%  This is a (PRE) block.  Make sure it's left aligned or your toc title will be off. 

\section*{\texttt{Geoquant::At}}

\label{f0}\begin{quotation} {\small{\begin{verbatim} 
Geoquant*  Geoquant::At(Simplex  s1,  ...)
  \end{verbatim}
}}
\end{quotation}
\subsection*{Key Words}

\begin{quotation} geoquant, recalculate, dependent, triposition, simplex\end{quotation}

\subsection*{Authors}

\begin{itemize}\item  Joseph Thomas
\end{itemize}
\begin{quotation} \end{quotation}
\subsection*{Introduction}

\begin{quotation} The \texttt{At} function is defined for every type of geoquant as a way to retrieve that quantity. Once the quantity is retrieved, a value can be set or asked of the quantity. A quantity is retrieved by providing a list of simplices that describe the position of the quantity in the triangulation.\end{quotation}

\subsection*{Subsidiaries}

\begin{quotation} Functions: \end{quotation}
\begin{itemize}
\item  getSerialNumber
\end{itemize}
\begin{quotation} Global Variables: mapLocal Variables: possible list of simplices\end{quotation}

\subsection*{Description}

\begin{quotation} The \texttt{At} function is a little di\"{\i}\&not;erent for every type of geoquant, but in all cases it is a static function for that class that serves as an object retrieval in place of a constructor. The function takes as a parameter a list of simplices which may be di\"{\i}\&not;erent for each type of geoquant. The list is the natural description of where the quantity is in the triangulation. For example, a radius is described by a vertex, whereas an angle is described as a vertex on a certain face. The At function returns a pointer to the requested quantity.\end{quotation}
\begin{quotation} When the \texttt{At} function is called, it searches a local map for a quantity with the given list of simplices.  If it is found, a pointer to that quantity in the map is simply returned.  If it is not found, the quantity is constructed and placed into the map.  If the construction of the object requires other types of quantities not yet created, then these will be constructed automatically at this time. Lastly, this quantity is returned.\end{quotation}
\begin{quotation} The constructor is hidden from the user for several reasons. The \"{\i}\&not;rst is that this avoids redundant construction and the need for an encapsulating object to hold a large set of geoquants (like the Geometry class in a previous version).  In the same vein, the need for an initial build step and a required order of construction is removed. In addition, this is an e\"{\i}\&not;ciency improvement as quantities that are never requested are never created, decreasing memory use and large dependency trees which can take a while for an \texttt{invalidate} to traverse.\end{quotation}

\subsection*{Practicum}

\begin{quotation} Example:{\small{\begin{verbatim} 
//  Get  the  Radius  quantity  from  the  first  vertex  in  the  triangulation.
Radius  *r  =  Radius::At(Triangulation::vertexTable[0]);
//  Get  the  angle  of  vertex  v  incident  on  face  f
Vertex  v;
Face  f;
...
EuclideanAngle  *ang  =  EuclideanAngle::At(v,  f);
  \end{verbatim}
}}
\end{quotation}
\subsection*{Limitations}

\begin{quotation} The \texttt{At} function is limited in that a speci\"{\i}\&not;c set of simplices will always return the exact same object.  While this is in fact the design goal, this can limit one\^as ability to modify an object as a change in one place will a\"{\i}\&not;ect its use elsewhere in the code. The function also will require the user to handle pointers, a powerful yet fragile and sometimes daunting aspect of the programming language.\end{quotation}

\subsection*{Revisions}

\begin{itemize}\item  subversion 761, 6/12/09: A working copy of \texttt{At} and the Geoquant system.
\end{itemize}

\subsection*{Testing}

\begin{quotation} The \texttt{At} function was tested in small modularized systems, then tested in a three dimensional \"{\i}\&not;ow, which required many varied uses of \texttt{At}.   Some retrieved quantities had their values set while others had their values accessed and compared with what mathematica calculations predicted.\end{quotation}

\subsection*{Future Work}

\begin{quotation} No future work is planned at this time.\end{quotation}
    
% html: End of file: `clean.html'

%%%%% END OF DOCUMENT BODY %%%%%
% In the future, we might want to put some additional data here, such
% as when the documentation was converted from wiki to TeX.
%

\end{document}
}}%
%BeginExpansion
%html2tex: Version  2.7 of June 17, 2008.
%Written by  F.J. Faase.  http://www.iwriteiam.nl/

\documentclass[10pt]{article}%
\usepackage{amssymb}
\usepackage{geometry}
\usepackage{indentfirst}
\usepackage{amsmath}
\usepackage{amsfonts}
\usepackage{graphicx}%
\setcounter{MaxMatrixCols}{30}
%TCIDATA{OutputFilter=latex2.dll}
%TCIDATA{Version=5.00.0.2606}
%TCIDATA{CSTFile=40 LaTeX article.cst}
%TCIDATA{Created=Friday, March 30, 2007 00:21:27}
%TCIDATA{LastRevised=Wednesday, June 10, 2009 11:42:33}
%TCIDATA{<META NAME="GraphicsSave" CONTENT="32">}
%TCIDATA{<META NAME="SaveForMode" CONTENT="1">}
%TCIDATA{BibliographyScheme=Manual}
%TCIDATA{<META NAME="DocumentShell" CONTENT="Standard LaTeX\Blank - Standard LaTeX Article">}
%TCIDATA{Language=American English}
\newtheorem{theorem}{Theorem}
\newtheorem{acknowledgement}[theorem]{Acknowledgement}
\newtheorem{algorithm}[theorem]{Algorithm}
\newtheorem{axiom}[theorem]{Axiom}
\newtheorem{case}[theorem]{Case}
\newtheorem{claim}[theorem]{Claim}
\newtheorem{conclusion}[theorem]{Conclusion}
\newtheorem{condition}[theorem]{Condition}
\newtheorem{conjecture}[theorem]{Conjecture}
\newtheorem{corollary}[theorem]{Corollary}
\newtheorem{criterion}[theorem]{Criterion}
\newtheorem{definition}[theorem]{Definition}
\newtheorem{example}[theorem]{Example}
\newtheorem{exercise}[theorem]{Exercise}
\newtheorem{lemma}[theorem]{Lemma}
\newtheorem{notation}[theorem]{Notation}
\newtheorem{problem}[theorem]{Problem}
\newtheorem{proposition}[theorem]{Proposition}
\newtheorem{remark}[theorem]{Remark}
\newtheorem{solution}[theorem]{Solution}
\newtheorem{summary}[theorem]{Summary}
\newenvironment{proof}[1][Proof]{\noindent\textbf{#1.} }{\ \rule{0.5em}{0.5em}}
\geometry{left=1in,right=1in,top=1in,bottom=1in}

\begin{document}

%%%%% BEGINNING OF DOCUMENT BODY %%%%%
% html: Beginning of file: `clean.html'
% DOCTYPE HTML PUBLIC "-//W3C//DTD HTML 4.01//EN"
%  This is a (PRE) block.  Make sure it's left aligned or your toc title will be off. 

\section*{\texttt{Geoquant::At}}

\label{f0}\begin{quotation} {\small{\begin{verbatim} 
Geoquant*  Geoquant::At(Simplex  s1,  ...)
  \end{verbatim}
}}
\end{quotation}
\subsection*{Key Words}

\begin{quotation} geoquant, recalculate, dependent, triposition, simplex\end{quotation}

\subsection*{Authors}

\begin{itemize}\item  Joseph Thomas
\end{itemize}
\begin{quotation} \end{quotation}
\subsection*{Introduction}

\begin{quotation} The \texttt{At} function is defined for every type of geoquant as a way to retrieve that quantity. Once the quantity is retrieved, a value can be set or asked of the quantity. A quantity is retrieved by providing a list of simplices that describe the position of the quantity in the triangulation.\end{quotation}

\subsection*{Subsidiaries}

\begin{quotation} Functions: \end{quotation}
\begin{itemize}
\item  getSerialNumber
\end{itemize}
\begin{quotation} Global Variables: mapLocal Variables: possible list of simplices\end{quotation}

\subsection*{Description}

\begin{quotation} The \texttt{At} function is a little di\"{\i}\&not;erent for every type of geoquant, but in all cases it is a static function for that class that serves as an object retrieval in place of a constructor. The function takes as a parameter a list of simplices which may be di\"{\i}\&not;erent for each type of geoquant. The list is the natural description of where the quantity is in the triangulation. For example, a radius is described by a vertex, whereas an angle is described as a vertex on a certain face. The At function returns a pointer to the requested quantity.\end{quotation}
\begin{quotation} When the \texttt{At} function is called, it searches a local map for a quantity with the given list of simplices.  If it is found, a pointer to that quantity in the map is simply returned.  If it is not found, the quantity is constructed and placed into the map.  If the construction of the object requires other types of quantities not yet created, then these will be constructed automatically at this time. Lastly, this quantity is returned.\end{quotation}
\begin{quotation} The constructor is hidden from the user for several reasons. The \"{\i}\&not;rst is that this avoids redundant construction and the need for an encapsulating object to hold a large set of geoquants (like the Geometry class in a previous version).  In the same vein, the need for an initial build step and a required order of construction is removed. In addition, this is an e\"{\i}\&not;ciency improvement as quantities that are never requested are never created, decreasing memory use and large dependency trees which can take a while for an \texttt{invalidate} to traverse.\end{quotation}

\subsection*{Practicum}

\begin{quotation} Example:{\small{\begin{verbatim} 
//  Get  the  Radius  quantity  from  the  first  vertex  in  the  triangulation.
Radius  *r  =  Radius::At(Triangulation::vertexTable[0]);
//  Get  the  angle  of  vertex  v  incident  on  face  f
Vertex  v;
Face  f;
...
EuclideanAngle  *ang  =  EuclideanAngle::At(v,  f);
  \end{verbatim}
}}
\end{quotation}
\subsection*{Limitations}

\begin{quotation} The \texttt{At} function is limited in that a speci\"{\i}\&not;c set of simplices will always return the exact same object.  While this is in fact the design goal, this can limit one\^as ability to modify an object as a change in one place will a\"{\i}\&not;ect its use elsewhere in the code. The function also will require the user to handle pointers, a powerful yet fragile and sometimes daunting aspect of the programming language.\end{quotation}

\subsection*{Revisions}

\begin{itemize}\item  subversion 761, 6/12/09: A working copy of \texttt{At} and the Geoquant system.
\end{itemize}

\subsection*{Testing}

\begin{quotation} The \texttt{At} function was tested in small modularized systems, then tested in a three dimensional \"{\i}\&not;ow, which required many varied uses of \texttt{At}.   Some retrieved quantities had their values set while others had their values accessed and compared with what mathematica calculations predicted.\end{quotation}

\subsection*{Future Work}

\begin{quotation} No future work is planned at this time.\end{quotation}
    
% html: End of file: `clean.html'

%%%%% END OF DOCUMENT BODY %%%%%
% In the future, we might want to put some additional data here, such
% as when the documentation was converted from wiki to TeX.
%

\end{document}
%
%EndExpansion

\bigskip 

\bigskip 

%TCIMACRO{\QSubDoc{Include hij_k}{%TCIDATA{LaTeXparent=0,0,functions.tex}}}%
%BeginExpansion
%TCIDATA{LaTeXparent=0,0,functions.tex}%
%EndExpansion

\bigskip

\bigskip

%TCIMACRO{\QSubDoc{Include hijk_l}{%TCIDATA{Version=5.00.0.2606}
%TCIDATA{LaTeXparent=1,1,functions.tex}
                      

\section*{\texttt{FaceHeight::FaceHeight\label{Face Height FUNCTION}}}

\subsection*{Function Prototype}

\texttt{double FaceHeight( Vertex vi, Vertex vj, Vertex vk, Vertex vl)}

\subsection*{Key Words}

Face height, edge height, geoquant.

\subsection*{Authors}

Daniel Champion

\subsection*{Introduction}

The function \texttt{FaceHeight}\ calculates the face height to the center
of a tetrahedron. \ 

\subsection*{Subsidiaries}

\textbf{Functions:}

\qquad\texttt{Geometry::dihedralAngle}

\qquad \texttt{EdgeHeight}

\qquad\texttt{listIntersection}

\textbf{Global Variables: }\ radii, etas.

\textbf{Local Variables:} \ Vertex vi, vj, vk, vl.

\subsection*{Description}

The calculation of \texttt{FaceHeight} involves the simple formula:%
\begin{equation*}
\text{\texttt{FaceHeight(vi, vj, vk, vl)}}=
\end{equation*}

\begin{equation*}
\frac{\left( \text{\texttt{EdgeHeight(vi,vj,vl)}}-\text{\texttt{%
EdgeHeight(vi,vj,vk)}}\cos (\beta _{ij,kl})\right) }{\sin \left( \beta
_{ij,kl}\right) }
\end{equation*}%
where $\beta _{ij,kl}$ is the dihedral angle along edge $\left\{
vi,vj\right\} $ of tetrahedron $\left\{ vi,vj,vk,vl\right\} $. \ A geometric
interpretation of this quantity is a follows. \ Given a decorated
tetrahedron (tetrahedron with radii and eta values assigned to the vertices
and edges respectively), the center of this tetrahedron can be calculated as
the common power point of its embedding into three-dimensional Euclidean
space. \ The perpendicular distance from this center point to the face $%
\left\{ vi,vj,vk\right\} $ is exactly \texttt{FaceHeight (vi, vj, vk, vl)}.
\ Take note that the first three vertices in the function call correspond to
the preferred face, and the fourth vertex in the function call identifies
the tetrahedron. \ 

A primary use of this function is in the calculation of several quantities
needed for the \texttt{CurvaturePartial} function used in the optimization
of the normalized Einstein-Hilbert-Regge functional.

\subsection*{Practicum}

An example of the use of this function is in the calculation of the dual
areas, \texttt{DualAreaSegment}, to an edge of a three dimensional
triangulation.

\qquad \texttt{double DualAreaSegment( Vertex vi, Vertex vj, Vertex vk,
Vertex vl)}

\qquad\qquad\texttt{\{}

\qquad \qquad \texttt{double result =
0.5*(EdgeHeight(vi,vj,vk)*FaceHeight(vi,vj,vk,vl)}

\qquad \qquad \qquad \texttt{+EdgeHeight(vi,vj,vl)*FaceHeight(vi,vj,vl,vk));}

\qquad\qquad\texttt{return result;}

\qquad\qquad\texttt{\}}

\subsection*{Limitations}

\texttt{faceHeight} must receive as input four vertices of a tetrahedron of
the triangulation. \ Moreover, the first three vertices in the function call
identify a face and can be in any order, however the fourth vertex in the
function call identifies the tetrahedron and can not be permuted with the
other three vertices. \ 

\subsection*{Revisions}

subversion 757, 6/8/09, \texttt{FaceHeight} created.

subversion 1055, 3/12/10, \texttt{FaceHeight}\ converted to a geoquant.

\subsection*{Testing}

This function was not tested.

\subsection*{Future Work}

This function has been incorporated into the Geometry class geoquants, and
thus this entry needs to be updated. \ 
}}%
%BeginExpansion
%TCIDATA{Version=5.00.0.2606}
%TCIDATA{LaTeXparent=1,1,functions.tex}
                      

\section*{\texttt{FaceHeight::FaceHeight\label{Face Height FUNCTION}}}

\subsection*{Function Prototype}

\texttt{double FaceHeight( Vertex vi, Vertex vj, Vertex vk, Vertex vl)}

\subsection*{Key Words}

Face height, edge height, geoquant.

\subsection*{Authors}

Daniel Champion

\subsection*{Introduction}

The function \texttt{FaceHeight}\ calculates the face height to the center
of a tetrahedron. \ 

\subsection*{Subsidiaries}

\textbf{Functions:}

\qquad\texttt{Geometry::dihedralAngle}

\qquad \texttt{EdgeHeight}

\qquad\texttt{listIntersection}

\textbf{Global Variables: }\ radii, etas.

\textbf{Local Variables:} \ Vertex vi, vj, vk, vl.

\subsection*{Description}

The calculation of \texttt{FaceHeight} involves the simple formula:%
\begin{equation*}
\text{\texttt{FaceHeight(vi, vj, vk, vl)}}=
\end{equation*}

\begin{equation*}
\frac{\left( \text{\texttt{EdgeHeight(vi,vj,vl)}}-\text{\texttt{%
EdgeHeight(vi,vj,vk)}}\cos (\beta _{ij,kl})\right) }{\sin \left( \beta
_{ij,kl}\right) }
\end{equation*}%
where $\beta _{ij,kl}$ is the dihedral angle along edge $\left\{
vi,vj\right\} $ of tetrahedron $\left\{ vi,vj,vk,vl\right\} $. \ A geometric
interpretation of this quantity is a follows. \ Given a decorated
tetrahedron (tetrahedron with radii and eta values assigned to the vertices
and edges respectively), the center of this tetrahedron can be calculated as
the common power point of its embedding into three-dimensional Euclidean
space. \ The perpendicular distance from this center point to the face $%
\left\{ vi,vj,vk\right\} $ is exactly \texttt{FaceHeight (vi, vj, vk, vl)}.
\ Take note that the first three vertices in the function call correspond to
the preferred face, and the fourth vertex in the function call identifies
the tetrahedron. \ 

A primary use of this function is in the calculation of several quantities
needed for the \texttt{CurvaturePartial} function used in the optimization
of the normalized Einstein-Hilbert-Regge functional.

\subsection*{Practicum}

An example of the use of this function is in the calculation of the dual
areas, \texttt{DualAreaSegment}, to an edge of a three dimensional
triangulation.

\qquad \texttt{double DualAreaSegment( Vertex vi, Vertex vj, Vertex vk,
Vertex vl)}

\qquad\qquad\texttt{\{}

\qquad \qquad \texttt{double result =
0.5*(EdgeHeight(vi,vj,vk)*FaceHeight(vi,vj,vk,vl)}

\qquad \qquad \qquad \texttt{+EdgeHeight(vi,vj,vl)*FaceHeight(vi,vj,vl,vk));}

\qquad\qquad\texttt{return result;}

\qquad\qquad\texttt{\}}

\subsection*{Limitations}

\texttt{faceHeight} must receive as input four vertices of a tetrahedron of
the triangulation. \ Moreover, the first three vertices in the function call
identify a face and can be in any order, however the fourth vertex in the
function call identifies the tetrahedron and can not be permuted with the
other three vertices. \ 

\subsection*{Revisions}

subversion 757, 6/8/09, \texttt{FaceHeight} created.

subversion 1055, 3/12/10, \texttt{FaceHeight}\ converted to a geoquant.

\subsection*{Testing}

This function was not tested.

\subsection*{Future Work}

This function has been incorporated into the Geometry class geoquants, and
thus this entry needs to be updated. \ 
%
%EndExpansion

\bigskip

\bigskip

%TCIMACRO{\QSubDoc{Include Lij_star}{%TCIDATA{Version=5.00.0.2606}
%TCIDATA{LaTeXparent=1,1,functions.tex}
                      

\section*{\texttt{DualArea::DualArea}}

\subsection*{Function Prototype}

\texttt{double DualArea(Edge e)}

\subsection*{Key Words}

Dual area, curvature, partial derivative, edge, geoquant.

\subsection*{Authors}

Daniel Champion

\subsection*{Introduction}

\texttt{DualArea} calculates the dual area of an edge.

\subsection*{Subsidiaries}

\subsubsection*{Functions: \ }

\qquad \texttt{DualAreaSegment}

\qquad \qquad \texttt{FaceHeight}

\qquad \qquad \qquad \texttt{EdgeHeight}

\qquad \qquad \qquad \qquad \texttt{PartialEdge}

\subsubsection*{Global Variables: \ }

\qquad Only radii and eta values are needed.

\subsubsection*{Local Variables: \ }

\qquad none

\subsection*{Description}

\texttt{DualArea} is defined as:%
\begin{equation*}
\text{\texttt{DualArea(edge e\_ij))}}=l_{ij}^{\ast }=\sum_{\substack{ \text{%
all tetrahedra }(i,j,k,l) \\ \text{containing edge }(i,j)\text{.}}}A_{ij,kl},
\end{equation*}%
where $A_{ij,kl}$ is computed with the function \texttt{DualAreaSegment}
applied to the vertices of the tetrahedron being summed over. \ Note that 
\texttt{DualAreaSegment} utilizes the functions \texttt{FaceHeight}, \texttt{%
EdgeHeight}, and \texttt{partialEdge}, however only the radii and eta values
are needed to calculate all of these quantities. \ 

This function was created for use in the \texttt{CurvaturePartial} function
which serves an essential role in calculating the second derivatives of the
Einstein-Hilbert-Regge functional (\texttt{EHRSecondPartial}). \ The second
order partial derivatives of the EHR functional are used in the optimization
of the EHR functional using Newton's method. \ \texttt{DualArea} will
eventually be used in the study of laplacians. \ 

\subsection*{Practicum}

Currently \texttt{DualArea} is only used to calculate the partial derivative
of curvature with respect to $\log $ radius. \ The following example
calculates the partial derivative of the curvature at vertex V with respect
to the $\log $ radius $r_{l}$ corresponding to vertex Vprime (adjacent to V).

\bigskip

\qquad\texttt{double sum = 0.0;}

\qquad\texttt{double dihedral\_sum = 0.0;}

\qquad\texttt{Vprime = Triangulation::vertexTable[l];}

\qquad\texttt{E =
Triangulation::edgeTable[listIntersection(V.getLocalEdges(),}

\qquad\qquad\texttt{Vprime.getLocalEdges())[0]];}

\qquad\texttt{// This assumes that there is a unique edge between two
vertices.}

\qquad\texttt{local\_tetra = E.getLocalTetras();}

\qquad\texttt{for (int m=0; m \TEXTsymbol{<} (*(local\_tetra)).size(); ++m)
\{}

\qquad\qquad\texttt{T = Triangulation::tetraTable[local\_tetra-\TEXTsymbol{>}%
at(m)];}

\qquad\qquad\texttt{dihedral\_sum += Geometry::dihedralAngle(E,T);}

\qquad\texttt{\}}

\qquad \texttt{result =
DualArea(E)/(Geometry::Length(E))-(2*PI-dihedral\_sum)}

\qquad \qquad \texttt{*(pow(Geometry::Radius(V),
2)*pow(Geometry::Radius(Vprime),2)}

\qquad \qquad \texttt{%
*(1-pow(Geometry::Eta(E),2)))/pow(Geometry::Length(E),3);}

\subsection*{Limitations}

\texttt{DualArea} can operate on any and all edges of a 3D triangulation
however it is only appropriate for triangulations where tetrahedra have
distinct edges. \ 

\subsection*{Revisions}

Subversion 676, 5/15/09, \texttt{DualArea} created within \texttt{%
Newtons\_Method}.

subversion 1055, 3/12/10, \texttt{DualArea}\ converted to a geoquant.

\subsection*{Testing}

\texttt{Lij\_star} has not been tested. \ 

\subsection*{Future Work}

\texttt{Lij\_star} should be moved to a more appropriate section of the
code. \ A more general volume function should be created that would take any
simplex object (including a boolean for dual simplices) and return the
appropriate volume. \ This general volume function would be an excellent
location for \texttt{Lij\_star}. \ It should be tested some time as well. \ 
}}%
%BeginExpansion
%TCIDATA{Version=5.00.0.2606}
%TCIDATA{LaTeXparent=1,1,functions.tex}
                      

\section*{\texttt{DualArea::DualArea}}

\subsection*{Function Prototype}

\texttt{double DualArea(Edge e)}

\subsection*{Key Words}

Dual area, curvature, partial derivative, edge, geoquant.

\subsection*{Authors}

Daniel Champion

\subsection*{Introduction}

\texttt{DualArea} calculates the dual area of an edge.

\subsection*{Subsidiaries}

\subsubsection*{Functions: \ }

\qquad \texttt{DualAreaSegment}

\qquad \qquad \texttt{FaceHeight}

\qquad \qquad \qquad \texttt{EdgeHeight}

\qquad \qquad \qquad \qquad \texttt{PartialEdge}

\subsubsection*{Global Variables: \ }

\qquad Only radii and eta values are needed.

\subsubsection*{Local Variables: \ }

\qquad none

\subsection*{Description}

\texttt{DualArea} is defined as:%
\begin{equation*}
\text{\texttt{DualArea(edge e\_ij))}}=l_{ij}^{\ast }=\sum_{\substack{ \text{%
all tetrahedra }(i,j,k,l) \\ \text{containing edge }(i,j)\text{.}}}A_{ij,kl},
\end{equation*}%
where $A_{ij,kl}$ is computed with the function \texttt{DualAreaSegment}
applied to the vertices of the tetrahedron being summed over. \ Note that 
\texttt{DualAreaSegment} utilizes the functions \texttt{FaceHeight}, \texttt{%
EdgeHeight}, and \texttt{partialEdge}, however only the radii and eta values
are needed to calculate all of these quantities. \ 

This function was created for use in the \texttt{CurvaturePartial} function
which serves an essential role in calculating the second derivatives of the
Einstein-Hilbert-Regge functional (\texttt{EHRSecondPartial}). \ The second
order partial derivatives of the EHR functional are used in the optimization
of the EHR functional using Newton's method. \ \texttt{DualArea} will
eventually be used in the study of laplacians. \ 

\subsection*{Practicum}

Currently \texttt{DualArea} is only used to calculate the partial derivative
of curvature with respect to $\log $ radius. \ The following example
calculates the partial derivative of the curvature at vertex V with respect
to the $\log $ radius $r_{l}$ corresponding to vertex Vprime (adjacent to V).

\bigskip

\qquad\texttt{double sum = 0.0;}

\qquad\texttt{double dihedral\_sum = 0.0;}

\qquad\texttt{Vprime = Triangulation::vertexTable[l];}

\qquad\texttt{E =
Triangulation::edgeTable[listIntersection(V.getLocalEdges(),}

\qquad\qquad\texttt{Vprime.getLocalEdges())[0]];}

\qquad\texttt{// This assumes that there is a unique edge between two
vertices.}

\qquad\texttt{local\_tetra = E.getLocalTetras();}

\qquad\texttt{for (int m=0; m \TEXTsymbol{<} (*(local\_tetra)).size(); ++m)
\{}

\qquad\qquad\texttt{T = Triangulation::tetraTable[local\_tetra-\TEXTsymbol{>}%
at(m)];}

\qquad\qquad\texttt{dihedral\_sum += Geometry::dihedralAngle(E,T);}

\qquad\texttt{\}}

\qquad \texttt{result =
DualArea(E)/(Geometry::Length(E))-(2*PI-dihedral\_sum)}

\qquad \qquad \texttt{*(pow(Geometry::Radius(V),
2)*pow(Geometry::Radius(Vprime),2)}

\qquad \qquad \texttt{%
*(1-pow(Geometry::Eta(E),2)))/pow(Geometry::Length(E),3);}

\subsection*{Limitations}

\texttt{DualArea} can operate on any and all edges of a 3D triangulation
however it is only appropriate for triangulations where tetrahedra have
distinct edges. \ 

\subsection*{Revisions}

Subversion 676, 5/15/09, \texttt{DualArea} created within \texttt{%
Newtons\_Method}.

subversion 1055, 3/12/10, \texttt{DualArea}\ converted to a geoquant.

\subsection*{Testing}

\texttt{Lij\_star} has not been tested. \ 

\subsection*{Future Work}

\texttt{Lij\_star} should be moved to a more appropriate section of the
code. \ A more general volume function should be created that would take any
simplex object (including a boolean for dual simplices) and return the
appropriate volume. \ This general volume function would be an excellent
location for \texttt{Lij\_star}. \ It should be tested some time as well. \ 
%
%EndExpansion

\bigskip

\bigskip 

%TCIMACRO{%
%\QSubDoc{Include makeTriangulationFile}{%html2tex: Version  2.7 of June 17, 2008.
%Written by  F.J. Faase.  http://www.iwriteiam.nl/

clean.html (21) : file `make3DTriangulationFile.html' does not exist.
clean.html (93) : file `readTriangulationFile.html' does not exist.
\documentclass[10pt]{article}%
\usepackage{amssymb}
\usepackage{geometry}
\usepackage{indentfirst}
\usepackage{amsmath}
\usepackage{amsfonts}
\usepackage{graphicx}%
\setcounter{MaxMatrixCols}{30}
%TCIDATA{OutputFilter=latex2.dll}
%TCIDATA{Version=5.00.0.2606}
%TCIDATA{CSTFile=40 LaTeX article.cst}
%TCIDATA{Created=Friday, March 30, 2007 00:21:27}
%TCIDATA{LastRevised=Wednesday, June 10, 2009 11:42:33}
%TCIDATA{<META NAME="GraphicsSave" CONTENT="32">}
%TCIDATA{<META NAME="SaveForMode" CONTENT="1">}
%TCIDATA{BibliographyScheme=Manual}
%TCIDATA{<META NAME="DocumentShell" CONTENT="Standard LaTeX\Blank - Standard LaTeX Article">}
%TCIDATA{Language=American English}
\newtheorem{theorem}{Theorem}
\newtheorem{acknowledgement}[theorem]{Acknowledgement}
\newtheorem{algorithm}[theorem]{Algorithm}
\newtheorem{axiom}[theorem]{Axiom}
\newtheorem{case}[theorem]{Case}
\newtheorem{claim}[theorem]{Claim}
\newtheorem{conclusion}[theorem]{Conclusion}
\newtheorem{condition}[theorem]{Condition}
\newtheorem{conjecture}[theorem]{Conjecture}
\newtheorem{corollary}[theorem]{Corollary}
\newtheorem{criterion}[theorem]{Criterion}
\newtheorem{definition}[theorem]{Definition}
\newtheorem{example}[theorem]{Example}
\newtheorem{exercise}[theorem]{Exercise}
\newtheorem{lemma}[theorem]{Lemma}
\newtheorem{notation}[theorem]{Notation}
\newtheorem{problem}[theorem]{Problem}
\newtheorem{proposition}[theorem]{Proposition}
\newtheorem{remark}[theorem]{Remark}
\newtheorem{solution}[theorem]{Solution}
\newtheorem{summary}[theorem]{Summary}
\newenvironment{proof}[1][Proof]{\noindent\textbf{#1.} }{\ \rule{0.5em}{0.5em}}
\geometry{left=1in,right=1in,top=1in,bottom=1in}

\begin{document}

%%%%% BEGINNING OF DOCUMENT BODY %%%%%
% html: Beginning of file: `clean.html'
% DOCTYPE HTML PUBLIC "-//W3C//DTD HTML 4.01//EN"
%  This is a (PRE) block.  Make sure it's left aligned or your toc title will be off. 

\section*{\texttt{makeTriangulationFile}}

\label{f0}{\small{\begin{verbatim} 
void makeTriangulationfile(char* fileIN, char* fileOUT)
\end{verbatim}
}}

\subsection*{Keywords}

\begin{quotation} triangulation, Lutz, simplices\end{quotation}

\subsection*{Authors}

\begin{itemize}\item  Alex Henniges
\item  Mitch Wilson
\end{itemize}

\subsection*{Introduction}

\begin{quotation} The \texttt{makeTriangulationFile} function converts a text file, given by \texttt{fileIn}, in the \texttt{Lutz} format to the standard format, printed to \texttt{fileOUT}. The file in standard format can then be read into the system to build the triangulation.\end{quotation}

\subsection*{Subsidiaries}

\begin{quotation} Functions:\end{quotation}
\begin{itemize}
\item  Pair::positionOf
\item  Pair::contains
\item  Pair::isInTuple
\end{itemize}
\begin{quotation} Global Variables:\end{quotation}
\begin{quotation} Local Variables: \texttt{fileIN}, \texttt{fileOUT}\end{quotation}

\subsection*{Description}

\begin{quotation} This function is used to convert one format to another format that we consider to be the standard for reading in a triangulation. We have dubbed the format we are converting from \texttt{Lutz}. This is based on the source we retrieve this format from, http://www.math.tu-berlin.de/diskregeom/stellar/\footnote{See URL http://www.math.tu-berlin.de/diskregeom/stellar/} . \end{quotation}
\begin{quotation} The \texttt{Lutz} format provides a simpler interface than our standard format, and can therefore allow for a user to create a quick triangulation. The idea is to provide only the index of every vertex on each face of the triangulation. No information about edges or adjacencies need to be given. The file should begin with a ``=\"{} followed by \"{}\mbox{$[$}\"{} and \"{}\mbox{$]$}'''s to contain the triangulation and each face. An example \texttt{Lutz} format for a tetrahedron is given \mbox{$[$}\#Practicum below\mbox{$]$}. \end{quotation}
\begin{quotation} Note that the \texttt{makeTriangulationFile} is used only for two-dimensional triangulations, and that for three-dimensions, one should use make3DTriangulationFile.\end{quotation}

\subsection*{Pracicum}

\begin{quotation} {\small{\begin{verbatim} 
  // Convert the tetrahedron written in Lutz format to a file in standard format.
  makeTriangulationFile("./tetra_lutz.txt", "./tetra_standard.txt");
  
  // Now read in the triangulation from standard format.
  readTriangulationFile("./tetra_standard.txt");
  \end{verbatim}
}}
\end{quotation}\begin{quotation} The \texttt{Lutz} format may look like:{\small{\begin{verbatim} 
=[[1,2,3],[1,2,4],[2,3,4],[1,3,4]]
  \end{verbatim}
}}
\end{quotation}\begin{quotation} The \texttt{makeTriangulationFile} would then create a file with:{\small{\begin{verbatim} 
Vertex: 1
2 3 4 
1 2 4 
1 2 4 
Vertex: 2
1 3 4 
1 3 5 
1 2 3 
Vertex: 3
1 2 4 
2 3 6 
1 3 4 
Vertex: 4
1 2 3 
4 5 6 
2 3 4 
Edge: 1
1 2
2 3 4 5 
1 2 
Edge: 2
1 3
1 3 4 6 
1 4 
Edge: 3
2 3
1 2 5 6 
1 3 
Edge: 4
1 4
1 2 5 6 
2 4 
Edge: 5
2 4
1 3 4 6 
2 3 
Edge: 6
3 4
2 3 4 5 
3 4 
Face: 1
1 2 3 
1 2 3 
2 3 4 
Face: 2
1 2 4 
1 4 5 
1 3 4 
Face: 3
2 3 4 
3 5 6 
1 2 4 
Face: 4
1 3 4 
2 4 6 
1 2 3 
  \end{verbatim}
}}
\end{quotation}
\subsection*{Limitations}

\begin{quotation} The limitation with the \texttt{Lutz} format that prevents it from being considered the standard format is that the user cannot create the most general of triangulations. To be more specific, it is impossible with the \texttt{Lutz} format to specify for there to be two edges with the same vertices.\end{quotation}
\begin{quotation} A limitation of the \texttt{makeTriangulationFile} is that its requirements are unintuitive. There should be no ``='' required, for example. Another limitation is that despite collecting enough information to build the triangulation, the function instead writes this to a file, requiring the user to subsequently call the function readTriangulationFile.\end{quotation}

\subsection*{Revisions}

\begin{itemize}\item  subversion 545, 9/29/08: Added the \texttt{makeTriangulationFile} function.
\end{itemize}

\subsection*{Testing}

\begin{quotation} This function has been tested through frequent use.\end{quotation}

\subsection*{Future Work}

\begin{itemize}\item  7/1 - Improve the format system.
\item  7/1 - Create the triangulation without performing a conversion to another file.
\end{itemize}
    
% html: End of file: `clean.html'

%%%%% END OF DOCUMENT BODY %%%%%
% In the future, we might want to put some additional data here, such
% as when the documentation was converted from wiki to TeX.
%

\end{document}
}}%
%BeginExpansion
%html2tex: Version  2.7 of June 17, 2008.
%Written by  F.J. Faase.  http://www.iwriteiam.nl/

clean.html (21) : file `make3DTriangulationFile.html' does not exist.
clean.html (93) : file `readTriangulationFile.html' does not exist.
\documentclass[10pt]{article}%
\usepackage{amssymb}
\usepackage{geometry}
\usepackage{indentfirst}
\usepackage{amsmath}
\usepackage{amsfonts}
\usepackage{graphicx}%
\setcounter{MaxMatrixCols}{30}
%TCIDATA{OutputFilter=latex2.dll}
%TCIDATA{Version=5.00.0.2606}
%TCIDATA{CSTFile=40 LaTeX article.cst}
%TCIDATA{Created=Friday, March 30, 2007 00:21:27}
%TCIDATA{LastRevised=Wednesday, June 10, 2009 11:42:33}
%TCIDATA{<META NAME="GraphicsSave" CONTENT="32">}
%TCIDATA{<META NAME="SaveForMode" CONTENT="1">}
%TCIDATA{BibliographyScheme=Manual}
%TCIDATA{<META NAME="DocumentShell" CONTENT="Standard LaTeX\Blank - Standard LaTeX Article">}
%TCIDATA{Language=American English}
\newtheorem{theorem}{Theorem}
\newtheorem{acknowledgement}[theorem]{Acknowledgement}
\newtheorem{algorithm}[theorem]{Algorithm}
\newtheorem{axiom}[theorem]{Axiom}
\newtheorem{case}[theorem]{Case}
\newtheorem{claim}[theorem]{Claim}
\newtheorem{conclusion}[theorem]{Conclusion}
\newtheorem{condition}[theorem]{Condition}
\newtheorem{conjecture}[theorem]{Conjecture}
\newtheorem{corollary}[theorem]{Corollary}
\newtheorem{criterion}[theorem]{Criterion}
\newtheorem{definition}[theorem]{Definition}
\newtheorem{example}[theorem]{Example}
\newtheorem{exercise}[theorem]{Exercise}
\newtheorem{lemma}[theorem]{Lemma}
\newtheorem{notation}[theorem]{Notation}
\newtheorem{problem}[theorem]{Problem}
\newtheorem{proposition}[theorem]{Proposition}
\newtheorem{remark}[theorem]{Remark}
\newtheorem{solution}[theorem]{Solution}
\newtheorem{summary}[theorem]{Summary}
\newenvironment{proof}[1][Proof]{\noindent\textbf{#1.} }{\ \rule{0.5em}{0.5em}}
\geometry{left=1in,right=1in,top=1in,bottom=1in}

\begin{document}

%%%%% BEGINNING OF DOCUMENT BODY %%%%%
% html: Beginning of file: `clean.html'
% DOCTYPE HTML PUBLIC "-//W3C//DTD HTML 4.01//EN"
%  This is a (PRE) block.  Make sure it's left aligned or your toc title will be off. 

\section*{\texttt{makeTriangulationFile}}

\label{f0}{\small{\begin{verbatim} 
void makeTriangulationfile(char* fileIN, char* fileOUT)
\end{verbatim}
}}

\subsection*{Keywords}

\begin{quotation} triangulation, Lutz, simplices\end{quotation}

\subsection*{Authors}

\begin{itemize}\item  Alex Henniges
\item  Mitch Wilson
\end{itemize}

\subsection*{Introduction}

\begin{quotation} The \texttt{makeTriangulationFile} function converts a text file, given by \texttt{fileIn}, in the \texttt{Lutz} format to the standard format, printed to \texttt{fileOUT}. The file in standard format can then be read into the system to build the triangulation.\end{quotation}

\subsection*{Subsidiaries}

\begin{quotation} Functions:\end{quotation}
\begin{itemize}
\item  Pair::positionOf
\item  Pair::contains
\item  Pair::isInTuple
\end{itemize}
\begin{quotation} Global Variables:\end{quotation}
\begin{quotation} Local Variables: \texttt{fileIN}, \texttt{fileOUT}\end{quotation}

\subsection*{Description}

\begin{quotation} This function is used to convert one format to another format that we consider to be the standard for reading in a triangulation. We have dubbed the format we are converting from \texttt{Lutz}. This is based on the source we retrieve this format from, http://www.math.tu-berlin.de/diskregeom/stellar/\footnote{See URL http://www.math.tu-berlin.de/diskregeom/stellar/} . \end{quotation}
\begin{quotation} The \texttt{Lutz} format provides a simpler interface than our standard format, and can therefore allow for a user to create a quick triangulation. The idea is to provide only the index of every vertex on each face of the triangulation. No information about edges or adjacencies need to be given. The file should begin with a ``=\"{} followed by \"{}\mbox{$[$}\"{} and \"{}\mbox{$]$}'''s to contain the triangulation and each face. An example \texttt{Lutz} format for a tetrahedron is given \mbox{$[$}\#Practicum below\mbox{$]$}. \end{quotation}
\begin{quotation} Note that the \texttt{makeTriangulationFile} is used only for two-dimensional triangulations, and that for three-dimensions, one should use make3DTriangulationFile.\end{quotation}

\subsection*{Pracicum}

\begin{quotation} {\small{\begin{verbatim} 
  // Convert the tetrahedron written in Lutz format to a file in standard format.
  makeTriangulationFile("./tetra_lutz.txt", "./tetra_standard.txt");
  
  // Now read in the triangulation from standard format.
  readTriangulationFile("./tetra_standard.txt");
  \end{verbatim}
}}
\end{quotation}\begin{quotation} The \texttt{Lutz} format may look like:{\small{\begin{verbatim} 
=[[1,2,3],[1,2,4],[2,3,4],[1,3,4]]
  \end{verbatim}
}}
\end{quotation}\begin{quotation} The \texttt{makeTriangulationFile} would then create a file with:{\small{\begin{verbatim} 
Vertex: 1
2 3 4 
1 2 4 
1 2 4 
Vertex: 2
1 3 4 
1 3 5 
1 2 3 
Vertex: 3
1 2 4 
2 3 6 
1 3 4 
Vertex: 4
1 2 3 
4 5 6 
2 3 4 
Edge: 1
1 2
2 3 4 5 
1 2 
Edge: 2
1 3
1 3 4 6 
1 4 
Edge: 3
2 3
1 2 5 6 
1 3 
Edge: 4
1 4
1 2 5 6 
2 4 
Edge: 5
2 4
1 3 4 6 
2 3 
Edge: 6
3 4
2 3 4 5 
3 4 
Face: 1
1 2 3 
1 2 3 
2 3 4 
Face: 2
1 2 4 
1 4 5 
1 3 4 
Face: 3
2 3 4 
3 5 6 
1 2 4 
Face: 4
1 3 4 
2 4 6 
1 2 3 
  \end{verbatim}
}}
\end{quotation}
\subsection*{Limitations}

\begin{quotation} The limitation with the \texttt{Lutz} format that prevents it from being considered the standard format is that the user cannot create the most general of triangulations. To be more specific, it is impossible with the \texttt{Lutz} format to specify for there to be two edges with the same vertices.\end{quotation}
\begin{quotation} A limitation of the \texttt{makeTriangulationFile} is that its requirements are unintuitive. There should be no ``='' required, for example. Another limitation is that despite collecting enough information to build the triangulation, the function instead writes this to a file, requiring the user to subsequently call the function readTriangulationFile.\end{quotation}

\subsection*{Revisions}

\begin{itemize}\item  subversion 545, 9/29/08: Added the \texttt{makeTriangulationFile} function.
\end{itemize}

\subsection*{Testing}

\begin{quotation} This function has been tested through frequent use.\end{quotation}

\subsection*{Future Work}

\begin{itemize}\item  7/1 - Improve the format system.
\item  7/1 - Create the triangulation without performing a conversion to another file.
\end{itemize}
    
% html: End of file: `clean.html'

%%%%% END OF DOCUMENT BODY %%%%%
% In the future, we might want to put some additional data here, such
% as when the documentation was converted from wiki to TeX.
%

\end{document}
%
%EndExpansion

\bigskip 

\bigskip 

%TCIMACRO{%
%\QSubDoc{Include NewtonsMethodOptimize}{%html2tex: Version  2.7 of June 17, 2008.
%Written by  F.J. Faase.  http://www.iwriteiam.nl/

\documentclass[10pt]{article}%
\usepackage{amssymb}
\usepackage{geometry}
\usepackage{indentfirst}
\usepackage{amsmath}
\usepackage{amsfonts}
\usepackage{graphicx}%
\setcounter{MaxMatrixCols}{30}
%TCIDATA{OutputFilter=latex2.dll}
%TCIDATA{Version=5.00.0.2606}
%TCIDATA{CSTFile=40 LaTeX article.cst}
%TCIDATA{Created=Friday, March 30, 2007 00:21:27}
%TCIDATA{LastRevised=Wednesday, June 10, 2009 11:42:33}
%TCIDATA{<META NAME="GraphicsSave" CONTENT="32">}
%TCIDATA{<META NAME="SaveForMode" CONTENT="1">}
%TCIDATA{BibliographyScheme=Manual}
%TCIDATA{<META NAME="DocumentShell" CONTENT="Standard LaTeX\Blank - Standard LaTeX Article">}
%TCIDATA{Language=American English}
\newtheorem{theorem}{Theorem}
\newtheorem{acknowledgement}[theorem]{Acknowledgement}
\newtheorem{algorithm}[theorem]{Algorithm}
\newtheorem{axiom}[theorem]{Axiom}
\newtheorem{case}[theorem]{Case}
\newtheorem{claim}[theorem]{Claim}
\newtheorem{conclusion}[theorem]{Conclusion}
\newtheorem{condition}[theorem]{Condition}
\newtheorem{conjecture}[theorem]{Conjecture}
\newtheorem{corollary}[theorem]{Corollary}
\newtheorem{criterion}[theorem]{Criterion}
\newtheorem{definition}[theorem]{Definition}
\newtheorem{example}[theorem]{Example}
\newtheorem{exercise}[theorem]{Exercise}
\newtheorem{lemma}[theorem]{Lemma}
\newtheorem{notation}[theorem]{Notation}
\newtheorem{problem}[theorem]{Problem}
\newtheorem{proposition}[theorem]{Proposition}
\newtheorem{remark}[theorem]{Remark}
\newtheorem{solution}[theorem]{Solution}
\newtheorem{summary}[theorem]{Summary}
\newenvironment{proof}[1][Proof]{\noindent\textbf{#1.} }{\ \rule{0.5em}{0.5em}}
\geometry{left=1in,right=1in,top=1in,bottom=1in}

\begin{document}

%%%%% BEGINNING OF DOCUMENT BODY %%%%%
% html: Beginning of file: `clean.html'
% DOCTYPE HTML PUBLIC "-//W3C//DTD HTML 4.01//EN"
%  This is a (PRE) block.  Make sure it's left aligned or your toc title will be off. 

\section*{\texttt{NewtonsMethod::optimize}}

\label{f0}{\small{\begin{verbatim} 
void optimize(double initial[], double soln[])
\end{verbatim}
}}

\subsection*{Keywords}

\begin{quotation} Newtons Method, optimize, extremum, gradient, hessian\end{quotation}

\subsection*{Authors}

\begin{itemize}\item  Alex Henniges
\end{itemize}

\subsection*{Introduction}

\begin{quotation} The \texttt{optimize} function of the NewtonsMethod class is designed to find either a maximum or minimum of a functional near a given point.\end{quotation}

\subsection*{Subsidiaries}

\begin{quotation} Functions:\end{quotation}
\begin{itemize}
\item  \texttt{NewtonsMethod::step}
\end{itemize}
\begin{quotation} \end{quotation}
\subsection*{Description}

\begin{quotation} The \texttt{optimize} function is called once by the user and it will continue to loop until an extremum of the functional is found. The functional is given in the constructor for NewtonsMethod. The initial point is the first parameter of the \texttt{optimize} function and the solution point is placed in the second parameter. This means that if one cannot be found, the function will loop without end. This is unlike the \texttt{step} function used within \texttt{optimize} that can also be used by a client program to gain much greater flexibility in the optimization process, such as more leeway on when to stop and allowing for data collection in between. See the \texttt{step} function for a description of how the optimization is performed.\end{quotation}

\subsection*{Practicum}

\begin{quotation} Example:{\small{\begin{verbatim} 
    double func(double vars[]) {
       double val = 1 - pow(vars[0], 2) / 4 - pow(vars[1], 2) / 9;
       return sqrt(val);
    }
    
    NewtonsMethod *nm = new NewtonsMethod(func, 2);
    double initial[] = {1, 1};
    double soln[2];

    nm->optimize(initial, soln);
  \end{verbatim}
}}
\end{quotation}
\subsection*{Limitations}

\begin{quotation} The \texttt{optimize} function is limited in its termination condition. This must be a constant over any use of the \texttt{optimize} function. It is also limited in that it may not terminate at all and the user will be forced to quit the program. Instead of modifying the function, these limitations are addressed by the \texttt{step} function which trades simplicity in terms of number of lines for greater flexibility.\end{quotation}

\subsection*{Revisions}

\begin{itemize}\item  subversion 876 7/16/09: Added a NewtonsMethod class for general maximizing.
\item  subversion 906 8/3/09: Changed the name of the function maximize to optimize in the NewtonsMethod class.
\end{itemize}

\subsection*{Testing}

\begin{quotation} Newtons Method has been tested using several functions of 1 or 2 variables including the Gaussian function. It has been tested with both approximating the gradient and hessian and when both are given explicitly.\end{quotation}

\subsection*{Future Work}

\begin{itemize}\item  8/4 - Add the ability to only move partially in the direction of the gradient.
\end{itemize}
    
% html: End of file: `clean.html'

%%%%% END OF DOCUMENT BODY %%%%%
% In the future, we might want to put some additional data here, such
% as when the documentation was converted from wiki to TeX.
%

\end{document}
}}%
%BeginExpansion
%html2tex: Version  2.7 of June 17, 2008.
%Written by  F.J. Faase.  http://www.iwriteiam.nl/

\documentclass[10pt]{article}%
\usepackage{amssymb}
\usepackage{geometry}
\usepackage{indentfirst}
\usepackage{amsmath}
\usepackage{amsfonts}
\usepackage{graphicx}%
\setcounter{MaxMatrixCols}{30}
%TCIDATA{OutputFilter=latex2.dll}
%TCIDATA{Version=5.00.0.2606}
%TCIDATA{CSTFile=40 LaTeX article.cst}
%TCIDATA{Created=Friday, March 30, 2007 00:21:27}
%TCIDATA{LastRevised=Wednesday, June 10, 2009 11:42:33}
%TCIDATA{<META NAME="GraphicsSave" CONTENT="32">}
%TCIDATA{<META NAME="SaveForMode" CONTENT="1">}
%TCIDATA{BibliographyScheme=Manual}
%TCIDATA{<META NAME="DocumentShell" CONTENT="Standard LaTeX\Blank - Standard LaTeX Article">}
%TCIDATA{Language=American English}
\newtheorem{theorem}{Theorem}
\newtheorem{acknowledgement}[theorem]{Acknowledgement}
\newtheorem{algorithm}[theorem]{Algorithm}
\newtheorem{axiom}[theorem]{Axiom}
\newtheorem{case}[theorem]{Case}
\newtheorem{claim}[theorem]{Claim}
\newtheorem{conclusion}[theorem]{Conclusion}
\newtheorem{condition}[theorem]{Condition}
\newtheorem{conjecture}[theorem]{Conjecture}
\newtheorem{corollary}[theorem]{Corollary}
\newtheorem{criterion}[theorem]{Criterion}
\newtheorem{definition}[theorem]{Definition}
\newtheorem{example}[theorem]{Example}
\newtheorem{exercise}[theorem]{Exercise}
\newtheorem{lemma}[theorem]{Lemma}
\newtheorem{notation}[theorem]{Notation}
\newtheorem{problem}[theorem]{Problem}
\newtheorem{proposition}[theorem]{Proposition}
\newtheorem{remark}[theorem]{Remark}
\newtheorem{solution}[theorem]{Solution}
\newtheorem{summary}[theorem]{Summary}
\newenvironment{proof}[1][Proof]{\noindent\textbf{#1.} }{\ \rule{0.5em}{0.5em}}
\geometry{left=1in,right=1in,top=1in,bottom=1in}

\begin{document}

%%%%% BEGINNING OF DOCUMENT BODY %%%%%
% html: Beginning of file: `clean.html'
% DOCTYPE HTML PUBLIC "-//W3C//DTD HTML 4.01//EN"
%  This is a (PRE) block.  Make sure it's left aligned or your toc title will be off. 

\section*{\texttt{NewtonsMethod::optimize}}

\label{f0}{\small{\begin{verbatim} 
void optimize(double initial[], double soln[])
\end{verbatim}
}}

\subsection*{Keywords}

\begin{quotation} Newtons Method, optimize, extremum, gradient, hessian\end{quotation}

\subsection*{Authors}

\begin{itemize}\item  Alex Henniges
\end{itemize}

\subsection*{Introduction}

\begin{quotation} The \texttt{optimize} function of the NewtonsMethod class is designed to find either a maximum or minimum of a functional near a given point.\end{quotation}

\subsection*{Subsidiaries}

\begin{quotation} Functions:\end{quotation}
\begin{itemize}
\item  \texttt{NewtonsMethod::step}
\end{itemize}
\begin{quotation} \end{quotation}
\subsection*{Description}

\begin{quotation} The \texttt{optimize} function is called once by the user and it will continue to loop until an extremum of the functional is found. The functional is given in the constructor for NewtonsMethod. The initial point is the first parameter of the \texttt{optimize} function and the solution point is placed in the second parameter. This means that if one cannot be found, the function will loop without end. This is unlike the \texttt{step} function used within \texttt{optimize} that can also be used by a client program to gain much greater flexibility in the optimization process, such as more leeway on when to stop and allowing for data collection in between. See the \texttt{step} function for a description of how the optimization is performed.\end{quotation}

\subsection*{Practicum}

\begin{quotation} Example:{\small{\begin{verbatim} 
    double func(double vars[]) {
       double val = 1 - pow(vars[0], 2) / 4 - pow(vars[1], 2) / 9;
       return sqrt(val);
    }
    
    NewtonsMethod *nm = new NewtonsMethod(func, 2);
    double initial[] = {1, 1};
    double soln[2];

    nm->optimize(initial, soln);
  \end{verbatim}
}}
\end{quotation}
\subsection*{Limitations}

\begin{quotation} The \texttt{optimize} function is limited in its termination condition. This must be a constant over any use of the \texttt{optimize} function. It is also limited in that it may not terminate at all and the user will be forced to quit the program. Instead of modifying the function, these limitations are addressed by the \texttt{step} function which trades simplicity in terms of number of lines for greater flexibility.\end{quotation}

\subsection*{Revisions}

\begin{itemize}\item  subversion 876 7/16/09: Added a NewtonsMethod class for general maximizing.
\item  subversion 906 8/3/09: Changed the name of the function maximize to optimize in the NewtonsMethod class.
\end{itemize}

\subsection*{Testing}

\begin{quotation} Newtons Method has been tested using several functions of 1 or 2 variables including the Gaussian function. It has been tested with both approximating the gradient and hessian and when both are given explicitly.\end{quotation}

\subsection*{Future Work}

\begin{itemize}\item  8/4 - Add the ability to only move partially in the direction of the gradient.
\end{itemize}
    
% html: End of file: `clean.html'

%%%%% END OF DOCUMENT BODY %%%%%
% In the future, we might want to put some additional data here, such
% as when the documentation was converted from wiki to TeX.
%

\end{document}
%
%EndExpansion

\bigskip 

\bigskip 

%TCIMACRO{\QSubDoc{Include pause}{%TCIDATA{Version=5.00.0.2606}
%TCIDATA{LaTeXparent=0,0,functions.tex}
                      

%%%%% BEGINNING OF DOCUMENT BODY %%%%%
% html: Beginning of file: `clean.html'
% DOCTYPE HTML PUBLIC "-//W3C//DTD HTML 4.01//EN"
%  This is a (PRE) block.  Make sure it's left aligned or your toc title will be off. 

\section*{\texttt{pause}}

\label{f0}

\begin{quotation}
{\small }
\end{quotation}

\begin{verbatim}
{\small    void pause();
}
{\small    void pause(char *fmt, ...);
}
{\small    
}
\end{verbatim}

\subsection*{Key Words}

\begin{quotation}
pause, print
\end{quotation}

\subsection*{Authors}

\begin{quotation}
Alex Henniges
\end{quotation}

\subsection*{Introduction}

\begin{quotation}
The \texttt{pause} freezes the current process until the user presses the 
\textbf{enter} key. This function also allows the user to print information
at the pause line.
\end{quotation}

\subsection*{Subsidiaries}

\begin{quotation}
Functions:
\end{quotation}

\begin{itemize}
\item \texttt{vprintf}

\item \texttt{scanf}

\item \texttt{fflush}
\end{itemize}

\begin{quotation}
Global Variables:

Local Variables:
\end{quotation}

\subsection*{Description}

\begin{quotation}
The \texttt{pause} function is designed to place break points in the code
that will stop the process until the user presses the enter key. There are
several uses to this. A standard one is debugging as it can allow a
programmer to step through a procedure. While there are usually similar
debugging options in code editors, this function can be added and removed
easily from within the code. The second use is that the console for programs
will close immediately after execution with some editors. Without a way to
freeze the program, the console would close before the data could be read
and interpreted.

There are two options for this \texttt{pause} function. If the default pause
is used, the following message will be printed:``PAUSE...''Pressing the 
\textbf{enter} key will resume the process. The function can also print out
a message provided to it. This uses the \texttt{vprintf} function so that
the printed information can be formatted text. The user must still press 
\textbf{enter} to resume when this form is used. Pressing other keys will
not affect the program.

Historically, the project has used {\small }
\end{quotation}

\begin{verbatim}
{\small   system("PAUSE");
}
{\small   
}
\end{verbatim}

\begin{quotation}
to pause the program. However, this can only be used on a Windows machine, a
limiting factor that we wish to remove from the project.
\end{quotation}

\subsection*{Practicum}

\begin{quotation}
Example:{\small }
\end{quotation}

\begin{verbatim}
{\small   pause("Done...press enter to exit."); // PAUSE
}
{\small   
}
\end{verbatim}

\subsection*{Limitations}

\begin{quotation}
One limitation of the \texttt{pause} function is that it only resumes after
pressing the \textbf{enter} key. This is compared to the former pause
function (see above) that would resume after pressing any key. This could
also be considered an improvement.
\end{quotation}

\subsection*{Revisions}

\begin{itemize}
\item subversion 909, 8/4/09: Added the fully functional \texttt{pause}
function.
\end{itemize}

\subsection*{Testing}

\begin{quotation}
The \texttt{pause} function has been tested simply through using it
extensively.
\end{quotation}

\subsection*{Future Work}

\begin{quotation}
No future work is planned at this time.
\end{quotation}

% html: End of file: `clean.html'

%%%%% END OF DOCUMENT BODY %%%%%
% In the future, we might want to put some additional data here, such
% as when the documentation was converted from wiki to TeX.
%
}}%
%BeginExpansion
%TCIDATA{Version=5.00.0.2606}
%TCIDATA{LaTeXparent=0,0,functions.tex}
                      

%%%%% BEGINNING OF DOCUMENT BODY %%%%%
% html: Beginning of file: `clean.html'
% DOCTYPE HTML PUBLIC "-//W3C//DTD HTML 4.01//EN"
%  This is a (PRE) block.  Make sure it's left aligned or your toc title will be off. 

\section*{\texttt{pause}}

\label{f0}

\begin{quotation}
{\small }
\end{quotation}

\begin{verbatim}
{\small    void pause();
}
{\small    void pause(char *fmt, ...);
}
{\small    
}
\end{verbatim}

\subsection*{Key Words}

\begin{quotation}
pause, print
\end{quotation}

\subsection*{Authors}

\begin{quotation}
Alex Henniges
\end{quotation}

\subsection*{Introduction}

\begin{quotation}
The \texttt{pause} freezes the current process until the user presses the 
\textbf{enter} key. This function also allows the user to print information
at the pause line.
\end{quotation}

\subsection*{Subsidiaries}

\begin{quotation}
Functions:
\end{quotation}

\begin{itemize}
\item \texttt{vprintf}

\item \texttt{scanf}

\item \texttt{fflush}
\end{itemize}

\begin{quotation}
Global Variables:

Local Variables:
\end{quotation}

\subsection*{Description}

\begin{quotation}
The \texttt{pause} function is designed to place break points in the code
that will stop the process until the user presses the enter key. There are
several uses to this. A standard one is debugging as it can allow a
programmer to step through a procedure. While there are usually similar
debugging options in code editors, this function can be added and removed
easily from within the code. The second use is that the console for programs
will close immediately after execution with some editors. Without a way to
freeze the program, the console would close before the data could be read
and interpreted.

There are two options for this \texttt{pause} function. If the default pause
is used, the following message will be printed:``PAUSE...''Pressing the 
\textbf{enter} key will resume the process. The function can also print out
a message provided to it. This uses the \texttt{vprintf} function so that
the printed information can be formatted text. The user must still press 
\textbf{enter} to resume when this form is used. Pressing other keys will
not affect the program.

Historically, the project has used {\small }
\end{quotation}

\begin{verbatim}
{\small   system("PAUSE");
}
{\small   
}
\end{verbatim}

\begin{quotation}
to pause the program. However, this can only be used on a Windows machine, a
limiting factor that we wish to remove from the project.
\end{quotation}

\subsection*{Practicum}

\begin{quotation}
Example:{\small }
\end{quotation}

\begin{verbatim}
{\small   pause("Done...press enter to exit."); // PAUSE
}
{\small   
}
\end{verbatim}

\subsection*{Limitations}

\begin{quotation}
One limitation of the \texttt{pause} function is that it only resumes after
pressing the \textbf{enter} key. This is compared to the former pause
function (see above) that would resume after pressing any key. This could
also be considered an improvement.
\end{quotation}

\subsection*{Revisions}

\begin{itemize}
\item subversion 909, 8/4/09: Added the fully functional \texttt{pause}
function.
\end{itemize}

\subsection*{Testing}

\begin{quotation}
The \texttt{pause} function has been tested simply through using it
extensively.
\end{quotation}

\subsection*{Future Work}

\begin{quotation}
No future work is planned at this time.
\end{quotation}

% html: End of file: `clean.html'

%%%%% END OF DOCUMENT BODY %%%%%
% In the future, we might want to put some additional data here, such
% as when the documentation was converted from wiki to TeX.
%
%
%EndExpansion

\bigskip 

\bigskip 

%TCIMACRO{%
%\QSubDoc{Include print3DResultsStep}{%html2tex: Version  2.7 of June 17, 2008.
%Written by  F.J. Faase.  http://www.iwriteiam.nl/

clean.html (22) : file `printResultsStep.html' does not exist.
clean.html (22) : file `printResultsVertex.html' does not exist.
\documentclass[10pt]{article}%
\usepackage{amssymb}
\usepackage{geometry}
\usepackage{indentfirst}
\usepackage{amsmath}
\usepackage{amsfonts}
\usepackage{graphicx}%
\setcounter{MaxMatrixCols}{30}
%TCIDATA{OutputFilter=latex2.dll}
%TCIDATA{Version=5.00.0.2606}
%TCIDATA{CSTFile=40 LaTeX article.cst}
%TCIDATA{Created=Friday, March 30, 2007 00:21:27}
%TCIDATA{LastRevised=Wednesday, June 10, 2009 11:42:33}
%TCIDATA{<META NAME="GraphicsSave" CONTENT="32">}
%TCIDATA{<META NAME="SaveForMode" CONTENT="1">}
%TCIDATA{BibliographyScheme=Manual}
%TCIDATA{<META NAME="DocumentShell" CONTENT="Standard LaTeX\Blank - Standard LaTeX Article">}
%TCIDATA{Language=American English}
\newtheorem{theorem}{Theorem}
\newtheorem{acknowledgement}[theorem]{Acknowledgement}
\newtheorem{algorithm}[theorem]{Algorithm}
\newtheorem{axiom}[theorem]{Axiom}
\newtheorem{case}[theorem]{Case}
\newtheorem{claim}[theorem]{Claim}
\newtheorem{conclusion}[theorem]{Conclusion}
\newtheorem{condition}[theorem]{Condition}
\newtheorem{conjecture}[theorem]{Conjecture}
\newtheorem{corollary}[theorem]{Corollary}
\newtheorem{criterion}[theorem]{Criterion}
\newtheorem{definition}[theorem]{Definition}
\newtheorem{example}[theorem]{Example}
\newtheorem{exercise}[theorem]{Exercise}
\newtheorem{lemma}[theorem]{Lemma}
\newtheorem{notation}[theorem]{Notation}
\newtheorem{problem}[theorem]{Problem}
\newtheorem{proposition}[theorem]{Proposition}
\newtheorem{remark}[theorem]{Remark}
\newtheorem{solution}[theorem]{Solution}
\newtheorem{summary}[theorem]{Summary}
\newenvironment{proof}[1][Proof]{\noindent\textbf{#1.} }{\ \rule{0.5em}{0.5em}}
\geometry{left=1in,right=1in,top=1in,bottom=1in}

\begin{document}

%%%%% BEGINNING OF DOCUMENT BODY %%%%%
% html: Beginning of file: `clean.html'
% DOCTYPE HTML PUBLIC "-//W3C//DTD HTML 4.01//EN"
%  This is a (PRE) block.  Make sure it's left aligned or your toc title will be off. 

\section*{\texttt{print3DResultsStep}}

\label{f0}\begin{quotation} {\small{\begin{verbatim} 
   void print3DResultsStep(char* fileName, vector<double>* radii, vector<double>* curvs)
   \end{verbatim}
}}
\end{quotation}
\subsection*{Key Words}

\begin{quotation} radii, curvatures, file, flow, step, print, three-dimensional\end{quotation}

\subsection*{Authors}

\begin{quotation} Alex Henniges\end{quotation}

\subsection*{Introduction}

\begin{quotation} The \texttt{print3DResultsStep} function prints out the results of a curvature flow, with the results grouped by each step of the flow. These results will be written to the file given by \texttt{filename}.\end{quotation}

\subsection*{Subsidiaries}

\begin{quotation} Functions:\end{quotation}
\begin{quotation} Global Variables:\end{quotation}
\begin{quotation} Local Variables: \texttt{int vertSize}, \texttt{int numSteps}\end{quotation}

\subsection*{Description}

\begin{quotation} Prints the results of a curvature flow into the file given by \texttt{filename}. The results, that is, the radii and curvature values, are given by vectors of doubles. Most commonly, these vectors are taken from the Approximator class after the flow is run. The \texttt{print3DResultsStep} function determines the number of vertices of the current triangulation and the total number of steps are then derived from this and the size of the vectors.\end{quotation}
\begin{quotation} There are several ways to display the results. The \texttt{print3DResultsStep} function groups by step. This means that for each step of the curvature flow, the radii and curvature values for each vertex is printed. In addition, since Yamabe flow converges with respect to curvature divided by radius, this value is printed as well. Therefore, this function should be used with three-dimensional curvature flows. An example is shown below. Other formats are given by printResultsStep , printResultsVertex, printResultsNum.\end{quotation}

\subsection*{Practicum}

\begin{quotation} Example:{\small{\begin{verbatim} 
  // Print the results of a curvature flow with Approximator app into file "ODEResult.txt"
  print3DResultsStep("./ODEResults.txt", app->radiiHistory, app->curvHistory);
  \end{verbatim}
}}
\end{quotation}\begin{quotation} The output of such an example may then be{\small{\begin{verbatim}       
         :
         :
       Vertex   5     0.8324396       8.5301529       10.2471738
       Total Curvature: 44.5286316

       Step    74     Radius          Curvature        Curv:Radius
       -----------------------------------------------------
       Vertex   1     0.8883594       9.3071126       10.4767428
       Vertex   2     0.8725496       9.0880458       10.4155064
       Vertex   3     0.8579899       8.8858872       10.3566333
       Vertex   4     0.8448655       8.7033021       10.3014058
       Vertex   5     0.8333839       8.5432883       10.2513233
       Total Curvature: 44.5276360

       Step    75     Radius          Curvature        Curv:Radius
       -----------------------------------------------------
       Vertex   1     0.8873282       9.2928034       10.4727922
         :
         :
  \end{verbatim}
}}
\end{quotation}
\subsection*{Limitations}

\begin{quotation} Currently the \texttt{print3DResultsStep} function is limited in the information it prints. As our curvature flow has evolved to record additional information such as volumes, it may be time to explore a more robust form for displaying results. As there is considerable dependence on the Approximator for the data vectors, it may be wise to place this and similar functions in the Approximator class.\end{quotation}

\subsection*{Revisions}

\begin{itemize}\item  subversion 545, 9/29/08: Moved the printing of results out of calcFlow and into a new function.
\item  subversion 783, 6/18/09: Small modifications in response to changes in the Approximator class.
\end{itemize}

\subsection*{Testing}

\begin{quotation} The \texttt{print3DResultsStep} function was tested by running multiple curvature flows and printing the results. It was considered working when the format of the data was as desired.\end{quotation}

\subsection*{Future Work}

\begin{itemize}\item  6/29 - Recreate the print functions to print more data and be more flexible.
\item  6/29 - Move the print functions into the Approximator class.
\end{itemize}
    
% html: End of file: `clean.html'

%%%%% END OF DOCUMENT BODY %%%%%
% In the future, we might want to put some additional data here, such
% as when the documentation was converted from wiki to TeX.
%

\end{document}
}}%
%BeginExpansion
%html2tex: Version  2.7 of June 17, 2008.
%Written by  F.J. Faase.  http://www.iwriteiam.nl/

clean.html (22) : file `printResultsStep.html' does not exist.
clean.html (22) : file `printResultsVertex.html' does not exist.
\documentclass[10pt]{article}%
\usepackage{amssymb}
\usepackage{geometry}
\usepackage{indentfirst}
\usepackage{amsmath}
\usepackage{amsfonts}
\usepackage{graphicx}%
\setcounter{MaxMatrixCols}{30}
%TCIDATA{OutputFilter=latex2.dll}
%TCIDATA{Version=5.00.0.2606}
%TCIDATA{CSTFile=40 LaTeX article.cst}
%TCIDATA{Created=Friday, March 30, 2007 00:21:27}
%TCIDATA{LastRevised=Wednesday, June 10, 2009 11:42:33}
%TCIDATA{<META NAME="GraphicsSave" CONTENT="32">}
%TCIDATA{<META NAME="SaveForMode" CONTENT="1">}
%TCIDATA{BibliographyScheme=Manual}
%TCIDATA{<META NAME="DocumentShell" CONTENT="Standard LaTeX\Blank - Standard LaTeX Article">}
%TCIDATA{Language=American English}
\newtheorem{theorem}{Theorem}
\newtheorem{acknowledgement}[theorem]{Acknowledgement}
\newtheorem{algorithm}[theorem]{Algorithm}
\newtheorem{axiom}[theorem]{Axiom}
\newtheorem{case}[theorem]{Case}
\newtheorem{claim}[theorem]{Claim}
\newtheorem{conclusion}[theorem]{Conclusion}
\newtheorem{condition}[theorem]{Condition}
\newtheorem{conjecture}[theorem]{Conjecture}
\newtheorem{corollary}[theorem]{Corollary}
\newtheorem{criterion}[theorem]{Criterion}
\newtheorem{definition}[theorem]{Definition}
\newtheorem{example}[theorem]{Example}
\newtheorem{exercise}[theorem]{Exercise}
\newtheorem{lemma}[theorem]{Lemma}
\newtheorem{notation}[theorem]{Notation}
\newtheorem{problem}[theorem]{Problem}
\newtheorem{proposition}[theorem]{Proposition}
\newtheorem{remark}[theorem]{Remark}
\newtheorem{solution}[theorem]{Solution}
\newtheorem{summary}[theorem]{Summary}
\newenvironment{proof}[1][Proof]{\noindent\textbf{#1.} }{\ \rule{0.5em}{0.5em}}
\geometry{left=1in,right=1in,top=1in,bottom=1in}

\begin{document}

%%%%% BEGINNING OF DOCUMENT BODY %%%%%
% html: Beginning of file: `clean.html'
% DOCTYPE HTML PUBLIC "-//W3C//DTD HTML 4.01//EN"
%  This is a (PRE) block.  Make sure it's left aligned or your toc title will be off. 

\section*{\texttt{print3DResultsStep}}

\label{f0}\begin{quotation} {\small{\begin{verbatim} 
   void print3DResultsStep(char* fileName, vector<double>* radii, vector<double>* curvs)
   \end{verbatim}
}}
\end{quotation}
\subsection*{Key Words}

\begin{quotation} radii, curvatures, file, flow, step, print, three-dimensional\end{quotation}

\subsection*{Authors}

\begin{quotation} Alex Henniges\end{quotation}

\subsection*{Introduction}

\begin{quotation} The \texttt{print3DResultsStep} function prints out the results of a curvature flow, with the results grouped by each step of the flow. These results will be written to the file given by \texttt{filename}.\end{quotation}

\subsection*{Subsidiaries}

\begin{quotation} Functions:\end{quotation}
\begin{quotation} Global Variables:\end{quotation}
\begin{quotation} Local Variables: \texttt{int vertSize}, \texttt{int numSteps}\end{quotation}

\subsection*{Description}

\begin{quotation} Prints the results of a curvature flow into the file given by \texttt{filename}. The results, that is, the radii and curvature values, are given by vectors of doubles. Most commonly, these vectors are taken from the Approximator class after the flow is run. The \texttt{print3DResultsStep} function determines the number of vertices of the current triangulation and the total number of steps are then derived from this and the size of the vectors.\end{quotation}
\begin{quotation} There are several ways to display the results. The \texttt{print3DResultsStep} function groups by step. This means that for each step of the curvature flow, the radii and curvature values for each vertex is printed. In addition, since Yamabe flow converges with respect to curvature divided by radius, this value is printed as well. Therefore, this function should be used with three-dimensional curvature flows. An example is shown below. Other formats are given by printResultsStep , printResultsVertex, printResultsNum.\end{quotation}

\subsection*{Practicum}

\begin{quotation} Example:{\small{\begin{verbatim} 
  // Print the results of a curvature flow with Approximator app into file "ODEResult.txt"
  print3DResultsStep("./ODEResults.txt", app->radiiHistory, app->curvHistory);
  \end{verbatim}
}}
\end{quotation}\begin{quotation} The output of such an example may then be{\small{\begin{verbatim}       
         :
         :
       Vertex   5     0.8324396       8.5301529       10.2471738
       Total Curvature: 44.5286316

       Step    74     Radius          Curvature        Curv:Radius
       -----------------------------------------------------
       Vertex   1     0.8883594       9.3071126       10.4767428
       Vertex   2     0.8725496       9.0880458       10.4155064
       Vertex   3     0.8579899       8.8858872       10.3566333
       Vertex   4     0.8448655       8.7033021       10.3014058
       Vertex   5     0.8333839       8.5432883       10.2513233
       Total Curvature: 44.5276360

       Step    75     Radius          Curvature        Curv:Radius
       -----------------------------------------------------
       Vertex   1     0.8873282       9.2928034       10.4727922
         :
         :
  \end{verbatim}
}}
\end{quotation}
\subsection*{Limitations}

\begin{quotation} Currently the \texttt{print3DResultsStep} function is limited in the information it prints. As our curvature flow has evolved to record additional information such as volumes, it may be time to explore a more robust form for displaying results. As there is considerable dependence on the Approximator for the data vectors, it may be wise to place this and similar functions in the Approximator class.\end{quotation}

\subsection*{Revisions}

\begin{itemize}\item  subversion 545, 9/29/08: Moved the printing of results out of calcFlow and into a new function.
\item  subversion 783, 6/18/09: Small modifications in response to changes in the Approximator class.
\end{itemize}

\subsection*{Testing}

\begin{quotation} The \texttt{print3DResultsStep} function was tested by running multiple curvature flows and printing the results. It was considered working when the format of the data was as desired.\end{quotation}

\subsection*{Future Work}

\begin{itemize}\item  6/29 - Recreate the print functions to print more data and be more flexible.
\item  6/29 - Move the print functions into the Approximator class.
\end{itemize}
    
% html: End of file: `clean.html'

%%%%% END OF DOCUMENT BODY %%%%%
% In the future, we might want to put some additional data here, such
% as when the documentation was converted from wiki to TeX.
%

\end{document}
%
%EndExpansion

\bigskip 

\bigskip 

%TCIMACRO{\QSubDoc{Include printResultsNum}{%TCIDATA{Version=5.00.0.2606}
%TCIDATA{LaTeXparent=0,0,functions.tex}
                      

%%%%% BEGINNING OF DOCUMENT BODY %%%%%
% html: Beginning of file: `clean.html'
% DOCTYPE HTML PUBLIC "-//W3C//DTD HTML 4.01//EN"
%  This is a (PRE) block.  Make sure it's left aligned or your toc title will be off. 

\section*{\texttt{printResultsNum}}

\label{f0}{\small }
\begin{verbatim}
{\small 
        void printResultsNum(char* fileName, vector<double>* radii, vector<double>* curvs)
}
\end{verbatim}

\subsection*{Key Words}

radii, curvatures, file, flow, vertex, print

\subsection*{Authors}

Alex Henniges

\subsection*{Introduction}

The \texttt{printResultsNum} function prints out the results of a curvature
flow, with the results grouped by each vertex of the triangulation but
without labels. This format is used for when a program, (GUI, Matlab, etc)
needs to parse the data. These results will be written to the file given by 
\texttt{filename}.

\subsection*{Subsidiaries}

Functions:

Global Variables:

Local Variables: \texttt{int vertSize}, \texttt{int numSteps}

\subsection*{Description}

Prints the results of a curvature flow into the file given by \texttt{%
filename}. The results, that is, the radii and curvature values, are given
by vectors of doubles. Most commonly, these vectors are taken from the
Approximator class after the flow is run. The \texttt{printResultsNum}
function determines the number of vertices of the current triangulation and
the total number of steps are then derived from this and the size of the
vectors.

There are several ways to display the results. The \texttt{printResultsNum}
function groups by vertex but provides no labels. This means that for each
vertex of the triangulation, the radii (first column) and curvature (second
column) values for each step is given. This format is used for when a
program, (GUI, Matlab, etc) needs to parse the data. Therefore, it would be
difficult for a human to read, but allows the computer to do so much easier.
An example is shown below. Other formats are given by printResultsStep,
printResultsVertex, print3DResultsStep.

\subsection*{Practicum}

Example: {\small }
\begin{verbatim}
{\small     // Print the results of a curvature flow with
}
{\small     // Approximator app into file "ODEResult.txt"
}
{\small 
    printResultsNum("./ODEResults.txt", app->radiiHistory, app->curvHistory); 
}
\end{verbatim}

The output of such an example may then be {\small }
\begin{verbatim}
{\small          :
}
{\small          :
}
{\small       0.8272160717        3.1425515910
}
{\small       0.8272081392        3.1425294463
}
{\small       0.8272003900        3.1425078130
}
 
{\small       1.0000000000        3.1415926536
}
{\small       1.0000000000        3.5987926375
}
{\small       0.9954280002        3.5877016632
}
{\small       0.9909873062        3.5768691746
}
{\small       0.9866737711        3.5662905388
}
{\small          :
}
{\small          :
}
\end{verbatim}

\subsection*{Limitations}

Currently the \texttt{printResultsNum} function is limited in the
information it prints. As our curvature flow has evolved to record
additional information such as volumes, it may be time to explore a more
robust form for displaying results. As there is considerable dependence on
the Approximator for the data vectors, it may be wise to place this and
similar functions in the \mbox{$[$}CurvatureFlow Approximator\mbox{$]$}
class.

\subsection*{Revisions}

\textbf{\ subversion 545, 9/29/08: Moved the printing of results out of
calcFlow and into a new function. }  subversion 783, 6/18/09: Small
modifications in response to changes in the Approximator class.

\subsection*{Testing}

The \texttt{printResultsNum} function was tested by running multiple
curvature flows and printing the results. It was considered working when the
format of the data was as desired.

\subsection*{Future Work}

\textbf{\ 6/29 - Recreate the print functions to print more data and be more
flexible.}  6/29 - Move the print functions into the Approximator class.

% html: End of file: `clean.html'

%%%%% END OF DOCUMENT BODY %%%%%
% In the future, we might want to put some additional data here, such
% as when the documentation was converted from wiki to TeX.
%
}}%
%BeginExpansion
%TCIDATA{Version=5.00.0.2606}
%TCIDATA{LaTeXparent=0,0,functions.tex}
                      

%%%%% BEGINNING OF DOCUMENT BODY %%%%%
% html: Beginning of file: `clean.html'
% DOCTYPE HTML PUBLIC "-//W3C//DTD HTML 4.01//EN"
%  This is a (PRE) block.  Make sure it's left aligned or your toc title will be off. 

\section*{\texttt{printResultsNum}}

\label{f0}{\small }
\begin{verbatim}
{\small 
        void printResultsNum(char* fileName, vector<double>* radii, vector<double>* curvs)
}
\end{verbatim}

\subsection*{Key Words}

radii, curvatures, file, flow, vertex, print

\subsection*{Authors}

Alex Henniges

\subsection*{Introduction}

The \texttt{printResultsNum} function prints out the results of a curvature
flow, with the results grouped by each vertex of the triangulation but
without labels. This format is used for when a program, (GUI, Matlab, etc)
needs to parse the data. These results will be written to the file given by 
\texttt{filename}.

\subsection*{Subsidiaries}

Functions:

Global Variables:

Local Variables: \texttt{int vertSize}, \texttt{int numSteps}

\subsection*{Description}

Prints the results of a curvature flow into the file given by \texttt{%
filename}. The results, that is, the radii and curvature values, are given
by vectors of doubles. Most commonly, these vectors are taken from the
Approximator class after the flow is run. The \texttt{printResultsNum}
function determines the number of vertices of the current triangulation and
the total number of steps are then derived from this and the size of the
vectors.

There are several ways to display the results. The \texttt{printResultsNum}
function groups by vertex but provides no labels. This means that for each
vertex of the triangulation, the radii (first column) and curvature (second
column) values for each step is given. This format is used for when a
program, (GUI, Matlab, etc) needs to parse the data. Therefore, it would be
difficult for a human to read, but allows the computer to do so much easier.
An example is shown below. Other formats are given by printResultsStep,
printResultsVertex, print3DResultsStep.

\subsection*{Practicum}

Example: {\small }
\begin{verbatim}
{\small     // Print the results of a curvature flow with
}
{\small     // Approximator app into file "ODEResult.txt"
}
{\small 
    printResultsNum("./ODEResults.txt", app->radiiHistory, app->curvHistory); 
}
\end{verbatim}

The output of such an example may then be {\small }
\begin{verbatim}
{\small          :
}
{\small          :
}
{\small       0.8272160717        3.1425515910
}
{\small       0.8272081392        3.1425294463
}
{\small       0.8272003900        3.1425078130
}
 
{\small       1.0000000000        3.1415926536
}
{\small       1.0000000000        3.5987926375
}
{\small       0.9954280002        3.5877016632
}
{\small       0.9909873062        3.5768691746
}
{\small       0.9866737711        3.5662905388
}
{\small          :
}
{\small          :
}
\end{verbatim}

\subsection*{Limitations}

Currently the \texttt{printResultsNum} function is limited in the
information it prints. As our curvature flow has evolved to record
additional information such as volumes, it may be time to explore a more
robust form for displaying results. As there is considerable dependence on
the Approximator for the data vectors, it may be wise to place this and
similar functions in the \mbox{$[$}CurvatureFlow Approximator\mbox{$]$}
class.

\subsection*{Revisions}

\textbf{\ subversion 545, 9/29/08: Moved the printing of results out of
calcFlow and into a new function. }  subversion 783, 6/18/09: Small
modifications in response to changes in the Approximator class.

\subsection*{Testing}

The \texttt{printResultsNum} function was tested by running multiple
curvature flows and printing the results. It was considered working when the
format of the data was as desired.

\subsection*{Future Work}

\textbf{\ 6/29 - Recreate the print functions to print more data and be more
flexible.}  6/29 - Move the print functions into the Approximator class.

% html: End of file: `clean.html'

%%%%% END OF DOCUMENT BODY %%%%%
% In the future, we might want to put some additional data here, such
% as when the documentation was converted from wiki to TeX.
%
%
%EndExpansion

\bigskip 

\bigskip 

%TCIMACRO{%
%\QSubDoc{Include printResultsNumSteps}{%TCIDATA{Version=5.00.0.2606}
%TCIDATA{LaTeXparent=0,0,functions.tex}
                      

%%%%% BEGINNING OF DOCUMENT BODY %%%%%
% html: Beginning of file: `clean.html'
% DOCTYPE HTML PUBLIC "-//W3C//DTD HTML 4.01//EN"
%  This is a (PRE) block.  Make sure it's left aligned or your toc title will be off. 

\section*{\texttt{printResultsNumSteps}}

\label{f0}

\begin{quotation}
{\small }
\end{quotation}

\begin{verbatim}
{\small 
   void printResultsNumSteps(char* fileName, vector<double>* radii, vector<double>* curvs)
}
{\small    
}
\end{verbatim}

\subsection*{Key Words}

\begin{quotation}
radii, curvatures, file, flow, vertex, print
\end{quotation}

\subsection*{Authors}

\begin{quotation}
Alex Henniges
\end{quotation}

\subsection*{Introduction}

\begin{quotation}
The \texttt{printResultsNumSteps} function prints out the results of a
curvature flow, with the results grouped by each step of the triangulation
but without labels. This format is used for with the GUI to create a
polygonal representation of curvatures. These results will be written to the
file given by \texttt{filename}.
\end{quotation}

\subsection*{Subsidiaries}

\begin{quotation}
Functions:

Global Variables:

Local Variables: \texttt{int vertSize}, \texttt{int numSteps}
\end{quotation}

\subsection*{Description}

\begin{quotation}
Prints the results of a curvature flow into the file given by \texttt{%
filename}. The results are curvature divided by radii values, and are given
by vectors of doubles. Most commonly, these vectors are taken from the
Approximator class after the flow is run. The \texttt{printResultsNumSteps}
function determines the number of vertices of the current triangulation and
the total number of steps are then derived from this and the size of the
vectors.

There are several ways to display the results. The \texttt{%
printResultsNumSteps} function groups by step but provides no labels and
does not print out radii, but instead curvautre divided by radii. The
purpose for this format is to create the ``Polygon flows'' in the GUI.
Therefore, it would be difficult for a human to read, but allows the
computer to do so much easier. An example is shown below.
\end{quotation}

\subsection*{Practicum}

\begin{quotation}
Example:{\small }
\end{quotation}

\begin{verbatim}
{\small 
  // Print the results of a curvature flow with Approximator app into file "ODEResult.txt"
}
{\small 
  printResultsNumSteps("./ODEResults.txt", app->radiiHistory, app->curvHistory);
}
{\small   
}
\end{verbatim}

\begin{quotation}
The output of such an example may then be{\small }
\end{quotation}

\begin{verbatim}
{\small          :
}
{\small          :
}
{\small       3.1425515910
}
{\small       3.1425294463
}
{\small       3.1425078130
}
 
{\small       3.1415926536
}
{\small       3.5987926375
}
{\small       3.5877016632
}
{\small       3.5768691746
}
{\small       3.5662905388
}
{\small          :
}
{\small          :
}
{\small   
}
\end{verbatim}

\subsection*{Limitations}

\begin{quotation}
Unlike the other print functions, the purpose of \texttt{printResultsNumSteps%
} is to only display the curvature divided by radii, and so is not limited
in the information it prints. On the otherhand, an overhaul of the entire
printing system would likely involve modifying this function.
\end{quotation}

\subsection*{Revisions}

\begin{itemize}
\item subversion 545, 9/29/08: Moved the printing of results out of calcFlow
and into a new function.

\item subversion 783, 6/18/09: Small modifications in response to changes in
the Approximator class.
\end{itemize}

\subsection*{Testing}

\begin{quotation}
The \texttt{printResultsNumSteps} function was tested by running multiple
curvature flows and printing the results. It was considered working when the
format of the data was as desired.
\end{quotation}

\subsection*{Future Work}

\begin{itemize}
\item 6/29 - Recreate the print functions to print more data and be more
flexible.

\item 6/29 - Move the print functions into the Approximator class.
\end{itemize}

% html: End of file: `clean.html'

%%%%% END OF DOCUMENT BODY %%%%%
% In the future, we might want to put some additional data here, such
% as when the documentation was converted from wiki to TeX.
%
}}%
%BeginExpansion
%TCIDATA{Version=5.00.0.2606}
%TCIDATA{LaTeXparent=0,0,functions.tex}
                      

%%%%% BEGINNING OF DOCUMENT BODY %%%%%
% html: Beginning of file: `clean.html'
% DOCTYPE HTML PUBLIC "-//W3C//DTD HTML 4.01//EN"
%  This is a (PRE) block.  Make sure it's left aligned or your toc title will be off. 

\section*{\texttt{printResultsNumSteps}}

\label{f0}

\begin{quotation}
{\small }
\end{quotation}

\begin{verbatim}
{\small 
   void printResultsNumSteps(char* fileName, vector<double>* radii, vector<double>* curvs)
}
{\small    
}
\end{verbatim}

\subsection*{Key Words}

\begin{quotation}
radii, curvatures, file, flow, vertex, print
\end{quotation}

\subsection*{Authors}

\begin{quotation}
Alex Henniges
\end{quotation}

\subsection*{Introduction}

\begin{quotation}
The \texttt{printResultsNumSteps} function prints out the results of a
curvature flow, with the results grouped by each step of the triangulation
but without labels. This format is used for with the GUI to create a
polygonal representation of curvatures. These results will be written to the
file given by \texttt{filename}.
\end{quotation}

\subsection*{Subsidiaries}

\begin{quotation}
Functions:

Global Variables:

Local Variables: \texttt{int vertSize}, \texttt{int numSteps}
\end{quotation}

\subsection*{Description}

\begin{quotation}
Prints the results of a curvature flow into the file given by \texttt{%
filename}. The results are curvature divided by radii values, and are given
by vectors of doubles. Most commonly, these vectors are taken from the
Approximator class after the flow is run. The \texttt{printResultsNumSteps}
function determines the number of vertices of the current triangulation and
the total number of steps are then derived from this and the size of the
vectors.

There are several ways to display the results. The \texttt{%
printResultsNumSteps} function groups by step but provides no labels and
does not print out radii, but instead curvautre divided by radii. The
purpose for this format is to create the ``Polygon flows'' in the GUI.
Therefore, it would be difficult for a human to read, but allows the
computer to do so much easier. An example is shown below.
\end{quotation}

\subsection*{Practicum}

\begin{quotation}
Example:{\small }
\end{quotation}

\begin{verbatim}
{\small 
  // Print the results of a curvature flow with Approximator app into file "ODEResult.txt"
}
{\small 
  printResultsNumSteps("./ODEResults.txt", app->radiiHistory, app->curvHistory);
}
{\small   
}
\end{verbatim}

\begin{quotation}
The output of such an example may then be{\small }
\end{quotation}

\begin{verbatim}
{\small          :
}
{\small          :
}
{\small       3.1425515910
}
{\small       3.1425294463
}
{\small       3.1425078130
}
 
{\small       3.1415926536
}
{\small       3.5987926375
}
{\small       3.5877016632
}
{\small       3.5768691746
}
{\small       3.5662905388
}
{\small          :
}
{\small          :
}
{\small   
}
\end{verbatim}

\subsection*{Limitations}

\begin{quotation}
Unlike the other print functions, the purpose of \texttt{printResultsNumSteps%
} is to only display the curvature divided by radii, and so is not limited
in the information it prints. On the otherhand, an overhaul of the entire
printing system would likely involve modifying this function.
\end{quotation}

\subsection*{Revisions}

\begin{itemize}
\item subversion 545, 9/29/08: Moved the printing of results out of calcFlow
and into a new function.

\item subversion 783, 6/18/09: Small modifications in response to changes in
the Approximator class.
\end{itemize}

\subsection*{Testing}

\begin{quotation}
The \texttt{printResultsNumSteps} function was tested by running multiple
curvature flows and printing the results. It was considered working when the
format of the data was as desired.
\end{quotation}

\subsection*{Future Work}

\begin{itemize}
\item 6/29 - Recreate the print functions to print more data and be more
flexible.

\item 6/29 - Move the print functions into the Approximator class.
\end{itemize}

% html: End of file: `clean.html'

%%%%% END OF DOCUMENT BODY %%%%%
% In the future, we might want to put some additional data here, such
% as when the documentation was converted from wiki to TeX.
%
%
%EndExpansion

\bigskip 

\bigskip 

%TCIMACRO{\QSubDoc{Include printResultsStep}{%TCIDATA{Version=5.00.0.2606}
%TCIDATA{LaTeXparent=0,0,functions.tex}
                      

%%%%% BEGINNING OF DOCUMENT BODY %%%%%
% html: Beginning of file: `clean.html'
% DOCTYPE HTML PUBLIC "-//W3C//DTD HTML 4.01//EN"
%  This is a (PRE) block.  Make sure it's left aligned or your toc title will be off. 

\section*{\texttt{printResultsStep}}

\label{f0}

\begin{quotation}
{\small }
\end{quotation}

\begin{verbatim}
{\small 
   void printResultsStep(char* fileName, vector<double>* radii, vector<double>* curvs)
}
{\small    
}
\end{verbatim}

\subsection*{Key Words}

\begin{quotation}
radii, curvatures, file, flow, step, print
\end{quotation}

\subsection*{Authors}

\begin{quotation}
Alex Henniges
\end{quotation}

\subsection*{Introduction}

\begin{quotation}
The \texttt{printResultsStep} function prints out the results of a curvature
flow, with the results grouped by each step of the flow. These results will
be written to the file given by \texttt{filename}.
\end{quotation}

\subsection*{Subsidiaries}

\begin{quotation}
Functions:

Global Variables:

Local Variables: \texttt{int vertSize}, \texttt{int numSteps}
\end{quotation}

\subsection*{Description}

\begin{quotation}
Prints the results of a curvature flow into the file given by \texttt{%
filename}. The results, that is, the radii and curvature values, are given
by vectors of doubles. Most commonly, these vectors are taken from the
Approximator class after the flow is run. The \texttt{printResultsStep}
function determines the number of vertices of the current triangulation and
the total number of steps are then derived from this and the size of the
vectors.

There are several ways to display the results. The \texttt{printResultsStep}
function groups by step. This means that for each step of the curvature
flow, the radii and curvature values for each vertex is printed. An example
is shown below. Other formats are given by printResultsVertex,
printResultsNum, print3DResultsStep.
\end{quotation}

\subsection*{Practicum}

\begin{quotation}
Example:{\small }
\end{quotation}

\begin{verbatim}
{\small 
  // Print the results of a curvature flow with Approximator app into file "ODEResult.txt"
}
{\small 
  printResultsStep("./ODEResults.txt", app->radiiHistory, app->curvHistory);
}
{\small   
}
\end{verbatim}

\begin{quotation}
The output of such an example may then be{\small }
\end{quotation}

\begin{verbatim}
{\small          :    
}
{\small          :
}
{\small        Vertex   4     0.5397923       2.1411007
}
{\small        Total Curvature: 12.5663706
}
 
{\small        Step     9     Radius          Curvature
}
{\small        -----------------------------------------------------
}
{\small        Vertex   1     1.1681789       3.9076805
}
{\small        Vertex   2     0.9668779       3.5170408
}
{\small        Vertex   3     0.7605802       2.9772908
}
{\small        Vertex   4     0.5451929       2.1643586
}
{\small        Total Curvature: 12.5663706
}
 
{\small        Step    10     Radius          Curvature
}
{\small        -----------------------------------------------------
}
{\small        Vertex   1     1.1592296       3.8916213
}
{\small          :
}
{\small          :
}
{\small   
}
\end{verbatim}

\subsection*{Limitations}

\begin{quotation}
Currently the \texttt{printResultsStep} function is limited in the
information it prints. As our curvature flow has evolved to record
additional information such as volumes, it may be time to explore a more
robust form for displaying results. As there is considerable dependence on
the Approximator for the data vectors, it may be wise to place this and
similar functions in the Approximator class.
\end{quotation}

\subsection*{Revisions}

\begin{itemize}
\item subversion 545, 9/29/08: Moved the printing of results out of calcFlow
and into a new function.

\item subversion 783, 6/18/09: Small modifications in response to changes in
the Approximator class.
\end{itemize}

\subsection*{Testing}

\begin{quotation}
The \texttt{printResultsStep} function was tested by running multiple
curvature flows and printing the results. It was considered working when the
format of the data was as desired.
\end{quotation}

\subsection*{Future Work}

\begin{itemize}
\item 6/29 - Recreate the print functions to print more data and be more
flexible.

\item 6/29 - Move the print functions into the Approximator class.
\end{itemize}

% html: End of file: `clean.html'

%%%%% END OF DOCUMENT BODY %%%%%
% In the future, we might want to put some additional data here, such
% as when the documentation was converted from wiki to TeX.
%
}}%
%BeginExpansion
%TCIDATA{Version=5.00.0.2606}
%TCIDATA{LaTeXparent=0,0,functions.tex}
                      

%%%%% BEGINNING OF DOCUMENT BODY %%%%%
% html: Beginning of file: `clean.html'
% DOCTYPE HTML PUBLIC "-//W3C//DTD HTML 4.01//EN"
%  This is a (PRE) block.  Make sure it's left aligned or your toc title will be off. 

\section*{\texttt{printResultsStep}}

\label{f0}

\begin{quotation}
{\small }
\end{quotation}

\begin{verbatim}
{\small 
   void printResultsStep(char* fileName, vector<double>* radii, vector<double>* curvs)
}
{\small    
}
\end{verbatim}

\subsection*{Key Words}

\begin{quotation}
radii, curvatures, file, flow, step, print
\end{quotation}

\subsection*{Authors}

\begin{quotation}
Alex Henniges
\end{quotation}

\subsection*{Introduction}

\begin{quotation}
The \texttt{printResultsStep} function prints out the results of a curvature
flow, with the results grouped by each step of the flow. These results will
be written to the file given by \texttt{filename}.
\end{quotation}

\subsection*{Subsidiaries}

\begin{quotation}
Functions:

Global Variables:

Local Variables: \texttt{int vertSize}, \texttt{int numSteps}
\end{quotation}

\subsection*{Description}

\begin{quotation}
Prints the results of a curvature flow into the file given by \texttt{%
filename}. The results, that is, the radii and curvature values, are given
by vectors of doubles. Most commonly, these vectors are taken from the
Approximator class after the flow is run. The \texttt{printResultsStep}
function determines the number of vertices of the current triangulation and
the total number of steps are then derived from this and the size of the
vectors.

There are several ways to display the results. The \texttt{printResultsStep}
function groups by step. This means that for each step of the curvature
flow, the radii and curvature values for each vertex is printed. An example
is shown below. Other formats are given by printResultsVertex,
printResultsNum, print3DResultsStep.
\end{quotation}

\subsection*{Practicum}

\begin{quotation}
Example:{\small }
\end{quotation}

\begin{verbatim}
{\small 
  // Print the results of a curvature flow with Approximator app into file "ODEResult.txt"
}
{\small 
  printResultsStep("./ODEResults.txt", app->radiiHistory, app->curvHistory);
}
{\small   
}
\end{verbatim}

\begin{quotation}
The output of such an example may then be{\small }
\end{quotation}

\begin{verbatim}
{\small          :    
}
{\small          :
}
{\small        Vertex   4     0.5397923       2.1411007
}
{\small        Total Curvature: 12.5663706
}
 
{\small        Step     9     Radius          Curvature
}
{\small        -----------------------------------------------------
}
{\small        Vertex   1     1.1681789       3.9076805
}
{\small        Vertex   2     0.9668779       3.5170408
}
{\small        Vertex   3     0.7605802       2.9772908
}
{\small        Vertex   4     0.5451929       2.1643586
}
{\small        Total Curvature: 12.5663706
}
 
{\small        Step    10     Radius          Curvature
}
{\small        -----------------------------------------------------
}
{\small        Vertex   1     1.1592296       3.8916213
}
{\small          :
}
{\small          :
}
{\small   
}
\end{verbatim}

\subsection*{Limitations}

\begin{quotation}
Currently the \texttt{printResultsStep} function is limited in the
information it prints. As our curvature flow has evolved to record
additional information such as volumes, it may be time to explore a more
robust form for displaying results. As there is considerable dependence on
the Approximator for the data vectors, it may be wise to place this and
similar functions in the Approximator class.
\end{quotation}

\subsection*{Revisions}

\begin{itemize}
\item subversion 545, 9/29/08: Moved the printing of results out of calcFlow
and into a new function.

\item subversion 783, 6/18/09: Small modifications in response to changes in
the Approximator class.
\end{itemize}

\subsection*{Testing}

\begin{quotation}
The \texttt{printResultsStep} function was tested by running multiple
curvature flows and printing the results. It was considered working when the
format of the data was as desired.
\end{quotation}

\subsection*{Future Work}

\begin{itemize}
\item 6/29 - Recreate the print functions to print more data and be more
flexible.

\item 6/29 - Move the print functions into the Approximator class.
\end{itemize}

% html: End of file: `clean.html'

%%%%% END OF DOCUMENT BODY %%%%%
% In the future, we might want to put some additional data here, such
% as when the documentation was converted from wiki to TeX.
%
%
%EndExpansion

\bigskip 

\bigskip 

%TCIMACRO{%
%\QSubDoc{Include printResultsVertex}{%TCIDATA{Version=5.00.0.2606}
%TCIDATA{LaTeXparent=0,0,functions.tex}
                      

%%%%% BEGINNING OF DOCUMENT BODY %%%%%
% html: Beginning of file: `clean.html'
% DOCTYPE HTML PUBLIC "-//W3C//DTD HTML 4.01//EN"
%  This is a (PRE) block.  Make sure it's left aligned or your toc title will be off. 

\section*{\texttt{printResultsVertex}}

\label{f0}

\begin{quotation}
{\small }
\end{quotation}

\begin{verbatim}
{\small 
   void printResultsVertex(char* fileName, vector<double>* radii, vector<double>* curvs)
}
{\small    
}
\end{verbatim}

\subsection*{Key Words}

\begin{quotation}
radii, curvatures, file, flow, vertex, print
\end{quotation}

\subsection*{Authors}

\begin{quotation}
Alex Henniges
\end{quotation}

\subsection*{Introduction}

\begin{quotation}
The \texttt{printResultsVertex} function prints out the results of a
curvature flow, with the results grouped by each vertex of the
triangulation. These results will be written to the file given by \texttt{%
filename}.
\end{quotation}

\subsection*{Subsidiaries}

\begin{quotation}
Functions:

Global Variables:

Local Variables: \texttt{int vertSize}, \texttt{int numSteps}
\end{quotation}

\subsection*{Description}

\begin{quotation}
Prints the results of a curvature flow into the file given by \texttt{%
filename}. The results, that is, the radii and curvature values, are given
by vectors of doubles. Most commonly, these vectors are taken from the
Approximator class after the flow is run. The \texttt{printResultsVertex}
function determines the number of vertices of the current triangulation and
the total number of steps are then derived from this and the size of the
vectors.

There are several ways to display the results. The \texttt{printResultsVertex%
} function groups by vertex. This means that for each vertex of the
triangulation, the radii and curvature values for each step is given. An
example is shown below. Other formats are given by printResultsStep,
printResultsNum, print3DResultsStep.
\end{quotation}

\subsection*{Practicum}

\begin{quotation}
Example:{\small }
\end{quotation}

\begin{verbatim}
{\small 
  // Print the results of a curvature flow with Approximator app into file "ODEResult.txt"
}
{\small 
  printResultsVertex("./ODEResults.txt", app->radiiHistory, app->curvHistory);
}
{\small   
}
\end{verbatim}

\begin{quotation}
The output of such an example may then be{\small }
\end{quotation}

\begin{verbatim}
{\small          : 
}
{\small          :
}
{\small        Step  298        0.8272161        3.1425516
}
{\small        Step  299        0.8272081        3.1425294
}
{\small        Step  300        0.8272004        3.1425078
}
 
{\small        Vertex:   2        Radius                Curv
}
 
{\small        ---------------------------------
}
{\small        Step    0        1.0000000        3.1415927
}
{\small        Step    1        1.0000000        3.5987926
}
{\small        Step    2        0.9954280        3.5877017
}
{\small        Step    3        0.9909873        3.5768692
}
{\small        Step    4        0.9866738        3.5662905
}
{\small          :
}
{\small          :
}
{\small   
}
\end{verbatim}

\subsection*{Limitations}

\begin{quotation}
Currently the \texttt{printResultsVertex} function is limited in the
information it prints. As our curvature flow has evolved to record
additional information such as volumes, it may be time to explore a more
robust form for displaying results. As there is considerable dependence on
the Approximator for the data vectors, it may be wise to place this and
similar functions in the Approximator class.
\end{quotation}

\subsection*{Revisions}

\begin{itemize}
\item subversion 545, 9/29/08: Moved the printing of results out of calcFlow
and into a new function.

\item subversion 783, 6/18/09: Small modifications in response to changes in
the Approximator class.
\end{itemize}

\subsection*{Testing}

\begin{quotation}
The \texttt{printResultsVertex} function was tested by running multiple
curvature flows and printing the results. It was considered working when the
format of the data was as desired.
\end{quotation}

\subsection*{Future Work}

\begin{itemize}
\item 6/29 - Recreate the print functions to print more data and be more
flexible.

\item 6/29 - Move the print functions into the Approximator class.
\end{itemize}

% html: End of file: `clean.html'

%%%%% END OF DOCUMENT BODY %%%%%
% In the future, we might want to put some additional data here, such
% as when the documentation was converted from wiki to TeX.
%
}}%
%BeginExpansion
%TCIDATA{Version=5.00.0.2606}
%TCIDATA{LaTeXparent=0,0,functions.tex}
                      

%%%%% BEGINNING OF DOCUMENT BODY %%%%%
% html: Beginning of file: `clean.html'
% DOCTYPE HTML PUBLIC "-//W3C//DTD HTML 4.01//EN"
%  This is a (PRE) block.  Make sure it's left aligned or your toc title will be off. 

\section*{\texttt{printResultsVertex}}

\label{f0}

\begin{quotation}
{\small }
\end{quotation}

\begin{verbatim}
{\small 
   void printResultsVertex(char* fileName, vector<double>* radii, vector<double>* curvs)
}
{\small    
}
\end{verbatim}

\subsection*{Key Words}

\begin{quotation}
radii, curvatures, file, flow, vertex, print
\end{quotation}

\subsection*{Authors}

\begin{quotation}
Alex Henniges
\end{quotation}

\subsection*{Introduction}

\begin{quotation}
The \texttt{printResultsVertex} function prints out the results of a
curvature flow, with the results grouped by each vertex of the
triangulation. These results will be written to the file given by \texttt{%
filename}.
\end{quotation}

\subsection*{Subsidiaries}

\begin{quotation}
Functions:

Global Variables:

Local Variables: \texttt{int vertSize}, \texttt{int numSteps}
\end{quotation}

\subsection*{Description}

\begin{quotation}
Prints the results of a curvature flow into the file given by \texttt{%
filename}. The results, that is, the radii and curvature values, are given
by vectors of doubles. Most commonly, these vectors are taken from the
Approximator class after the flow is run. The \texttt{printResultsVertex}
function determines the number of vertices of the current triangulation and
the total number of steps are then derived from this and the size of the
vectors.

There are several ways to display the results. The \texttt{printResultsVertex%
} function groups by vertex. This means that for each vertex of the
triangulation, the radii and curvature values for each step is given. An
example is shown below. Other formats are given by printResultsStep,
printResultsNum, print3DResultsStep.
\end{quotation}

\subsection*{Practicum}

\begin{quotation}
Example:{\small }
\end{quotation}

\begin{verbatim}
{\small 
  // Print the results of a curvature flow with Approximator app into file "ODEResult.txt"
}
{\small 
  printResultsVertex("./ODEResults.txt", app->radiiHistory, app->curvHistory);
}
{\small   
}
\end{verbatim}

\begin{quotation}
The output of such an example may then be{\small }
\end{quotation}

\begin{verbatim}
{\small          : 
}
{\small          :
}
{\small        Step  298        0.8272161        3.1425516
}
{\small        Step  299        0.8272081        3.1425294
}
{\small        Step  300        0.8272004        3.1425078
}
 
{\small        Vertex:   2        Radius                Curv
}
 
{\small        ---------------------------------
}
{\small        Step    0        1.0000000        3.1415927
}
{\small        Step    1        1.0000000        3.5987926
}
{\small        Step    2        0.9954280        3.5877017
}
{\small        Step    3        0.9909873        3.5768692
}
{\small        Step    4        0.9866738        3.5662905
}
{\small          :
}
{\small          :
}
{\small   
}
\end{verbatim}

\subsection*{Limitations}

\begin{quotation}
Currently the \texttt{printResultsVertex} function is limited in the
information it prints. As our curvature flow has evolved to record
additional information such as volumes, it may be time to explore a more
robust form for displaying results. As there is considerable dependence on
the Approximator for the data vectors, it may be wise to place this and
similar functions in the Approximator class.
\end{quotation}

\subsection*{Revisions}

\begin{itemize}
\item subversion 545, 9/29/08: Moved the printing of results out of calcFlow
and into a new function.

\item subversion 783, 6/18/09: Small modifications in response to changes in
the Approximator class.
\end{itemize}

\subsection*{Testing}

\begin{quotation}
The \texttt{printResultsVertex} function was tested by running multiple
curvature flows and printing the results. It was considered working when the
format of the data was as desired.
\end{quotation}

\subsection*{Future Work}

\begin{itemize}
\item 6/29 - Recreate the print functions to print more data and be more
flexible.

\item 6/29 - Move the print functions into the Approximator class.
\end{itemize}

% html: End of file: `clean.html'

%%%%% END OF DOCUMENT BODY %%%%%
% In the future, we might want to put some additional data here, such
% as when the documentation was converted from wiki to TeX.
%
%
%EndExpansion

\bigskip 

\bigskip 

%TCIMACRO{%
%\QSubDoc{Include Total_Volume}{%TCIDATA{LaTeXparent=0,0,functions.tex}}}%
%BeginExpansion
%TCIDATA{LaTeXparent=0,0,functions.tex}%
%EndExpansion

\bigskip

\bigskip

%TCIMACRO{%
%\QSubDoc{Include Volume_Partial}{%TCIDATA{Version=5.00.0.2606}
%TCIDATA{LaTeXparent=1,1,functions.tex}
                      

\section*{\texttt{VolumePartial::VolumePartial}}

\subsection*{Function Prototype}

\texttt{double VolumePartial(Vertex v\_i, Tetra t)}

\subsection*{Key Words}

Volume, tetrahedron, vertex, radius, Cayley-Menger determinant, standard
form, geoquant.

\subsection*{Authors}

Daniel Champion

\subsection*{Introduction}

\texttt{VolumePartial} calculates the partial derivative of the volume of a
tetrahedron with respect to the logarithm of the radius of a vertex.

\subsection*{Subsidiaries}

\textbf{Functions:} \ 

\texttt{listDifference}

\texttt{listIntersection}

\texttt{Simplex::isAdjVertex}

\textbf{Global Variables:} \ radii, etas.

\textbf{Local Variables:}

\subsection*{Description}

The volume of a tetrahedron only depends on the lengths of its edges as
calculated from the Cayley-Menger determinant. \ Thus for a given
tetrahedron $t$, it's partial derivatives with respect to $\log $ radii will
vanish except for those radii corresponding to the vertices of $t$. \ The
function \texttt{isAdjVertex} of the simplex class determines this
condition. \ \texttt{VolumePartial} then proceeds with an initialization
procedure that labels the vertices and edges (radii and etas) in standard
form. \ Specifically, \texttt{Volume\_Partial} receives as inputs an integer 
\texttt{i} corresponding to a vertex (in the vertex table) which is labeled
vertex v1. \ The remaining vertices are labeled $v2,v3,v4$, and the edges $%
e12,e13,e14,e23,e24,e34$ are labeled preserving the structure implied by the
assignment of the vertices. \ The radii $r_{1}$, $r_{2}$,... and eta values $%
Eta_{12},Eta_{13},$... are assigned to the corresponding vertices and edges.
\ 

The formula for the partial derivative in terms of these standard form
variables was calculated in Mathematica using the Cayley-Menger determinant,
that is:%
\begin{equation*}
288V^{2}=\det\left[ 
\begin{array}{ccccc}
0 & 1 & 1 & 1 & 1 \\ 
1 & 0 & L_{12}^{2} & L_{13}^{2} & L_{14}^{2} \\ 
1 & L_{12}^{2} & 0 & L_{23}^{2} & L_{24}^{2} \\ 
1 & L_{13}^{2} & L_{23}^{2} & 0 & L_{34}^{2} \\ 
1 & L_{14}^{2} & L_{24}^{2} & L_{34}^{2} & 0%
\end{array}
\right] ,
\end{equation*}
where the lengths were determined from the radii and eta values using the
formula%
\begin{equation*}
L_{ij}^{2}=r_{i}^{2}+r_{j}^{2}+2r_{i}r_{j}Eta_{ij}.
\end{equation*}

The formula obtained from Mathematica was outputted into the C programming
language using the function CForm. \ 

This function was designed for use in the optimization of the
Einstein-Hilbert-Regge functional using Newton's method. \ In this procedure
the gradient of the EHR functional is needed which contains the partial
derivatives of the volume. \ 

\subsection*{Practicum}

Usage:

\texttt{VolumePartial(Vertex v\_i, Tetra t)}

The integer \texttt{i} corresponds to a vertex in the vertex table, that is%
\begin{equation*}
\text{\texttt{VolumePartial (v\_i, t)} }=\frac{\partial }{\partial \log r_{i}%
}Volume(t).
\end{equation*}

\subsection*{Limitations}

\texttt{VolumePartial} was designed to output the correct partial derivative
for any integer $i$ in the vertex table and any tetrahedron $t$ in the
triangulation. \ 

\subsection*{Revisions}

subversion 757, 7/7/09, \texttt{VolumePartial} created within
NewtonsMethod.cpp.

subversion 1055, 3/12/10, \texttt{VolumePartial} converted to a geoquant.

\subsection*{Testing}

The partial derivative of volume of several known tetrahedra were calculated
using \texttt{Volume\_Partial} and verified using Mathematica.

\subsection*{Future Work}

The procedure that initializes the tetrahedron into standard form should be
removed from this program and placed elsewhere. \ There are several
occurrences of this type of procedure that should be consolidated. \ 
}}%
%BeginExpansion
%TCIDATA{Version=5.00.0.2606}
%TCIDATA{LaTeXparent=1,1,functions.tex}
                      

\section*{\texttt{VolumePartial::VolumePartial}}

\subsection*{Function Prototype}

\texttt{double VolumePartial(Vertex v\_i, Tetra t)}

\subsection*{Key Words}

Volume, tetrahedron, vertex, radius, Cayley-Menger determinant, standard
form, geoquant.

\subsection*{Authors}

Daniel Champion

\subsection*{Introduction}

\texttt{VolumePartial} calculates the partial derivative of the volume of a
tetrahedron with respect to the logarithm of the radius of a vertex.

\subsection*{Subsidiaries}

\textbf{Functions:} \ 

\texttt{listDifference}

\texttt{listIntersection}

\texttt{Simplex::isAdjVertex}

\textbf{Global Variables:} \ radii, etas.

\textbf{Local Variables:}

\subsection*{Description}

The volume of a tetrahedron only depends on the lengths of its edges as
calculated from the Cayley-Menger determinant. \ Thus for a given
tetrahedron $t$, it's partial derivatives with respect to $\log $ radii will
vanish except for those radii corresponding to the vertices of $t$. \ The
function \texttt{isAdjVertex} of the simplex class determines this
condition. \ \texttt{VolumePartial} then proceeds with an initialization
procedure that labels the vertices and edges (radii and etas) in standard
form. \ Specifically, \texttt{Volume\_Partial} receives as inputs an integer 
\texttt{i} corresponding to a vertex (in the vertex table) which is labeled
vertex v1. \ The remaining vertices are labeled $v2,v3,v4$, and the edges $%
e12,e13,e14,e23,e24,e34$ are labeled preserving the structure implied by the
assignment of the vertices. \ The radii $r_{1}$, $r_{2}$,... and eta values $%
Eta_{12},Eta_{13},$... are assigned to the corresponding vertices and edges.
\ 

The formula for the partial derivative in terms of these standard form
variables was calculated in Mathematica using the Cayley-Menger determinant,
that is:%
\begin{equation*}
288V^{2}=\det\left[ 
\begin{array}{ccccc}
0 & 1 & 1 & 1 & 1 \\ 
1 & 0 & L_{12}^{2} & L_{13}^{2} & L_{14}^{2} \\ 
1 & L_{12}^{2} & 0 & L_{23}^{2} & L_{24}^{2} \\ 
1 & L_{13}^{2} & L_{23}^{2} & 0 & L_{34}^{2} \\ 
1 & L_{14}^{2} & L_{24}^{2} & L_{34}^{2} & 0%
\end{array}
\right] ,
\end{equation*}
where the lengths were determined from the radii and eta values using the
formula%
\begin{equation*}
L_{ij}^{2}=r_{i}^{2}+r_{j}^{2}+2r_{i}r_{j}Eta_{ij}.
\end{equation*}

The formula obtained from Mathematica was outputted into the C programming
language using the function CForm. \ 

This function was designed for use in the optimization of the
Einstein-Hilbert-Regge functional using Newton's method. \ In this procedure
the gradient of the EHR functional is needed which contains the partial
derivatives of the volume. \ 

\subsection*{Practicum}

Usage:

\texttt{VolumePartial(Vertex v\_i, Tetra t)}

The integer \texttt{i} corresponds to a vertex in the vertex table, that is%
\begin{equation*}
\text{\texttt{VolumePartial (v\_i, t)} }=\frac{\partial }{\partial \log r_{i}%
}Volume(t).
\end{equation*}

\subsection*{Limitations}

\texttt{VolumePartial} was designed to output the correct partial derivative
for any integer $i$ in the vertex table and any tetrahedron $t$ in the
triangulation. \ 

\subsection*{Revisions}

subversion 757, 7/7/09, \texttt{VolumePartial} created within
NewtonsMethod.cpp.

subversion 1055, 3/12/10, \texttt{VolumePartial} converted to a geoquant.

\subsection*{Testing}

The partial derivative of volume of several known tetrahedra were calculated
using \texttt{Volume\_Partial} and verified using Mathematica.

\subsection*{Future Work}

The procedure that initializes the tetrahedron into standard form should be
removed from this program and placed elsewhere. \ There are several
occurrences of this type of procedure that should be consolidated. \ 
%
%EndExpansion

\bigskip

\bigskip

%TCIMACRO{%
%\QSubDoc{Include Volume_Second_Partial}{%TCIDATA{Version=5.00.0.2606}
%TCIDATA{LaTeXparent=1,1,functions.tex}
                      

\section*{\texttt{VolumeSecondPartial::VolumeSecondPartial}}

\subsection*{Function Prototype}

\texttt{double VolumeSecondPartial(Vertex v\_i, Vertex v\_j, Tetra t)}

\subsection*{Key Words}

Volume, Hessian Matrix, Newton's Method, partial derivative,
Einstein-Hilbert-Regge functional, geoquant.

\subsection*{Authors}

Daniel Champion

\subsection*{Introduction}

\texttt{VolumeSecondPartial} calculates the second order partial derivatives
of the volume of a tetrahedron with respect to log radii for all pairs of
indices (not necessarily distinct) in the vertex table. \ 

\subsection*{Subsidiaries}

\textbf{Functions:}

\texttt{listDifference}

\texttt{listIntersection}

\texttt{Simplex::isAdjVertex}

\textbf{Global Variables:} \ radii, etas.

\textbf{Local Variables:}

\subsection*{Description}

The volume of a tetrahedron only depends on the lengths of its edges as
calculated from the Cayley-Menger determinant. \ Thus for a given
tetrahedron $t$, it's second order partial derivatives with respect to $\log 
$ radii will vanish except for pairs of radii (not necessarily distinct)
corresponding to the vertices of $t$. \ The first step in the implementation
of \texttt{VolumeSecondPartial} is the determination of the following
trichotomy for a pair $\left\{ i,j\right\} $ of indices in the vertex table:%
\begin{equation*}
\begin{array}{l}
\text{A. \ }i=j\text{ and }i\text{ is a vertex of tetrahedron }t \\ 
\text{B. \ }i\neq j\text{ and both }i\text{ and }j\text{ belong to }t \\ 
\text{C. \ at least one of }i\text{ or }j\text{ doesn't belong to }t.%
\end{array}%
\end{equation*}

Each condition of the trichotomy requires a distinct calculation to
determine the desired partial derivative. \ Nevertheless, the next step in
the implementation is to place the tetrahedron in "standard form" relative
to the indices $i$ and $j$ (for conditions A and B only). \ More
specifically, for condition A the radius for vertex $i$ is stored as $r_{1}$%
, and the remaining radii of the tetrahedron $t$ are assigned $r_{2},r_{3},$
and $r_{4}$ in no particular order. \ The eta values $Eta_{12},Eta_{13},...$
are then assigned preserving the preceding assignments. \ In the case of
condition B, the radii at vertices $i$ and $j$ are assigned to $r_{1}$ and $%
r_{2}$ respectively, and $r_{3}$, and $r_{4}$ the remaining radii of $t$. \
The eta values $Eta_{12},Eta_{13},...$ are again assigned preserving the
preceding assignments.

The formulas for the second order partial derivatives in terms of these
standard form variables was calculated in Mathematica using the
Cayley-Menger determinant, that is:%
\begin{equation*}
288V^{2}=\det\left[ 
\begin{array}{ccccc}
0 & 1 & 1 & 1 & 1 \\ 
1 & 0 & L_{12}^{2} & L_{13}^{2} & L_{14}^{2} \\ 
1 & L_{12}^{2} & 0 & L_{23}^{2} & L_{24}^{2} \\ 
1 & L_{13}^{2} & L_{23}^{2} & 0 & L_{34}^{2} \\ 
1 & L_{14}^{2} & L_{24}^{2} & L_{34}^{2} & 0%
\end{array}
\right] ,
\end{equation*}
where the lengths were determined from the radii and eta values using the
formula%
\begin{equation*}
L_{ij}^{2}=r_{i}^{2}+r_{j}^{2}+2r_{i}r_{j}Eta_{ij}.
\end{equation*}

The formula obtained from Mathematica was outputted into the C programming
language using the function CForm.

This function was designed for use in the optimization of the
Einstein-Hilbert-Regge functional using Newton's method. \ In this procedure
the Hessian matrix of the normalized EHR functional is needed, each entry of
which uses the second order partial derivatives of volume. \ See the entry
on \texttt{EHRSecondPartial}.

\subsection*{Practicum}

Usage:

\texttt{VolumeSecondPartial (Vertex v\_i, Vertex v\_j, Tetra t)}

The integers \texttt{i} and \texttt{j} correspond to vertices in the vertex
table and \texttt{t} is a tetrahedron in the triangulation. \ Specifically
the function returns:%
\begin{equation*}
\text{\texttt{VolumeSecondPartial(v\_i, v\_j, t)} }=\frac{\partial ^{2}}{%
\partial \log r_{i}\partial \log r_{j}}Volume(t).
\end{equation*}

\subsection*{Limitations}

\texttt{VolumeSecondPartial} is fully operational with no know limitations.
\ The function will output appropriate values when given indices $i,$ and $j$
in the vertex table, and a tetrahedron $t$. \ 

\subsection*{Revisions}

subversion 757, 7/9/09, \texttt{VolumeSecondPartial} created.

subversion 1055, 3/12/10, \texttt{VolumeSecondPartial} converted to a
geoquant.

\subsection*{Testing}

Several trials were run outputting the values of \texttt{VolumeSecondPartial}
for a variety of vertices and tetrahedra. \ These values were compared with
calculations performed on Mathematica. \ 

\subsection*{Future Work}

None planned.
}}%
%BeginExpansion
%TCIDATA{Version=5.00.0.2606}
%TCIDATA{LaTeXparent=1,1,functions.tex}
                      

\section*{\texttt{VolumeSecondPartial::VolumeSecondPartial}}

\subsection*{Function Prototype}

\texttt{double VolumeSecondPartial(Vertex v\_i, Vertex v\_j, Tetra t)}

\subsection*{Key Words}

Volume, Hessian Matrix, Newton's Method, partial derivative,
Einstein-Hilbert-Regge functional, geoquant.

\subsection*{Authors}

Daniel Champion

\subsection*{Introduction}

\texttt{VolumeSecondPartial} calculates the second order partial derivatives
of the volume of a tetrahedron with respect to log radii for all pairs of
indices (not necessarily distinct) in the vertex table. \ 

\subsection*{Subsidiaries}

\textbf{Functions:}

\texttt{listDifference}

\texttt{listIntersection}

\texttt{Simplex::isAdjVertex}

\textbf{Global Variables:} \ radii, etas.

\textbf{Local Variables:}

\subsection*{Description}

The volume of a tetrahedron only depends on the lengths of its edges as
calculated from the Cayley-Menger determinant. \ Thus for a given
tetrahedron $t$, it's second order partial derivatives with respect to $\log 
$ radii will vanish except for pairs of radii (not necessarily distinct)
corresponding to the vertices of $t$. \ The first step in the implementation
of \texttt{VolumeSecondPartial} is the determination of the following
trichotomy for a pair $\left\{ i,j\right\} $ of indices in the vertex table:%
\begin{equation*}
\begin{array}{l}
\text{A. \ }i=j\text{ and }i\text{ is a vertex of tetrahedron }t \\ 
\text{B. \ }i\neq j\text{ and both }i\text{ and }j\text{ belong to }t \\ 
\text{C. \ at least one of }i\text{ or }j\text{ doesn't belong to }t.%
\end{array}%
\end{equation*}

Each condition of the trichotomy requires a distinct calculation to
determine the desired partial derivative. \ Nevertheless, the next step in
the implementation is to place the tetrahedron in "standard form" relative
to the indices $i$ and $j$ (for conditions A and B only). \ More
specifically, for condition A the radius for vertex $i$ is stored as $r_{1}$%
, and the remaining radii of the tetrahedron $t$ are assigned $r_{2},r_{3},$
and $r_{4}$ in no particular order. \ The eta values $Eta_{12},Eta_{13},...$
are then assigned preserving the preceding assignments. \ In the case of
condition B, the radii at vertices $i$ and $j$ are assigned to $r_{1}$ and $%
r_{2}$ respectively, and $r_{3}$, and $r_{4}$ the remaining radii of $t$. \
The eta values $Eta_{12},Eta_{13},...$ are again assigned preserving the
preceding assignments.

The formulas for the second order partial derivatives in terms of these
standard form variables was calculated in Mathematica using the
Cayley-Menger determinant, that is:%
\begin{equation*}
288V^{2}=\det\left[ 
\begin{array}{ccccc}
0 & 1 & 1 & 1 & 1 \\ 
1 & 0 & L_{12}^{2} & L_{13}^{2} & L_{14}^{2} \\ 
1 & L_{12}^{2} & 0 & L_{23}^{2} & L_{24}^{2} \\ 
1 & L_{13}^{2} & L_{23}^{2} & 0 & L_{34}^{2} \\ 
1 & L_{14}^{2} & L_{24}^{2} & L_{34}^{2} & 0%
\end{array}
\right] ,
\end{equation*}
where the lengths were determined from the radii and eta values using the
formula%
\begin{equation*}
L_{ij}^{2}=r_{i}^{2}+r_{j}^{2}+2r_{i}r_{j}Eta_{ij}.
\end{equation*}

The formula obtained from Mathematica was outputted into the C programming
language using the function CForm.

This function was designed for use in the optimization of the
Einstein-Hilbert-Regge functional using Newton's method. \ In this procedure
the Hessian matrix of the normalized EHR functional is needed, each entry of
which uses the second order partial derivatives of volume. \ See the entry
on \texttt{EHRSecondPartial}.

\subsection*{Practicum}

Usage:

\texttt{VolumeSecondPartial (Vertex v\_i, Vertex v\_j, Tetra t)}

The integers \texttt{i} and \texttt{j} correspond to vertices in the vertex
table and \texttt{t} is a tetrahedron in the triangulation. \ Specifically
the function returns:%
\begin{equation*}
\text{\texttt{VolumeSecondPartial(v\_i, v\_j, t)} }=\frac{\partial ^{2}}{%
\partial \log r_{i}\partial \log r_{j}}Volume(t).
\end{equation*}

\subsection*{Limitations}

\texttt{VolumeSecondPartial} is fully operational with no know limitations.
\ The function will output appropriate values when given indices $i,$ and $j$
in the vertex table, and a tetrahedron $t$. \ 

\subsection*{Revisions}

subversion 757, 7/9/09, \texttt{VolumeSecondPartial} created.

subversion 1055, 3/12/10, \texttt{VolumeSecondPartial} converted to a
geoquant.

\subsection*{Testing}

Several trials were run outputting the values of \texttt{VolumeSecondPartial}
for a variety of vertices and tetrahedra. \ These values were compared with
calculations performed on Mathematica. \ 

\subsection*{Future Work}

None planned.
%
%EndExpansion
}}%
%BeginExpansion
%TCIDATA{Version=5.00.0.2606}
%TCIDATA{LaTeXparent=0,0,geocam.tex}
                      

\chapter{Functions}

%TCIMACRO{\QSubDoc{Include Aij_kl}{%TCIDATA{Version=5.00.0.2606}
%TCIDATA{LaTeXparent=1,1,functions.tex}
                      

\section*{\texttt{DualAreaSegment::DualAreaSegment}}

\subsection*{Function Prototype}

\texttt{double DualAreaSegment( Vertex vi, Vertex vj, Vertex vk, Vertex vl)}

\subsection*{Key Words}

Dual area, curvature, partial derivative, Einstein-Hilbert-Regge, geoquant.

\subsection*{Authors}

Daniel Champion

\subsection*{Introduction}

\texttt{DualAreaSegment} calculates the dual area to an edge of a
tetrahedron. \ 

\subsection*{Subsidiaries}

\textbf{Functions:}

\qquad \texttt{EdgeHeight}

\qquad \qquad \texttt{PartialEdge}

\qquad\qquad\texttt{Geometry::angle}

\qquad \texttt{FaceHeight}

\qquad\qquad\texttt{Geometry::dihedralAngle}

\textbf{Global Variables: \ }radii, etas

\textbf{Local Variables:} \ none.

\subsection*{Description}

\texttt{DualAreaSegment} is calculated with the formula:%
\begin{equation*}
\text{\texttt{DualAreaSegment(vi, vj, vk, vl)}}=
\end{equation*}%
\begin{equation*}
\frac{1}{2}\left( 
\begin{array}{c}
\text{\texttt{EdgeHeight(vi,vj,vk)}}\cdot \text{\texttt{%
FaceHeight(vi,vj,vk,vl)}} \\ 
+\text{\texttt{EdgeHeight(vi,vj,vl)}}\cdot \text{\texttt{%
FaceHeight(vi,vj,vl,vk)}}%
\end{array}%
\right) 
\end{equation*}%
\texttt{EdgeHeight} and \texttt{FaceHeight} are calculated with the
following formulae:%
\begin{align*}
\text{\texttt{EdgeHeight(vi, vj, vk)}}& =\frac{\left( \text{\texttt{%
PartialEdge(vi,vk)}}-\text{\texttt{PartialEdge(vi,vj)}}\cos \left( \alpha
_{i,jk}\right) \right) }{\sin \left( \alpha _{i,jk}\right) } \\
\text{\texttt{FaceHeight(vi, vj, vk, vl)}}& =\frac{\left( \text{\texttt{%
EdgeHeight(vi,vj,vl)}}-\text{\texttt{EdgeHeight(vi,vj,vk)}}\cos (\beta
_{ij,kl})\right) }{\sin \left( \beta _{ij,kl}\right) }
\end{align*}%
where $\alpha _{i,jk}$ is the angle at vertex $vi$ of triangle $\left\{
vi,vj,vk\right\} $, and $\beta _{ij,kl}$ is the dihedral angle along edge $%
\left\{ vi,vj\right\} $ of tetrahedron $\left\{ vi,vj,vk,vl\right\} $
(implemented with the functions \texttt{Geometry::angle} and \texttt{%
Geometry::dihedralAngle} respectively).

\texttt{DualAreaSegment} was created for the calculation performed in the
function \texttt{DualArea}, which is used in the computation of the partial
derivatives of curvature. \ These partial derivatives of curvature are used
in the calculation of the second order partial derivatives of the
Einstein-Hilbert-Regge functional for use in the optimization of said
functional using Newton's method. \ 

\subsection*{Practicum}

As an example of the usage of this function, we will calculate the dual area
to the edge $eij=\left\{ vi,vj\right\} $ (see entry: \texttt{DualArea}). \
To do this, we will sum the dual areas to each tetrahedron containing the
edge $eij$. \ 

\bigskip

\texttt{vector\TEXTsymbol{<}int\TEXTsymbol{>} sum\_over =
*(eij.getLocalTetras());}

\texttt{double sum = 0.0;}

\texttt{vector\TEXTsymbol{<}int\TEXTsymbol{>} T\_vertices, e\_vertices;}

\texttt{Tetra T;}

\texttt{Vertex vi,vj,vk,vl;}

\texttt{for(i=0; i\TEXTsymbol{<}sum\_over.size(); ++i) \{}

\qquad\texttt{T = Triangulation::tetraTable[sum\_over[i]];}

\qquad\texttt{T\_vertices = *(T.getLocalVertices());}

\qquad\texttt{e\_vertices = *(eij.getLocalVertices());}

\qquad\texttt{vi = Triangulation::vertexTable[e\_vertices[0]];}

\qquad\texttt{vj = Triangulation::vertexTable[e\_vertices[1]];}

\qquad\texttt{vk = Triangulation::vertexTable[listDifference(\&T\_vertices,
\&e\_vertices)[0]];}

\qquad\texttt{vl = Triangulation::vertexTable[listDifference(\&T\_vertices,
\&e\_vertices)[1]];}

\qquad \texttt{sum += DualAreaSegment(vi, vj, vk, vl);}

\qquad\texttt{\}}

\texttt{return sum;}

\subsection*{Limitations}

\texttt{DualAreaSegment} if fully operational and has no known limitations.
\ The function will output appropriate values provided it receives as input
four distinct vertices that define a tetrahedron.

\subsection*{Revisions}

subversion 757, 7/8/09, \texttt{DualAreaSegment} created.

subversion 1055, 3/12/10, \texttt{DualAreaSegment}\ converted to a geoquant.

\subsection*{Testing}

Trials were run and the calculations returned were verified by hand.

\subsection*{Future Work}

No future work planned.
}}%
%BeginExpansion
%TCIDATA{Version=5.00.0.2606}
%TCIDATA{LaTeXparent=1,1,functions.tex}
                      

\section*{\texttt{DualAreaSegment::DualAreaSegment}}

\subsection*{Function Prototype}

\texttt{double DualAreaSegment( Vertex vi, Vertex vj, Vertex vk, Vertex vl)}

\subsection*{Key Words}

Dual area, curvature, partial derivative, Einstein-Hilbert-Regge, geoquant.

\subsection*{Authors}

Daniel Champion

\subsection*{Introduction}

\texttt{DualAreaSegment} calculates the dual area to an edge of a
tetrahedron. \ 

\subsection*{Subsidiaries}

\textbf{Functions:}

\qquad \texttt{EdgeHeight}

\qquad \qquad \texttt{PartialEdge}

\qquad\qquad\texttt{Geometry::angle}

\qquad \texttt{FaceHeight}

\qquad\qquad\texttt{Geometry::dihedralAngle}

\textbf{Global Variables: \ }radii, etas

\textbf{Local Variables:} \ none.

\subsection*{Description}

\texttt{DualAreaSegment} is calculated with the formula:%
\begin{equation*}
\text{\texttt{DualAreaSegment(vi, vj, vk, vl)}}=
\end{equation*}%
\begin{equation*}
\frac{1}{2}\left( 
\begin{array}{c}
\text{\texttt{EdgeHeight(vi,vj,vk)}}\cdot \text{\texttt{%
FaceHeight(vi,vj,vk,vl)}} \\ 
+\text{\texttt{EdgeHeight(vi,vj,vl)}}\cdot \text{\texttt{%
FaceHeight(vi,vj,vl,vk)}}%
\end{array}%
\right) 
\end{equation*}%
\texttt{EdgeHeight} and \texttt{FaceHeight} are calculated with the
following formulae:%
\begin{align*}
\text{\texttt{EdgeHeight(vi, vj, vk)}}& =\frac{\left( \text{\texttt{%
PartialEdge(vi,vk)}}-\text{\texttt{PartialEdge(vi,vj)}}\cos \left( \alpha
_{i,jk}\right) \right) }{\sin \left( \alpha _{i,jk}\right) } \\
\text{\texttt{FaceHeight(vi, vj, vk, vl)}}& =\frac{\left( \text{\texttt{%
EdgeHeight(vi,vj,vl)}}-\text{\texttt{EdgeHeight(vi,vj,vk)}}\cos (\beta
_{ij,kl})\right) }{\sin \left( \beta _{ij,kl}\right) }
\end{align*}%
where $\alpha _{i,jk}$ is the angle at vertex $vi$ of triangle $\left\{
vi,vj,vk\right\} $, and $\beta _{ij,kl}$ is the dihedral angle along edge $%
\left\{ vi,vj\right\} $ of tetrahedron $\left\{ vi,vj,vk,vl\right\} $
(implemented with the functions \texttt{Geometry::angle} and \texttt{%
Geometry::dihedralAngle} respectively).

\texttt{DualAreaSegment} was created for the calculation performed in the
function \texttt{DualArea}, which is used in the computation of the partial
derivatives of curvature. \ These partial derivatives of curvature are used
in the calculation of the second order partial derivatives of the
Einstein-Hilbert-Regge functional for use in the optimization of said
functional using Newton's method. \ 

\subsection*{Practicum}

As an example of the usage of this function, we will calculate the dual area
to the edge $eij=\left\{ vi,vj\right\} $ (see entry: \texttt{DualArea}). \
To do this, we will sum the dual areas to each tetrahedron containing the
edge $eij$. \ 

\bigskip

\texttt{vector\TEXTsymbol{<}int\TEXTsymbol{>} sum\_over =
*(eij.getLocalTetras());}

\texttt{double sum = 0.0;}

\texttt{vector\TEXTsymbol{<}int\TEXTsymbol{>} T\_vertices, e\_vertices;}

\texttt{Tetra T;}

\texttt{Vertex vi,vj,vk,vl;}

\texttt{for(i=0; i\TEXTsymbol{<}sum\_over.size(); ++i) \{}

\qquad\texttt{T = Triangulation::tetraTable[sum\_over[i]];}

\qquad\texttt{T\_vertices = *(T.getLocalVertices());}

\qquad\texttt{e\_vertices = *(eij.getLocalVertices());}

\qquad\texttt{vi = Triangulation::vertexTable[e\_vertices[0]];}

\qquad\texttt{vj = Triangulation::vertexTable[e\_vertices[1]];}

\qquad\texttt{vk = Triangulation::vertexTable[listDifference(\&T\_vertices,
\&e\_vertices)[0]];}

\qquad\texttt{vl = Triangulation::vertexTable[listDifference(\&T\_vertices,
\&e\_vertices)[1]];}

\qquad \texttt{sum += DualAreaSegment(vi, vj, vk, vl);}

\qquad\texttt{\}}

\texttt{return sum;}

\subsection*{Limitations}

\texttt{DualAreaSegment} if fully operational and has no known limitations.
\ The function will output appropriate values provided it receives as input
four distinct vertices that define a tetrahedron.

\subsection*{Revisions}

subversion 757, 7/8/09, \texttt{DualAreaSegment} created.

subversion 1055, 3/12/10, \texttt{DualAreaSegment}\ converted to a geoquant.

\subsection*{Testing}

Trials were run and the calculations returned were verified by hand.

\subsection*{Future Work}

No future work planned.
%
%EndExpansion

\bigskip

\bigskip 

%TCIMACRO{\QSubDoc{Include ApproximatorRun}{%html2tex: Version  2.7 of June 17, 2008.
%Written by  F.J. Faase.  http://www.iwriteiam.nl/

clean.html (42) : unknown <strike>.
clean.html (42) : unknown </strike>.
clean.html (42) : unknown <strike>.
clean.html (42) : unknown </strike>.
\documentclass[10pt]{article}%
\usepackage{amssymb}
\usepackage{geometry}
\usepackage{indentfirst}
\usepackage{amsmath}
\usepackage{amsfonts}
\usepackage{graphicx}%
\setcounter{MaxMatrixCols}{30}
%TCIDATA{OutputFilter=latex2.dll}
%TCIDATA{Version=5.00.0.2606}
%TCIDATA{CSTFile=40 LaTeX article.cst}
%TCIDATA{Created=Friday, March 30, 2007 00:21:27}
%TCIDATA{LastRevised=Wednesday, June 10, 2009 11:42:33}
%TCIDATA{<META NAME="GraphicsSave" CONTENT="32">}
%TCIDATA{<META NAME="SaveForMode" CONTENT="1">}
%TCIDATA{BibliographyScheme=Manual}
%TCIDATA{<META NAME="DocumentShell" CONTENT="Standard LaTeX\Blank - Standard LaTeX Article">}
%TCIDATA{Language=American English}
\newtheorem{theorem}{Theorem}
\newtheorem{acknowledgement}[theorem]{Acknowledgement}
\newtheorem{algorithm}[theorem]{Algorithm}
\newtheorem{axiom}[theorem]{Axiom}
\newtheorem{case}[theorem]{Case}
\newtheorem{claim}[theorem]{Claim}
\newtheorem{conclusion}[theorem]{Conclusion}
\newtheorem{condition}[theorem]{Condition}
\newtheorem{conjecture}[theorem]{Conjecture}
\newtheorem{corollary}[theorem]{Corollary}
\newtheorem{criterion}[theorem]{Criterion}
\newtheorem{definition}[theorem]{Definition}
\newtheorem{example}[theorem]{Example}
\newtheorem{exercise}[theorem]{Exercise}
\newtheorem{lemma}[theorem]{Lemma}
\newtheorem{notation}[theorem]{Notation}
\newtheorem{problem}[theorem]{Problem}
\newtheorem{proposition}[theorem]{Proposition}
\newtheorem{remark}[theorem]{Remark}
\newtheorem{solution}[theorem]{Solution}
\newtheorem{summary}[theorem]{Summary}
\newenvironment{proof}[1][Proof]{\noindent\textbf{#1.} }{\ \rule{0.5em}{0.5em}}
\geometry{left=1in,right=1in,top=1in,bottom=1in}

\begin{document}

%%%%% BEGINNING OF DOCUMENT BODY %%%%%
% html: Beginning of file: `clean.html'
% DOCTYPE HTML PUBLIC "-//W3C//DTD HTML 4.01//EN"
%  This is a (PRE) block.  Make sure it's left aligned or your toc title will be off. 

\section*{\texttt{Approximator::run}}

\label{f0}\begin{quotation} {\small{\begin{verbatim} 
int run(int numsteps, double stepsize)        
int run(double precision, double stepsize)
int run(double precision, int maxNumSteps, double stepsize)
  \end{verbatim}
}}
\end{quotation}
\subsection*{Keywords}

\begin{quotation} flow, curvature, stepsize, precision, accuracy, approximator\end{quotation}

\subsection*{Authors}

\begin{itemize}\item  Joseph Thomas
\item  Alex Henniges
\end{itemize}

\subsection*{Introduction}

\begin{quotation} The \texttt{run} function of the Approximator class runs a system of differential equations representing a curvature flow for either a number of steps or until the values are within a desired precision. The system to use and how steps are performed is given in the constructor of the approximator. The type of run is based on the parameters given. The \texttt{run} function returns 0 on success or any other number if an error is encountered.\end{quotation}

\subsection*{Subsidiaries}

\begin{quotation} Functions: \end{quotation}
\begin{itemize}
\item  \texttt{Approximator::step}
\item  \texttt{Approximator::isPrecise}
\item  \texttt{Approximator::isAccurate} 
\item  \texttt{Approximator::getLatest}
\begin{enumerate}
\item  \texttt{Approximator::recordState}
\end{enumerate}
\end{itemize}
\begin{quotation} Global Variables: \texttt{radii}, \texttt{curvatures}\end{quotation}
\begin{quotation} Local Variables: \texttt{int numsteps}, \texttt{double stepsize}, \texttt{double precision}, \texttt{double accuracy}\end{quotation}

\subsection*{Description}

\begin{quotation} If the \texttt{run} function is given a number of steps, it will call its step function that number of times. In between steps, the \texttt{run} function will record the current state of any values that have been requested to be recorded (this is specified in the constructor).\end{quotation}
\begin{quotation} If the \texttt{run} function is given a precision, it will continue to call its step function until the desired quantities (curvature in two dimensions and curvature divide by radius in three dimensions) have converged within the precision bounds. Precision is defined to be the difference between subsequent values of a quantity. Therefore, precision is a measure of how much a value is changing. In between steps, the \texttt{run} function will record the current state of any values that have been requested to be recorded (this is specified in the constructor).        \end{quotation}
\begin{quotation} A flow can also be run with a precision and a max number of steps that will stop once one of the conditions is reached. The last parameter of any run indicates the step size of the flow. A lower step size will lead to more accurate steps, but a longer time to convergence.\end{quotation}
\begin{quotation} The \texttt{run} function and the overarching Approximator class exists as an improvement over the curvature flows of earlier versions of the Geocam project. The \texttt{run} function provides the skeleton that is similar for all types of curvature flows. Beyond the constructor, this should be the only thing a user calls from the Approximator class.\end{quotation}

\subsection*{Practicum}

\begin{quotation} Example:\end{quotation}{\small{\begin{verbatim} 
// Create an approximator that uses the Euler method on a Yamabe flow.
Approximator *app = new EulerApprox(Yamabe);

// Run a Yamabe flow for 300 steps with a stepsize of 0.01.
app->run(300, 0.01);
// Run with a precision bound of 0.000001 and a stepsize of 0.01
app->run(0.000001, 0.01);
\end{verbatim}
}}

\subsection*{Limitations}

\begin{quotation} The \texttt{run} function is limited in the systems of differential equations that it can run. It is designed to run with curvature flows and, when precision is used, expects the values to converge. If a precision run is performed on a flow that does not converge, the \texttt{run} function will not stop. If a new curvature flow is created whose convergence is not the usual (as in curvature divided by radius in Yamabe flow) then the \texttt{run} function will have to be modified to accommodate for this.\end{quotation}

\subsection*{Revisions}

\begin{itemize}\item  subversion 659, 5/1/09: Initial \texttt{run} function uploaded to the code.
\item  subversion 679, 6/3/09: \texttt{run} function modified to work with new Geometry structure.
\item  subversion 761, 6/12/09: \texttt{run} function modified to work with new quantity structure.
\item  subversion 787, 6/18/09: Added new \texttt{run} options to approximator. Removed accuracy. Checks for bad numbers.
\end{itemize}

\subsection*{Testing}

\begin{quotation} The function was tested by performing two and three dimensional flows on familiar triangulations. The start and end values for radii and curvature was then compared with our expected values. The expected values were obtained from the earlier curvature flows we had (see \mbox{$[$}\#Description Description\mbox{$]$} above). We also checked that the end values were within the precision and accuracy bounds when they were in effect. \end{quotation}

\subsection*{Future Work}

\begin{itemize}\item  6/17 - 
% <strike name="Future Work">
Add more run options (ex. precision and maxNumSteps).
% </strike name="Future Work">
 \textbf{Complete (6/18)}
\item  6/17 - 
% <strike name="Future Work">
Have a run stop the moment an undefined number appears.
% </strike name="Future Work">
 \textbf{Complete (6/18)}
\end{itemize}
\begin{quotation} No future work is planned at this time.\end{quotation}
    
% html: End of file: `clean.html'

%%%%% END OF DOCUMENT BODY %%%%%
% In the future, we might want to put some additional data here, such
% as when the documentation was converted from wiki to TeX.
%

\end{document}
}}%
%BeginExpansion
%html2tex: Version  2.7 of June 17, 2008.
%Written by  F.J. Faase.  http://www.iwriteiam.nl/

clean.html (42) : unknown <strike>.
clean.html (42) : unknown </strike>.
clean.html (42) : unknown <strike>.
clean.html (42) : unknown </strike>.
\documentclass[10pt]{article}%
\usepackage{amssymb}
\usepackage{geometry}
\usepackage{indentfirst}
\usepackage{amsmath}
\usepackage{amsfonts}
\usepackage{graphicx}%
\setcounter{MaxMatrixCols}{30}
%TCIDATA{OutputFilter=latex2.dll}
%TCIDATA{Version=5.00.0.2606}
%TCIDATA{CSTFile=40 LaTeX article.cst}
%TCIDATA{Created=Friday, March 30, 2007 00:21:27}
%TCIDATA{LastRevised=Wednesday, June 10, 2009 11:42:33}
%TCIDATA{<META NAME="GraphicsSave" CONTENT="32">}
%TCIDATA{<META NAME="SaveForMode" CONTENT="1">}
%TCIDATA{BibliographyScheme=Manual}
%TCIDATA{<META NAME="DocumentShell" CONTENT="Standard LaTeX\Blank - Standard LaTeX Article">}
%TCIDATA{Language=American English}
\newtheorem{theorem}{Theorem}
\newtheorem{acknowledgement}[theorem]{Acknowledgement}
\newtheorem{algorithm}[theorem]{Algorithm}
\newtheorem{axiom}[theorem]{Axiom}
\newtheorem{case}[theorem]{Case}
\newtheorem{claim}[theorem]{Claim}
\newtheorem{conclusion}[theorem]{Conclusion}
\newtheorem{condition}[theorem]{Condition}
\newtheorem{conjecture}[theorem]{Conjecture}
\newtheorem{corollary}[theorem]{Corollary}
\newtheorem{criterion}[theorem]{Criterion}
\newtheorem{definition}[theorem]{Definition}
\newtheorem{example}[theorem]{Example}
\newtheorem{exercise}[theorem]{Exercise}
\newtheorem{lemma}[theorem]{Lemma}
\newtheorem{notation}[theorem]{Notation}
\newtheorem{problem}[theorem]{Problem}
\newtheorem{proposition}[theorem]{Proposition}
\newtheorem{remark}[theorem]{Remark}
\newtheorem{solution}[theorem]{Solution}
\newtheorem{summary}[theorem]{Summary}
\newenvironment{proof}[1][Proof]{\noindent\textbf{#1.} }{\ \rule{0.5em}{0.5em}}
\geometry{left=1in,right=1in,top=1in,bottom=1in}

\begin{document}

%%%%% BEGINNING OF DOCUMENT BODY %%%%%
% html: Beginning of file: `clean.html'
% DOCTYPE HTML PUBLIC "-//W3C//DTD HTML 4.01//EN"
%  This is a (PRE) block.  Make sure it's left aligned or your toc title will be off. 

\section*{\texttt{Approximator::run}}

\label{f0}\begin{quotation} {\small{\begin{verbatim} 
int run(int numsteps, double stepsize)        
int run(double precision, double stepsize)
int run(double precision, int maxNumSteps, double stepsize)
  \end{verbatim}
}}
\end{quotation}
\subsection*{Keywords}

\begin{quotation} flow, curvature, stepsize, precision, accuracy, approximator\end{quotation}

\subsection*{Authors}

\begin{itemize}\item  Joseph Thomas
\item  Alex Henniges
\end{itemize}

\subsection*{Introduction}

\begin{quotation} The \texttt{run} function of the Approximator class runs a system of differential equations representing a curvature flow for either a number of steps or until the values are within a desired precision. The system to use and how steps are performed is given in the constructor of the approximator. The type of run is based on the parameters given. The \texttt{run} function returns 0 on success or any other number if an error is encountered.\end{quotation}

\subsection*{Subsidiaries}

\begin{quotation} Functions: \end{quotation}
\begin{itemize}
\item  \texttt{Approximator::step}
\item  \texttt{Approximator::isPrecise}
\item  \texttt{Approximator::isAccurate} 
\item  \texttt{Approximator::getLatest}
\begin{enumerate}
\item  \texttt{Approximator::recordState}
\end{enumerate}
\end{itemize}
\begin{quotation} Global Variables: \texttt{radii}, \texttt{curvatures}\end{quotation}
\begin{quotation} Local Variables: \texttt{int numsteps}, \texttt{double stepsize}, \texttt{double precision}, \texttt{double accuracy}\end{quotation}

\subsection*{Description}

\begin{quotation} If the \texttt{run} function is given a number of steps, it will call its step function that number of times. In between steps, the \texttt{run} function will record the current state of any values that have been requested to be recorded (this is specified in the constructor).\end{quotation}
\begin{quotation} If the \texttt{run} function is given a precision, it will continue to call its step function until the desired quantities (curvature in two dimensions and curvature divide by radius in three dimensions) have converged within the precision bounds. Precision is defined to be the difference between subsequent values of a quantity. Therefore, precision is a measure of how much a value is changing. In between steps, the \texttt{run} function will record the current state of any values that have been requested to be recorded (this is specified in the constructor).        \end{quotation}
\begin{quotation} A flow can also be run with a precision and a max number of steps that will stop once one of the conditions is reached. The last parameter of any run indicates the step size of the flow. A lower step size will lead to more accurate steps, but a longer time to convergence.\end{quotation}
\begin{quotation} The \texttt{run} function and the overarching Approximator class exists as an improvement over the curvature flows of earlier versions of the Geocam project. The \texttt{run} function provides the skeleton that is similar for all types of curvature flows. Beyond the constructor, this should be the only thing a user calls from the Approximator class.\end{quotation}

\subsection*{Practicum}

\begin{quotation} Example:\end{quotation}{\small{\begin{verbatim} 
// Create an approximator that uses the Euler method on a Yamabe flow.
Approximator *app = new EulerApprox(Yamabe);

// Run a Yamabe flow for 300 steps with a stepsize of 0.01.
app->run(300, 0.01);
// Run with a precision bound of 0.000001 and a stepsize of 0.01
app->run(0.000001, 0.01);
\end{verbatim}
}}

\subsection*{Limitations}

\begin{quotation} The \texttt{run} function is limited in the systems of differential equations that it can run. It is designed to run with curvature flows and, when precision is used, expects the values to converge. If a precision run is performed on a flow that does not converge, the \texttt{run} function will not stop. If a new curvature flow is created whose convergence is not the usual (as in curvature divided by radius in Yamabe flow) then the \texttt{run} function will have to be modified to accommodate for this.\end{quotation}

\subsection*{Revisions}

\begin{itemize}\item  subversion 659, 5/1/09: Initial \texttt{run} function uploaded to the code.
\item  subversion 679, 6/3/09: \texttt{run} function modified to work with new Geometry structure.
\item  subversion 761, 6/12/09: \texttt{run} function modified to work with new quantity structure.
\item  subversion 787, 6/18/09: Added new \texttt{run} options to approximator. Removed accuracy. Checks for bad numbers.
\end{itemize}

\subsection*{Testing}

\begin{quotation} The function was tested by performing two and three dimensional flows on familiar triangulations. The start and end values for radii and curvature was then compared with our expected values. The expected values were obtained from the earlier curvature flows we had (see \mbox{$[$}\#Description Description\mbox{$]$} above). We also checked that the end values were within the precision and accuracy bounds when they were in effect. \end{quotation}

\subsection*{Future Work}

\begin{itemize}\item  6/17 - 
% <strike name="Future Work">
Add more run options (ex. precision and maxNumSteps).
% </strike name="Future Work">
 \textbf{Complete (6/18)}
\item  6/17 - 
% <strike name="Future Work">
Have a run stop the moment an undefined number appears.
% </strike name="Future Work">
 \textbf{Complete (6/18)}
\end{itemize}
\begin{quotation} No future work is planned at this time.\end{quotation}
    
% html: End of file: `clean.html'

%%%%% END OF DOCUMENT BODY %%%%%
% In the future, we might want to put some additional data here, such
% as when the documentation was converted from wiki to TeX.
%

\end{document}
%
%EndExpansion

\bigskip 

\bigskip 

%TCIMACRO{%
%\QSubDoc{Include Curvature_Partial}{%TCIDATA{Version=5.00.0.2606}
%TCIDATA{LaTeXparent=1,1,functions.tex}
                      

\section*{\texttt{CurvaturePartial::CurvaturePartial}}

\subsection*{Function Prototype}

\texttt{double CurvaturePartial( Vertex v\_i, Vertex v\_l )}

\subsection*{Key Words}

Curvature, Einstein-Hilbert-Regge, functional, partial derivative, geoquant.

\subsection*{Authors}

Daniel Champion

\subsection*{Introduction}

CurvaturePartial calculates the partial derivative of the curvature at a
vertex with respect to the log radius of another (possibly the same) vertex.
\ 

\subsection*{Subsidiaries}

Functions:

\qquad\texttt{isAdjVertex}

\qquad \texttt{DualArea}

\qquad \qquad \texttt{DualAreaSegment}

\qquad \qquad \qquad \texttt{FaceHeight}

\qquad \qquad \qquad \qquad \texttt{EdgeHeight}

\qquad \qquad \qquad \qquad \qquad \texttt{PartialEdge}

\qquad\texttt{listDifference}

\qquad\texttt{listIntersection}

Global Variables: \ curvature, dihedralAngle, eta, length, radius

Local Variables: none.

\subsection*{Description}

\texttt{CurvaturePartial} receives as inputs two vertices $v_{i}$ and $v_{l}$%
. \ The first corresponds to the vertex of interest, the second corresponds
to the vertex of differentiation. \ That is,%
\begin{equation*}
\text{\texttt{CurvaturePartial (v\_i, v\_l)}}=\frac{\partial }{\partial \log
r_{l}}K_{i},
\end{equation*}%
where $r_{l}$ is the radius at vertex $v_{l}$, and $K_{i}$ is the curvature
at vertex $v_{i}$. \ 

The function begins implementation by determining the relationship between $i
$ and $l$ via the trichotomy $v_{i}=v_{l}$, $v_{i}$ is adjacent to $v_{l}$,
or $v_{i}$ and $v_{l}$ are not endpoints of any edge. \ Each of the three
cases are calculated differently. \ The general formula for the variation of
curvature w.r.t. log radii was calculated by Prof. David Glickenstein and is
available at arXiv:0906.1560v1:%
\begin{equation*}
\delta K_{i}=-\sum_{edges\text{ }\left\{ i,j\right\} }\left( 2\frac{%
l_{ij}^{\ast }}{l_{ij}}-\frac{r_{i}^{2}r_{j}^{2}\left( 1-\eta
_{ij}^{2}\right) }{l_{ij}^{2}}K_{ij}\right) \left( \delta f_{j}-\delta
f_{i}\right) +K_{i}\delta f_{i},
\end{equation*}%
where $f_{i}=\log r_{i}$, $l_{ij}$ is the length of the edge $\left\{
i,j\right\} $, and $l_{ij}^{\ast }$ is the dual area calculated with the
function \texttt{DualArea}.

When $i=l$, the formula for the partial derivative $\frac{\partial}{%
\partial\log r_{l}}K_{i}$ becomes:%
\begin{equation*}
\frac{\partial}{\partial\log r_{i}}K_{i}=\sum_{edges\text{ }\left\{
i,j\right\} }\left( 2\frac{l_{ij}^{\ast}}{l_{ij}}-\frac{r_{i}^{2}r_{j}^{2}%
\left( 1-\eta_{ij}^{2}\right) }{l_{ij}^{2}}K_{ij}\right) +K_{i}.
\end{equation*}

When $v_{i}$ is adjacent to $v_{l}$ only one term in the sum survives:%
\begin{equation*}
\frac{\partial}{\partial\log r_{l}}K_{i}=-\left( 2\frac{l_{il}^{\ast}}{l_{il}%
}-\frac{r_{i}^{2}r_{l}^{2}\left( 1-\eta_{il}^{2}\right) }{l_{il}^{2}}%
K_{il}\right) .
\end{equation*}

When $v_{i}$ and $v_{j}$ are not endpoints of any edge the partial
derivative is zero. \ 

This function was created to assist in the computation of the second
derivatives of the normalized Einstein-Hilbert-Regge functional:%
\begin{equation*}
EHR=\frac{\sum K_{i}}{\sqrt[3]{\sum\limits_{tetra\text{ }t}Vol(t)}}.
\end{equation*}

Surprisingly the first derivatives of the normalized $EHR$ functional do not
require the formula for the partial derivative of curvature since 
\begin{equation*}
\frac{\partial EHR}{\partial\log r_{i}}=K_{i}.
\end{equation*}

However, the second order partial derivatives of $EHR$ certainly require the
formulas for $\frac{\partial}{\partial\log r_{l}}K_{i}$ given above. \ These
second order partial derivatives are used to construct a Hessian matrix
which is then used the optimization of the $EHR$ functional using Newton's
method (implemented by \texttt{Newtons\_Method}).

\subsection*{Practicum}

When called, \texttt{CurvaturePartial(v\_i, v\_l)} returns the partial
derivative $\frac{\partial }{\partial \log r_{l}}K_{i}$. \ An example of its
usage is in the calculation of the second order partial derivatives of the
normalized $EHR$ functional. \ For this example let 
\begin{align*}
\text{\texttt{VolSumPartial\_i}}& =\sum_{tetra\text{ }t}\frac{\partial V_{t}%
}{\partial \log r_{i}}, \\
\text{\texttt{VolSumPartial\_j}}& =\sum_{tetra\text{ }t}\frac{\partial V_{t}%
}{\partial \log r_{j}}, \\
\text{\texttt{VolSumSecondPartial}}& =\sum_{tetra\text{ }t}\frac{\partial
^{2}V_{t}}{\partial \log r_{i}\partial \log r_{j}} \\
K& =\sum_{i}K_{i}, \\
V& =\sum_{tetra\text{ }t}V_{t}.
\end{align*}%
Then the second order partial derivative $\frac{\partial ^{2}EHR}{\partial
\log r_{i}\partial \log r_{j}}$ is calculated by:

\bigskip

\qquad \texttt{result = pow(V,
(-4.0/3.0))*(1.0/3.0)*(3*V*CurvaturePartial(i,j)}

\qquad\qquad\qquad \texttt{%
-Geometry::curvature(Triangulation::vertexTable[i])*VolSumPartial\_j}

\qquad\qquad\qquad \texttt{%
-Geometry::curvature(Triangulation::vertexTable[j])*VolSumPartial\_i}

\qquad\qquad\qquad\texttt{+(4.0/3.0)*K*pow(V,
-1.0)*VolSumPartial\_i*VolSumPartial\_j}

\qquad\qquad\qquad\texttt{-K*VolSumSecondPartial);}

\bigskip

\subsection*{Limitations}

The function \texttt{CurvaturePartial} is operational for all pairs of input
integers $i$ and $l$ that are in the vertex table. \ If one of the arguments
is not in the vertex table, the function will output zero. \ 

\subsection*{Revisions}

subversion 757, 7/6/09, \texttt{CurvaturePartial} created.

subversion 1055, 3/12/10, \texttt{CurvaturePartial}\ converted to a geoquant.

\subsection*{Testing}

This function has not been tested.

\subsection*{Future Work}

Using the calculation of the partial derivative of curvature in Mathematica,
it should be compared to the output from \texttt{CurvaturePartial}.
}}%
%BeginExpansion
%TCIDATA{Version=5.00.0.2606}
%TCIDATA{LaTeXparent=1,1,functions.tex}
                      

\section*{\texttt{CurvaturePartial::CurvaturePartial}}

\subsection*{Function Prototype}

\texttt{double CurvaturePartial( Vertex v\_i, Vertex v\_l )}

\subsection*{Key Words}

Curvature, Einstein-Hilbert-Regge, functional, partial derivative, geoquant.

\subsection*{Authors}

Daniel Champion

\subsection*{Introduction}

CurvaturePartial calculates the partial derivative of the curvature at a
vertex with respect to the log radius of another (possibly the same) vertex.
\ 

\subsection*{Subsidiaries}

Functions:

\qquad\texttt{isAdjVertex}

\qquad \texttt{DualArea}

\qquad \qquad \texttt{DualAreaSegment}

\qquad \qquad \qquad \texttt{FaceHeight}

\qquad \qquad \qquad \qquad \texttt{EdgeHeight}

\qquad \qquad \qquad \qquad \qquad \texttt{PartialEdge}

\qquad\texttt{listDifference}

\qquad\texttt{listIntersection}

Global Variables: \ curvature, dihedralAngle, eta, length, radius

Local Variables: none.

\subsection*{Description}

\texttt{CurvaturePartial} receives as inputs two vertices $v_{i}$ and $v_{l}$%
. \ The first corresponds to the vertex of interest, the second corresponds
to the vertex of differentiation. \ That is,%
\begin{equation*}
\text{\texttt{CurvaturePartial (v\_i, v\_l)}}=\frac{\partial }{\partial \log
r_{l}}K_{i},
\end{equation*}%
where $r_{l}$ is the radius at vertex $v_{l}$, and $K_{i}$ is the curvature
at vertex $v_{i}$. \ 

The function begins implementation by determining the relationship between $i
$ and $l$ via the trichotomy $v_{i}=v_{l}$, $v_{i}$ is adjacent to $v_{l}$,
or $v_{i}$ and $v_{l}$ are not endpoints of any edge. \ Each of the three
cases are calculated differently. \ The general formula for the variation of
curvature w.r.t. log radii was calculated by Prof. David Glickenstein and is
available at arXiv:0906.1560v1:%
\begin{equation*}
\delta K_{i}=-\sum_{edges\text{ }\left\{ i,j\right\} }\left( 2\frac{%
l_{ij}^{\ast }}{l_{ij}}-\frac{r_{i}^{2}r_{j}^{2}\left( 1-\eta
_{ij}^{2}\right) }{l_{ij}^{2}}K_{ij}\right) \left( \delta f_{j}-\delta
f_{i}\right) +K_{i}\delta f_{i},
\end{equation*}%
where $f_{i}=\log r_{i}$, $l_{ij}$ is the length of the edge $\left\{
i,j\right\} $, and $l_{ij}^{\ast }$ is the dual area calculated with the
function \texttt{DualArea}.

When $i=l$, the formula for the partial derivative $\frac{\partial}{%
\partial\log r_{l}}K_{i}$ becomes:%
\begin{equation*}
\frac{\partial}{\partial\log r_{i}}K_{i}=\sum_{edges\text{ }\left\{
i,j\right\} }\left( 2\frac{l_{ij}^{\ast}}{l_{ij}}-\frac{r_{i}^{2}r_{j}^{2}%
\left( 1-\eta_{ij}^{2}\right) }{l_{ij}^{2}}K_{ij}\right) +K_{i}.
\end{equation*}

When $v_{i}$ is adjacent to $v_{l}$ only one term in the sum survives:%
\begin{equation*}
\frac{\partial}{\partial\log r_{l}}K_{i}=-\left( 2\frac{l_{il}^{\ast}}{l_{il}%
}-\frac{r_{i}^{2}r_{l}^{2}\left( 1-\eta_{il}^{2}\right) }{l_{il}^{2}}%
K_{il}\right) .
\end{equation*}

When $v_{i}$ and $v_{j}$ are not endpoints of any edge the partial
derivative is zero. \ 

This function was created to assist in the computation of the second
derivatives of the normalized Einstein-Hilbert-Regge functional:%
\begin{equation*}
EHR=\frac{\sum K_{i}}{\sqrt[3]{\sum\limits_{tetra\text{ }t}Vol(t)}}.
\end{equation*}

Surprisingly the first derivatives of the normalized $EHR$ functional do not
require the formula for the partial derivative of curvature since 
\begin{equation*}
\frac{\partial EHR}{\partial\log r_{i}}=K_{i}.
\end{equation*}

However, the second order partial derivatives of $EHR$ certainly require the
formulas for $\frac{\partial}{\partial\log r_{l}}K_{i}$ given above. \ These
second order partial derivatives are used to construct a Hessian matrix
which is then used the optimization of the $EHR$ functional using Newton's
method (implemented by \texttt{Newtons\_Method}).

\subsection*{Practicum}

When called, \texttt{CurvaturePartial(v\_i, v\_l)} returns the partial
derivative $\frac{\partial }{\partial \log r_{l}}K_{i}$. \ An example of its
usage is in the calculation of the second order partial derivatives of the
normalized $EHR$ functional. \ For this example let 
\begin{align*}
\text{\texttt{VolSumPartial\_i}}& =\sum_{tetra\text{ }t}\frac{\partial V_{t}%
}{\partial \log r_{i}}, \\
\text{\texttt{VolSumPartial\_j}}& =\sum_{tetra\text{ }t}\frac{\partial V_{t}%
}{\partial \log r_{j}}, \\
\text{\texttt{VolSumSecondPartial}}& =\sum_{tetra\text{ }t}\frac{\partial
^{2}V_{t}}{\partial \log r_{i}\partial \log r_{j}} \\
K& =\sum_{i}K_{i}, \\
V& =\sum_{tetra\text{ }t}V_{t}.
\end{align*}%
Then the second order partial derivative $\frac{\partial ^{2}EHR}{\partial
\log r_{i}\partial \log r_{j}}$ is calculated by:

\bigskip

\qquad \texttt{result = pow(V,
(-4.0/3.0))*(1.0/3.0)*(3*V*CurvaturePartial(i,j)}

\qquad\qquad\qquad \texttt{%
-Geometry::curvature(Triangulation::vertexTable[i])*VolSumPartial\_j}

\qquad\qquad\qquad \texttt{%
-Geometry::curvature(Triangulation::vertexTable[j])*VolSumPartial\_i}

\qquad\qquad\qquad\texttt{+(4.0/3.0)*K*pow(V,
-1.0)*VolSumPartial\_i*VolSumPartial\_j}

\qquad\qquad\qquad\texttt{-K*VolSumSecondPartial);}

\bigskip

\subsection*{Limitations}

The function \texttt{CurvaturePartial} is operational for all pairs of input
integers $i$ and $l$ that are in the vertex table. \ If one of the arguments
is not in the vertex table, the function will output zero. \ 

\subsection*{Revisions}

subversion 757, 7/6/09, \texttt{CurvaturePartial} created.

subversion 1055, 3/12/10, \texttt{CurvaturePartial}\ converted to a geoquant.

\subsection*{Testing}

This function has not been tested.

\subsection*{Future Work}

Using the calculation of the partial derivative of curvature in Mathematica,
it should be compared to the output from \texttt{CurvaturePartial}.
%
%EndExpansion

\bigskip

\bigskip

%TCIMACRO{\QSubDoc{Include dij}{%TCIDATA{Version=5.00.0.2606}
%TCIDATA{LaTeXparent=1,1,functions.tex}
                      

\section*{\texttt{PartialEdge::PartialEdge}}

\subsection*{Function Prototype}

\texttt{double PartialEdge( Vertex vi, Vertex vj)}

\subsection*{Key Words}

Partial length, geoquant.

\subsection*{Authors}

Daniel Champion, ???

\subsection*{Introduction}

The function \texttt{PartialEdge} calculates the distance from a vertex to
the center of an edge (as determined by the center of a decorated triangle).

\subsection*{Subsidiaries}

\textbf{Functions:}

\qquad\texttt{Geometry::length}

\qquad\texttt{listIntersection}

\textbf{Global Variables:} \ radii, etas.

\textbf{Local Variables:} \ Vertex vi, vj.

\subsection*{Description}

The function \texttt{PartialEdge} is calculated with the simple formula:%
\begin{equation*}
\mathtt{PartialEdge}\text{\texttt{(vi,vj)}}=\frac{%
L_{ij}^{2}+r_{i}^{2}-r_{j}^{2}}{2L_{ij}},
\end{equation*}%
where $r_{i},r_{j}$ are the radii at vertices \texttt{vi}, and \texttt{vj}
respectively, and $L_{ij}$ is the length of the edge $\left\{ vi,vj\right\} $%
. \ Notice that this formula is not symmetric in $i$ and $j$. \ 

This function plays an important role in several areas of the project
including curvature, Dirichlet energy, and the optimization of the
Einstein-Hilbert-Regge functional. \ \texttt{PartialEdge} is used in the
calculation of several quantities used in the implementation of the \texttt{%
CurvaturePartial} function, which is used in the optimization of the
normalized Einstein-Hilbert-Regge functional.

\subsection*{Practicum}

An example of the use of this function is in the calculation of the edge
height function \texttt{EdgeHeight}:

\qquad \texttt{double EdgeHeight( Vertex vi, Vertex vj, Vertex vk) \{}

\qquad\qquad\texttt{Face fijk;}

\qquad\qquad\texttt{vector\TEXTsymbol{<}int\TEXTsymbol{>} temp\_ij = }

\qquad\qquad\qquad\texttt{listIntersection(vi.getLocalFaces(),
vj.getLocalFaces());}

\qquad\qquad\texttt{vector\TEXTsymbol{<}int\TEXTsymbol{>} temp = }

\qquad\qquad\qquad\texttt{listIntersection( \&temp\_ij, vk.getLocalFaces());}

\qquad\qquad\texttt{fijk = Triangulation::faceTable[temp[0]];}

\qquad \qquad \texttt{double result = (PartialEdge(vi, vk)-PartialEdge(vi,vj)%
}

\qquad\qquad\qquad\texttt{*cos(Geometry::angle(vi,
fijk)))/sin(Geometry::angle(vi, fijk));}

\qquad\qquad\texttt{return result;}

\qquad\qquad\texttt{\}}

\subsection*{Limitations}

\texttt{PartialEdge} must receive as input two vertices that define an edge
in the triangulation. \ \texttt{PartialEdge} returns distinct values for
each permutation of the input vertices. \ 

\subsection*{Revisions}

subversion 757, 7/13/09, \texttt{PartialEdge} created.

subversion 1055, 3/12/10, \texttt{PartialEdge}\ converted to a geoquant.

\subsection*{Testing}

\texttt{dij} was tested by working out several examples by hand.

\subsection*{Future Work}

This function has been added to the Geometry class geoquants, and thus this
entry needs to be updated.
}}%
%BeginExpansion
%TCIDATA{Version=5.00.0.2606}
%TCIDATA{LaTeXparent=1,1,functions.tex}
                      

\section*{\texttt{PartialEdge::PartialEdge}}

\subsection*{Function Prototype}

\texttt{double PartialEdge( Vertex vi, Vertex vj)}

\subsection*{Key Words}

Partial length, geoquant.

\subsection*{Authors}

Daniel Champion, ???

\subsection*{Introduction}

The function \texttt{PartialEdge} calculates the distance from a vertex to
the center of an edge (as determined by the center of a decorated triangle).

\subsection*{Subsidiaries}

\textbf{Functions:}

\qquad\texttt{Geometry::length}

\qquad\texttt{listIntersection}

\textbf{Global Variables:} \ radii, etas.

\textbf{Local Variables:} \ Vertex vi, vj.

\subsection*{Description}

The function \texttt{PartialEdge} is calculated with the simple formula:%
\begin{equation*}
\mathtt{PartialEdge}\text{\texttt{(vi,vj)}}=\frac{%
L_{ij}^{2}+r_{i}^{2}-r_{j}^{2}}{2L_{ij}},
\end{equation*}%
where $r_{i},r_{j}$ are the radii at vertices \texttt{vi}, and \texttt{vj}
respectively, and $L_{ij}$ is the length of the edge $\left\{ vi,vj\right\} $%
. \ Notice that this formula is not symmetric in $i$ and $j$. \ 

This function plays an important role in several areas of the project
including curvature, Dirichlet energy, and the optimization of the
Einstein-Hilbert-Regge functional. \ \texttt{PartialEdge} is used in the
calculation of several quantities used in the implementation of the \texttt{%
CurvaturePartial} function, which is used in the optimization of the
normalized Einstein-Hilbert-Regge functional.

\subsection*{Practicum}

An example of the use of this function is in the calculation of the edge
height function \texttt{EdgeHeight}:

\qquad \texttt{double EdgeHeight( Vertex vi, Vertex vj, Vertex vk) \{}

\qquad\qquad\texttt{Face fijk;}

\qquad\qquad\texttt{vector\TEXTsymbol{<}int\TEXTsymbol{>} temp\_ij = }

\qquad\qquad\qquad\texttt{listIntersection(vi.getLocalFaces(),
vj.getLocalFaces());}

\qquad\qquad\texttt{vector\TEXTsymbol{<}int\TEXTsymbol{>} temp = }

\qquad\qquad\qquad\texttt{listIntersection( \&temp\_ij, vk.getLocalFaces());}

\qquad\qquad\texttt{fijk = Triangulation::faceTable[temp[0]];}

\qquad \qquad \texttt{double result = (PartialEdge(vi, vk)-PartialEdge(vi,vj)%
}

\qquad\qquad\qquad\texttt{*cos(Geometry::angle(vi,
fijk)))/sin(Geometry::angle(vi, fijk));}

\qquad\qquad\texttt{return result;}

\qquad\qquad\texttt{\}}

\subsection*{Limitations}

\texttt{PartialEdge} must receive as input two vertices that define an edge
in the triangulation. \ \texttt{PartialEdge} returns distinct values for
each permutation of the input vertices. \ 

\subsection*{Revisions}

subversion 757, 7/13/09, \texttt{PartialEdge} created.

subversion 1055, 3/12/10, \texttt{PartialEdge}\ converted to a geoquant.

\subsection*{Testing}

\texttt{dij} was tested by working out several examples by hand.

\subsection*{Future Work}

This function has been added to the Geometry class geoquants, and thus this
entry needs to be updated.
%
%EndExpansion

\bigskip

\bigskip

%TCIMACRO{\QSubDoc{Include EHR}{%TCIDATA{Version=5.00.0.2606}
%TCIDATA{LaTeXparent=1,1,functions.tex}
                      

\section*{\texttt{EHR}}

\subsection*{Function Prototype}

\texttt{double Example\_Function (Vertex, Edge1, Edge2, double, int,...)}

\subsection*{Key Words}

Enter all key words to this function here.

\subsection*{Authors}

Type the primary authors here. Example:

Daniel "Cliff Jumper" Champion

\subsection*{Introduction}

In this space provide a brief introduction to familiarize the reader with
the function listed above. \ 

\subsection*{Subsidiaries}

List all functions used by this function; list all variables (and types)
used by this function. \ Indent to show hierarchy.

Functions:

\qquad Function1

\qquad Function2

\qquad\qquad Function2.1

\qquad\qquad Function2.2

\qquad\qquad\qquad Function2.3

Global Variables:

Local Variables:

\subsection*{Description}

Begin this section with a detailed description of the function. \ For simple
functions provide sufficient theory to define the function, otherwise
outline the theory and cite "calculations were performed in Mathematica..."
etc. if applicable. \ 

Conclude this section with an explanation of why this function exists. \
This would include initial motivation for the creation of the function, as
well as all primary programs (functions) that utilize the function. \ A
brief history of the function can also be given if it serves to explain why
the function exists. \ 

\subsection*{Practicum}

Place any and all practical information about the function in this section.
\ Provide a short example of the use of this function if appropriate. \ This
should be written in code or pseudo-code written in the format below:

\bigskip

\qquad\texttt{begin repeat;}

\qquad\qquad\texttt{result = n+5;}

\qquad\qquad\texttt{end if result \TEXTsymbol{>} 5;}

\qquad\qquad\texttt{n=n+1;}

\qquad\texttt{end repeat;}

\bigskip

\subsection*{Limitations}

Provide a description of the limitations of the function. \ It should be
clear what works and doesn't work about the function from reading this
section. \ 

\subsection*{Revisions}

List the major revisions to the function with dates and a one sentence
comment. \ Example:

subversion 617, 6/8/09, \texttt{Example\_Function} created with severe
limitations.

subversion 618, 6/9/09, \texttt{Example\_Function} was fully commented and
initial testing complete.

subversion 619, 6/10/09, \texttt{Example\_Function} was augmented to utilize
the Geometry class.

\subsection*{Testing}

Describe how the function was tested. \ Include dates and names of test
results if possible.

\subsection*{Future Work}

In this section, describe what changes or increased functionality are
desired for this function. \ It may be helpful to address some of the items
listed in the "Limitations" section.
}}%
%BeginExpansion
%TCIDATA{Version=5.00.0.2606}
%TCIDATA{LaTeXparent=1,1,functions.tex}
                      

\section*{\texttt{EHR}}

\subsection*{Function Prototype}

\texttt{double Example\_Function (Vertex, Edge1, Edge2, double, int,...)}

\subsection*{Key Words}

Enter all key words to this function here.

\subsection*{Authors}

Type the primary authors here. Example:

Daniel "Cliff Jumper" Champion

\subsection*{Introduction}

In this space provide a brief introduction to familiarize the reader with
the function listed above. \ 

\subsection*{Subsidiaries}

List all functions used by this function; list all variables (and types)
used by this function. \ Indent to show hierarchy.

Functions:

\qquad Function1

\qquad Function2

\qquad\qquad Function2.1

\qquad\qquad Function2.2

\qquad\qquad\qquad Function2.3

Global Variables:

Local Variables:

\subsection*{Description}

Begin this section with a detailed description of the function. \ For simple
functions provide sufficient theory to define the function, otherwise
outline the theory and cite "calculations were performed in Mathematica..."
etc. if applicable. \ 

Conclude this section with an explanation of why this function exists. \
This would include initial motivation for the creation of the function, as
well as all primary programs (functions) that utilize the function. \ A
brief history of the function can also be given if it serves to explain why
the function exists. \ 

\subsection*{Practicum}

Place any and all practical information about the function in this section.
\ Provide a short example of the use of this function if appropriate. \ This
should be written in code or pseudo-code written in the format below:

\bigskip

\qquad\texttt{begin repeat;}

\qquad\qquad\texttt{result = n+5;}

\qquad\qquad\texttt{end if result \TEXTsymbol{>} 5;}

\qquad\qquad\texttt{n=n+1;}

\qquad\texttt{end repeat;}

\bigskip

\subsection*{Limitations}

Provide a description of the limitations of the function. \ It should be
clear what works and doesn't work about the function from reading this
section. \ 

\subsection*{Revisions}

List the major revisions to the function with dates and a one sentence
comment. \ Example:

subversion 617, 6/8/09, \texttt{Example\_Function} created with severe
limitations.

subversion 618, 6/9/09, \texttt{Example\_Function} was fully commented and
initial testing complete.

subversion 619, 6/10/09, \texttt{Example\_Function} was augmented to utilize
the Geometry class.

\subsection*{Testing}

Describe how the function was tested. \ Include dates and names of test
results if possible.

\subsection*{Future Work}

In this section, describe what changes or increased functionality are
desired for this function. \ It may be helpful to address some of the items
listed in the "Limitations" section.
%
%EndExpansion

\bigskip

\bigskip

%TCIMACRO{\QSubDoc{Include EHR_Partial}{%TCIDATA{Version=5.00.0.2606}
%TCIDATA{LaTeXparent=1,1,functions.tex}
                      

\section*{\texttt{EHRPartial::EHRPartial}}

\subsection*{Function Prototype}

\texttt{double EHRPartial(int i)}

\subsection*{Key Words}

Einstein-Hilbert-Regge, functional, Newton's method, partial derivative,
geoquant.

\subsection*{Authors}

Daniel Champion

\subsection*{Introduction}

\texttt{EHRPartial} calculates the partial derivative of the normalized
Einstein-Hilbert-Regge functional with respect to log radii. \ 

\subsection*{Subsidiaries}

\textbf{Functions:}

\qquad \texttt{TotalVolume}

\qquad \texttt{TotalCurvature}

\qquad \texttt{VolumePartial}

\qquad\qquad\texttt{listDifference}

\qquad\qquad\texttt{listIntersection}

\textbf{Global Variables:} radii, etas, curvature, volume

\textbf{Local Variables:} \ none

\subsection*{Description}

The normalized Einstein-Hilbert-Regge functional is given by the expression:%
\begin{equation*}
EHR=\frac{\sum\limits_{j}K_{j}}{\sqrt[3]{\sum\limits_{tetra\text{ }t}V_{t}}},
\end{equation*}%
where $K_{i}$ is the curvature at vertex $j$, and $V_{t}$ is the volume of
tetrahedron $t$. \ It can be shown (see arXiv:0906.1560v1) that 
\begin{equation*}
\frac{\partial }{\partial \log r_{i}}\left( \sum\limits_{j}K_{j}\right)
=K_{i},
\end{equation*}%
hence the partial derivative of the normalized EHR functional becomes:%
\begin{align*}
\frac{\partial }{\partial \log r_{i}}EHR& =\frac{K_{i}\sqrt[3]{%
\sum\limits_{tetra\text{ }t}V_{t}}-\frac{1}{3}\left( \sum\limits_{tetra\text{
}t}V_{t}\right) ^{-\frac{2}{3}}\sum\limits_{tetra\text{ }t}\frac{\partial
V_{t}}{\partial \log r_{i}}\sum\limits_{j}K_{j}}{\left( \sum\limits_{tetra%
\text{ }t}V_{t}\right) ^{\frac{2}{3}}} \\
& =V^{-\frac{4}{3}}\left( K_{i}V-\frac{1}{3}K\sum\limits_{tetra\text{ }t}%
\frac{\partial V_{t}}{\partial \log r_{i}}\right) ,
\end{align*}%
where $V$ is the total volume of all tetrahedra in the triangulation and $K$
is the sum of the curvatures over all vertices in the triangulation. \ 
\texttt{EHRPartial (v\_i)} calculates $\frac{\partial }{\partial \log r_{i}}%
EHR$. \ 

The primary use of this function is in the calculation of the negative
gradient of the EHR functional for use in optimization of the functional
using Newton's method. \ The formula for $\frac{\partial }{\partial \log
r_{i}}EHR$ given above was also used in the calculation of the second order
partial derivatives of the EHR functional, implemented in \texttt{%
EHRSecondPartial}.

\subsection*{Practicum}

As an example of the use of this function, the calculation of the gradient
of the EHR functional will be calculated. \ The negative gradient will be
outputted as the array \texttt{EHRneg\_gradient}.

\bigskip

\qquad\texttt{double EHRneg\_gradient[Triangulation::vertexTable.size()];}

\qquad \texttt{for(int i=0; i \TEXTsymbol{<}
Triangulation::vertexTable.size(); ++i) \{}

\qquad \qquad \texttt{Vertex v\_i = Triangulation::vertexTable[i+1];}

\qquad \qquad \texttt{EHRneg\_gradient[i] = -1.0*EHRPartial(v\_i);}

\qquad\qquad\texttt{\}}

\subsection*{Limitations}

The function \texttt{EHRPartial} is fully functional with no known
limitations. \ It will return appropriate values so long as it is called
with an integer in the vertex table. \ 

\subsection*{Revisions}

List the major revisions to the function with dates and a one sentence
comment. \ Example:

subversion 757, 7/7/09, \texttt{EHRPartial} created.

subversion 1055, 3/12/10, \texttt{EHRPartial}\ converted to a geoquant.

\subsection*{Testing}

This function has not been tested.

\subsection*{Future Work}

A testing regime should be instituted for this function. \ 
}}%
%BeginExpansion
%TCIDATA{Version=5.00.0.2606}
%TCIDATA{LaTeXparent=1,1,functions.tex}
                      

\section*{\texttt{EHRPartial::EHRPartial}}

\subsection*{Function Prototype}

\texttt{double EHRPartial(int i)}

\subsection*{Key Words}

Einstein-Hilbert-Regge, functional, Newton's method, partial derivative,
geoquant.

\subsection*{Authors}

Daniel Champion

\subsection*{Introduction}

\texttt{EHRPartial} calculates the partial derivative of the normalized
Einstein-Hilbert-Regge functional with respect to log radii. \ 

\subsection*{Subsidiaries}

\textbf{Functions:}

\qquad \texttt{TotalVolume}

\qquad \texttt{TotalCurvature}

\qquad \texttt{VolumePartial}

\qquad\qquad\texttt{listDifference}

\qquad\qquad\texttt{listIntersection}

\textbf{Global Variables:} radii, etas, curvature, volume

\textbf{Local Variables:} \ none

\subsection*{Description}

The normalized Einstein-Hilbert-Regge functional is given by the expression:%
\begin{equation*}
EHR=\frac{\sum\limits_{j}K_{j}}{\sqrt[3]{\sum\limits_{tetra\text{ }t}V_{t}}},
\end{equation*}%
where $K_{i}$ is the curvature at vertex $j$, and $V_{t}$ is the volume of
tetrahedron $t$. \ It can be shown (see arXiv:0906.1560v1) that 
\begin{equation*}
\frac{\partial }{\partial \log r_{i}}\left( \sum\limits_{j}K_{j}\right)
=K_{i},
\end{equation*}%
hence the partial derivative of the normalized EHR functional becomes:%
\begin{align*}
\frac{\partial }{\partial \log r_{i}}EHR& =\frac{K_{i}\sqrt[3]{%
\sum\limits_{tetra\text{ }t}V_{t}}-\frac{1}{3}\left( \sum\limits_{tetra\text{
}t}V_{t}\right) ^{-\frac{2}{3}}\sum\limits_{tetra\text{ }t}\frac{\partial
V_{t}}{\partial \log r_{i}}\sum\limits_{j}K_{j}}{\left( \sum\limits_{tetra%
\text{ }t}V_{t}\right) ^{\frac{2}{3}}} \\
& =V^{-\frac{4}{3}}\left( K_{i}V-\frac{1}{3}K\sum\limits_{tetra\text{ }t}%
\frac{\partial V_{t}}{\partial \log r_{i}}\right) ,
\end{align*}%
where $V$ is the total volume of all tetrahedra in the triangulation and $K$
is the sum of the curvatures over all vertices in the triangulation. \ 
\texttt{EHRPartial (v\_i)} calculates $\frac{\partial }{\partial \log r_{i}}%
EHR$. \ 

The primary use of this function is in the calculation of the negative
gradient of the EHR functional for use in optimization of the functional
using Newton's method. \ The formula for $\frac{\partial }{\partial \log
r_{i}}EHR$ given above was also used in the calculation of the second order
partial derivatives of the EHR functional, implemented in \texttt{%
EHRSecondPartial}.

\subsection*{Practicum}

As an example of the use of this function, the calculation of the gradient
of the EHR functional will be calculated. \ The negative gradient will be
outputted as the array \texttt{EHRneg\_gradient}.

\bigskip

\qquad\texttt{double EHRneg\_gradient[Triangulation::vertexTable.size()];}

\qquad \texttt{for(int i=0; i \TEXTsymbol{<}
Triangulation::vertexTable.size(); ++i) \{}

\qquad \qquad \texttt{Vertex v\_i = Triangulation::vertexTable[i+1];}

\qquad \qquad \texttt{EHRneg\_gradient[i] = -1.0*EHRPartial(v\_i);}

\qquad\qquad\texttt{\}}

\subsection*{Limitations}

The function \texttt{EHRPartial} is fully functional with no known
limitations. \ It will return appropriate values so long as it is called
with an integer in the vertex table. \ 

\subsection*{Revisions}

List the major revisions to the function with dates and a one sentence
comment. \ Example:

subversion 757, 7/7/09, \texttt{EHRPartial} created.

subversion 1055, 3/12/10, \texttt{EHRPartial}\ converted to a geoquant.

\subsection*{Testing}

This function has not been tested.

\subsection*{Future Work}

A testing regime should be instituted for this function. \ 
%
%EndExpansion

\bigskip

\bigskip

%TCIMACRO{%
%\QSubDoc{Include EHR_Second_Partial}{%TCIDATA{Version=5.00.0.2606}
%TCIDATA{LaTeXparent=1,1,functions.tex}
                      

\section*{\texttt{EHRSecondPartial::EHRSecondPartial}}

\subsection*{Function Prototype}

\texttt{double EHRSecondPartial (Vertex v\_i, Vertex v\_j)}

\subsection*{Key Words}

Einstein-Hilbert-Regge, functional, partial derivative, Hessian, geoquant.

\subsection*{Authors}

Daniel Champion

\subsection*{Introduction}

\texttt{EHRSecondPartial} calculates the second order partial derivatives of
the normalized Einstein-Hilbert-Regge functional with respect to log radii.
\ 

\subsection*{Subsidiaries}

\textbf{Functions:}

\qquad \texttt{CurvaturePartial}

\qquad\qquad\texttt{isAdjVertex}

\qquad \qquad \texttt{DualArea}

\qquad \qquad \qquad \texttt{DualAreaSegment}

\qquad \qquad \qquad \qquad \texttt{FaceHeight}

\qquad \qquad \qquad \qquad \qquad \texttt{EdgeHeight}

\qquad \qquad \qquad \qquad \qquad \qquad \texttt{PartialEdge}

\qquad\qquad\texttt{listDifference}

\qquad\qquad\texttt{listIntersection}

\qquad \texttt{TotalCurvature}

\qquad\qquad\texttt{Geometry::curvature}

\qquad \texttt{TotalVolume}

\qquad\qquad\texttt{Geometry::volume}

\qquad \texttt{VolumePartial}

\qquad\qquad\texttt{listDifference}

\qquad\qquad\texttt{listIntersection.}

\qquad \texttt{VolumeSecondPartial}

\textbf{Global Variables: }\ radii, etas.

\textbf{Local Variables:} \ none.

\subsection*{Description}

The normalized Einstein-Hilbert-Regge functional is given by the expression:%
\begin{equation*}
EHR=\frac{\sum\limits_{j}K_{j}}{\sqrt[3]{\sum\limits_{tetra\text{ }t}V_{t}}},
\end{equation*}
where $K_{i}$ is the curvature at vertex $j$, and $V_{t}$ is the volume of
tetrahedron $t$. \ It can be shown (see arXiv:0906.1560v1) that 
\begin{equation*}
\frac{\partial}{\partial\log r_{i}}\left( \sum\limits_{j}K_{j}\right) =K_{i},
\end{equation*}
hence the partial derivative of the normalized EHR functional simplifies to
become:%
\begin{equation*}
\frac{\partial}{\partial\log r_{i}}EHR=V^{-\frac{4}{3}}\left( K_{i}V-\frac {1%
}{3}K\sum\limits_{tetra\text{ }t}\frac{\partial V_{t}}{\partial\log r_{i}}%
\right) ,
\end{equation*}
where $V$ is the total volume of all tetrahedra in the triangulation and $K$
is the sum of the curvatures over all vertices in the triangulation. \
Differentiating this result with respect to $\log r_{j}$ yields:%
\begin{equation*}
\frac{\partial^{2}}{\partial\log r_{i}\partial\log r_{j}}EHR=V^{-\frac{4}{3}%
}\left( 
\begin{array}{c}
V\frac{\partial K_{i}}{\partial\log r_{j}}-\frac{1}{3}K_{i}\sum\limits_{t}%
\frac{\partial V_{t}}{\partial\log r_{j}}-\frac{1}{3}K_{j}\sum\limits_{t}%
\frac{\partial V_{t}}{\partial\log r_{i}} \\ 
+\frac{4}{9}KV^{-1}\sum\limits_{t}\frac{\partial V_{t}}{\partial\log r_{j}}%
\sum\limits_{t}\frac{\partial V_{t}}{\partial\log r_{i}}-\frac{1}{3}%
K\sum\limits_{t}\frac{\partial^{2}V_{t}}{\partial\log r_{i}\partial\log r_{j}%
}%
\end{array}
\right) .
\end{equation*}

When called, \texttt{EHRSecondPartial} calculates the formula above, that is:%
\begin{equation*}
\text{\texttt{EHR\_Second\_Partial (i,j)}}=\frac{\partial ^{2}}{\partial
\log r_{i}\partial \log r_{j}}EHR.
\end{equation*}

The use of this function is in the population of the Hessian matrix for the
normalized EHR functional. \ This Hessian matrix is used in the optimization
of the EHR functional using Newton's method.

\subsection*{Practicum}

As an example of the usage of \texttt{EHRSecondPartial}, the Hessian matrix
of the normalized EHR functional will be populated. \ In this example, the
Hessian matrix is the array EHRhessian. \ The example reduced computation
time by only calling \texttt{EHRSecondPartial} for the upper triangular
portion of the EHRhessian array, and symmetrically copies the entries above
the diagonal to the corresponding location below the diagonal. \
Furthermore, in C++ arrays begin indexing at zero, however the vertices of
the triangulations begin indexing at 1, requiring a shift of one in the
population step. \ Note that triangulations that do not label the vertices
consecutively will not be compatible with the following code. \ 

\bigskip

\qquad\texttt{double
EHRhessian[Triangulation::vertexTable.size()][Triangulation::vertexTable.size()];%
}

\qquad\texttt{for(int i = 0; i \TEXTsymbol{<}
Triangulation::vertexTable.size(); ++i) \{}

\qquad \qquad \texttt{for(int j = 0; j \TEXTsymbol{<}
Triangulation::vertexTable.size(); ++j) \{}

\qquad \qquad \qquad \texttt{Vertex vi = Triangulation::vertexYable[i+1];}

\qquad \qquad \qquad \texttt{Vertex vj = Triangulation::vertexTable[j+1];}

\qquad \qquad \qquad \texttt{if (i \TEXTsymbol{<}= j) \{}

\qquad \qquad \qquad \qquad \texttt{EHRhessian[i][j]=EHRSecondPartial( vi ,
vj );}

\qquad\qquad\qquad\qquad\texttt{EHRhessian[j][i]=EHRhessian[i][j];}

\qquad\qquad\qquad\qquad\texttt{\}}

\qquad\qquad\qquad\texttt{\}}

\qquad\qquad\texttt{\}}

\subsection*{Limitations}

\texttt{EHRSecondPartial} is fully operational with no known limitations. \
The function will output appropriate values provided it receives as inputs a
pair of integers in the vertex table. \ 

\subsection*{Revisions}

subversion 757, 7/7/09, \texttt{EHRSecondPartial} created.

subversion 1055, 3/12/10, \texttt{EHRSecondPartial}\ converted to a geoquant.

\subsection*{Testing}

This function has not been tested.

\subsection*{Future Work}

A testing regime should be instituted for this function. \ 
}}%
%BeginExpansion
%TCIDATA{Version=5.00.0.2606}
%TCIDATA{LaTeXparent=1,1,functions.tex}
                      

\section*{\texttt{EHRSecondPartial::EHRSecondPartial}}

\subsection*{Function Prototype}

\texttt{double EHRSecondPartial (Vertex v\_i, Vertex v\_j)}

\subsection*{Key Words}

Einstein-Hilbert-Regge, functional, partial derivative, Hessian, geoquant.

\subsection*{Authors}

Daniel Champion

\subsection*{Introduction}

\texttt{EHRSecondPartial} calculates the second order partial derivatives of
the normalized Einstein-Hilbert-Regge functional with respect to log radii.
\ 

\subsection*{Subsidiaries}

\textbf{Functions:}

\qquad \texttt{CurvaturePartial}

\qquad\qquad\texttt{isAdjVertex}

\qquad \qquad \texttt{DualArea}

\qquad \qquad \qquad \texttt{DualAreaSegment}

\qquad \qquad \qquad \qquad \texttt{FaceHeight}

\qquad \qquad \qquad \qquad \qquad \texttt{EdgeHeight}

\qquad \qquad \qquad \qquad \qquad \qquad \texttt{PartialEdge}

\qquad\qquad\texttt{listDifference}

\qquad\qquad\texttt{listIntersection}

\qquad \texttt{TotalCurvature}

\qquad\qquad\texttt{Geometry::curvature}

\qquad \texttt{TotalVolume}

\qquad\qquad\texttt{Geometry::volume}

\qquad \texttt{VolumePartial}

\qquad\qquad\texttt{listDifference}

\qquad\qquad\texttt{listIntersection.}

\qquad \texttt{VolumeSecondPartial}

\textbf{Global Variables: }\ radii, etas.

\textbf{Local Variables:} \ none.

\subsection*{Description}

The normalized Einstein-Hilbert-Regge functional is given by the expression:%
\begin{equation*}
EHR=\frac{\sum\limits_{j}K_{j}}{\sqrt[3]{\sum\limits_{tetra\text{ }t}V_{t}}},
\end{equation*}
where $K_{i}$ is the curvature at vertex $j$, and $V_{t}$ is the volume of
tetrahedron $t$. \ It can be shown (see arXiv:0906.1560v1) that 
\begin{equation*}
\frac{\partial}{\partial\log r_{i}}\left( \sum\limits_{j}K_{j}\right) =K_{i},
\end{equation*}
hence the partial derivative of the normalized EHR functional simplifies to
become:%
\begin{equation*}
\frac{\partial}{\partial\log r_{i}}EHR=V^{-\frac{4}{3}}\left( K_{i}V-\frac {1%
}{3}K\sum\limits_{tetra\text{ }t}\frac{\partial V_{t}}{\partial\log r_{i}}%
\right) ,
\end{equation*}
where $V$ is the total volume of all tetrahedra in the triangulation and $K$
is the sum of the curvatures over all vertices in the triangulation. \
Differentiating this result with respect to $\log r_{j}$ yields:%
\begin{equation*}
\frac{\partial^{2}}{\partial\log r_{i}\partial\log r_{j}}EHR=V^{-\frac{4}{3}%
}\left( 
\begin{array}{c}
V\frac{\partial K_{i}}{\partial\log r_{j}}-\frac{1}{3}K_{i}\sum\limits_{t}%
\frac{\partial V_{t}}{\partial\log r_{j}}-\frac{1}{3}K_{j}\sum\limits_{t}%
\frac{\partial V_{t}}{\partial\log r_{i}} \\ 
+\frac{4}{9}KV^{-1}\sum\limits_{t}\frac{\partial V_{t}}{\partial\log r_{j}}%
\sum\limits_{t}\frac{\partial V_{t}}{\partial\log r_{i}}-\frac{1}{3}%
K\sum\limits_{t}\frac{\partial^{2}V_{t}}{\partial\log r_{i}\partial\log r_{j}%
}%
\end{array}
\right) .
\end{equation*}

When called, \texttt{EHRSecondPartial} calculates the formula above, that is:%
\begin{equation*}
\text{\texttt{EHR\_Second\_Partial (i,j)}}=\frac{\partial ^{2}}{\partial
\log r_{i}\partial \log r_{j}}EHR.
\end{equation*}

The use of this function is in the population of the Hessian matrix for the
normalized EHR functional. \ This Hessian matrix is used in the optimization
of the EHR functional using Newton's method.

\subsection*{Practicum}

As an example of the usage of \texttt{EHRSecondPartial}, the Hessian matrix
of the normalized EHR functional will be populated. \ In this example, the
Hessian matrix is the array EHRhessian. \ The example reduced computation
time by only calling \texttt{EHRSecondPartial} for the upper triangular
portion of the EHRhessian array, and symmetrically copies the entries above
the diagonal to the corresponding location below the diagonal. \
Furthermore, in C++ arrays begin indexing at zero, however the vertices of
the triangulations begin indexing at 1, requiring a shift of one in the
population step. \ Note that triangulations that do not label the vertices
consecutively will not be compatible with the following code. \ 

\bigskip

\qquad\texttt{double
EHRhessian[Triangulation::vertexTable.size()][Triangulation::vertexTable.size()];%
}

\qquad\texttt{for(int i = 0; i \TEXTsymbol{<}
Triangulation::vertexTable.size(); ++i) \{}

\qquad \qquad \texttt{for(int j = 0; j \TEXTsymbol{<}
Triangulation::vertexTable.size(); ++j) \{}

\qquad \qquad \qquad \texttt{Vertex vi = Triangulation::vertexYable[i+1];}

\qquad \qquad \qquad \texttt{Vertex vj = Triangulation::vertexTable[j+1];}

\qquad \qquad \qquad \texttt{if (i \TEXTsymbol{<}= j) \{}

\qquad \qquad \qquad \qquad \texttt{EHRhessian[i][j]=EHRSecondPartial( vi ,
vj );}

\qquad\qquad\qquad\qquad\texttt{EHRhessian[j][i]=EHRhessian[i][j];}

\qquad\qquad\qquad\qquad\texttt{\}}

\qquad\qquad\qquad\texttt{\}}

\qquad\qquad\texttt{\}}

\subsection*{Limitations}

\texttt{EHRSecondPartial} is fully operational with no known limitations. \
The function will output appropriate values provided it receives as inputs a
pair of integers in the vertex table. \ 

\subsection*{Revisions}

subversion 757, 7/7/09, \texttt{EHRSecondPartial} created.

subversion 1055, 3/12/10, \texttt{EHRSecondPartial}\ converted to a geoquant.

\subsection*{Testing}

This function has not been tested.

\subsection*{Future Work}

A testing regime should be instituted for this function. \ 
%
%EndExpansion

\bigskip

\bigskip 

%TCIMACRO{\QSubDoc{Include flip}{%html2tex: Version  2.7 of June 17, 2008.
%Written by  F.J. Faase.  http://www.iwriteiam.nl/

\documentclass[10pt]{article}%
\usepackage{amssymb}
\usepackage{geometry}
\usepackage{indentfirst}
\usepackage{amsmath}
\usepackage{amsfonts}
\usepackage{graphicx}%
\setcounter{MaxMatrixCols}{30}
%TCIDATA{OutputFilter=latex2.dll}
%TCIDATA{Version=5.00.0.2606}
%TCIDATA{CSTFile=40 LaTeX article.cst}
%TCIDATA{Created=Friday, March 30, 2007 00:21:27}
%TCIDATA{LastRevised=Wednesday, June 10, 2009 11:42:33}
%TCIDATA{<META NAME="GraphicsSave" CONTENT="32">}
%TCIDATA{<META NAME="SaveForMode" CONTENT="1">}
%TCIDATA{BibliographyScheme=Manual}
%TCIDATA{<META NAME="DocumentShell" CONTENT="Standard LaTeX\Blank - Standard LaTeX Article">}
%TCIDATA{Language=American English}
\newtheorem{theorem}{Theorem}
\newtheorem{acknowledgement}[theorem]{Acknowledgement}
\newtheorem{algorithm}[theorem]{Algorithm}
\newtheorem{axiom}[theorem]{Axiom}
\newtheorem{case}[theorem]{Case}
\newtheorem{claim}[theorem]{Claim}
\newtheorem{conclusion}[theorem]{Conclusion}
\newtheorem{condition}[theorem]{Condition}
\newtheorem{conjecture}[theorem]{Conjecture}
\newtheorem{corollary}[theorem]{Corollary}
\newtheorem{criterion}[theorem]{Criterion}
\newtheorem{definition}[theorem]{Definition}
\newtheorem{example}[theorem]{Example}
\newtheorem{exercise}[theorem]{Exercise}
\newtheorem{lemma}[theorem]{Lemma}
\newtheorem{notation}[theorem]{Notation}
\newtheorem{problem}[theorem]{Problem}
\newtheorem{proposition}[theorem]{Proposition}
\newtheorem{remark}[theorem]{Remark}
\newtheorem{solution}[theorem]{Solution}
\newtheorem{summary}[theorem]{Summary}
\newenvironment{proof}[1][Proof]{\noindent\textbf{#1.} }{\ \rule{0.5em}{0.5em}}
\geometry{left=1in,right=1in,top=1in,bottom=1in}

\begin{document}

%%%%% BEGINNING OF DOCUMENT BODY %%%%%
% html: Beginning of file: `clean.html'
% DOCTYPE HTML PUBLIC "-//W3C//DTD HTML 4.01//EN"
%  This is a (PRE) block.  Make sure it's left aligned or your toc title will be off. 

\section*{\texttt{flip}}

\label{f0}{\small{\begin{verbatim} 
Edge flip(Edge e)
\end{verbatim}
}}

\subsection*{Keywords}

\begin{quotation} flip, delaunay\end{quotation}

\subsection*{Authors}

\begin{quotation} Kurt Norwood\end{quotation}

\subsection*{Introduction}

\begin{quotation} The \texttt{flip} function takes a single Edge as a parameter and performs a flip on it. This involves determining the new length of the edge after the flip and changing the topological information in the edge being flipped as well as all of the edge's adjacent simplices. This can be thought of as taking two triangles which share an edge (the parameter to flip) and making two new triangles which share an edge between the two vertices which were previously non-adjacent.\end{quotation}

\subsection*{Subsidiaries}

\begin{quotation} Functions:\end{quotation}
{\small{\begin{verbatim} 
    void flipPP(struct simps b)

    void flipPN(struct simps b)

    void flipNN(struct simps b)

    void topo_flip(Edge, struct simps)

    bool prep_for_flip(Edge, struct simps*)
\end{verbatim}
}}
\begin{quotation} Global Variables:Local Variables:\end{quotation}
{\small{\begin{verbatim} 
    Edge e
\end{verbatim}
}}

\subsection*{Description}

\begin{quotation} flip begins by calling the prep\_for\_flip function, that will setup the struct given to it to contain all the important information necessary for the flip to occur, such as indices for the different simplices and the lengths of the triangles' edges, and the two angles which are not incident on the edge being flipped. The struct looks like:\end{quotation}{\small{\begin{verbatim} 
struct simps {
       int v0, v1, v2, v3, e0, e1, e2, e3, e4, f0, f1;
       double e0_len, e1_len, e2_len, e3_len, e4_len;
       double a0, a2;
};
\end{verbatim}
}}
\begin{quotation} With all this information known, the next step is to determine the type of flip that is to occur. The possibilities are broken up three ways: positive positive (PP), positive negative (PN), negative negative (NN); based on the initial condition of the two triangles. This will determine which of flipPP, flipPN, flipNN is called. Within these function is logic which should compute the new edge length and assign it to the edge e, and determine the positive/negative configuration of the two triangles and assign the appropriate boolean value to each face.\end{quotation}
\begin{quotation} With the new edge length computed and assigned, the topo\_flip function is called which performs the rearrangement of all the adjacencies of the different simplices which are adjacent to edge e.\end{quotation}
\begin{quotation} The edge is returned.\end{quotation}

\subsection*{Practicum}

\begin{quotation} \end{quotation}{\small{\begin{verbatim} 
  Edge e;
  e = Triangulation::edgeTable[indexOfE];
  e = flip(e);
\end{verbatim}
}}
\begin{quotation} one thing to note is that in future implementations the edge being given as the parameter may be different than the one returned\end{quotation}

\subsection*{Limitations}

\begin{quotation} The biggest limitation of the flip function is that it currently only works for bistellar flips. If higher dimensional flips are required this function will need to be modified heavily.\end{quotation}

\subsection*{Revisions}

\begin{quotation} ------------------------------------------------------------------------r816 \mbox{$|$} kortox \mbox{$|$} 2009-06-29 12:41:33 -0700 (Mon, 29 Jun 2009) \mbox{$|$} 1 line\end{quotation}
\begin{quotation} have all the new\_flip stuff up to date and working with the new geometry classes\end{quotation}
\begin{quotation} ------------------------------------------------------------------------r795 \mbox{$|$} kortox \mbox{$|$} 2009-06-18 17:58:30 -0700 (Thu, 18 Jun 2009) \mbox{$|$} 5 lines\end{quotation}
\begin{quotation} anyway, this is a project for devopment of the flip algorithm, so far it contains a new flip function which is intended to replace the flip function that was previously in Triangulation/triangulationmorphs.cpp\end{quotation}
\begin{quotation} main currently contains some test functions that can be called one at a time manually and should produce output that can indicate how the flip function is performing, this testing really needs to be improved\end{quotation}

\subsection*{Testing}

\begin{quotation} Initially testing was done inefficiently by manually analyzing what was written by the writeTriangulationFile  function. Now that we have a way to display the triangulation, we can select an edge and flip it in the display and see that the flip occurred correctly. Granted this should at sometime in the future be automated, but for now if there is an issue we can try to debug it with the display.\end{quotation}

\subsection*{Future Work}

\begin{quotation} \end{quotation}\begin{itemize}\item  Adding the ability to flip in higher dimensions. This would involve altering the function to take a Simplex object instead of an edge so that it is more general.
\end{itemize}
\begin{quotation} \end{quotation}\begin{itemize}\item  We'll most likely want to have the function add an edge to the triangulation instead of just reposition the edge given, since this will lend itself better to the possible addition of 3-1 flips. Related to this would also be changing the return type to be a vector of Simplex objects for generality's sake.
\end{itemize}
\begin{quotation} \end{quotation}\begin{itemize}\item  Moving the whole thing to a different file with an appropriate name other than new\_flip
\end{itemize}
    
% html: End of file: `clean.html'

%%%%% END OF DOCUMENT BODY %%%%%
% In the future, we might want to put some additional data here, such
% as when the documentation was converted from wiki to TeX.
%

\end{document}
}}%
%BeginExpansion
%html2tex: Version  2.7 of June 17, 2008.
%Written by  F.J. Faase.  http://www.iwriteiam.nl/

\documentclass[10pt]{article}%
\usepackage{amssymb}
\usepackage{geometry}
\usepackage{indentfirst}
\usepackage{amsmath}
\usepackage{amsfonts}
\usepackage{graphicx}%
\setcounter{MaxMatrixCols}{30}
%TCIDATA{OutputFilter=latex2.dll}
%TCIDATA{Version=5.00.0.2606}
%TCIDATA{CSTFile=40 LaTeX article.cst}
%TCIDATA{Created=Friday, March 30, 2007 00:21:27}
%TCIDATA{LastRevised=Wednesday, June 10, 2009 11:42:33}
%TCIDATA{<META NAME="GraphicsSave" CONTENT="32">}
%TCIDATA{<META NAME="SaveForMode" CONTENT="1">}
%TCIDATA{BibliographyScheme=Manual}
%TCIDATA{<META NAME="DocumentShell" CONTENT="Standard LaTeX\Blank - Standard LaTeX Article">}
%TCIDATA{Language=American English}
\newtheorem{theorem}{Theorem}
\newtheorem{acknowledgement}[theorem]{Acknowledgement}
\newtheorem{algorithm}[theorem]{Algorithm}
\newtheorem{axiom}[theorem]{Axiom}
\newtheorem{case}[theorem]{Case}
\newtheorem{claim}[theorem]{Claim}
\newtheorem{conclusion}[theorem]{Conclusion}
\newtheorem{condition}[theorem]{Condition}
\newtheorem{conjecture}[theorem]{Conjecture}
\newtheorem{corollary}[theorem]{Corollary}
\newtheorem{criterion}[theorem]{Criterion}
\newtheorem{definition}[theorem]{Definition}
\newtheorem{example}[theorem]{Example}
\newtheorem{exercise}[theorem]{Exercise}
\newtheorem{lemma}[theorem]{Lemma}
\newtheorem{notation}[theorem]{Notation}
\newtheorem{problem}[theorem]{Problem}
\newtheorem{proposition}[theorem]{Proposition}
\newtheorem{remark}[theorem]{Remark}
\newtheorem{solution}[theorem]{Solution}
\newtheorem{summary}[theorem]{Summary}
\newenvironment{proof}[1][Proof]{\noindent\textbf{#1.} }{\ \rule{0.5em}{0.5em}}
\geometry{left=1in,right=1in,top=1in,bottom=1in}

\begin{document}

%%%%% BEGINNING OF DOCUMENT BODY %%%%%
% html: Beginning of file: `clean.html'
% DOCTYPE HTML PUBLIC "-//W3C//DTD HTML 4.01//EN"
%  This is a (PRE) block.  Make sure it's left aligned or your toc title will be off. 

\section*{\texttt{flip}}

\label{f0}{\small{\begin{verbatim} 
Edge flip(Edge e)
\end{verbatim}
}}

\subsection*{Keywords}

\begin{quotation} flip, delaunay\end{quotation}

\subsection*{Authors}

\begin{quotation} Kurt Norwood\end{quotation}

\subsection*{Introduction}

\begin{quotation} The \texttt{flip} function takes a single Edge as a parameter and performs a flip on it. This involves determining the new length of the edge after the flip and changing the topological information in the edge being flipped as well as all of the edge's adjacent simplices. This can be thought of as taking two triangles which share an edge (the parameter to flip) and making two new triangles which share an edge between the two vertices which were previously non-adjacent.\end{quotation}

\subsection*{Subsidiaries}

\begin{quotation} Functions:\end{quotation}
{\small{\begin{verbatim} 
    void flipPP(struct simps b)

    void flipPN(struct simps b)

    void flipNN(struct simps b)

    void topo_flip(Edge, struct simps)

    bool prep_for_flip(Edge, struct simps*)
\end{verbatim}
}}
\begin{quotation} Global Variables:Local Variables:\end{quotation}
{\small{\begin{verbatim} 
    Edge e
\end{verbatim}
}}

\subsection*{Description}

\begin{quotation} flip begins by calling the prep\_for\_flip function, that will setup the struct given to it to contain all the important information necessary for the flip to occur, such as indices for the different simplices and the lengths of the triangles' edges, and the two angles which are not incident on the edge being flipped. The struct looks like:\end{quotation}{\small{\begin{verbatim} 
struct simps {
       int v0, v1, v2, v3, e0, e1, e2, e3, e4, f0, f1;
       double e0_len, e1_len, e2_len, e3_len, e4_len;
       double a0, a2;
};
\end{verbatim}
}}
\begin{quotation} With all this information known, the next step is to determine the type of flip that is to occur. The possibilities are broken up three ways: positive positive (PP), positive negative (PN), negative negative (NN); based on the initial condition of the two triangles. This will determine which of flipPP, flipPN, flipNN is called. Within these function is logic which should compute the new edge length and assign it to the edge e, and determine the positive/negative configuration of the two triangles and assign the appropriate boolean value to each face.\end{quotation}
\begin{quotation} With the new edge length computed and assigned, the topo\_flip function is called which performs the rearrangement of all the adjacencies of the different simplices which are adjacent to edge e.\end{quotation}
\begin{quotation} The edge is returned.\end{quotation}

\subsection*{Practicum}

\begin{quotation} \end{quotation}{\small{\begin{verbatim} 
  Edge e;
  e = Triangulation::edgeTable[indexOfE];
  e = flip(e);
\end{verbatim}
}}
\begin{quotation} one thing to note is that in future implementations the edge being given as the parameter may be different than the one returned\end{quotation}

\subsection*{Limitations}

\begin{quotation} The biggest limitation of the flip function is that it currently only works for bistellar flips. If higher dimensional flips are required this function will need to be modified heavily.\end{quotation}

\subsection*{Revisions}

\begin{quotation} ------------------------------------------------------------------------r816 \mbox{$|$} kortox \mbox{$|$} 2009-06-29 12:41:33 -0700 (Mon, 29 Jun 2009) \mbox{$|$} 1 line\end{quotation}
\begin{quotation} have all the new\_flip stuff up to date and working with the new geometry classes\end{quotation}
\begin{quotation} ------------------------------------------------------------------------r795 \mbox{$|$} kortox \mbox{$|$} 2009-06-18 17:58:30 -0700 (Thu, 18 Jun 2009) \mbox{$|$} 5 lines\end{quotation}
\begin{quotation} anyway, this is a project for devopment of the flip algorithm, so far it contains a new flip function which is intended to replace the flip function that was previously in Triangulation/triangulationmorphs.cpp\end{quotation}
\begin{quotation} main currently contains some test functions that can be called one at a time manually and should produce output that can indicate how the flip function is performing, this testing really needs to be improved\end{quotation}

\subsection*{Testing}

\begin{quotation} Initially testing was done inefficiently by manually analyzing what was written by the writeTriangulationFile  function. Now that we have a way to display the triangulation, we can select an edge and flip it in the display and see that the flip occurred correctly. Granted this should at sometime in the future be automated, but for now if there is an issue we can try to debug it with the display.\end{quotation}

\subsection*{Future Work}

\begin{quotation} \end{quotation}\begin{itemize}\item  Adding the ability to flip in higher dimensions. This would involve altering the function to take a Simplex object instead of an edge so that it is more general.
\end{itemize}
\begin{quotation} \end{quotation}\begin{itemize}\item  We'll most likely want to have the function add an edge to the triangulation instead of just reposition the edge given, since this will lend itself better to the possible addition of 3-1 flips. Related to this would also be changing the return type to be a vector of Simplex objects for generality's sake.
\end{itemize}
\begin{quotation} \end{quotation}\begin{itemize}\item  Moving the whole thing to a different file with an appropriate name other than new\_flip
\end{itemize}
    
% html: End of file: `clean.html'

%%%%% END OF DOCUMENT BODY %%%%%
% In the future, we might want to put some additional data here, such
% as when the documentation was converted from wiki to TeX.
%

\end{document}
%
%EndExpansion

\bigskip 

\bigskip 

%TCIMACRO{\QSubDoc{Include GeoquantAt}{%html2tex: Version  2.7 of June 17, 2008.
%Written by  F.J. Faase.  http://www.iwriteiam.nl/

\documentclass[10pt]{article}%
\usepackage{amssymb}
\usepackage{geometry}
\usepackage{indentfirst}
\usepackage{amsmath}
\usepackage{amsfonts}
\usepackage{graphicx}%
\setcounter{MaxMatrixCols}{30}
%TCIDATA{OutputFilter=latex2.dll}
%TCIDATA{Version=5.00.0.2606}
%TCIDATA{CSTFile=40 LaTeX article.cst}
%TCIDATA{Created=Friday, March 30, 2007 00:21:27}
%TCIDATA{LastRevised=Wednesday, June 10, 2009 11:42:33}
%TCIDATA{<META NAME="GraphicsSave" CONTENT="32">}
%TCIDATA{<META NAME="SaveForMode" CONTENT="1">}
%TCIDATA{BibliographyScheme=Manual}
%TCIDATA{<META NAME="DocumentShell" CONTENT="Standard LaTeX\Blank - Standard LaTeX Article">}
%TCIDATA{Language=American English}
\newtheorem{theorem}{Theorem}
\newtheorem{acknowledgement}[theorem]{Acknowledgement}
\newtheorem{algorithm}[theorem]{Algorithm}
\newtheorem{axiom}[theorem]{Axiom}
\newtheorem{case}[theorem]{Case}
\newtheorem{claim}[theorem]{Claim}
\newtheorem{conclusion}[theorem]{Conclusion}
\newtheorem{condition}[theorem]{Condition}
\newtheorem{conjecture}[theorem]{Conjecture}
\newtheorem{corollary}[theorem]{Corollary}
\newtheorem{criterion}[theorem]{Criterion}
\newtheorem{definition}[theorem]{Definition}
\newtheorem{example}[theorem]{Example}
\newtheorem{exercise}[theorem]{Exercise}
\newtheorem{lemma}[theorem]{Lemma}
\newtheorem{notation}[theorem]{Notation}
\newtheorem{problem}[theorem]{Problem}
\newtheorem{proposition}[theorem]{Proposition}
\newtheorem{remark}[theorem]{Remark}
\newtheorem{solution}[theorem]{Solution}
\newtheorem{summary}[theorem]{Summary}
\newenvironment{proof}[1][Proof]{\noindent\textbf{#1.} }{\ \rule{0.5em}{0.5em}}
\geometry{left=1in,right=1in,top=1in,bottom=1in}

\begin{document}

%%%%% BEGINNING OF DOCUMENT BODY %%%%%
% html: Beginning of file: `clean.html'
% DOCTYPE HTML PUBLIC "-//W3C//DTD HTML 4.01//EN"
%  This is a (PRE) block.  Make sure it's left aligned or your toc title will be off. 

\section*{\texttt{Geoquant::At}}

\label{f0}\begin{quotation} {\small{\begin{verbatim} 
Geoquant*  Geoquant::At(Simplex  s1,  ...)
  \end{verbatim}
}}
\end{quotation}
\subsection*{Key Words}

\begin{quotation} geoquant, recalculate, dependent, triposition, simplex\end{quotation}

\subsection*{Authors}

\begin{itemize}\item  Joseph Thomas
\end{itemize}
\begin{quotation} \end{quotation}
\subsection*{Introduction}

\begin{quotation} The \texttt{At} function is defined for every type of geoquant as a way to retrieve that quantity. Once the quantity is retrieved, a value can be set or asked of the quantity. A quantity is retrieved by providing a list of simplices that describe the position of the quantity in the triangulation.\end{quotation}

\subsection*{Subsidiaries}

\begin{quotation} Functions: \end{quotation}
\begin{itemize}
\item  getSerialNumber
\end{itemize}
\begin{quotation} Global Variables: mapLocal Variables: possible list of simplices\end{quotation}

\subsection*{Description}

\begin{quotation} The \texttt{At} function is a little di\"{\i}\&not;erent for every type of geoquant, but in all cases it is a static function for that class that serves as an object retrieval in place of a constructor. The function takes as a parameter a list of simplices which may be di\"{\i}\&not;erent for each type of geoquant. The list is the natural description of where the quantity is in the triangulation. For example, a radius is described by a vertex, whereas an angle is described as a vertex on a certain face. The At function returns a pointer to the requested quantity.\end{quotation}
\begin{quotation} When the \texttt{At} function is called, it searches a local map for a quantity with the given list of simplices.  If it is found, a pointer to that quantity in the map is simply returned.  If it is not found, the quantity is constructed and placed into the map.  If the construction of the object requires other types of quantities not yet created, then these will be constructed automatically at this time. Lastly, this quantity is returned.\end{quotation}
\begin{quotation} The constructor is hidden from the user for several reasons. The \"{\i}\&not;rst is that this avoids redundant construction and the need for an encapsulating object to hold a large set of geoquants (like the Geometry class in a previous version).  In the same vein, the need for an initial build step and a required order of construction is removed. In addition, this is an e\"{\i}\&not;ciency improvement as quantities that are never requested are never created, decreasing memory use and large dependency trees which can take a while for an \texttt{invalidate} to traverse.\end{quotation}

\subsection*{Practicum}

\begin{quotation} Example:{\small{\begin{verbatim} 
//  Get  the  Radius  quantity  from  the  first  vertex  in  the  triangulation.
Radius  *r  =  Radius::At(Triangulation::vertexTable[0]);
//  Get  the  angle  of  vertex  v  incident  on  face  f
Vertex  v;
Face  f;
...
EuclideanAngle  *ang  =  EuclideanAngle::At(v,  f);
  \end{verbatim}
}}
\end{quotation}
\subsection*{Limitations}

\begin{quotation} The \texttt{At} function is limited in that a speci\"{\i}\&not;c set of simplices will always return the exact same object.  While this is in fact the design goal, this can limit one\^as ability to modify an object as a change in one place will a\"{\i}\&not;ect its use elsewhere in the code. The function also will require the user to handle pointers, a powerful yet fragile and sometimes daunting aspect of the programming language.\end{quotation}

\subsection*{Revisions}

\begin{itemize}\item  subversion 761, 6/12/09: A working copy of \texttt{At} and the Geoquant system.
\end{itemize}

\subsection*{Testing}

\begin{quotation} The \texttt{At} function was tested in small modularized systems, then tested in a three dimensional \"{\i}\&not;ow, which required many varied uses of \texttt{At}.   Some retrieved quantities had their values set while others had their values accessed and compared with what mathematica calculations predicted.\end{quotation}

\subsection*{Future Work}

\begin{quotation} No future work is planned at this time.\end{quotation}
    
% html: End of file: `clean.html'

%%%%% END OF DOCUMENT BODY %%%%%
% In the future, we might want to put some additional data here, such
% as when the documentation was converted from wiki to TeX.
%

\end{document}
}}%
%BeginExpansion
%html2tex: Version  2.7 of June 17, 2008.
%Written by  F.J. Faase.  http://www.iwriteiam.nl/

\documentclass[10pt]{article}%
\usepackage{amssymb}
\usepackage{geometry}
\usepackage{indentfirst}
\usepackage{amsmath}
\usepackage{amsfonts}
\usepackage{graphicx}%
\setcounter{MaxMatrixCols}{30}
%TCIDATA{OutputFilter=latex2.dll}
%TCIDATA{Version=5.00.0.2606}
%TCIDATA{CSTFile=40 LaTeX article.cst}
%TCIDATA{Created=Friday, March 30, 2007 00:21:27}
%TCIDATA{LastRevised=Wednesday, June 10, 2009 11:42:33}
%TCIDATA{<META NAME="GraphicsSave" CONTENT="32">}
%TCIDATA{<META NAME="SaveForMode" CONTENT="1">}
%TCIDATA{BibliographyScheme=Manual}
%TCIDATA{<META NAME="DocumentShell" CONTENT="Standard LaTeX\Blank - Standard LaTeX Article">}
%TCIDATA{Language=American English}
\newtheorem{theorem}{Theorem}
\newtheorem{acknowledgement}[theorem]{Acknowledgement}
\newtheorem{algorithm}[theorem]{Algorithm}
\newtheorem{axiom}[theorem]{Axiom}
\newtheorem{case}[theorem]{Case}
\newtheorem{claim}[theorem]{Claim}
\newtheorem{conclusion}[theorem]{Conclusion}
\newtheorem{condition}[theorem]{Condition}
\newtheorem{conjecture}[theorem]{Conjecture}
\newtheorem{corollary}[theorem]{Corollary}
\newtheorem{criterion}[theorem]{Criterion}
\newtheorem{definition}[theorem]{Definition}
\newtheorem{example}[theorem]{Example}
\newtheorem{exercise}[theorem]{Exercise}
\newtheorem{lemma}[theorem]{Lemma}
\newtheorem{notation}[theorem]{Notation}
\newtheorem{problem}[theorem]{Problem}
\newtheorem{proposition}[theorem]{Proposition}
\newtheorem{remark}[theorem]{Remark}
\newtheorem{solution}[theorem]{Solution}
\newtheorem{summary}[theorem]{Summary}
\newenvironment{proof}[1][Proof]{\noindent\textbf{#1.} }{\ \rule{0.5em}{0.5em}}
\geometry{left=1in,right=1in,top=1in,bottom=1in}

\begin{document}

%%%%% BEGINNING OF DOCUMENT BODY %%%%%
% html: Beginning of file: `clean.html'
% DOCTYPE HTML PUBLIC "-//W3C//DTD HTML 4.01//EN"
%  This is a (PRE) block.  Make sure it's left aligned or your toc title will be off. 

\section*{\texttt{Geoquant::At}}

\label{f0}\begin{quotation} {\small{\begin{verbatim} 
Geoquant*  Geoquant::At(Simplex  s1,  ...)
  \end{verbatim}
}}
\end{quotation}
\subsection*{Key Words}

\begin{quotation} geoquant, recalculate, dependent, triposition, simplex\end{quotation}

\subsection*{Authors}

\begin{itemize}\item  Joseph Thomas
\end{itemize}
\begin{quotation} \end{quotation}
\subsection*{Introduction}

\begin{quotation} The \texttt{At} function is defined for every type of geoquant as a way to retrieve that quantity. Once the quantity is retrieved, a value can be set or asked of the quantity. A quantity is retrieved by providing a list of simplices that describe the position of the quantity in the triangulation.\end{quotation}

\subsection*{Subsidiaries}

\begin{quotation} Functions: \end{quotation}
\begin{itemize}
\item  getSerialNumber
\end{itemize}
\begin{quotation} Global Variables: mapLocal Variables: possible list of simplices\end{quotation}

\subsection*{Description}

\begin{quotation} The \texttt{At} function is a little di\"{\i}\&not;erent for every type of geoquant, but in all cases it is a static function for that class that serves as an object retrieval in place of a constructor. The function takes as a parameter a list of simplices which may be di\"{\i}\&not;erent for each type of geoquant. The list is the natural description of where the quantity is in the triangulation. For example, a radius is described by a vertex, whereas an angle is described as a vertex on a certain face. The At function returns a pointer to the requested quantity.\end{quotation}
\begin{quotation} When the \texttt{At} function is called, it searches a local map for a quantity with the given list of simplices.  If it is found, a pointer to that quantity in the map is simply returned.  If it is not found, the quantity is constructed and placed into the map.  If the construction of the object requires other types of quantities not yet created, then these will be constructed automatically at this time. Lastly, this quantity is returned.\end{quotation}
\begin{quotation} The constructor is hidden from the user for several reasons. The \"{\i}\&not;rst is that this avoids redundant construction and the need for an encapsulating object to hold a large set of geoquants (like the Geometry class in a previous version).  In the same vein, the need for an initial build step and a required order of construction is removed. In addition, this is an e\"{\i}\&not;ciency improvement as quantities that are never requested are never created, decreasing memory use and large dependency trees which can take a while for an \texttt{invalidate} to traverse.\end{quotation}

\subsection*{Practicum}

\begin{quotation} Example:{\small{\begin{verbatim} 
//  Get  the  Radius  quantity  from  the  first  vertex  in  the  triangulation.
Radius  *r  =  Radius::At(Triangulation::vertexTable[0]);
//  Get  the  angle  of  vertex  v  incident  on  face  f
Vertex  v;
Face  f;
...
EuclideanAngle  *ang  =  EuclideanAngle::At(v,  f);
  \end{verbatim}
}}
\end{quotation}
\subsection*{Limitations}

\begin{quotation} The \texttt{At} function is limited in that a speci\"{\i}\&not;c set of simplices will always return the exact same object.  While this is in fact the design goal, this can limit one\^as ability to modify an object as a change in one place will a\"{\i}\&not;ect its use elsewhere in the code. The function also will require the user to handle pointers, a powerful yet fragile and sometimes daunting aspect of the programming language.\end{quotation}

\subsection*{Revisions}

\begin{itemize}\item  subversion 761, 6/12/09: A working copy of \texttt{At} and the Geoquant system.
\end{itemize}

\subsection*{Testing}

\begin{quotation} The \texttt{At} function was tested in small modularized systems, then tested in a three dimensional \"{\i}\&not;ow, which required many varied uses of \texttt{At}.   Some retrieved quantities had their values set while others had their values accessed and compared with what mathematica calculations predicted.\end{quotation}

\subsection*{Future Work}

\begin{quotation} No future work is planned at this time.\end{quotation}
    
% html: End of file: `clean.html'

%%%%% END OF DOCUMENT BODY %%%%%
% In the future, we might want to put some additional data here, such
% as when the documentation was converted from wiki to TeX.
%

\end{document}
%
%EndExpansion

\bigskip 

\bigskip 

%TCIMACRO{\QSubDoc{Include hij_k}{%TCIDATA{LaTeXparent=0,0,functions.tex}}}%
%BeginExpansion
%TCIDATA{LaTeXparent=0,0,functions.tex}%
%EndExpansion

\bigskip

\bigskip

%TCIMACRO{\QSubDoc{Include hijk_l}{%TCIDATA{Version=5.00.0.2606}
%TCIDATA{LaTeXparent=1,1,functions.tex}
                      

\section*{\texttt{FaceHeight::FaceHeight\label{Face Height FUNCTION}}}

\subsection*{Function Prototype}

\texttt{double FaceHeight( Vertex vi, Vertex vj, Vertex vk, Vertex vl)}

\subsection*{Key Words}

Face height, edge height, geoquant.

\subsection*{Authors}

Daniel Champion

\subsection*{Introduction}

The function \texttt{FaceHeight}\ calculates the face height to the center
of a tetrahedron. \ 

\subsection*{Subsidiaries}

\textbf{Functions:}

\qquad\texttt{Geometry::dihedralAngle}

\qquad \texttt{EdgeHeight}

\qquad\texttt{listIntersection}

\textbf{Global Variables: }\ radii, etas.

\textbf{Local Variables:} \ Vertex vi, vj, vk, vl.

\subsection*{Description}

The calculation of \texttt{FaceHeight} involves the simple formula:%
\begin{equation*}
\text{\texttt{FaceHeight(vi, vj, vk, vl)}}=
\end{equation*}

\begin{equation*}
\frac{\left( \text{\texttt{EdgeHeight(vi,vj,vl)}}-\text{\texttt{%
EdgeHeight(vi,vj,vk)}}\cos (\beta _{ij,kl})\right) }{\sin \left( \beta
_{ij,kl}\right) }
\end{equation*}%
where $\beta _{ij,kl}$ is the dihedral angle along edge $\left\{
vi,vj\right\} $ of tetrahedron $\left\{ vi,vj,vk,vl\right\} $. \ A geometric
interpretation of this quantity is a follows. \ Given a decorated
tetrahedron (tetrahedron with radii and eta values assigned to the vertices
and edges respectively), the center of this tetrahedron can be calculated as
the common power point of its embedding into three-dimensional Euclidean
space. \ The perpendicular distance from this center point to the face $%
\left\{ vi,vj,vk\right\} $ is exactly \texttt{FaceHeight (vi, vj, vk, vl)}.
\ Take note that the first three vertices in the function call correspond to
the preferred face, and the fourth vertex in the function call identifies
the tetrahedron. \ 

A primary use of this function is in the calculation of several quantities
needed for the \texttt{CurvaturePartial} function used in the optimization
of the normalized Einstein-Hilbert-Regge functional.

\subsection*{Practicum}

An example of the use of this function is in the calculation of the dual
areas, \texttt{DualAreaSegment}, to an edge of a three dimensional
triangulation.

\qquad \texttt{double DualAreaSegment( Vertex vi, Vertex vj, Vertex vk,
Vertex vl)}

\qquad\qquad\texttt{\{}

\qquad \qquad \texttt{double result =
0.5*(EdgeHeight(vi,vj,vk)*FaceHeight(vi,vj,vk,vl)}

\qquad \qquad \qquad \texttt{+EdgeHeight(vi,vj,vl)*FaceHeight(vi,vj,vl,vk));}

\qquad\qquad\texttt{return result;}

\qquad\qquad\texttt{\}}

\subsection*{Limitations}

\texttt{faceHeight} must receive as input four vertices of a tetrahedron of
the triangulation. \ Moreover, the first three vertices in the function call
identify a face and can be in any order, however the fourth vertex in the
function call identifies the tetrahedron and can not be permuted with the
other three vertices. \ 

\subsection*{Revisions}

subversion 757, 6/8/09, \texttt{FaceHeight} created.

subversion 1055, 3/12/10, \texttt{FaceHeight}\ converted to a geoquant.

\subsection*{Testing}

This function was not tested.

\subsection*{Future Work}

This function has been incorporated into the Geometry class geoquants, and
thus this entry needs to be updated. \ 
}}%
%BeginExpansion
%TCIDATA{Version=5.00.0.2606}
%TCIDATA{LaTeXparent=1,1,functions.tex}
                      

\section*{\texttt{FaceHeight::FaceHeight\label{Face Height FUNCTION}}}

\subsection*{Function Prototype}

\texttt{double FaceHeight( Vertex vi, Vertex vj, Vertex vk, Vertex vl)}

\subsection*{Key Words}

Face height, edge height, geoquant.

\subsection*{Authors}

Daniel Champion

\subsection*{Introduction}

The function \texttt{FaceHeight}\ calculates the face height to the center
of a tetrahedron. \ 

\subsection*{Subsidiaries}

\textbf{Functions:}

\qquad\texttt{Geometry::dihedralAngle}

\qquad \texttt{EdgeHeight}

\qquad\texttt{listIntersection}

\textbf{Global Variables: }\ radii, etas.

\textbf{Local Variables:} \ Vertex vi, vj, vk, vl.

\subsection*{Description}

The calculation of \texttt{FaceHeight} involves the simple formula:%
\begin{equation*}
\text{\texttt{FaceHeight(vi, vj, vk, vl)}}=
\end{equation*}

\begin{equation*}
\frac{\left( \text{\texttt{EdgeHeight(vi,vj,vl)}}-\text{\texttt{%
EdgeHeight(vi,vj,vk)}}\cos (\beta _{ij,kl})\right) }{\sin \left( \beta
_{ij,kl}\right) }
\end{equation*}%
where $\beta _{ij,kl}$ is the dihedral angle along edge $\left\{
vi,vj\right\} $ of tetrahedron $\left\{ vi,vj,vk,vl\right\} $. \ A geometric
interpretation of this quantity is a follows. \ Given a decorated
tetrahedron (tetrahedron with radii and eta values assigned to the vertices
and edges respectively), the center of this tetrahedron can be calculated as
the common power point of its embedding into three-dimensional Euclidean
space. \ The perpendicular distance from this center point to the face $%
\left\{ vi,vj,vk\right\} $ is exactly \texttt{FaceHeight (vi, vj, vk, vl)}.
\ Take note that the first three vertices in the function call correspond to
the preferred face, and the fourth vertex in the function call identifies
the tetrahedron. \ 

A primary use of this function is in the calculation of several quantities
needed for the \texttt{CurvaturePartial} function used in the optimization
of the normalized Einstein-Hilbert-Regge functional.

\subsection*{Practicum}

An example of the use of this function is in the calculation of the dual
areas, \texttt{DualAreaSegment}, to an edge of a three dimensional
triangulation.

\qquad \texttt{double DualAreaSegment( Vertex vi, Vertex vj, Vertex vk,
Vertex vl)}

\qquad\qquad\texttt{\{}

\qquad \qquad \texttt{double result =
0.5*(EdgeHeight(vi,vj,vk)*FaceHeight(vi,vj,vk,vl)}

\qquad \qquad \qquad \texttt{+EdgeHeight(vi,vj,vl)*FaceHeight(vi,vj,vl,vk));}

\qquad\qquad\texttt{return result;}

\qquad\qquad\texttt{\}}

\subsection*{Limitations}

\texttt{faceHeight} must receive as input four vertices of a tetrahedron of
the triangulation. \ Moreover, the first three vertices in the function call
identify a face and can be in any order, however the fourth vertex in the
function call identifies the tetrahedron and can not be permuted with the
other three vertices. \ 

\subsection*{Revisions}

subversion 757, 6/8/09, \texttt{FaceHeight} created.

subversion 1055, 3/12/10, \texttt{FaceHeight}\ converted to a geoquant.

\subsection*{Testing}

This function was not tested.

\subsection*{Future Work}

This function has been incorporated into the Geometry class geoquants, and
thus this entry needs to be updated. \ 
%
%EndExpansion

\bigskip

\bigskip

%TCIMACRO{\QSubDoc{Include Lij_star}{%TCIDATA{Version=5.00.0.2606}
%TCIDATA{LaTeXparent=1,1,functions.tex}
                      

\section*{\texttt{DualArea::DualArea}}

\subsection*{Function Prototype}

\texttt{double DualArea(Edge e)}

\subsection*{Key Words}

Dual area, curvature, partial derivative, edge, geoquant.

\subsection*{Authors}

Daniel Champion

\subsection*{Introduction}

\texttt{DualArea} calculates the dual area of an edge.

\subsection*{Subsidiaries}

\subsubsection*{Functions: \ }

\qquad \texttt{DualAreaSegment}

\qquad \qquad \texttt{FaceHeight}

\qquad \qquad \qquad \texttt{EdgeHeight}

\qquad \qquad \qquad \qquad \texttt{PartialEdge}

\subsubsection*{Global Variables: \ }

\qquad Only radii and eta values are needed.

\subsubsection*{Local Variables: \ }

\qquad none

\subsection*{Description}

\texttt{DualArea} is defined as:%
\begin{equation*}
\text{\texttt{DualArea(edge e\_ij))}}=l_{ij}^{\ast }=\sum_{\substack{ \text{%
all tetrahedra }(i,j,k,l) \\ \text{containing edge }(i,j)\text{.}}}A_{ij,kl},
\end{equation*}%
where $A_{ij,kl}$ is computed with the function \texttt{DualAreaSegment}
applied to the vertices of the tetrahedron being summed over. \ Note that 
\texttt{DualAreaSegment} utilizes the functions \texttt{FaceHeight}, \texttt{%
EdgeHeight}, and \texttt{partialEdge}, however only the radii and eta values
are needed to calculate all of these quantities. \ 

This function was created for use in the \texttt{CurvaturePartial} function
which serves an essential role in calculating the second derivatives of the
Einstein-Hilbert-Regge functional (\texttt{EHRSecondPartial}). \ The second
order partial derivatives of the EHR functional are used in the optimization
of the EHR functional using Newton's method. \ \texttt{DualArea} will
eventually be used in the study of laplacians. \ 

\subsection*{Practicum}

Currently \texttt{DualArea} is only used to calculate the partial derivative
of curvature with respect to $\log $ radius. \ The following example
calculates the partial derivative of the curvature at vertex V with respect
to the $\log $ radius $r_{l}$ corresponding to vertex Vprime (adjacent to V).

\bigskip

\qquad\texttt{double sum = 0.0;}

\qquad\texttt{double dihedral\_sum = 0.0;}

\qquad\texttt{Vprime = Triangulation::vertexTable[l];}

\qquad\texttt{E =
Triangulation::edgeTable[listIntersection(V.getLocalEdges(),}

\qquad\qquad\texttt{Vprime.getLocalEdges())[0]];}

\qquad\texttt{// This assumes that there is a unique edge between two
vertices.}

\qquad\texttt{local\_tetra = E.getLocalTetras();}

\qquad\texttt{for (int m=0; m \TEXTsymbol{<} (*(local\_tetra)).size(); ++m)
\{}

\qquad\qquad\texttt{T = Triangulation::tetraTable[local\_tetra-\TEXTsymbol{>}%
at(m)];}

\qquad\qquad\texttt{dihedral\_sum += Geometry::dihedralAngle(E,T);}

\qquad\texttt{\}}

\qquad \texttt{result =
DualArea(E)/(Geometry::Length(E))-(2*PI-dihedral\_sum)}

\qquad \qquad \texttt{*(pow(Geometry::Radius(V),
2)*pow(Geometry::Radius(Vprime),2)}

\qquad \qquad \texttt{%
*(1-pow(Geometry::Eta(E),2)))/pow(Geometry::Length(E),3);}

\subsection*{Limitations}

\texttt{DualArea} can operate on any and all edges of a 3D triangulation
however it is only appropriate for triangulations where tetrahedra have
distinct edges. \ 

\subsection*{Revisions}

Subversion 676, 5/15/09, \texttt{DualArea} created within \texttt{%
Newtons\_Method}.

subversion 1055, 3/12/10, \texttt{DualArea}\ converted to a geoquant.

\subsection*{Testing}

\texttt{Lij\_star} has not been tested. \ 

\subsection*{Future Work}

\texttt{Lij\_star} should be moved to a more appropriate section of the
code. \ A more general volume function should be created that would take any
simplex object (including a boolean for dual simplices) and return the
appropriate volume. \ This general volume function would be an excellent
location for \texttt{Lij\_star}. \ It should be tested some time as well. \ 
}}%
%BeginExpansion
%TCIDATA{Version=5.00.0.2606}
%TCIDATA{LaTeXparent=1,1,functions.tex}
                      

\section*{\texttt{DualArea::DualArea}}

\subsection*{Function Prototype}

\texttt{double DualArea(Edge e)}

\subsection*{Key Words}

Dual area, curvature, partial derivative, edge, geoquant.

\subsection*{Authors}

Daniel Champion

\subsection*{Introduction}

\texttt{DualArea} calculates the dual area of an edge.

\subsection*{Subsidiaries}

\subsubsection*{Functions: \ }

\qquad \texttt{DualAreaSegment}

\qquad \qquad \texttt{FaceHeight}

\qquad \qquad \qquad \texttt{EdgeHeight}

\qquad \qquad \qquad \qquad \texttt{PartialEdge}

\subsubsection*{Global Variables: \ }

\qquad Only radii and eta values are needed.

\subsubsection*{Local Variables: \ }

\qquad none

\subsection*{Description}

\texttt{DualArea} is defined as:%
\begin{equation*}
\text{\texttt{DualArea(edge e\_ij))}}=l_{ij}^{\ast }=\sum_{\substack{ \text{%
all tetrahedra }(i,j,k,l) \\ \text{containing edge }(i,j)\text{.}}}A_{ij,kl},
\end{equation*}%
where $A_{ij,kl}$ is computed with the function \texttt{DualAreaSegment}
applied to the vertices of the tetrahedron being summed over. \ Note that 
\texttt{DualAreaSegment} utilizes the functions \texttt{FaceHeight}, \texttt{%
EdgeHeight}, and \texttt{partialEdge}, however only the radii and eta values
are needed to calculate all of these quantities. \ 

This function was created for use in the \texttt{CurvaturePartial} function
which serves an essential role in calculating the second derivatives of the
Einstein-Hilbert-Regge functional (\texttt{EHRSecondPartial}). \ The second
order partial derivatives of the EHR functional are used in the optimization
of the EHR functional using Newton's method. \ \texttt{DualArea} will
eventually be used in the study of laplacians. \ 

\subsection*{Practicum}

Currently \texttt{DualArea} is only used to calculate the partial derivative
of curvature with respect to $\log $ radius. \ The following example
calculates the partial derivative of the curvature at vertex V with respect
to the $\log $ radius $r_{l}$ corresponding to vertex Vprime (adjacent to V).

\bigskip

\qquad\texttt{double sum = 0.0;}

\qquad\texttt{double dihedral\_sum = 0.0;}

\qquad\texttt{Vprime = Triangulation::vertexTable[l];}

\qquad\texttt{E =
Triangulation::edgeTable[listIntersection(V.getLocalEdges(),}

\qquad\qquad\texttt{Vprime.getLocalEdges())[0]];}

\qquad\texttt{// This assumes that there is a unique edge between two
vertices.}

\qquad\texttt{local\_tetra = E.getLocalTetras();}

\qquad\texttt{for (int m=0; m \TEXTsymbol{<} (*(local\_tetra)).size(); ++m)
\{}

\qquad\qquad\texttt{T = Triangulation::tetraTable[local\_tetra-\TEXTsymbol{>}%
at(m)];}

\qquad\qquad\texttt{dihedral\_sum += Geometry::dihedralAngle(E,T);}

\qquad\texttt{\}}

\qquad \texttt{result =
DualArea(E)/(Geometry::Length(E))-(2*PI-dihedral\_sum)}

\qquad \qquad \texttt{*(pow(Geometry::Radius(V),
2)*pow(Geometry::Radius(Vprime),2)}

\qquad \qquad \texttt{%
*(1-pow(Geometry::Eta(E),2)))/pow(Geometry::Length(E),3);}

\subsection*{Limitations}

\texttt{DualArea} can operate on any and all edges of a 3D triangulation
however it is only appropriate for triangulations where tetrahedra have
distinct edges. \ 

\subsection*{Revisions}

Subversion 676, 5/15/09, \texttt{DualArea} created within \texttt{%
Newtons\_Method}.

subversion 1055, 3/12/10, \texttt{DualArea}\ converted to a geoquant.

\subsection*{Testing}

\texttt{Lij\_star} has not been tested. \ 

\subsection*{Future Work}

\texttt{Lij\_star} should be moved to a more appropriate section of the
code. \ A more general volume function should be created that would take any
simplex object (including a boolean for dual simplices) and return the
appropriate volume. \ This general volume function would be an excellent
location for \texttt{Lij\_star}. \ It should be tested some time as well. \ 
%
%EndExpansion

\bigskip

\bigskip 

%TCIMACRO{%
%\QSubDoc{Include makeTriangulationFile}{%html2tex: Version  2.7 of June 17, 2008.
%Written by  F.J. Faase.  http://www.iwriteiam.nl/

clean.html (21) : file `make3DTriangulationFile.html' does not exist.
clean.html (93) : file `readTriangulationFile.html' does not exist.
\documentclass[10pt]{article}%
\usepackage{amssymb}
\usepackage{geometry}
\usepackage{indentfirst}
\usepackage{amsmath}
\usepackage{amsfonts}
\usepackage{graphicx}%
\setcounter{MaxMatrixCols}{30}
%TCIDATA{OutputFilter=latex2.dll}
%TCIDATA{Version=5.00.0.2606}
%TCIDATA{CSTFile=40 LaTeX article.cst}
%TCIDATA{Created=Friday, March 30, 2007 00:21:27}
%TCIDATA{LastRevised=Wednesday, June 10, 2009 11:42:33}
%TCIDATA{<META NAME="GraphicsSave" CONTENT="32">}
%TCIDATA{<META NAME="SaveForMode" CONTENT="1">}
%TCIDATA{BibliographyScheme=Manual}
%TCIDATA{<META NAME="DocumentShell" CONTENT="Standard LaTeX\Blank - Standard LaTeX Article">}
%TCIDATA{Language=American English}
\newtheorem{theorem}{Theorem}
\newtheorem{acknowledgement}[theorem]{Acknowledgement}
\newtheorem{algorithm}[theorem]{Algorithm}
\newtheorem{axiom}[theorem]{Axiom}
\newtheorem{case}[theorem]{Case}
\newtheorem{claim}[theorem]{Claim}
\newtheorem{conclusion}[theorem]{Conclusion}
\newtheorem{condition}[theorem]{Condition}
\newtheorem{conjecture}[theorem]{Conjecture}
\newtheorem{corollary}[theorem]{Corollary}
\newtheorem{criterion}[theorem]{Criterion}
\newtheorem{definition}[theorem]{Definition}
\newtheorem{example}[theorem]{Example}
\newtheorem{exercise}[theorem]{Exercise}
\newtheorem{lemma}[theorem]{Lemma}
\newtheorem{notation}[theorem]{Notation}
\newtheorem{problem}[theorem]{Problem}
\newtheorem{proposition}[theorem]{Proposition}
\newtheorem{remark}[theorem]{Remark}
\newtheorem{solution}[theorem]{Solution}
\newtheorem{summary}[theorem]{Summary}
\newenvironment{proof}[1][Proof]{\noindent\textbf{#1.} }{\ \rule{0.5em}{0.5em}}
\geometry{left=1in,right=1in,top=1in,bottom=1in}

\begin{document}

%%%%% BEGINNING OF DOCUMENT BODY %%%%%
% html: Beginning of file: `clean.html'
% DOCTYPE HTML PUBLIC "-//W3C//DTD HTML 4.01//EN"
%  This is a (PRE) block.  Make sure it's left aligned or your toc title will be off. 

\section*{\texttt{makeTriangulationFile}}

\label{f0}{\small{\begin{verbatim} 
void makeTriangulationfile(char* fileIN, char* fileOUT)
\end{verbatim}
}}

\subsection*{Keywords}

\begin{quotation} triangulation, Lutz, simplices\end{quotation}

\subsection*{Authors}

\begin{itemize}\item  Alex Henniges
\item  Mitch Wilson
\end{itemize}

\subsection*{Introduction}

\begin{quotation} The \texttt{makeTriangulationFile} function converts a text file, given by \texttt{fileIn}, in the \texttt{Lutz} format to the standard format, printed to \texttt{fileOUT}. The file in standard format can then be read into the system to build the triangulation.\end{quotation}

\subsection*{Subsidiaries}

\begin{quotation} Functions:\end{quotation}
\begin{itemize}
\item  Pair::positionOf
\item  Pair::contains
\item  Pair::isInTuple
\end{itemize}
\begin{quotation} Global Variables:\end{quotation}
\begin{quotation} Local Variables: \texttt{fileIN}, \texttt{fileOUT}\end{quotation}

\subsection*{Description}

\begin{quotation} This function is used to convert one format to another format that we consider to be the standard for reading in a triangulation. We have dubbed the format we are converting from \texttt{Lutz}. This is based on the source we retrieve this format from, http://www.math.tu-berlin.de/diskregeom/stellar/\footnote{See URL http://www.math.tu-berlin.de/diskregeom/stellar/} . \end{quotation}
\begin{quotation} The \texttt{Lutz} format provides a simpler interface than our standard format, and can therefore allow for a user to create a quick triangulation. The idea is to provide only the index of every vertex on each face of the triangulation. No information about edges or adjacencies need to be given. The file should begin with a ``=\"{} followed by \"{}\mbox{$[$}\"{} and \"{}\mbox{$]$}'''s to contain the triangulation and each face. An example \texttt{Lutz} format for a tetrahedron is given \mbox{$[$}\#Practicum below\mbox{$]$}. \end{quotation}
\begin{quotation} Note that the \texttt{makeTriangulationFile} is used only for two-dimensional triangulations, and that for three-dimensions, one should use make3DTriangulationFile.\end{quotation}

\subsection*{Pracicum}

\begin{quotation} {\small{\begin{verbatim} 
  // Convert the tetrahedron written in Lutz format to a file in standard format.
  makeTriangulationFile("./tetra_lutz.txt", "./tetra_standard.txt");
  
  // Now read in the triangulation from standard format.
  readTriangulationFile("./tetra_standard.txt");
  \end{verbatim}
}}
\end{quotation}\begin{quotation} The \texttt{Lutz} format may look like:{\small{\begin{verbatim} 
=[[1,2,3],[1,2,4],[2,3,4],[1,3,4]]
  \end{verbatim}
}}
\end{quotation}\begin{quotation} The \texttt{makeTriangulationFile} would then create a file with:{\small{\begin{verbatim} 
Vertex: 1
2 3 4 
1 2 4 
1 2 4 
Vertex: 2
1 3 4 
1 3 5 
1 2 3 
Vertex: 3
1 2 4 
2 3 6 
1 3 4 
Vertex: 4
1 2 3 
4 5 6 
2 3 4 
Edge: 1
1 2
2 3 4 5 
1 2 
Edge: 2
1 3
1 3 4 6 
1 4 
Edge: 3
2 3
1 2 5 6 
1 3 
Edge: 4
1 4
1 2 5 6 
2 4 
Edge: 5
2 4
1 3 4 6 
2 3 
Edge: 6
3 4
2 3 4 5 
3 4 
Face: 1
1 2 3 
1 2 3 
2 3 4 
Face: 2
1 2 4 
1 4 5 
1 3 4 
Face: 3
2 3 4 
3 5 6 
1 2 4 
Face: 4
1 3 4 
2 4 6 
1 2 3 
  \end{verbatim}
}}
\end{quotation}
\subsection*{Limitations}

\begin{quotation} The limitation with the \texttt{Lutz} format that prevents it from being considered the standard format is that the user cannot create the most general of triangulations. To be more specific, it is impossible with the \texttt{Lutz} format to specify for there to be two edges with the same vertices.\end{quotation}
\begin{quotation} A limitation of the \texttt{makeTriangulationFile} is that its requirements are unintuitive. There should be no ``='' required, for example. Another limitation is that despite collecting enough information to build the triangulation, the function instead writes this to a file, requiring the user to subsequently call the function readTriangulationFile.\end{quotation}

\subsection*{Revisions}

\begin{itemize}\item  subversion 545, 9/29/08: Added the \texttt{makeTriangulationFile} function.
\end{itemize}

\subsection*{Testing}

\begin{quotation} This function has been tested through frequent use.\end{quotation}

\subsection*{Future Work}

\begin{itemize}\item  7/1 - Improve the format system.
\item  7/1 - Create the triangulation without performing a conversion to another file.
\end{itemize}
    
% html: End of file: `clean.html'

%%%%% END OF DOCUMENT BODY %%%%%
% In the future, we might want to put some additional data here, such
% as when the documentation was converted from wiki to TeX.
%

\end{document}
}}%
%BeginExpansion
%html2tex: Version  2.7 of June 17, 2008.
%Written by  F.J. Faase.  http://www.iwriteiam.nl/

clean.html (21) : file `make3DTriangulationFile.html' does not exist.
clean.html (93) : file `readTriangulationFile.html' does not exist.
\documentclass[10pt]{article}%
\usepackage{amssymb}
\usepackage{geometry}
\usepackage{indentfirst}
\usepackage{amsmath}
\usepackage{amsfonts}
\usepackage{graphicx}%
\setcounter{MaxMatrixCols}{30}
%TCIDATA{OutputFilter=latex2.dll}
%TCIDATA{Version=5.00.0.2606}
%TCIDATA{CSTFile=40 LaTeX article.cst}
%TCIDATA{Created=Friday, March 30, 2007 00:21:27}
%TCIDATA{LastRevised=Wednesday, June 10, 2009 11:42:33}
%TCIDATA{<META NAME="GraphicsSave" CONTENT="32">}
%TCIDATA{<META NAME="SaveForMode" CONTENT="1">}
%TCIDATA{BibliographyScheme=Manual}
%TCIDATA{<META NAME="DocumentShell" CONTENT="Standard LaTeX\Blank - Standard LaTeX Article">}
%TCIDATA{Language=American English}
\newtheorem{theorem}{Theorem}
\newtheorem{acknowledgement}[theorem]{Acknowledgement}
\newtheorem{algorithm}[theorem]{Algorithm}
\newtheorem{axiom}[theorem]{Axiom}
\newtheorem{case}[theorem]{Case}
\newtheorem{claim}[theorem]{Claim}
\newtheorem{conclusion}[theorem]{Conclusion}
\newtheorem{condition}[theorem]{Condition}
\newtheorem{conjecture}[theorem]{Conjecture}
\newtheorem{corollary}[theorem]{Corollary}
\newtheorem{criterion}[theorem]{Criterion}
\newtheorem{definition}[theorem]{Definition}
\newtheorem{example}[theorem]{Example}
\newtheorem{exercise}[theorem]{Exercise}
\newtheorem{lemma}[theorem]{Lemma}
\newtheorem{notation}[theorem]{Notation}
\newtheorem{problem}[theorem]{Problem}
\newtheorem{proposition}[theorem]{Proposition}
\newtheorem{remark}[theorem]{Remark}
\newtheorem{solution}[theorem]{Solution}
\newtheorem{summary}[theorem]{Summary}
\newenvironment{proof}[1][Proof]{\noindent\textbf{#1.} }{\ \rule{0.5em}{0.5em}}
\geometry{left=1in,right=1in,top=1in,bottom=1in}

\begin{document}

%%%%% BEGINNING OF DOCUMENT BODY %%%%%
% html: Beginning of file: `clean.html'
% DOCTYPE HTML PUBLIC "-//W3C//DTD HTML 4.01//EN"
%  This is a (PRE) block.  Make sure it's left aligned or your toc title will be off. 

\section*{\texttt{makeTriangulationFile}}

\label{f0}{\small{\begin{verbatim} 
void makeTriangulationfile(char* fileIN, char* fileOUT)
\end{verbatim}
}}

\subsection*{Keywords}

\begin{quotation} triangulation, Lutz, simplices\end{quotation}

\subsection*{Authors}

\begin{itemize}\item  Alex Henniges
\item  Mitch Wilson
\end{itemize}

\subsection*{Introduction}

\begin{quotation} The \texttt{makeTriangulationFile} function converts a text file, given by \texttt{fileIn}, in the \texttt{Lutz} format to the standard format, printed to \texttt{fileOUT}. The file in standard format can then be read into the system to build the triangulation.\end{quotation}

\subsection*{Subsidiaries}

\begin{quotation} Functions:\end{quotation}
\begin{itemize}
\item  Pair::positionOf
\item  Pair::contains
\item  Pair::isInTuple
\end{itemize}
\begin{quotation} Global Variables:\end{quotation}
\begin{quotation} Local Variables: \texttt{fileIN}, \texttt{fileOUT}\end{quotation}

\subsection*{Description}

\begin{quotation} This function is used to convert one format to another format that we consider to be the standard for reading in a triangulation. We have dubbed the format we are converting from \texttt{Lutz}. This is based on the source we retrieve this format from, http://www.math.tu-berlin.de/diskregeom/stellar/\footnote{See URL http://www.math.tu-berlin.de/diskregeom/stellar/} . \end{quotation}
\begin{quotation} The \texttt{Lutz} format provides a simpler interface than our standard format, and can therefore allow for a user to create a quick triangulation. The idea is to provide only the index of every vertex on each face of the triangulation. No information about edges or adjacencies need to be given. The file should begin with a ``=\"{} followed by \"{}\mbox{$[$}\"{} and \"{}\mbox{$]$}'''s to contain the triangulation and each face. An example \texttt{Lutz} format for a tetrahedron is given \mbox{$[$}\#Practicum below\mbox{$]$}. \end{quotation}
\begin{quotation} Note that the \texttt{makeTriangulationFile} is used only for two-dimensional triangulations, and that for three-dimensions, one should use make3DTriangulationFile.\end{quotation}

\subsection*{Pracicum}

\begin{quotation} {\small{\begin{verbatim} 
  // Convert the tetrahedron written in Lutz format to a file in standard format.
  makeTriangulationFile("./tetra_lutz.txt", "./tetra_standard.txt");
  
  // Now read in the triangulation from standard format.
  readTriangulationFile("./tetra_standard.txt");
  \end{verbatim}
}}
\end{quotation}\begin{quotation} The \texttt{Lutz} format may look like:{\small{\begin{verbatim} 
=[[1,2,3],[1,2,4],[2,3,4],[1,3,4]]
  \end{verbatim}
}}
\end{quotation}\begin{quotation} The \texttt{makeTriangulationFile} would then create a file with:{\small{\begin{verbatim} 
Vertex: 1
2 3 4 
1 2 4 
1 2 4 
Vertex: 2
1 3 4 
1 3 5 
1 2 3 
Vertex: 3
1 2 4 
2 3 6 
1 3 4 
Vertex: 4
1 2 3 
4 5 6 
2 3 4 
Edge: 1
1 2
2 3 4 5 
1 2 
Edge: 2
1 3
1 3 4 6 
1 4 
Edge: 3
2 3
1 2 5 6 
1 3 
Edge: 4
1 4
1 2 5 6 
2 4 
Edge: 5
2 4
1 3 4 6 
2 3 
Edge: 6
3 4
2 3 4 5 
3 4 
Face: 1
1 2 3 
1 2 3 
2 3 4 
Face: 2
1 2 4 
1 4 5 
1 3 4 
Face: 3
2 3 4 
3 5 6 
1 2 4 
Face: 4
1 3 4 
2 4 6 
1 2 3 
  \end{verbatim}
}}
\end{quotation}
\subsection*{Limitations}

\begin{quotation} The limitation with the \texttt{Lutz} format that prevents it from being considered the standard format is that the user cannot create the most general of triangulations. To be more specific, it is impossible with the \texttt{Lutz} format to specify for there to be two edges with the same vertices.\end{quotation}
\begin{quotation} A limitation of the \texttt{makeTriangulationFile} is that its requirements are unintuitive. There should be no ``='' required, for example. Another limitation is that despite collecting enough information to build the triangulation, the function instead writes this to a file, requiring the user to subsequently call the function readTriangulationFile.\end{quotation}

\subsection*{Revisions}

\begin{itemize}\item  subversion 545, 9/29/08: Added the \texttt{makeTriangulationFile} function.
\end{itemize}

\subsection*{Testing}

\begin{quotation} This function has been tested through frequent use.\end{quotation}

\subsection*{Future Work}

\begin{itemize}\item  7/1 - Improve the format system.
\item  7/1 - Create the triangulation without performing a conversion to another file.
\end{itemize}
    
% html: End of file: `clean.html'

%%%%% END OF DOCUMENT BODY %%%%%
% In the future, we might want to put some additional data here, such
% as when the documentation was converted from wiki to TeX.
%

\end{document}
%
%EndExpansion

\bigskip 

\bigskip 

%TCIMACRO{%
%\QSubDoc{Include NewtonsMethodOptimize}{%html2tex: Version  2.7 of June 17, 2008.
%Written by  F.J. Faase.  http://www.iwriteiam.nl/

\documentclass[10pt]{article}%
\usepackage{amssymb}
\usepackage{geometry}
\usepackage{indentfirst}
\usepackage{amsmath}
\usepackage{amsfonts}
\usepackage{graphicx}%
\setcounter{MaxMatrixCols}{30}
%TCIDATA{OutputFilter=latex2.dll}
%TCIDATA{Version=5.00.0.2606}
%TCIDATA{CSTFile=40 LaTeX article.cst}
%TCIDATA{Created=Friday, March 30, 2007 00:21:27}
%TCIDATA{LastRevised=Wednesday, June 10, 2009 11:42:33}
%TCIDATA{<META NAME="GraphicsSave" CONTENT="32">}
%TCIDATA{<META NAME="SaveForMode" CONTENT="1">}
%TCIDATA{BibliographyScheme=Manual}
%TCIDATA{<META NAME="DocumentShell" CONTENT="Standard LaTeX\Blank - Standard LaTeX Article">}
%TCIDATA{Language=American English}
\newtheorem{theorem}{Theorem}
\newtheorem{acknowledgement}[theorem]{Acknowledgement}
\newtheorem{algorithm}[theorem]{Algorithm}
\newtheorem{axiom}[theorem]{Axiom}
\newtheorem{case}[theorem]{Case}
\newtheorem{claim}[theorem]{Claim}
\newtheorem{conclusion}[theorem]{Conclusion}
\newtheorem{condition}[theorem]{Condition}
\newtheorem{conjecture}[theorem]{Conjecture}
\newtheorem{corollary}[theorem]{Corollary}
\newtheorem{criterion}[theorem]{Criterion}
\newtheorem{definition}[theorem]{Definition}
\newtheorem{example}[theorem]{Example}
\newtheorem{exercise}[theorem]{Exercise}
\newtheorem{lemma}[theorem]{Lemma}
\newtheorem{notation}[theorem]{Notation}
\newtheorem{problem}[theorem]{Problem}
\newtheorem{proposition}[theorem]{Proposition}
\newtheorem{remark}[theorem]{Remark}
\newtheorem{solution}[theorem]{Solution}
\newtheorem{summary}[theorem]{Summary}
\newenvironment{proof}[1][Proof]{\noindent\textbf{#1.} }{\ \rule{0.5em}{0.5em}}
\geometry{left=1in,right=1in,top=1in,bottom=1in}

\begin{document}

%%%%% BEGINNING OF DOCUMENT BODY %%%%%
% html: Beginning of file: `clean.html'
% DOCTYPE HTML PUBLIC "-//W3C//DTD HTML 4.01//EN"
%  This is a (PRE) block.  Make sure it's left aligned or your toc title will be off. 

\section*{\texttt{NewtonsMethod::optimize}}

\label{f0}{\small{\begin{verbatim} 
void optimize(double initial[], double soln[])
\end{verbatim}
}}

\subsection*{Keywords}

\begin{quotation} Newtons Method, optimize, extremum, gradient, hessian\end{quotation}

\subsection*{Authors}

\begin{itemize}\item  Alex Henniges
\end{itemize}

\subsection*{Introduction}

\begin{quotation} The \texttt{optimize} function of the NewtonsMethod class is designed to find either a maximum or minimum of a functional near a given point.\end{quotation}

\subsection*{Subsidiaries}

\begin{quotation} Functions:\end{quotation}
\begin{itemize}
\item  \texttt{NewtonsMethod::step}
\end{itemize}
\begin{quotation} \end{quotation}
\subsection*{Description}

\begin{quotation} The \texttt{optimize} function is called once by the user and it will continue to loop until an extremum of the functional is found. The functional is given in the constructor for NewtonsMethod. The initial point is the first parameter of the \texttt{optimize} function and the solution point is placed in the second parameter. This means that if one cannot be found, the function will loop without end. This is unlike the \texttt{step} function used within \texttt{optimize} that can also be used by a client program to gain much greater flexibility in the optimization process, such as more leeway on when to stop and allowing for data collection in between. See the \texttt{step} function for a description of how the optimization is performed.\end{quotation}

\subsection*{Practicum}

\begin{quotation} Example:{\small{\begin{verbatim} 
    double func(double vars[]) {
       double val = 1 - pow(vars[0], 2) / 4 - pow(vars[1], 2) / 9;
       return sqrt(val);
    }
    
    NewtonsMethod *nm = new NewtonsMethod(func, 2);
    double initial[] = {1, 1};
    double soln[2];

    nm->optimize(initial, soln);
  \end{verbatim}
}}
\end{quotation}
\subsection*{Limitations}

\begin{quotation} The \texttt{optimize} function is limited in its termination condition. This must be a constant over any use of the \texttt{optimize} function. It is also limited in that it may not terminate at all and the user will be forced to quit the program. Instead of modifying the function, these limitations are addressed by the \texttt{step} function which trades simplicity in terms of number of lines for greater flexibility.\end{quotation}

\subsection*{Revisions}

\begin{itemize}\item  subversion 876 7/16/09: Added a NewtonsMethod class for general maximizing.
\item  subversion 906 8/3/09: Changed the name of the function maximize to optimize in the NewtonsMethod class.
\end{itemize}

\subsection*{Testing}

\begin{quotation} Newtons Method has been tested using several functions of 1 or 2 variables including the Gaussian function. It has been tested with both approximating the gradient and hessian and when both are given explicitly.\end{quotation}

\subsection*{Future Work}

\begin{itemize}\item  8/4 - Add the ability to only move partially in the direction of the gradient.
\end{itemize}
    
% html: End of file: `clean.html'

%%%%% END OF DOCUMENT BODY %%%%%
% In the future, we might want to put some additional data here, such
% as when the documentation was converted from wiki to TeX.
%

\end{document}
}}%
%BeginExpansion
%html2tex: Version  2.7 of June 17, 2008.
%Written by  F.J. Faase.  http://www.iwriteiam.nl/

\documentclass[10pt]{article}%
\usepackage{amssymb}
\usepackage{geometry}
\usepackage{indentfirst}
\usepackage{amsmath}
\usepackage{amsfonts}
\usepackage{graphicx}%
\setcounter{MaxMatrixCols}{30}
%TCIDATA{OutputFilter=latex2.dll}
%TCIDATA{Version=5.00.0.2606}
%TCIDATA{CSTFile=40 LaTeX article.cst}
%TCIDATA{Created=Friday, March 30, 2007 00:21:27}
%TCIDATA{LastRevised=Wednesday, June 10, 2009 11:42:33}
%TCIDATA{<META NAME="GraphicsSave" CONTENT="32">}
%TCIDATA{<META NAME="SaveForMode" CONTENT="1">}
%TCIDATA{BibliographyScheme=Manual}
%TCIDATA{<META NAME="DocumentShell" CONTENT="Standard LaTeX\Blank - Standard LaTeX Article">}
%TCIDATA{Language=American English}
\newtheorem{theorem}{Theorem}
\newtheorem{acknowledgement}[theorem]{Acknowledgement}
\newtheorem{algorithm}[theorem]{Algorithm}
\newtheorem{axiom}[theorem]{Axiom}
\newtheorem{case}[theorem]{Case}
\newtheorem{claim}[theorem]{Claim}
\newtheorem{conclusion}[theorem]{Conclusion}
\newtheorem{condition}[theorem]{Condition}
\newtheorem{conjecture}[theorem]{Conjecture}
\newtheorem{corollary}[theorem]{Corollary}
\newtheorem{criterion}[theorem]{Criterion}
\newtheorem{definition}[theorem]{Definition}
\newtheorem{example}[theorem]{Example}
\newtheorem{exercise}[theorem]{Exercise}
\newtheorem{lemma}[theorem]{Lemma}
\newtheorem{notation}[theorem]{Notation}
\newtheorem{problem}[theorem]{Problem}
\newtheorem{proposition}[theorem]{Proposition}
\newtheorem{remark}[theorem]{Remark}
\newtheorem{solution}[theorem]{Solution}
\newtheorem{summary}[theorem]{Summary}
\newenvironment{proof}[1][Proof]{\noindent\textbf{#1.} }{\ \rule{0.5em}{0.5em}}
\geometry{left=1in,right=1in,top=1in,bottom=1in}

\begin{document}

%%%%% BEGINNING OF DOCUMENT BODY %%%%%
% html: Beginning of file: `clean.html'
% DOCTYPE HTML PUBLIC "-//W3C//DTD HTML 4.01//EN"
%  This is a (PRE) block.  Make sure it's left aligned or your toc title will be off. 

\section*{\texttt{NewtonsMethod::optimize}}

\label{f0}{\small{\begin{verbatim} 
void optimize(double initial[], double soln[])
\end{verbatim}
}}

\subsection*{Keywords}

\begin{quotation} Newtons Method, optimize, extremum, gradient, hessian\end{quotation}

\subsection*{Authors}

\begin{itemize}\item  Alex Henniges
\end{itemize}

\subsection*{Introduction}

\begin{quotation} The \texttt{optimize} function of the NewtonsMethod class is designed to find either a maximum or minimum of a functional near a given point.\end{quotation}

\subsection*{Subsidiaries}

\begin{quotation} Functions:\end{quotation}
\begin{itemize}
\item  \texttt{NewtonsMethod::step}
\end{itemize}
\begin{quotation} \end{quotation}
\subsection*{Description}

\begin{quotation} The \texttt{optimize} function is called once by the user and it will continue to loop until an extremum of the functional is found. The functional is given in the constructor for NewtonsMethod. The initial point is the first parameter of the \texttt{optimize} function and the solution point is placed in the second parameter. This means that if one cannot be found, the function will loop without end. This is unlike the \texttt{step} function used within \texttt{optimize} that can also be used by a client program to gain much greater flexibility in the optimization process, such as more leeway on when to stop and allowing for data collection in between. See the \texttt{step} function for a description of how the optimization is performed.\end{quotation}

\subsection*{Practicum}

\begin{quotation} Example:{\small{\begin{verbatim} 
    double func(double vars[]) {
       double val = 1 - pow(vars[0], 2) / 4 - pow(vars[1], 2) / 9;
       return sqrt(val);
    }
    
    NewtonsMethod *nm = new NewtonsMethod(func, 2);
    double initial[] = {1, 1};
    double soln[2];

    nm->optimize(initial, soln);
  \end{verbatim}
}}
\end{quotation}
\subsection*{Limitations}

\begin{quotation} The \texttt{optimize} function is limited in its termination condition. This must be a constant over any use of the \texttt{optimize} function. It is also limited in that it may not terminate at all and the user will be forced to quit the program. Instead of modifying the function, these limitations are addressed by the \texttt{step} function which trades simplicity in terms of number of lines for greater flexibility.\end{quotation}

\subsection*{Revisions}

\begin{itemize}\item  subversion 876 7/16/09: Added a NewtonsMethod class for general maximizing.
\item  subversion 906 8/3/09: Changed the name of the function maximize to optimize in the NewtonsMethod class.
\end{itemize}

\subsection*{Testing}

\begin{quotation} Newtons Method has been tested using several functions of 1 or 2 variables including the Gaussian function. It has been tested with both approximating the gradient and hessian and when both are given explicitly.\end{quotation}

\subsection*{Future Work}

\begin{itemize}\item  8/4 - Add the ability to only move partially in the direction of the gradient.
\end{itemize}
    
% html: End of file: `clean.html'

%%%%% END OF DOCUMENT BODY %%%%%
% In the future, we might want to put some additional data here, such
% as when the documentation was converted from wiki to TeX.
%

\end{document}
%
%EndExpansion

\bigskip 

\bigskip 

%TCIMACRO{\QSubDoc{Include pause}{%TCIDATA{Version=5.00.0.2606}
%TCIDATA{LaTeXparent=0,0,functions.tex}
                      

%%%%% BEGINNING OF DOCUMENT BODY %%%%%
% html: Beginning of file: `clean.html'
% DOCTYPE HTML PUBLIC "-//W3C//DTD HTML 4.01//EN"
%  This is a (PRE) block.  Make sure it's left aligned or your toc title will be off. 

\section*{\texttt{pause}}

\label{f0}

\begin{quotation}
{\small }
\end{quotation}

\begin{verbatim}
{\small    void pause();
}
{\small    void pause(char *fmt, ...);
}
{\small    
}
\end{verbatim}

\subsection*{Key Words}

\begin{quotation}
pause, print
\end{quotation}

\subsection*{Authors}

\begin{quotation}
Alex Henniges
\end{quotation}

\subsection*{Introduction}

\begin{quotation}
The \texttt{pause} freezes the current process until the user presses the 
\textbf{enter} key. This function also allows the user to print information
at the pause line.
\end{quotation}

\subsection*{Subsidiaries}

\begin{quotation}
Functions:
\end{quotation}

\begin{itemize}
\item \texttt{vprintf}

\item \texttt{scanf}

\item \texttt{fflush}
\end{itemize}

\begin{quotation}
Global Variables:

Local Variables:
\end{quotation}

\subsection*{Description}

\begin{quotation}
The \texttt{pause} function is designed to place break points in the code
that will stop the process until the user presses the enter key. There are
several uses to this. A standard one is debugging as it can allow a
programmer to step through a procedure. While there are usually similar
debugging options in code editors, this function can be added and removed
easily from within the code. The second use is that the console for programs
will close immediately after execution with some editors. Without a way to
freeze the program, the console would close before the data could be read
and interpreted.

There are two options for this \texttt{pause} function. If the default pause
is used, the following message will be printed:``PAUSE...''Pressing the 
\textbf{enter} key will resume the process. The function can also print out
a message provided to it. This uses the \texttt{vprintf} function so that
the printed information can be formatted text. The user must still press 
\textbf{enter} to resume when this form is used. Pressing other keys will
not affect the program.

Historically, the project has used {\small }
\end{quotation}

\begin{verbatim}
{\small   system("PAUSE");
}
{\small   
}
\end{verbatim}

\begin{quotation}
to pause the program. However, this can only be used on a Windows machine, a
limiting factor that we wish to remove from the project.
\end{quotation}

\subsection*{Practicum}

\begin{quotation}
Example:{\small }
\end{quotation}

\begin{verbatim}
{\small   pause("Done...press enter to exit."); // PAUSE
}
{\small   
}
\end{verbatim}

\subsection*{Limitations}

\begin{quotation}
One limitation of the \texttt{pause} function is that it only resumes after
pressing the \textbf{enter} key. This is compared to the former pause
function (see above) that would resume after pressing any key. This could
also be considered an improvement.
\end{quotation}

\subsection*{Revisions}

\begin{itemize}
\item subversion 909, 8/4/09: Added the fully functional \texttt{pause}
function.
\end{itemize}

\subsection*{Testing}

\begin{quotation}
The \texttt{pause} function has been tested simply through using it
extensively.
\end{quotation}

\subsection*{Future Work}

\begin{quotation}
No future work is planned at this time.
\end{quotation}

% html: End of file: `clean.html'

%%%%% END OF DOCUMENT BODY %%%%%
% In the future, we might want to put some additional data here, such
% as when the documentation was converted from wiki to TeX.
%
}}%
%BeginExpansion
%TCIDATA{Version=5.00.0.2606}
%TCIDATA{LaTeXparent=0,0,functions.tex}
                      

%%%%% BEGINNING OF DOCUMENT BODY %%%%%
% html: Beginning of file: `clean.html'
% DOCTYPE HTML PUBLIC "-//W3C//DTD HTML 4.01//EN"
%  This is a (PRE) block.  Make sure it's left aligned or your toc title will be off. 

\section*{\texttt{pause}}

\label{f0}

\begin{quotation}
{\small }
\end{quotation}

\begin{verbatim}
{\small    void pause();
}
{\small    void pause(char *fmt, ...);
}
{\small    
}
\end{verbatim}

\subsection*{Key Words}

\begin{quotation}
pause, print
\end{quotation}

\subsection*{Authors}

\begin{quotation}
Alex Henniges
\end{quotation}

\subsection*{Introduction}

\begin{quotation}
The \texttt{pause} freezes the current process until the user presses the 
\textbf{enter} key. This function also allows the user to print information
at the pause line.
\end{quotation}

\subsection*{Subsidiaries}

\begin{quotation}
Functions:
\end{quotation}

\begin{itemize}
\item \texttt{vprintf}

\item \texttt{scanf}

\item \texttt{fflush}
\end{itemize}

\begin{quotation}
Global Variables:

Local Variables:
\end{quotation}

\subsection*{Description}

\begin{quotation}
The \texttt{pause} function is designed to place break points in the code
that will stop the process until the user presses the enter key. There are
several uses to this. A standard one is debugging as it can allow a
programmer to step through a procedure. While there are usually similar
debugging options in code editors, this function can be added and removed
easily from within the code. The second use is that the console for programs
will close immediately after execution with some editors. Without a way to
freeze the program, the console would close before the data could be read
and interpreted.

There are two options for this \texttt{pause} function. If the default pause
is used, the following message will be printed:``PAUSE...''Pressing the 
\textbf{enter} key will resume the process. The function can also print out
a message provided to it. This uses the \texttt{vprintf} function so that
the printed information can be formatted text. The user must still press 
\textbf{enter} to resume when this form is used. Pressing other keys will
not affect the program.

Historically, the project has used {\small }
\end{quotation}

\begin{verbatim}
{\small   system("PAUSE");
}
{\small   
}
\end{verbatim}

\begin{quotation}
to pause the program. However, this can only be used on a Windows machine, a
limiting factor that we wish to remove from the project.
\end{quotation}

\subsection*{Practicum}

\begin{quotation}
Example:{\small }
\end{quotation}

\begin{verbatim}
{\small   pause("Done...press enter to exit."); // PAUSE
}
{\small   
}
\end{verbatim}

\subsection*{Limitations}

\begin{quotation}
One limitation of the \texttt{pause} function is that it only resumes after
pressing the \textbf{enter} key. This is compared to the former pause
function (see above) that would resume after pressing any key. This could
also be considered an improvement.
\end{quotation}

\subsection*{Revisions}

\begin{itemize}
\item subversion 909, 8/4/09: Added the fully functional \texttt{pause}
function.
\end{itemize}

\subsection*{Testing}

\begin{quotation}
The \texttt{pause} function has been tested simply through using it
extensively.
\end{quotation}

\subsection*{Future Work}

\begin{quotation}
No future work is planned at this time.
\end{quotation}

% html: End of file: `clean.html'

%%%%% END OF DOCUMENT BODY %%%%%
% In the future, we might want to put some additional data here, such
% as when the documentation was converted from wiki to TeX.
%
%
%EndExpansion

\bigskip 

\bigskip 

%TCIMACRO{%
%\QSubDoc{Include print3DResultsStep}{%html2tex: Version  2.7 of June 17, 2008.
%Written by  F.J. Faase.  http://www.iwriteiam.nl/

clean.html (22) : file `printResultsStep.html' does not exist.
clean.html (22) : file `printResultsVertex.html' does not exist.
\documentclass[10pt]{article}%
\usepackage{amssymb}
\usepackage{geometry}
\usepackage{indentfirst}
\usepackage{amsmath}
\usepackage{amsfonts}
\usepackage{graphicx}%
\setcounter{MaxMatrixCols}{30}
%TCIDATA{OutputFilter=latex2.dll}
%TCIDATA{Version=5.00.0.2606}
%TCIDATA{CSTFile=40 LaTeX article.cst}
%TCIDATA{Created=Friday, March 30, 2007 00:21:27}
%TCIDATA{LastRevised=Wednesday, June 10, 2009 11:42:33}
%TCIDATA{<META NAME="GraphicsSave" CONTENT="32">}
%TCIDATA{<META NAME="SaveForMode" CONTENT="1">}
%TCIDATA{BibliographyScheme=Manual}
%TCIDATA{<META NAME="DocumentShell" CONTENT="Standard LaTeX\Blank - Standard LaTeX Article">}
%TCIDATA{Language=American English}
\newtheorem{theorem}{Theorem}
\newtheorem{acknowledgement}[theorem]{Acknowledgement}
\newtheorem{algorithm}[theorem]{Algorithm}
\newtheorem{axiom}[theorem]{Axiom}
\newtheorem{case}[theorem]{Case}
\newtheorem{claim}[theorem]{Claim}
\newtheorem{conclusion}[theorem]{Conclusion}
\newtheorem{condition}[theorem]{Condition}
\newtheorem{conjecture}[theorem]{Conjecture}
\newtheorem{corollary}[theorem]{Corollary}
\newtheorem{criterion}[theorem]{Criterion}
\newtheorem{definition}[theorem]{Definition}
\newtheorem{example}[theorem]{Example}
\newtheorem{exercise}[theorem]{Exercise}
\newtheorem{lemma}[theorem]{Lemma}
\newtheorem{notation}[theorem]{Notation}
\newtheorem{problem}[theorem]{Problem}
\newtheorem{proposition}[theorem]{Proposition}
\newtheorem{remark}[theorem]{Remark}
\newtheorem{solution}[theorem]{Solution}
\newtheorem{summary}[theorem]{Summary}
\newenvironment{proof}[1][Proof]{\noindent\textbf{#1.} }{\ \rule{0.5em}{0.5em}}
\geometry{left=1in,right=1in,top=1in,bottom=1in}

\begin{document}

%%%%% BEGINNING OF DOCUMENT BODY %%%%%
% html: Beginning of file: `clean.html'
% DOCTYPE HTML PUBLIC "-//W3C//DTD HTML 4.01//EN"
%  This is a (PRE) block.  Make sure it's left aligned or your toc title will be off. 

\section*{\texttt{print3DResultsStep}}

\label{f0}\begin{quotation} {\small{\begin{verbatim} 
   void print3DResultsStep(char* fileName, vector<double>* radii, vector<double>* curvs)
   \end{verbatim}
}}
\end{quotation}
\subsection*{Key Words}

\begin{quotation} radii, curvatures, file, flow, step, print, three-dimensional\end{quotation}

\subsection*{Authors}

\begin{quotation} Alex Henniges\end{quotation}

\subsection*{Introduction}

\begin{quotation} The \texttt{print3DResultsStep} function prints out the results of a curvature flow, with the results grouped by each step of the flow. These results will be written to the file given by \texttt{filename}.\end{quotation}

\subsection*{Subsidiaries}

\begin{quotation} Functions:\end{quotation}
\begin{quotation} Global Variables:\end{quotation}
\begin{quotation} Local Variables: \texttt{int vertSize}, \texttt{int numSteps}\end{quotation}

\subsection*{Description}

\begin{quotation} Prints the results of a curvature flow into the file given by \texttt{filename}. The results, that is, the radii and curvature values, are given by vectors of doubles. Most commonly, these vectors are taken from the Approximator class after the flow is run. The \texttt{print3DResultsStep} function determines the number of vertices of the current triangulation and the total number of steps are then derived from this and the size of the vectors.\end{quotation}
\begin{quotation} There are several ways to display the results. The \texttt{print3DResultsStep} function groups by step. This means that for each step of the curvature flow, the radii and curvature values for each vertex is printed. In addition, since Yamabe flow converges with respect to curvature divided by radius, this value is printed as well. Therefore, this function should be used with three-dimensional curvature flows. An example is shown below. Other formats are given by printResultsStep , printResultsVertex, printResultsNum.\end{quotation}

\subsection*{Practicum}

\begin{quotation} Example:{\small{\begin{verbatim} 
  // Print the results of a curvature flow with Approximator app into file "ODEResult.txt"
  print3DResultsStep("./ODEResults.txt", app->radiiHistory, app->curvHistory);
  \end{verbatim}
}}
\end{quotation}\begin{quotation} The output of such an example may then be{\small{\begin{verbatim}       
         :
         :
       Vertex   5     0.8324396       8.5301529       10.2471738
       Total Curvature: 44.5286316

       Step    74     Radius          Curvature        Curv:Radius
       -----------------------------------------------------
       Vertex   1     0.8883594       9.3071126       10.4767428
       Vertex   2     0.8725496       9.0880458       10.4155064
       Vertex   3     0.8579899       8.8858872       10.3566333
       Vertex   4     0.8448655       8.7033021       10.3014058
       Vertex   5     0.8333839       8.5432883       10.2513233
       Total Curvature: 44.5276360

       Step    75     Radius          Curvature        Curv:Radius
       -----------------------------------------------------
       Vertex   1     0.8873282       9.2928034       10.4727922
         :
         :
  \end{verbatim}
}}
\end{quotation}
\subsection*{Limitations}

\begin{quotation} Currently the \texttt{print3DResultsStep} function is limited in the information it prints. As our curvature flow has evolved to record additional information such as volumes, it may be time to explore a more robust form for displaying results. As there is considerable dependence on the Approximator for the data vectors, it may be wise to place this and similar functions in the Approximator class.\end{quotation}

\subsection*{Revisions}

\begin{itemize}\item  subversion 545, 9/29/08: Moved the printing of results out of calcFlow and into a new function.
\item  subversion 783, 6/18/09: Small modifications in response to changes in the Approximator class.
\end{itemize}

\subsection*{Testing}

\begin{quotation} The \texttt{print3DResultsStep} function was tested by running multiple curvature flows and printing the results. It was considered working when the format of the data was as desired.\end{quotation}

\subsection*{Future Work}

\begin{itemize}\item  6/29 - Recreate the print functions to print more data and be more flexible.
\item  6/29 - Move the print functions into the Approximator class.
\end{itemize}
    
% html: End of file: `clean.html'

%%%%% END OF DOCUMENT BODY %%%%%
% In the future, we might want to put some additional data here, such
% as when the documentation was converted from wiki to TeX.
%

\end{document}
}}%
%BeginExpansion
%html2tex: Version  2.7 of June 17, 2008.
%Written by  F.J. Faase.  http://www.iwriteiam.nl/

clean.html (22) : file `printResultsStep.html' does not exist.
clean.html (22) : file `printResultsVertex.html' does not exist.
\documentclass[10pt]{article}%
\usepackage{amssymb}
\usepackage{geometry}
\usepackage{indentfirst}
\usepackage{amsmath}
\usepackage{amsfonts}
\usepackage{graphicx}%
\setcounter{MaxMatrixCols}{30}
%TCIDATA{OutputFilter=latex2.dll}
%TCIDATA{Version=5.00.0.2606}
%TCIDATA{CSTFile=40 LaTeX article.cst}
%TCIDATA{Created=Friday, March 30, 2007 00:21:27}
%TCIDATA{LastRevised=Wednesday, June 10, 2009 11:42:33}
%TCIDATA{<META NAME="GraphicsSave" CONTENT="32">}
%TCIDATA{<META NAME="SaveForMode" CONTENT="1">}
%TCIDATA{BibliographyScheme=Manual}
%TCIDATA{<META NAME="DocumentShell" CONTENT="Standard LaTeX\Blank - Standard LaTeX Article">}
%TCIDATA{Language=American English}
\newtheorem{theorem}{Theorem}
\newtheorem{acknowledgement}[theorem]{Acknowledgement}
\newtheorem{algorithm}[theorem]{Algorithm}
\newtheorem{axiom}[theorem]{Axiom}
\newtheorem{case}[theorem]{Case}
\newtheorem{claim}[theorem]{Claim}
\newtheorem{conclusion}[theorem]{Conclusion}
\newtheorem{condition}[theorem]{Condition}
\newtheorem{conjecture}[theorem]{Conjecture}
\newtheorem{corollary}[theorem]{Corollary}
\newtheorem{criterion}[theorem]{Criterion}
\newtheorem{definition}[theorem]{Definition}
\newtheorem{example}[theorem]{Example}
\newtheorem{exercise}[theorem]{Exercise}
\newtheorem{lemma}[theorem]{Lemma}
\newtheorem{notation}[theorem]{Notation}
\newtheorem{problem}[theorem]{Problem}
\newtheorem{proposition}[theorem]{Proposition}
\newtheorem{remark}[theorem]{Remark}
\newtheorem{solution}[theorem]{Solution}
\newtheorem{summary}[theorem]{Summary}
\newenvironment{proof}[1][Proof]{\noindent\textbf{#1.} }{\ \rule{0.5em}{0.5em}}
\geometry{left=1in,right=1in,top=1in,bottom=1in}

\begin{document}

%%%%% BEGINNING OF DOCUMENT BODY %%%%%
% html: Beginning of file: `clean.html'
% DOCTYPE HTML PUBLIC "-//W3C//DTD HTML 4.01//EN"
%  This is a (PRE) block.  Make sure it's left aligned or your toc title will be off. 

\section*{\texttt{print3DResultsStep}}

\label{f0}\begin{quotation} {\small{\begin{verbatim} 
   void print3DResultsStep(char* fileName, vector<double>* radii, vector<double>* curvs)
   \end{verbatim}
}}
\end{quotation}
\subsection*{Key Words}

\begin{quotation} radii, curvatures, file, flow, step, print, three-dimensional\end{quotation}

\subsection*{Authors}

\begin{quotation} Alex Henniges\end{quotation}

\subsection*{Introduction}

\begin{quotation} The \texttt{print3DResultsStep} function prints out the results of a curvature flow, with the results grouped by each step of the flow. These results will be written to the file given by \texttt{filename}.\end{quotation}

\subsection*{Subsidiaries}

\begin{quotation} Functions:\end{quotation}
\begin{quotation} Global Variables:\end{quotation}
\begin{quotation} Local Variables: \texttt{int vertSize}, \texttt{int numSteps}\end{quotation}

\subsection*{Description}

\begin{quotation} Prints the results of a curvature flow into the file given by \texttt{filename}. The results, that is, the radii and curvature values, are given by vectors of doubles. Most commonly, these vectors are taken from the Approximator class after the flow is run. The \texttt{print3DResultsStep} function determines the number of vertices of the current triangulation and the total number of steps are then derived from this and the size of the vectors.\end{quotation}
\begin{quotation} There are several ways to display the results. The \texttt{print3DResultsStep} function groups by step. This means that for each step of the curvature flow, the radii and curvature values for each vertex is printed. In addition, since Yamabe flow converges with respect to curvature divided by radius, this value is printed as well. Therefore, this function should be used with three-dimensional curvature flows. An example is shown below. Other formats are given by printResultsStep , printResultsVertex, printResultsNum.\end{quotation}

\subsection*{Practicum}

\begin{quotation} Example:{\small{\begin{verbatim} 
  // Print the results of a curvature flow with Approximator app into file "ODEResult.txt"
  print3DResultsStep("./ODEResults.txt", app->radiiHistory, app->curvHistory);
  \end{verbatim}
}}
\end{quotation}\begin{quotation} The output of such an example may then be{\small{\begin{verbatim}       
         :
         :
       Vertex   5     0.8324396       8.5301529       10.2471738
       Total Curvature: 44.5286316

       Step    74     Radius          Curvature        Curv:Radius
       -----------------------------------------------------
       Vertex   1     0.8883594       9.3071126       10.4767428
       Vertex   2     0.8725496       9.0880458       10.4155064
       Vertex   3     0.8579899       8.8858872       10.3566333
       Vertex   4     0.8448655       8.7033021       10.3014058
       Vertex   5     0.8333839       8.5432883       10.2513233
       Total Curvature: 44.5276360

       Step    75     Radius          Curvature        Curv:Radius
       -----------------------------------------------------
       Vertex   1     0.8873282       9.2928034       10.4727922
         :
         :
  \end{verbatim}
}}
\end{quotation}
\subsection*{Limitations}

\begin{quotation} Currently the \texttt{print3DResultsStep} function is limited in the information it prints. As our curvature flow has evolved to record additional information such as volumes, it may be time to explore a more robust form for displaying results. As there is considerable dependence on the Approximator for the data vectors, it may be wise to place this and similar functions in the Approximator class.\end{quotation}

\subsection*{Revisions}

\begin{itemize}\item  subversion 545, 9/29/08: Moved the printing of results out of calcFlow and into a new function.
\item  subversion 783, 6/18/09: Small modifications in response to changes in the Approximator class.
\end{itemize}

\subsection*{Testing}

\begin{quotation} The \texttt{print3DResultsStep} function was tested by running multiple curvature flows and printing the results. It was considered working when the format of the data was as desired.\end{quotation}

\subsection*{Future Work}

\begin{itemize}\item  6/29 - Recreate the print functions to print more data and be more flexible.
\item  6/29 - Move the print functions into the Approximator class.
\end{itemize}
    
% html: End of file: `clean.html'

%%%%% END OF DOCUMENT BODY %%%%%
% In the future, we might want to put some additional data here, such
% as when the documentation was converted from wiki to TeX.
%

\end{document}
%
%EndExpansion

\bigskip 

\bigskip 

%TCIMACRO{\QSubDoc{Include printResultsNum}{%TCIDATA{Version=5.00.0.2606}
%TCIDATA{LaTeXparent=0,0,functions.tex}
                      

%%%%% BEGINNING OF DOCUMENT BODY %%%%%
% html: Beginning of file: `clean.html'
% DOCTYPE HTML PUBLIC "-//W3C//DTD HTML 4.01//EN"
%  This is a (PRE) block.  Make sure it's left aligned or your toc title will be off. 

\section*{\texttt{printResultsNum}}

\label{f0}{\small }
\begin{verbatim}
{\small 
        void printResultsNum(char* fileName, vector<double>* radii, vector<double>* curvs)
}
\end{verbatim}

\subsection*{Key Words}

radii, curvatures, file, flow, vertex, print

\subsection*{Authors}

Alex Henniges

\subsection*{Introduction}

The \texttt{printResultsNum} function prints out the results of a curvature
flow, with the results grouped by each vertex of the triangulation but
without labels. This format is used for when a program, (GUI, Matlab, etc)
needs to parse the data. These results will be written to the file given by 
\texttt{filename}.

\subsection*{Subsidiaries}

Functions:

Global Variables:

Local Variables: \texttt{int vertSize}, \texttt{int numSteps}

\subsection*{Description}

Prints the results of a curvature flow into the file given by \texttt{%
filename}. The results, that is, the radii and curvature values, are given
by vectors of doubles. Most commonly, these vectors are taken from the
Approximator class after the flow is run. The \texttt{printResultsNum}
function determines the number of vertices of the current triangulation and
the total number of steps are then derived from this and the size of the
vectors.

There are several ways to display the results. The \texttt{printResultsNum}
function groups by vertex but provides no labels. This means that for each
vertex of the triangulation, the radii (first column) and curvature (second
column) values for each step is given. This format is used for when a
program, (GUI, Matlab, etc) needs to parse the data. Therefore, it would be
difficult for a human to read, but allows the computer to do so much easier.
An example is shown below. Other formats are given by printResultsStep,
printResultsVertex, print3DResultsStep.

\subsection*{Practicum}

Example: {\small }
\begin{verbatim}
{\small     // Print the results of a curvature flow with
}
{\small     // Approximator app into file "ODEResult.txt"
}
{\small 
    printResultsNum("./ODEResults.txt", app->radiiHistory, app->curvHistory); 
}
\end{verbatim}

The output of such an example may then be {\small }
\begin{verbatim}
{\small          :
}
{\small          :
}
{\small       0.8272160717        3.1425515910
}
{\small       0.8272081392        3.1425294463
}
{\small       0.8272003900        3.1425078130
}
 
{\small       1.0000000000        3.1415926536
}
{\small       1.0000000000        3.5987926375
}
{\small       0.9954280002        3.5877016632
}
{\small       0.9909873062        3.5768691746
}
{\small       0.9866737711        3.5662905388
}
{\small          :
}
{\small          :
}
\end{verbatim}

\subsection*{Limitations}

Currently the \texttt{printResultsNum} function is limited in the
information it prints. As our curvature flow has evolved to record
additional information such as volumes, it may be time to explore a more
robust form for displaying results. As there is considerable dependence on
the Approximator for the data vectors, it may be wise to place this and
similar functions in the \mbox{$[$}CurvatureFlow Approximator\mbox{$]$}
class.

\subsection*{Revisions}

\textbf{\ subversion 545, 9/29/08: Moved the printing of results out of
calcFlow and into a new function. }  subversion 783, 6/18/09: Small
modifications in response to changes in the Approximator class.

\subsection*{Testing}

The \texttt{printResultsNum} function was tested by running multiple
curvature flows and printing the results. It was considered working when the
format of the data was as desired.

\subsection*{Future Work}

\textbf{\ 6/29 - Recreate the print functions to print more data and be more
flexible.}  6/29 - Move the print functions into the Approximator class.

% html: End of file: `clean.html'

%%%%% END OF DOCUMENT BODY %%%%%
% In the future, we might want to put some additional data here, such
% as when the documentation was converted from wiki to TeX.
%
}}%
%BeginExpansion
%TCIDATA{Version=5.00.0.2606}
%TCIDATA{LaTeXparent=0,0,functions.tex}
                      

%%%%% BEGINNING OF DOCUMENT BODY %%%%%
% html: Beginning of file: `clean.html'
% DOCTYPE HTML PUBLIC "-//W3C//DTD HTML 4.01//EN"
%  This is a (PRE) block.  Make sure it's left aligned or your toc title will be off. 

\section*{\texttt{printResultsNum}}

\label{f0}{\small }
\begin{verbatim}
{\small 
        void printResultsNum(char* fileName, vector<double>* radii, vector<double>* curvs)
}
\end{verbatim}

\subsection*{Key Words}

radii, curvatures, file, flow, vertex, print

\subsection*{Authors}

Alex Henniges

\subsection*{Introduction}

The \texttt{printResultsNum} function prints out the results of a curvature
flow, with the results grouped by each vertex of the triangulation but
without labels. This format is used for when a program, (GUI, Matlab, etc)
needs to parse the data. These results will be written to the file given by 
\texttt{filename}.

\subsection*{Subsidiaries}

Functions:

Global Variables:

Local Variables: \texttt{int vertSize}, \texttt{int numSteps}

\subsection*{Description}

Prints the results of a curvature flow into the file given by \texttt{%
filename}. The results, that is, the radii and curvature values, are given
by vectors of doubles. Most commonly, these vectors are taken from the
Approximator class after the flow is run. The \texttt{printResultsNum}
function determines the number of vertices of the current triangulation and
the total number of steps are then derived from this and the size of the
vectors.

There are several ways to display the results. The \texttt{printResultsNum}
function groups by vertex but provides no labels. This means that for each
vertex of the triangulation, the radii (first column) and curvature (second
column) values for each step is given. This format is used for when a
program, (GUI, Matlab, etc) needs to parse the data. Therefore, it would be
difficult for a human to read, but allows the computer to do so much easier.
An example is shown below. Other formats are given by printResultsStep,
printResultsVertex, print3DResultsStep.

\subsection*{Practicum}

Example: {\small }
\begin{verbatim}
{\small     // Print the results of a curvature flow with
}
{\small     // Approximator app into file "ODEResult.txt"
}
{\small 
    printResultsNum("./ODEResults.txt", app->radiiHistory, app->curvHistory); 
}
\end{verbatim}

The output of such an example may then be {\small }
\begin{verbatim}
{\small          :
}
{\small          :
}
{\small       0.8272160717        3.1425515910
}
{\small       0.8272081392        3.1425294463
}
{\small       0.8272003900        3.1425078130
}
 
{\small       1.0000000000        3.1415926536
}
{\small       1.0000000000        3.5987926375
}
{\small       0.9954280002        3.5877016632
}
{\small       0.9909873062        3.5768691746
}
{\small       0.9866737711        3.5662905388
}
{\small          :
}
{\small          :
}
\end{verbatim}

\subsection*{Limitations}

Currently the \texttt{printResultsNum} function is limited in the
information it prints. As our curvature flow has evolved to record
additional information such as volumes, it may be time to explore a more
robust form for displaying results. As there is considerable dependence on
the Approximator for the data vectors, it may be wise to place this and
similar functions in the \mbox{$[$}CurvatureFlow Approximator\mbox{$]$}
class.

\subsection*{Revisions}

\textbf{\ subversion 545, 9/29/08: Moved the printing of results out of
calcFlow and into a new function. }  subversion 783, 6/18/09: Small
modifications in response to changes in the Approximator class.

\subsection*{Testing}

The \texttt{printResultsNum} function was tested by running multiple
curvature flows and printing the results. It was considered working when the
format of the data was as desired.

\subsection*{Future Work}

\textbf{\ 6/29 - Recreate the print functions to print more data and be more
flexible.}  6/29 - Move the print functions into the Approximator class.

% html: End of file: `clean.html'

%%%%% END OF DOCUMENT BODY %%%%%
% In the future, we might want to put some additional data here, such
% as when the documentation was converted from wiki to TeX.
%
%
%EndExpansion

\bigskip 

\bigskip 

%TCIMACRO{%
%\QSubDoc{Include printResultsNumSteps}{%TCIDATA{Version=5.00.0.2606}
%TCIDATA{LaTeXparent=0,0,functions.tex}
                      

%%%%% BEGINNING OF DOCUMENT BODY %%%%%
% html: Beginning of file: `clean.html'
% DOCTYPE HTML PUBLIC "-//W3C//DTD HTML 4.01//EN"
%  This is a (PRE) block.  Make sure it's left aligned or your toc title will be off. 

\section*{\texttt{printResultsNumSteps}}

\label{f0}

\begin{quotation}
{\small }
\end{quotation}

\begin{verbatim}
{\small 
   void printResultsNumSteps(char* fileName, vector<double>* radii, vector<double>* curvs)
}
{\small    
}
\end{verbatim}

\subsection*{Key Words}

\begin{quotation}
radii, curvatures, file, flow, vertex, print
\end{quotation}

\subsection*{Authors}

\begin{quotation}
Alex Henniges
\end{quotation}

\subsection*{Introduction}

\begin{quotation}
The \texttt{printResultsNumSteps} function prints out the results of a
curvature flow, with the results grouped by each step of the triangulation
but without labels. This format is used for with the GUI to create a
polygonal representation of curvatures. These results will be written to the
file given by \texttt{filename}.
\end{quotation}

\subsection*{Subsidiaries}

\begin{quotation}
Functions:

Global Variables:

Local Variables: \texttt{int vertSize}, \texttt{int numSteps}
\end{quotation}

\subsection*{Description}

\begin{quotation}
Prints the results of a curvature flow into the file given by \texttt{%
filename}. The results are curvature divided by radii values, and are given
by vectors of doubles. Most commonly, these vectors are taken from the
Approximator class after the flow is run. The \texttt{printResultsNumSteps}
function determines the number of vertices of the current triangulation and
the total number of steps are then derived from this and the size of the
vectors.

There are several ways to display the results. The \texttt{%
printResultsNumSteps} function groups by step but provides no labels and
does not print out radii, but instead curvautre divided by radii. The
purpose for this format is to create the ``Polygon flows'' in the GUI.
Therefore, it would be difficult for a human to read, but allows the
computer to do so much easier. An example is shown below.
\end{quotation}

\subsection*{Practicum}

\begin{quotation}
Example:{\small }
\end{quotation}

\begin{verbatim}
{\small 
  // Print the results of a curvature flow with Approximator app into file "ODEResult.txt"
}
{\small 
  printResultsNumSteps("./ODEResults.txt", app->radiiHistory, app->curvHistory);
}
{\small   
}
\end{verbatim}

\begin{quotation}
The output of such an example may then be{\small }
\end{quotation}

\begin{verbatim}
{\small          :
}
{\small          :
}
{\small       3.1425515910
}
{\small       3.1425294463
}
{\small       3.1425078130
}
 
{\small       3.1415926536
}
{\small       3.5987926375
}
{\small       3.5877016632
}
{\small       3.5768691746
}
{\small       3.5662905388
}
{\small          :
}
{\small          :
}
{\small   
}
\end{verbatim}

\subsection*{Limitations}

\begin{quotation}
Unlike the other print functions, the purpose of \texttt{printResultsNumSteps%
} is to only display the curvature divided by radii, and so is not limited
in the information it prints. On the otherhand, an overhaul of the entire
printing system would likely involve modifying this function.
\end{quotation}

\subsection*{Revisions}

\begin{itemize}
\item subversion 545, 9/29/08: Moved the printing of results out of calcFlow
and into a new function.

\item subversion 783, 6/18/09: Small modifications in response to changes in
the Approximator class.
\end{itemize}

\subsection*{Testing}

\begin{quotation}
The \texttt{printResultsNumSteps} function was tested by running multiple
curvature flows and printing the results. It was considered working when the
format of the data was as desired.
\end{quotation}

\subsection*{Future Work}

\begin{itemize}
\item 6/29 - Recreate the print functions to print more data and be more
flexible.

\item 6/29 - Move the print functions into the Approximator class.
\end{itemize}

% html: End of file: `clean.html'

%%%%% END OF DOCUMENT BODY %%%%%
% In the future, we might want to put some additional data here, such
% as when the documentation was converted from wiki to TeX.
%
}}%
%BeginExpansion
%TCIDATA{Version=5.00.0.2606}
%TCIDATA{LaTeXparent=0,0,functions.tex}
                      

%%%%% BEGINNING OF DOCUMENT BODY %%%%%
% html: Beginning of file: `clean.html'
% DOCTYPE HTML PUBLIC "-//W3C//DTD HTML 4.01//EN"
%  This is a (PRE) block.  Make sure it's left aligned or your toc title will be off. 

\section*{\texttt{printResultsNumSteps}}

\label{f0}

\begin{quotation}
{\small }
\end{quotation}

\begin{verbatim}
{\small 
   void printResultsNumSteps(char* fileName, vector<double>* radii, vector<double>* curvs)
}
{\small    
}
\end{verbatim}

\subsection*{Key Words}

\begin{quotation}
radii, curvatures, file, flow, vertex, print
\end{quotation}

\subsection*{Authors}

\begin{quotation}
Alex Henniges
\end{quotation}

\subsection*{Introduction}

\begin{quotation}
The \texttt{printResultsNumSteps} function prints out the results of a
curvature flow, with the results grouped by each step of the triangulation
but without labels. This format is used for with the GUI to create a
polygonal representation of curvatures. These results will be written to the
file given by \texttt{filename}.
\end{quotation}

\subsection*{Subsidiaries}

\begin{quotation}
Functions:

Global Variables:

Local Variables: \texttt{int vertSize}, \texttt{int numSteps}
\end{quotation}

\subsection*{Description}

\begin{quotation}
Prints the results of a curvature flow into the file given by \texttt{%
filename}. The results are curvature divided by radii values, and are given
by vectors of doubles. Most commonly, these vectors are taken from the
Approximator class after the flow is run. The \texttt{printResultsNumSteps}
function determines the number of vertices of the current triangulation and
the total number of steps are then derived from this and the size of the
vectors.

There are several ways to display the results. The \texttt{%
printResultsNumSteps} function groups by step but provides no labels and
does not print out radii, but instead curvautre divided by radii. The
purpose for this format is to create the ``Polygon flows'' in the GUI.
Therefore, it would be difficult for a human to read, but allows the
computer to do so much easier. An example is shown below.
\end{quotation}

\subsection*{Practicum}

\begin{quotation}
Example:{\small }
\end{quotation}

\begin{verbatim}
{\small 
  // Print the results of a curvature flow with Approximator app into file "ODEResult.txt"
}
{\small 
  printResultsNumSteps("./ODEResults.txt", app->radiiHistory, app->curvHistory);
}
{\small   
}
\end{verbatim}

\begin{quotation}
The output of such an example may then be{\small }
\end{quotation}

\begin{verbatim}
{\small          :
}
{\small          :
}
{\small       3.1425515910
}
{\small       3.1425294463
}
{\small       3.1425078130
}
 
{\small       3.1415926536
}
{\small       3.5987926375
}
{\small       3.5877016632
}
{\small       3.5768691746
}
{\small       3.5662905388
}
{\small          :
}
{\small          :
}
{\small   
}
\end{verbatim}

\subsection*{Limitations}

\begin{quotation}
Unlike the other print functions, the purpose of \texttt{printResultsNumSteps%
} is to only display the curvature divided by radii, and so is not limited
in the information it prints. On the otherhand, an overhaul of the entire
printing system would likely involve modifying this function.
\end{quotation}

\subsection*{Revisions}

\begin{itemize}
\item subversion 545, 9/29/08: Moved the printing of results out of calcFlow
and into a new function.

\item subversion 783, 6/18/09: Small modifications in response to changes in
the Approximator class.
\end{itemize}

\subsection*{Testing}

\begin{quotation}
The \texttt{printResultsNumSteps} function was tested by running multiple
curvature flows and printing the results. It was considered working when the
format of the data was as desired.
\end{quotation}

\subsection*{Future Work}

\begin{itemize}
\item 6/29 - Recreate the print functions to print more data and be more
flexible.

\item 6/29 - Move the print functions into the Approximator class.
\end{itemize}

% html: End of file: `clean.html'

%%%%% END OF DOCUMENT BODY %%%%%
% In the future, we might want to put some additional data here, such
% as when the documentation was converted from wiki to TeX.
%
%
%EndExpansion

\bigskip 

\bigskip 

%TCIMACRO{\QSubDoc{Include printResultsStep}{%TCIDATA{Version=5.00.0.2606}
%TCIDATA{LaTeXparent=0,0,functions.tex}
                      

%%%%% BEGINNING OF DOCUMENT BODY %%%%%
% html: Beginning of file: `clean.html'
% DOCTYPE HTML PUBLIC "-//W3C//DTD HTML 4.01//EN"
%  This is a (PRE) block.  Make sure it's left aligned or your toc title will be off. 

\section*{\texttt{printResultsStep}}

\label{f0}

\begin{quotation}
{\small }
\end{quotation}

\begin{verbatim}
{\small 
   void printResultsStep(char* fileName, vector<double>* radii, vector<double>* curvs)
}
{\small    
}
\end{verbatim}

\subsection*{Key Words}

\begin{quotation}
radii, curvatures, file, flow, step, print
\end{quotation}

\subsection*{Authors}

\begin{quotation}
Alex Henniges
\end{quotation}

\subsection*{Introduction}

\begin{quotation}
The \texttt{printResultsStep} function prints out the results of a curvature
flow, with the results grouped by each step of the flow. These results will
be written to the file given by \texttt{filename}.
\end{quotation}

\subsection*{Subsidiaries}

\begin{quotation}
Functions:

Global Variables:

Local Variables: \texttt{int vertSize}, \texttt{int numSteps}
\end{quotation}

\subsection*{Description}

\begin{quotation}
Prints the results of a curvature flow into the file given by \texttt{%
filename}. The results, that is, the radii and curvature values, are given
by vectors of doubles. Most commonly, these vectors are taken from the
Approximator class after the flow is run. The \texttt{printResultsStep}
function determines the number of vertices of the current triangulation and
the total number of steps are then derived from this and the size of the
vectors.

There are several ways to display the results. The \texttt{printResultsStep}
function groups by step. This means that for each step of the curvature
flow, the radii and curvature values for each vertex is printed. An example
is shown below. Other formats are given by printResultsVertex,
printResultsNum, print3DResultsStep.
\end{quotation}

\subsection*{Practicum}

\begin{quotation}
Example:{\small }
\end{quotation}

\begin{verbatim}
{\small 
  // Print the results of a curvature flow with Approximator app into file "ODEResult.txt"
}
{\small 
  printResultsStep("./ODEResults.txt", app->radiiHistory, app->curvHistory);
}
{\small   
}
\end{verbatim}

\begin{quotation}
The output of such an example may then be{\small }
\end{quotation}

\begin{verbatim}
{\small          :    
}
{\small          :
}
{\small        Vertex   4     0.5397923       2.1411007
}
{\small        Total Curvature: 12.5663706
}
 
{\small        Step     9     Radius          Curvature
}
{\small        -----------------------------------------------------
}
{\small        Vertex   1     1.1681789       3.9076805
}
{\small        Vertex   2     0.9668779       3.5170408
}
{\small        Vertex   3     0.7605802       2.9772908
}
{\small        Vertex   4     0.5451929       2.1643586
}
{\small        Total Curvature: 12.5663706
}
 
{\small        Step    10     Radius          Curvature
}
{\small        -----------------------------------------------------
}
{\small        Vertex   1     1.1592296       3.8916213
}
{\small          :
}
{\small          :
}
{\small   
}
\end{verbatim}

\subsection*{Limitations}

\begin{quotation}
Currently the \texttt{printResultsStep} function is limited in the
information it prints. As our curvature flow has evolved to record
additional information such as volumes, it may be time to explore a more
robust form for displaying results. As there is considerable dependence on
the Approximator for the data vectors, it may be wise to place this and
similar functions in the Approximator class.
\end{quotation}

\subsection*{Revisions}

\begin{itemize}
\item subversion 545, 9/29/08: Moved the printing of results out of calcFlow
and into a new function.

\item subversion 783, 6/18/09: Small modifications in response to changes in
the Approximator class.
\end{itemize}

\subsection*{Testing}

\begin{quotation}
The \texttt{printResultsStep} function was tested by running multiple
curvature flows and printing the results. It was considered working when the
format of the data was as desired.
\end{quotation}

\subsection*{Future Work}

\begin{itemize}
\item 6/29 - Recreate the print functions to print more data and be more
flexible.

\item 6/29 - Move the print functions into the Approximator class.
\end{itemize}

% html: End of file: `clean.html'

%%%%% END OF DOCUMENT BODY %%%%%
% In the future, we might want to put some additional data here, such
% as when the documentation was converted from wiki to TeX.
%
}}%
%BeginExpansion
%TCIDATA{Version=5.00.0.2606}
%TCIDATA{LaTeXparent=0,0,functions.tex}
                      

%%%%% BEGINNING OF DOCUMENT BODY %%%%%
% html: Beginning of file: `clean.html'
% DOCTYPE HTML PUBLIC "-//W3C//DTD HTML 4.01//EN"
%  This is a (PRE) block.  Make sure it's left aligned or your toc title will be off. 

\section*{\texttt{printResultsStep}}

\label{f0}

\begin{quotation}
{\small }
\end{quotation}

\begin{verbatim}
{\small 
   void printResultsStep(char* fileName, vector<double>* radii, vector<double>* curvs)
}
{\small    
}
\end{verbatim}

\subsection*{Key Words}

\begin{quotation}
radii, curvatures, file, flow, step, print
\end{quotation}

\subsection*{Authors}

\begin{quotation}
Alex Henniges
\end{quotation}

\subsection*{Introduction}

\begin{quotation}
The \texttt{printResultsStep} function prints out the results of a curvature
flow, with the results grouped by each step of the flow. These results will
be written to the file given by \texttt{filename}.
\end{quotation}

\subsection*{Subsidiaries}

\begin{quotation}
Functions:

Global Variables:

Local Variables: \texttt{int vertSize}, \texttt{int numSteps}
\end{quotation}

\subsection*{Description}

\begin{quotation}
Prints the results of a curvature flow into the file given by \texttt{%
filename}. The results, that is, the radii and curvature values, are given
by vectors of doubles. Most commonly, these vectors are taken from the
Approximator class after the flow is run. The \texttt{printResultsStep}
function determines the number of vertices of the current triangulation and
the total number of steps are then derived from this and the size of the
vectors.

There are several ways to display the results. The \texttt{printResultsStep}
function groups by step. This means that for each step of the curvature
flow, the radii and curvature values for each vertex is printed. An example
is shown below. Other formats are given by printResultsVertex,
printResultsNum, print3DResultsStep.
\end{quotation}

\subsection*{Practicum}

\begin{quotation}
Example:{\small }
\end{quotation}

\begin{verbatim}
{\small 
  // Print the results of a curvature flow with Approximator app into file "ODEResult.txt"
}
{\small 
  printResultsStep("./ODEResults.txt", app->radiiHistory, app->curvHistory);
}
{\small   
}
\end{verbatim}

\begin{quotation}
The output of such an example may then be{\small }
\end{quotation}

\begin{verbatim}
{\small          :    
}
{\small          :
}
{\small        Vertex   4     0.5397923       2.1411007
}
{\small        Total Curvature: 12.5663706
}
 
{\small        Step     9     Radius          Curvature
}
{\small        -----------------------------------------------------
}
{\small        Vertex   1     1.1681789       3.9076805
}
{\small        Vertex   2     0.9668779       3.5170408
}
{\small        Vertex   3     0.7605802       2.9772908
}
{\small        Vertex   4     0.5451929       2.1643586
}
{\small        Total Curvature: 12.5663706
}
 
{\small        Step    10     Radius          Curvature
}
{\small        -----------------------------------------------------
}
{\small        Vertex   1     1.1592296       3.8916213
}
{\small          :
}
{\small          :
}
{\small   
}
\end{verbatim}

\subsection*{Limitations}

\begin{quotation}
Currently the \texttt{printResultsStep} function is limited in the
information it prints. As our curvature flow has evolved to record
additional information such as volumes, it may be time to explore a more
robust form for displaying results. As there is considerable dependence on
the Approximator for the data vectors, it may be wise to place this and
similar functions in the Approximator class.
\end{quotation}

\subsection*{Revisions}

\begin{itemize}
\item subversion 545, 9/29/08: Moved the printing of results out of calcFlow
and into a new function.

\item subversion 783, 6/18/09: Small modifications in response to changes in
the Approximator class.
\end{itemize}

\subsection*{Testing}

\begin{quotation}
The \texttt{printResultsStep} function was tested by running multiple
curvature flows and printing the results. It was considered working when the
format of the data was as desired.
\end{quotation}

\subsection*{Future Work}

\begin{itemize}
\item 6/29 - Recreate the print functions to print more data and be more
flexible.

\item 6/29 - Move the print functions into the Approximator class.
\end{itemize}

% html: End of file: `clean.html'

%%%%% END OF DOCUMENT BODY %%%%%
% In the future, we might want to put some additional data here, such
% as when the documentation was converted from wiki to TeX.
%
%
%EndExpansion

\bigskip 

\bigskip 

%TCIMACRO{%
%\QSubDoc{Include printResultsVertex}{%TCIDATA{Version=5.00.0.2606}
%TCIDATA{LaTeXparent=0,0,functions.tex}
                      

%%%%% BEGINNING OF DOCUMENT BODY %%%%%
% html: Beginning of file: `clean.html'
% DOCTYPE HTML PUBLIC "-//W3C//DTD HTML 4.01//EN"
%  This is a (PRE) block.  Make sure it's left aligned or your toc title will be off. 

\section*{\texttt{printResultsVertex}}

\label{f0}

\begin{quotation}
{\small }
\end{quotation}

\begin{verbatim}
{\small 
   void printResultsVertex(char* fileName, vector<double>* radii, vector<double>* curvs)
}
{\small    
}
\end{verbatim}

\subsection*{Key Words}

\begin{quotation}
radii, curvatures, file, flow, vertex, print
\end{quotation}

\subsection*{Authors}

\begin{quotation}
Alex Henniges
\end{quotation}

\subsection*{Introduction}

\begin{quotation}
The \texttt{printResultsVertex} function prints out the results of a
curvature flow, with the results grouped by each vertex of the
triangulation. These results will be written to the file given by \texttt{%
filename}.
\end{quotation}

\subsection*{Subsidiaries}

\begin{quotation}
Functions:

Global Variables:

Local Variables: \texttt{int vertSize}, \texttt{int numSteps}
\end{quotation}

\subsection*{Description}

\begin{quotation}
Prints the results of a curvature flow into the file given by \texttt{%
filename}. The results, that is, the radii and curvature values, are given
by vectors of doubles. Most commonly, these vectors are taken from the
Approximator class after the flow is run. The \texttt{printResultsVertex}
function determines the number of vertices of the current triangulation and
the total number of steps are then derived from this and the size of the
vectors.

There are several ways to display the results. The \texttt{printResultsVertex%
} function groups by vertex. This means that for each vertex of the
triangulation, the radii and curvature values for each step is given. An
example is shown below. Other formats are given by printResultsStep,
printResultsNum, print3DResultsStep.
\end{quotation}

\subsection*{Practicum}

\begin{quotation}
Example:{\small }
\end{quotation}

\begin{verbatim}
{\small 
  // Print the results of a curvature flow with Approximator app into file "ODEResult.txt"
}
{\small 
  printResultsVertex("./ODEResults.txt", app->radiiHistory, app->curvHistory);
}
{\small   
}
\end{verbatim}

\begin{quotation}
The output of such an example may then be{\small }
\end{quotation}

\begin{verbatim}
{\small          : 
}
{\small          :
}
{\small        Step  298        0.8272161        3.1425516
}
{\small        Step  299        0.8272081        3.1425294
}
{\small        Step  300        0.8272004        3.1425078
}
 
{\small        Vertex:   2        Radius                Curv
}
 
{\small        ---------------------------------
}
{\small        Step    0        1.0000000        3.1415927
}
{\small        Step    1        1.0000000        3.5987926
}
{\small        Step    2        0.9954280        3.5877017
}
{\small        Step    3        0.9909873        3.5768692
}
{\small        Step    4        0.9866738        3.5662905
}
{\small          :
}
{\small          :
}
{\small   
}
\end{verbatim}

\subsection*{Limitations}

\begin{quotation}
Currently the \texttt{printResultsVertex} function is limited in the
information it prints. As our curvature flow has evolved to record
additional information such as volumes, it may be time to explore a more
robust form for displaying results. As there is considerable dependence on
the Approximator for the data vectors, it may be wise to place this and
similar functions in the Approximator class.
\end{quotation}

\subsection*{Revisions}

\begin{itemize}
\item subversion 545, 9/29/08: Moved the printing of results out of calcFlow
and into a new function.

\item subversion 783, 6/18/09: Small modifications in response to changes in
the Approximator class.
\end{itemize}

\subsection*{Testing}

\begin{quotation}
The \texttt{printResultsVertex} function was tested by running multiple
curvature flows and printing the results. It was considered working when the
format of the data was as desired.
\end{quotation}

\subsection*{Future Work}

\begin{itemize}
\item 6/29 - Recreate the print functions to print more data and be more
flexible.

\item 6/29 - Move the print functions into the Approximator class.
\end{itemize}

% html: End of file: `clean.html'

%%%%% END OF DOCUMENT BODY %%%%%
% In the future, we might want to put some additional data here, such
% as when the documentation was converted from wiki to TeX.
%
}}%
%BeginExpansion
%TCIDATA{Version=5.00.0.2606}
%TCIDATA{LaTeXparent=0,0,functions.tex}
                      

%%%%% BEGINNING OF DOCUMENT BODY %%%%%
% html: Beginning of file: `clean.html'
% DOCTYPE HTML PUBLIC "-//W3C//DTD HTML 4.01//EN"
%  This is a (PRE) block.  Make sure it's left aligned or your toc title will be off. 

\section*{\texttt{printResultsVertex}}

\label{f0}

\begin{quotation}
{\small }
\end{quotation}

\begin{verbatim}
{\small 
   void printResultsVertex(char* fileName, vector<double>* radii, vector<double>* curvs)
}
{\small    
}
\end{verbatim}

\subsection*{Key Words}

\begin{quotation}
radii, curvatures, file, flow, vertex, print
\end{quotation}

\subsection*{Authors}

\begin{quotation}
Alex Henniges
\end{quotation}

\subsection*{Introduction}

\begin{quotation}
The \texttt{printResultsVertex} function prints out the results of a
curvature flow, with the results grouped by each vertex of the
triangulation. These results will be written to the file given by \texttt{%
filename}.
\end{quotation}

\subsection*{Subsidiaries}

\begin{quotation}
Functions:

Global Variables:

Local Variables: \texttt{int vertSize}, \texttt{int numSteps}
\end{quotation}

\subsection*{Description}

\begin{quotation}
Prints the results of a curvature flow into the file given by \texttt{%
filename}. The results, that is, the radii and curvature values, are given
by vectors of doubles. Most commonly, these vectors are taken from the
Approximator class after the flow is run. The \texttt{printResultsVertex}
function determines the number of vertices of the current triangulation and
the total number of steps are then derived from this and the size of the
vectors.

There are several ways to display the results. The \texttt{printResultsVertex%
} function groups by vertex. This means that for each vertex of the
triangulation, the radii and curvature values for each step is given. An
example is shown below. Other formats are given by printResultsStep,
printResultsNum, print3DResultsStep.
\end{quotation}

\subsection*{Practicum}

\begin{quotation}
Example:{\small }
\end{quotation}

\begin{verbatim}
{\small 
  // Print the results of a curvature flow with Approximator app into file "ODEResult.txt"
}
{\small 
  printResultsVertex("./ODEResults.txt", app->radiiHistory, app->curvHistory);
}
{\small   
}
\end{verbatim}

\begin{quotation}
The output of such an example may then be{\small }
\end{quotation}

\begin{verbatim}
{\small          : 
}
{\small          :
}
{\small        Step  298        0.8272161        3.1425516
}
{\small        Step  299        0.8272081        3.1425294
}
{\small        Step  300        0.8272004        3.1425078
}
 
{\small        Vertex:   2        Radius                Curv
}
 
{\small        ---------------------------------
}
{\small        Step    0        1.0000000        3.1415927
}
{\small        Step    1        1.0000000        3.5987926
}
{\small        Step    2        0.9954280        3.5877017
}
{\small        Step    3        0.9909873        3.5768692
}
{\small        Step    4        0.9866738        3.5662905
}
{\small          :
}
{\small          :
}
{\small   
}
\end{verbatim}

\subsection*{Limitations}

\begin{quotation}
Currently the \texttt{printResultsVertex} function is limited in the
information it prints. As our curvature flow has evolved to record
additional information such as volumes, it may be time to explore a more
robust form for displaying results. As there is considerable dependence on
the Approximator for the data vectors, it may be wise to place this and
similar functions in the Approximator class.
\end{quotation}

\subsection*{Revisions}

\begin{itemize}
\item subversion 545, 9/29/08: Moved the printing of results out of calcFlow
and into a new function.

\item subversion 783, 6/18/09: Small modifications in response to changes in
the Approximator class.
\end{itemize}

\subsection*{Testing}

\begin{quotation}
The \texttt{printResultsVertex} function was tested by running multiple
curvature flows and printing the results. It was considered working when the
format of the data was as desired.
\end{quotation}

\subsection*{Future Work}

\begin{itemize}
\item 6/29 - Recreate the print functions to print more data and be more
flexible.

\item 6/29 - Move the print functions into the Approximator class.
\end{itemize}

% html: End of file: `clean.html'

%%%%% END OF DOCUMENT BODY %%%%%
% In the future, we might want to put some additional data here, such
% as when the documentation was converted from wiki to TeX.
%
%
%EndExpansion

\bigskip 

\bigskip 

%TCIMACRO{%
%\QSubDoc{Include Total_Volume}{%TCIDATA{LaTeXparent=0,0,functions.tex}}}%
%BeginExpansion
%TCIDATA{LaTeXparent=0,0,functions.tex}%
%EndExpansion

\bigskip

\bigskip

%TCIMACRO{%
%\QSubDoc{Include Volume_Partial}{%TCIDATA{Version=5.00.0.2606}
%TCIDATA{LaTeXparent=1,1,functions.tex}
                      

\section*{\texttt{VolumePartial::VolumePartial}}

\subsection*{Function Prototype}

\texttt{double VolumePartial(Vertex v\_i, Tetra t)}

\subsection*{Key Words}

Volume, tetrahedron, vertex, radius, Cayley-Menger determinant, standard
form, geoquant.

\subsection*{Authors}

Daniel Champion

\subsection*{Introduction}

\texttt{VolumePartial} calculates the partial derivative of the volume of a
tetrahedron with respect to the logarithm of the radius of a vertex.

\subsection*{Subsidiaries}

\textbf{Functions:} \ 

\texttt{listDifference}

\texttt{listIntersection}

\texttt{Simplex::isAdjVertex}

\textbf{Global Variables:} \ radii, etas.

\textbf{Local Variables:}

\subsection*{Description}

The volume of a tetrahedron only depends on the lengths of its edges as
calculated from the Cayley-Menger determinant. \ Thus for a given
tetrahedron $t$, it's partial derivatives with respect to $\log $ radii will
vanish except for those radii corresponding to the vertices of $t$. \ The
function \texttt{isAdjVertex} of the simplex class determines this
condition. \ \texttt{VolumePartial} then proceeds with an initialization
procedure that labels the vertices and edges (radii and etas) in standard
form. \ Specifically, \texttt{Volume\_Partial} receives as inputs an integer 
\texttt{i} corresponding to a vertex (in the vertex table) which is labeled
vertex v1. \ The remaining vertices are labeled $v2,v3,v4$, and the edges $%
e12,e13,e14,e23,e24,e34$ are labeled preserving the structure implied by the
assignment of the vertices. \ The radii $r_{1}$, $r_{2}$,... and eta values $%
Eta_{12},Eta_{13},$... are assigned to the corresponding vertices and edges.
\ 

The formula for the partial derivative in terms of these standard form
variables was calculated in Mathematica using the Cayley-Menger determinant,
that is:%
\begin{equation*}
288V^{2}=\det\left[ 
\begin{array}{ccccc}
0 & 1 & 1 & 1 & 1 \\ 
1 & 0 & L_{12}^{2} & L_{13}^{2} & L_{14}^{2} \\ 
1 & L_{12}^{2} & 0 & L_{23}^{2} & L_{24}^{2} \\ 
1 & L_{13}^{2} & L_{23}^{2} & 0 & L_{34}^{2} \\ 
1 & L_{14}^{2} & L_{24}^{2} & L_{34}^{2} & 0%
\end{array}
\right] ,
\end{equation*}
where the lengths were determined from the radii and eta values using the
formula%
\begin{equation*}
L_{ij}^{2}=r_{i}^{2}+r_{j}^{2}+2r_{i}r_{j}Eta_{ij}.
\end{equation*}

The formula obtained from Mathematica was outputted into the C programming
language using the function CForm. \ 

This function was designed for use in the optimization of the
Einstein-Hilbert-Regge functional using Newton's method. \ In this procedure
the gradient of the EHR functional is needed which contains the partial
derivatives of the volume. \ 

\subsection*{Practicum}

Usage:

\texttt{VolumePartial(Vertex v\_i, Tetra t)}

The integer \texttt{i} corresponds to a vertex in the vertex table, that is%
\begin{equation*}
\text{\texttt{VolumePartial (v\_i, t)} }=\frac{\partial }{\partial \log r_{i}%
}Volume(t).
\end{equation*}

\subsection*{Limitations}

\texttt{VolumePartial} was designed to output the correct partial derivative
for any integer $i$ in the vertex table and any tetrahedron $t$ in the
triangulation. \ 

\subsection*{Revisions}

subversion 757, 7/7/09, \texttt{VolumePartial} created within
NewtonsMethod.cpp.

subversion 1055, 3/12/10, \texttt{VolumePartial} converted to a geoquant.

\subsection*{Testing}

The partial derivative of volume of several known tetrahedra were calculated
using \texttt{Volume\_Partial} and verified using Mathematica.

\subsection*{Future Work}

The procedure that initializes the tetrahedron into standard form should be
removed from this program and placed elsewhere. \ There are several
occurrences of this type of procedure that should be consolidated. \ 
}}%
%BeginExpansion
%TCIDATA{Version=5.00.0.2606}
%TCIDATA{LaTeXparent=1,1,functions.tex}
                      

\section*{\texttt{VolumePartial::VolumePartial}}

\subsection*{Function Prototype}

\texttt{double VolumePartial(Vertex v\_i, Tetra t)}

\subsection*{Key Words}

Volume, tetrahedron, vertex, radius, Cayley-Menger determinant, standard
form, geoquant.

\subsection*{Authors}

Daniel Champion

\subsection*{Introduction}

\texttt{VolumePartial} calculates the partial derivative of the volume of a
tetrahedron with respect to the logarithm of the radius of a vertex.

\subsection*{Subsidiaries}

\textbf{Functions:} \ 

\texttt{listDifference}

\texttt{listIntersection}

\texttt{Simplex::isAdjVertex}

\textbf{Global Variables:} \ radii, etas.

\textbf{Local Variables:}

\subsection*{Description}

The volume of a tetrahedron only depends on the lengths of its edges as
calculated from the Cayley-Menger determinant. \ Thus for a given
tetrahedron $t$, it's partial derivatives with respect to $\log $ radii will
vanish except for those radii corresponding to the vertices of $t$. \ The
function \texttt{isAdjVertex} of the simplex class determines this
condition. \ \texttt{VolumePartial} then proceeds with an initialization
procedure that labels the vertices and edges (radii and etas) in standard
form. \ Specifically, \texttt{Volume\_Partial} receives as inputs an integer 
\texttt{i} corresponding to a vertex (in the vertex table) which is labeled
vertex v1. \ The remaining vertices are labeled $v2,v3,v4$, and the edges $%
e12,e13,e14,e23,e24,e34$ are labeled preserving the structure implied by the
assignment of the vertices. \ The radii $r_{1}$, $r_{2}$,... and eta values $%
Eta_{12},Eta_{13},$... are assigned to the corresponding vertices and edges.
\ 

The formula for the partial derivative in terms of these standard form
variables was calculated in Mathematica using the Cayley-Menger determinant,
that is:%
\begin{equation*}
288V^{2}=\det\left[ 
\begin{array}{ccccc}
0 & 1 & 1 & 1 & 1 \\ 
1 & 0 & L_{12}^{2} & L_{13}^{2} & L_{14}^{2} \\ 
1 & L_{12}^{2} & 0 & L_{23}^{2} & L_{24}^{2} \\ 
1 & L_{13}^{2} & L_{23}^{2} & 0 & L_{34}^{2} \\ 
1 & L_{14}^{2} & L_{24}^{2} & L_{34}^{2} & 0%
\end{array}
\right] ,
\end{equation*}
where the lengths were determined from the radii and eta values using the
formula%
\begin{equation*}
L_{ij}^{2}=r_{i}^{2}+r_{j}^{2}+2r_{i}r_{j}Eta_{ij}.
\end{equation*}

The formula obtained from Mathematica was outputted into the C programming
language using the function CForm. \ 

This function was designed for use in the optimization of the
Einstein-Hilbert-Regge functional using Newton's method. \ In this procedure
the gradient of the EHR functional is needed which contains the partial
derivatives of the volume. \ 

\subsection*{Practicum}

Usage:

\texttt{VolumePartial(Vertex v\_i, Tetra t)}

The integer \texttt{i} corresponds to a vertex in the vertex table, that is%
\begin{equation*}
\text{\texttt{VolumePartial (v\_i, t)} }=\frac{\partial }{\partial \log r_{i}%
}Volume(t).
\end{equation*}

\subsection*{Limitations}

\texttt{VolumePartial} was designed to output the correct partial derivative
for any integer $i$ in the vertex table and any tetrahedron $t$ in the
triangulation. \ 

\subsection*{Revisions}

subversion 757, 7/7/09, \texttt{VolumePartial} created within
NewtonsMethod.cpp.

subversion 1055, 3/12/10, \texttt{VolumePartial} converted to a geoquant.

\subsection*{Testing}

The partial derivative of volume of several known tetrahedra were calculated
using \texttt{Volume\_Partial} and verified using Mathematica.

\subsection*{Future Work}

The procedure that initializes the tetrahedron into standard form should be
removed from this program and placed elsewhere. \ There are several
occurrences of this type of procedure that should be consolidated. \ 
%
%EndExpansion

\bigskip

\bigskip

%TCIMACRO{%
%\QSubDoc{Include Volume_Second_Partial}{%TCIDATA{Version=5.00.0.2606}
%TCIDATA{LaTeXparent=1,1,functions.tex}
                      

\section*{\texttt{VolumeSecondPartial::VolumeSecondPartial}}

\subsection*{Function Prototype}

\texttt{double VolumeSecondPartial(Vertex v\_i, Vertex v\_j, Tetra t)}

\subsection*{Key Words}

Volume, Hessian Matrix, Newton's Method, partial derivative,
Einstein-Hilbert-Regge functional, geoquant.

\subsection*{Authors}

Daniel Champion

\subsection*{Introduction}

\texttt{VolumeSecondPartial} calculates the second order partial derivatives
of the volume of a tetrahedron with respect to log radii for all pairs of
indices (not necessarily distinct) in the vertex table. \ 

\subsection*{Subsidiaries}

\textbf{Functions:}

\texttt{listDifference}

\texttt{listIntersection}

\texttt{Simplex::isAdjVertex}

\textbf{Global Variables:} \ radii, etas.

\textbf{Local Variables:}

\subsection*{Description}

The volume of a tetrahedron only depends on the lengths of its edges as
calculated from the Cayley-Menger determinant. \ Thus for a given
tetrahedron $t$, it's second order partial derivatives with respect to $\log 
$ radii will vanish except for pairs of radii (not necessarily distinct)
corresponding to the vertices of $t$. \ The first step in the implementation
of \texttt{VolumeSecondPartial} is the determination of the following
trichotomy for a pair $\left\{ i,j\right\} $ of indices in the vertex table:%
\begin{equation*}
\begin{array}{l}
\text{A. \ }i=j\text{ and }i\text{ is a vertex of tetrahedron }t \\ 
\text{B. \ }i\neq j\text{ and both }i\text{ and }j\text{ belong to }t \\ 
\text{C. \ at least one of }i\text{ or }j\text{ doesn't belong to }t.%
\end{array}%
\end{equation*}

Each condition of the trichotomy requires a distinct calculation to
determine the desired partial derivative. \ Nevertheless, the next step in
the implementation is to place the tetrahedron in "standard form" relative
to the indices $i$ and $j$ (for conditions A and B only). \ More
specifically, for condition A the radius for vertex $i$ is stored as $r_{1}$%
, and the remaining radii of the tetrahedron $t$ are assigned $r_{2},r_{3},$
and $r_{4}$ in no particular order. \ The eta values $Eta_{12},Eta_{13},...$
are then assigned preserving the preceding assignments. \ In the case of
condition B, the radii at vertices $i$ and $j$ are assigned to $r_{1}$ and $%
r_{2}$ respectively, and $r_{3}$, and $r_{4}$ the remaining radii of $t$. \
The eta values $Eta_{12},Eta_{13},...$ are again assigned preserving the
preceding assignments.

The formulas for the second order partial derivatives in terms of these
standard form variables was calculated in Mathematica using the
Cayley-Menger determinant, that is:%
\begin{equation*}
288V^{2}=\det\left[ 
\begin{array}{ccccc}
0 & 1 & 1 & 1 & 1 \\ 
1 & 0 & L_{12}^{2} & L_{13}^{2} & L_{14}^{2} \\ 
1 & L_{12}^{2} & 0 & L_{23}^{2} & L_{24}^{2} \\ 
1 & L_{13}^{2} & L_{23}^{2} & 0 & L_{34}^{2} \\ 
1 & L_{14}^{2} & L_{24}^{2} & L_{34}^{2} & 0%
\end{array}
\right] ,
\end{equation*}
where the lengths were determined from the radii and eta values using the
formula%
\begin{equation*}
L_{ij}^{2}=r_{i}^{2}+r_{j}^{2}+2r_{i}r_{j}Eta_{ij}.
\end{equation*}

The formula obtained from Mathematica was outputted into the C programming
language using the function CForm.

This function was designed for use in the optimization of the
Einstein-Hilbert-Regge functional using Newton's method. \ In this procedure
the Hessian matrix of the normalized EHR functional is needed, each entry of
which uses the second order partial derivatives of volume. \ See the entry
on \texttt{EHRSecondPartial}.

\subsection*{Practicum}

Usage:

\texttt{VolumeSecondPartial (Vertex v\_i, Vertex v\_j, Tetra t)}

The integers \texttt{i} and \texttt{j} correspond to vertices in the vertex
table and \texttt{t} is a tetrahedron in the triangulation. \ Specifically
the function returns:%
\begin{equation*}
\text{\texttt{VolumeSecondPartial(v\_i, v\_j, t)} }=\frac{\partial ^{2}}{%
\partial \log r_{i}\partial \log r_{j}}Volume(t).
\end{equation*}

\subsection*{Limitations}

\texttt{VolumeSecondPartial} is fully operational with no know limitations.
\ The function will output appropriate values when given indices $i,$ and $j$
in the vertex table, and a tetrahedron $t$. \ 

\subsection*{Revisions}

subversion 757, 7/9/09, \texttt{VolumeSecondPartial} created.

subversion 1055, 3/12/10, \texttt{VolumeSecondPartial} converted to a
geoquant.

\subsection*{Testing}

Several trials were run outputting the values of \texttt{VolumeSecondPartial}
for a variety of vertices and tetrahedra. \ These values were compared with
calculations performed on Mathematica. \ 

\subsection*{Future Work}

None planned.
}}%
%BeginExpansion
%TCIDATA{Version=5.00.0.2606}
%TCIDATA{LaTeXparent=1,1,functions.tex}
                      

\section*{\texttt{VolumeSecondPartial::VolumeSecondPartial}}

\subsection*{Function Prototype}

\texttt{double VolumeSecondPartial(Vertex v\_i, Vertex v\_j, Tetra t)}

\subsection*{Key Words}

Volume, Hessian Matrix, Newton's Method, partial derivative,
Einstein-Hilbert-Regge functional, geoquant.

\subsection*{Authors}

Daniel Champion

\subsection*{Introduction}

\texttt{VolumeSecondPartial} calculates the second order partial derivatives
of the volume of a tetrahedron with respect to log radii for all pairs of
indices (not necessarily distinct) in the vertex table. \ 

\subsection*{Subsidiaries}

\textbf{Functions:}

\texttt{listDifference}

\texttt{listIntersection}

\texttt{Simplex::isAdjVertex}

\textbf{Global Variables:} \ radii, etas.

\textbf{Local Variables:}

\subsection*{Description}

The volume of a tetrahedron only depends on the lengths of its edges as
calculated from the Cayley-Menger determinant. \ Thus for a given
tetrahedron $t$, it's second order partial derivatives with respect to $\log 
$ radii will vanish except for pairs of radii (not necessarily distinct)
corresponding to the vertices of $t$. \ The first step in the implementation
of \texttt{VolumeSecondPartial} is the determination of the following
trichotomy for a pair $\left\{ i,j\right\} $ of indices in the vertex table:%
\begin{equation*}
\begin{array}{l}
\text{A. \ }i=j\text{ and }i\text{ is a vertex of tetrahedron }t \\ 
\text{B. \ }i\neq j\text{ and both }i\text{ and }j\text{ belong to }t \\ 
\text{C. \ at least one of }i\text{ or }j\text{ doesn't belong to }t.%
\end{array}%
\end{equation*}

Each condition of the trichotomy requires a distinct calculation to
determine the desired partial derivative. \ Nevertheless, the next step in
the implementation is to place the tetrahedron in "standard form" relative
to the indices $i$ and $j$ (for conditions A and B only). \ More
specifically, for condition A the radius for vertex $i$ is stored as $r_{1}$%
, and the remaining radii of the tetrahedron $t$ are assigned $r_{2},r_{3},$
and $r_{4}$ in no particular order. \ The eta values $Eta_{12},Eta_{13},...$
are then assigned preserving the preceding assignments. \ In the case of
condition B, the radii at vertices $i$ and $j$ are assigned to $r_{1}$ and $%
r_{2}$ respectively, and $r_{3}$, and $r_{4}$ the remaining radii of $t$. \
The eta values $Eta_{12},Eta_{13},...$ are again assigned preserving the
preceding assignments.

The formulas for the second order partial derivatives in terms of these
standard form variables was calculated in Mathematica using the
Cayley-Menger determinant, that is:%
\begin{equation*}
288V^{2}=\det\left[ 
\begin{array}{ccccc}
0 & 1 & 1 & 1 & 1 \\ 
1 & 0 & L_{12}^{2} & L_{13}^{2} & L_{14}^{2} \\ 
1 & L_{12}^{2} & 0 & L_{23}^{2} & L_{24}^{2} \\ 
1 & L_{13}^{2} & L_{23}^{2} & 0 & L_{34}^{2} \\ 
1 & L_{14}^{2} & L_{24}^{2} & L_{34}^{2} & 0%
\end{array}
\right] ,
\end{equation*}
where the lengths were determined from the radii and eta values using the
formula%
\begin{equation*}
L_{ij}^{2}=r_{i}^{2}+r_{j}^{2}+2r_{i}r_{j}Eta_{ij}.
\end{equation*}

The formula obtained from Mathematica was outputted into the C programming
language using the function CForm.

This function was designed for use in the optimization of the
Einstein-Hilbert-Regge functional using Newton's method. \ In this procedure
the Hessian matrix of the normalized EHR functional is needed, each entry of
which uses the second order partial derivatives of volume. \ See the entry
on \texttt{EHRSecondPartial}.

\subsection*{Practicum}

Usage:

\texttt{VolumeSecondPartial (Vertex v\_i, Vertex v\_j, Tetra t)}

The integers \texttt{i} and \texttt{j} correspond to vertices in the vertex
table and \texttt{t} is a tetrahedron in the triangulation. \ Specifically
the function returns:%
\begin{equation*}
\text{\texttt{VolumeSecondPartial(v\_i, v\_j, t)} }=\frac{\partial ^{2}}{%
\partial \log r_{i}\partial \log r_{j}}Volume(t).
\end{equation*}

\subsection*{Limitations}

\texttt{VolumeSecondPartial} is fully operational with no know limitations.
\ The function will output appropriate values when given indices $i,$ and $j$
in the vertex table, and a tetrahedron $t$. \ 

\subsection*{Revisions}

subversion 757, 7/9/09, \texttt{VolumeSecondPartial} created.

subversion 1055, 3/12/10, \texttt{VolumeSecondPartial} converted to a
geoquant.

\subsection*{Testing}

Several trials were run outputting the values of \texttt{VolumeSecondPartial}
for a variety of vertices and tetrahedra. \ These values were compared with
calculations performed on Mathematica. \ 

\subsection*{Future Work}

None planned.
%
%EndExpansion
%
%EndExpansion

%TCIMACRO{\QSubDoc{Include classes}{%TCIDATA{Version=5.00.0.2606}
%TCIDATA{LaTeXparent=0,0,geocam.tex}
                      

\chapter{Classes}

%TCIMACRO{\QSubDoc{Include Approximator}{%TCIDATA{Version=5.00.0.2606}
%TCIDATA{LaTeXparent=0,0,classes.tex}
                      

%%%%% BEGINNING OF DOCUMENT BODY %%%%%
% html: Beginning of file: `clean.html'
% DOCTYPE HTML PUBLIC "-//W3C//DTD HTML 4.01//EN"
%  This is a (PRE) block.  Make sure it's left aligned or your toc title will be off. 

\section*{\texttt{Approximator}}

\label{f0}

\subsection*{Key Words}

\begin{quotation}
curvature flow, differential equations, Euler's method, Runga Kutta
\end{quotation}

\subsection*{Authors}

\begin{itemize}
\item Joe Thomas

\item Alex Henniges
\end{itemize}

\subsection*{Introduction}

\begin{quotation}
The Approximator class runs a curvature flow using one of several methods.
The class itself is abstract and the method is chosen by the instantiating
object.
\end{quotation}

\subsection*{Subsidiaries}

\begin{quotation}
Functions:
\end{quotation}

\begin{itemize}
\item \mbox{$[$}Functions\#Approximator::run run\mbox{$]$}
\end{itemize}

\begin{quotation}
Sub-classes:
\end{quotation}

\begin{itemize}
\item EulerApprox

\item RungaApprox
\end{itemize}

\begin{quotation}
Public Variables:
\end{quotation}

\begin{itemize}
\item radiiHistory

\item curvHistory

\item areaHistory

\item volumeHistory
\end{itemize}

\subsection*{Description}

\begin{quotation}
The \texttt{Approximator} class is the shell for running a curvature flow.
The class provides the functionality to determine what system of
differential equations to use, how to perform a step, and even which values
to record for later use. The system is provided by the user at run-time and
is defined to be a function that takes in an empty array of \texttt{double}s
and fills the array with the values calculated in the system of equations.
The \texttt{Approximator} class is abstract with an abstract method \texttt{%
step}. A class that extends \texttt{Approximator} implements \texttt{step}
with the method of approximation to use (i.e Euler's method). Lastly, the 
\texttt{Approximator} stores values after each \texttt{step} of a flow
according to which values were requested at construction. Values include
radii, curvatures, areas, and volumes. These histories can then be accessed
directly from the \texttt{Approximator} object.
\end{quotation}

\subsection*{Constructor}

\begin{quotation}
The constructor takes in a function that defines a system of differential
equations and a string of characters representing what histories to record.
For the function to be a \texttt{sysdiffeq} it must not return a value and
its only parameter is an array of doubles that will be filled in with values
after the function completes. The history string must be nul-terminated and
consisting of only valid characters. The valid characters currently are:
\end{quotation}

\begin{itemize}
\item r - Record radii

\item 2 - Record two-dimensional curvatures

\item 3 - Record three-dimensional curvatures

\item a - Record areas

\item v - Record volumes
\end{itemize}

\begin{quotation}
One cannot list both two- and three-dimensional curvatures.{\small }
\end{quotation}

\begin{verbatim}
{\small   typedef void (*sysdiffeq)(double derivs[]);
}
{\small   Approximator(sysdiffeq funct, char* histories);
}
{\small   
}
\end{verbatim}

\subsection*{Practicum}

\begin{quotation}
This example will show how to run a Yamabe curvature flow on the pentachoron
using precision and accuracy bounds (see \mbox{$[$}Functions\#run run%
\mbox{$]$}) while recording radii, curvatures, and volumes. It will show how
to initialize the system and also how to print out results at the end.%
{\small }
\end{quotation}

\begin{verbatim}
{\small int main(int argc, char** argv) {
}
{\small    map<int, Vertex>::iterator vit;
}
{\small    map<int, Edge>::iterator eit;
}
{\small    map<int, Face>::iterator fit;
}
{\small    map<int, Tetra>::iterator tit;
}
{\small      
}
{\small    vector<int> edges;
}
{\small    vector<int> faces;
}
{\small    vector<int> tetras;
}
{\small     
}
{\small     
}
{\small    time_t start, end;
}
{\small    
}
{\small    // File to read in triangulation from.
}
{\small 
   char from[] = "./Triangulation Files/3D Manifolds/Lutz Format/pentachoron.txt";
}
{\small    // File to convert to proper format.
}
{\small    char to[] = "./Triangulation Files/manifold converted.txt";
}
{\small    // Convert, then read in triangulation.
}
{\small    make3DTriangulationFile(from, to);
}
{\small    read3DTriangulationFile(to);
}
 
{\small    int vertSize = Triangulation::vertexTable.size();
}
{\small    int edgeSize = Triangulation::edgeTable.size();
}
{\small    
}
{\small    // Set the radii
}
{\small    for(int i = 1; i <= vertSize; i++) {
}
{\small 
      Radius::At(Triangulation::vertexTable[i])->setValue(1 + (0.5 - i/5.0) );        
}
{\small    }
}
{\small    // Set the etas
}
{\small    for(int i = 1; i <= edgeSize; i++) {
}
{\small        Eta::At(Triangulation::edgeTable[i])->setValue(1.0);
}
{\small    }
}
 
{\small 
   // Construct an Approximator object that uses the Euler method and Yamabe flow while
}
{\small    // recording radii, 3D curvatures, and volumes.
}
{\small    Approximator *app = new EulerApprox((sysdiffeq) Yamabe, "r3v");
}
 
{\small 
   // Run the Yamabe flow with precision and accuracy bounds of 0.0001 and stepsize of 0.01
}
{\small    app->run(0.0001, 0.0001, 0.01);
}
 
{\small    // Print out radii, curvatures and volumes
}
{\small 
   printResultsStep("./Triangulation Files/ODE Result.txt", &(app->radiiHistory), &(app->curvHistory));
}
{\small 
   printResultsVolumes("./Triangulation Files/Volumes.txt", &(app->volumeHistory));
}
 
{\small    return 0;
}
{\small }
}
\end{verbatim}

% html: End of file: `clean.html'

%%%%% END OF DOCUMENT BODY %%%%%
% In the future, we might want to put some additional data here, such
% as when the documentation was converted from wiki to TeX.
%
}}%
%BeginExpansion
%TCIDATA{Version=5.00.0.2606}
%TCIDATA{LaTeXparent=0,0,classes.tex}
                      

%%%%% BEGINNING OF DOCUMENT BODY %%%%%
% html: Beginning of file: `clean.html'
% DOCTYPE HTML PUBLIC "-//W3C//DTD HTML 4.01//EN"
%  This is a (PRE) block.  Make sure it's left aligned or your toc title will be off. 

\section*{\texttt{Approximator}}

\label{f0}

\subsection*{Key Words}

\begin{quotation}
curvature flow, differential equations, Euler's method, Runga Kutta
\end{quotation}

\subsection*{Authors}

\begin{itemize}
\item Joe Thomas

\item Alex Henniges
\end{itemize}

\subsection*{Introduction}

\begin{quotation}
The Approximator class runs a curvature flow using one of several methods.
The class itself is abstract and the method is chosen by the instantiating
object.
\end{quotation}

\subsection*{Subsidiaries}

\begin{quotation}
Functions:
\end{quotation}

\begin{itemize}
\item \mbox{$[$}Functions\#Approximator::run run\mbox{$]$}
\end{itemize}

\begin{quotation}
Sub-classes:
\end{quotation}

\begin{itemize}
\item EulerApprox

\item RungaApprox
\end{itemize}

\begin{quotation}
Public Variables:
\end{quotation}

\begin{itemize}
\item radiiHistory

\item curvHistory

\item areaHistory

\item volumeHistory
\end{itemize}

\subsection*{Description}

\begin{quotation}
The \texttt{Approximator} class is the shell for running a curvature flow.
The class provides the functionality to determine what system of
differential equations to use, how to perform a step, and even which values
to record for later use. The system is provided by the user at run-time and
is defined to be a function that takes in an empty array of \texttt{double}s
and fills the array with the values calculated in the system of equations.
The \texttt{Approximator} class is abstract with an abstract method \texttt{%
step}. A class that extends \texttt{Approximator} implements \texttt{step}
with the method of approximation to use (i.e Euler's method). Lastly, the 
\texttt{Approximator} stores values after each \texttt{step} of a flow
according to which values were requested at construction. Values include
radii, curvatures, areas, and volumes. These histories can then be accessed
directly from the \texttt{Approximator} object.
\end{quotation}

\subsection*{Constructor}

\begin{quotation}
The constructor takes in a function that defines a system of differential
equations and a string of characters representing what histories to record.
For the function to be a \texttt{sysdiffeq} it must not return a value and
its only parameter is an array of doubles that will be filled in with values
after the function completes. The history string must be nul-terminated and
consisting of only valid characters. The valid characters currently are:
\end{quotation}

\begin{itemize}
\item r - Record radii

\item 2 - Record two-dimensional curvatures

\item 3 - Record three-dimensional curvatures

\item a - Record areas

\item v - Record volumes
\end{itemize}

\begin{quotation}
One cannot list both two- and three-dimensional curvatures.{\small }
\end{quotation}

\begin{verbatim}
{\small   typedef void (*sysdiffeq)(double derivs[]);
}
{\small   Approximator(sysdiffeq funct, char* histories);
}
{\small   
}
\end{verbatim}

\subsection*{Practicum}

\begin{quotation}
This example will show how to run a Yamabe curvature flow on the pentachoron
using precision and accuracy bounds (see \mbox{$[$}Functions\#run run%
\mbox{$]$}) while recording radii, curvatures, and volumes. It will show how
to initialize the system and also how to print out results at the end.%
{\small }
\end{quotation}

\begin{verbatim}
{\small int main(int argc, char** argv) {
}
{\small    map<int, Vertex>::iterator vit;
}
{\small    map<int, Edge>::iterator eit;
}
{\small    map<int, Face>::iterator fit;
}
{\small    map<int, Tetra>::iterator tit;
}
{\small      
}
{\small    vector<int> edges;
}
{\small    vector<int> faces;
}
{\small    vector<int> tetras;
}
{\small     
}
{\small     
}
{\small    time_t start, end;
}
{\small    
}
{\small    // File to read in triangulation from.
}
{\small 
   char from[] = "./Triangulation Files/3D Manifolds/Lutz Format/pentachoron.txt";
}
{\small    // File to convert to proper format.
}
{\small    char to[] = "./Triangulation Files/manifold converted.txt";
}
{\small    // Convert, then read in triangulation.
}
{\small    make3DTriangulationFile(from, to);
}
{\small    read3DTriangulationFile(to);
}
 
{\small    int vertSize = Triangulation::vertexTable.size();
}
{\small    int edgeSize = Triangulation::edgeTable.size();
}
{\small    
}
{\small    // Set the radii
}
{\small    for(int i = 1; i <= vertSize; i++) {
}
{\small 
      Radius::At(Triangulation::vertexTable[i])->setValue(1 + (0.5 - i/5.0) );        
}
{\small    }
}
{\small    // Set the etas
}
{\small    for(int i = 1; i <= edgeSize; i++) {
}
{\small        Eta::At(Triangulation::edgeTable[i])->setValue(1.0);
}
{\small    }
}
 
{\small 
   // Construct an Approximator object that uses the Euler method and Yamabe flow while
}
{\small    // recording radii, 3D curvatures, and volumes.
}
{\small    Approximator *app = new EulerApprox((sysdiffeq) Yamabe, "r3v");
}
 
{\small 
   // Run the Yamabe flow with precision and accuracy bounds of 0.0001 and stepsize of 0.01
}
{\small    app->run(0.0001, 0.0001, 0.01);
}
 
{\small    // Print out radii, curvatures and volumes
}
{\small 
   printResultsStep("./Triangulation Files/ODE Result.txt", &(app->radiiHistory), &(app->curvHistory));
}
{\small 
   printResultsVolumes("./Triangulation Files/Volumes.txt", &(app->volumeHistory));
}
 
{\small    return 0;
}
{\small }
}
\end{verbatim}

% html: End of file: `clean.html'

%%%%% END OF DOCUMENT BODY %%%%%
% In the future, we might want to put some additional data here, such
% as when the documentation was converted from wiki to TeX.
%
%
%EndExpansion

\bigskip 

\bigskip 

%TCIMACRO{\QSubDoc{Include NewtonsMethod}{%TCIDATA{Version=5.00.0.2606}
%TCIDATA{LaTeXparent=0,0,classes.tex}
                      

%%%%% BEGINNING OF DOCUMENT BODY %%%%%
% html: Beginning of file: `clean.html'
% DOCTYPE HTML PUBLIC "-//W3C//DTD HTML 4.01//EN"
%  This is a (PRE) block.  Make sure it's left aligned or your toc title will be off. 

\section*{\texttt{NewtonsMethod}}

\label{f0}

\subsection*{Key Words}

\begin{quotation}
gradient, hessian, extrema
\end{quotation}

\subsection*{Authors}

\begin{itemize}
\item Alex Henniges

\item Dan Champion
\end{itemize}

\subsection*{Introduction}

\begin{quotation}
The \texttt{NewtonsMethod} class is used to find an extremum of a given
functional. In addition to the functional given, the user can provide the
gradient or hessian function. If not, these are approximated during run-time.
\end{quotation}

\subsection*{Subsidiaries}

\begin{quotation}
Functions:
\end{quotation}

\begin{itemize}
\item NewtonsMethod::maximize

\item NewtonsMethod::step

\item NewtonsMethod::setPrintFunc

\item NewtonsMethod::printInfo
\end{itemize}

\subsection*{Description}

\begin{quotation}
As a class, \texttt{NewtonsMethod} is used as a general way to perform
Newton's method on a function in order to find its extrema. Newton's method
will find an extrema much faster than Euler's method, but is also more
complicated. In order for Newton's method to work, it requires the first and
second-order partial derivatives in addition to the original function. If
the user knows these explicitly, they can be passed in the \mbox{$[$}%
\#Constructor constructor\mbox{$]$} and should lead to more accurate and
possible quicker calculations. If the first and secord-partial derivatives
are not provided, then they will be approximated using quotients.

For our purposes within the \texttt{NewtonsMethod} class, the original
function is defined to take as a parameter an array of \texttt{double}s that
represent the values of each variable. The function should return a double.
The gradient function is defined to take an array of doubles that also
represent the point at which the partial derivatives should be calculated.
In addition, it takes another array of doubles for the partial derivatives
to be placed in. Lastly, a hessian function is also defined to take an array
of doubles for the point at which the calculation is being done. It also
takes a two-dimensional array of double for the second-order partial
derivatives to be placed in. Both the gradient and hessian function do not
return a value.
\end{quotation}

\subsection*{Constructor}

\begin{quotation}
There are three constructors for the \texttt{NewtonsMethod} class, allowing
for a combination of potential functions that can be given explicitly. In
addition, every constructor must be given an integer that indicates the
number of variables given to the functional.{\small }
\end{quotation}

\begin{verbatim}
{\small     typedef double (*orig_function)(double vars[])
}
{\small     typedef void   (*gradient)(double vars[], double sol[])
}
{\small     typedef void   (*hessian)(double vars[], double *sol[])
}
 
{\small     NewtonsMethod(orig_function func, int numVars)
}
{\small     NewtonsMethod(orig_function func, gradient df, int numVars)
}
{\small 
    NewtonsMethod(orig_function func, gradient df, hessian d2f, int numVars)
}
{\small   
}
\end{verbatim}

\subsection*{Practicum}

\begin{quotation}
Below is a full example of how to use the \texttt{NewtonsMethod} class to
find the minimum of an ellipse. In this case, the gradient and hessian are
not given. The maximum found is at (0,0).{\small }
\end{quotation}

\begin{verbatim}
{\small    // This function takes two variables.
}
{\small    // f(x, y) = (1 - x^/4 - y^2/9) ^(1/2)
}
{\small    double ellipse(double vars[]) {
}
{\small        double val = 1 - pow(vars[0], 2) / 4 - pow(vars[1], 2) / 9;
}
{\small        return sqrt(val);
}
{\small    }
}
 
{\small    int main(int arg, char** argv) {
}
{\small     // Create the NewtonsMethod object, 2 variables
}
{\small     NewtonsMethod *nm = new NewtonsMethod(ellipse, 2);
}
{\small     // Build the array that holds the initial values.
}
{\small     double initial[] = {0.1, 2.5};
}
{\small     // Build the array that will hold the final solution.
}
{\small     double soln[2];
}
{\small     // Run the maximize function
}
{\small     nm->maximize(initial, soln);
}
 
{\small     // Display the results
}
{\small     printf("\nSolution: %.10f, %.10f\n", soln[0], soln[1]);
}
 
{\small     return 0;
}
{\small    }
}
{\small   
}
\end{verbatim}

\begin{quotation}
Using the same ellipse, one can use the \texttt{step} function instead of 
\texttt{maximize} to gain greater flexibility over the procedure. In this
case, we also print out useful information after each step.{\small }
\end{quotation}

\begin{verbatim}
{\small      // This function takes two variables.
}
{\small    double ellipse(double vars[]) {
}
{\small        double val = 1 - pow(vars[0], 2) / 4 - pow(vars[1], 2) / 9;
}
{\small        return sqrt(val);
}
{\small    }
}
 
{\small    int main(int arg, char** argv) {
}
{\small     NewtonsMethod *nm = new NewtonsMethod(ellipse, 2);
}
{\small     double x_n[] = {1, 1};
}
{\small     int i = 1;
}
{\small     fprintf(stdout, "Initial\n-----------------\n");
}
{\small     for(int j = 0; j < 2; j++) {
}
{\small       fprintf(stdout, "x_n_%d[%d] = %f\n", i, j, x_n[j]);
}
{\small     }
}
{\small 
    // Continue with the procedure until the length of the gradient is 
}
{\small     // less than 0.000001.
}
{\small     while(nm->step(x_n) > 0.000001) {
}
{\small       fprintf(stdout, "\n***** Step %d *****\n", i++);
}
{\small       nm->printInfo(stdout);
}
{\small       for(int j = 0; j < 2; j++) {
}
{\small         fprintf(stdout, "x_n_%d[%d] = %f\n", i, j, x_n[j]);
}
{\small       }
}
{\small     }
}
{\small     printf("\nSolution: %.10f\n", x_n[0]);
}
 
{\small     return 0;
}
{\small    }
}
{\small   
}
\end{verbatim}

\begin{quotation}
In this example, we use a one variable Gaussian function, but provide a
gradient and hessian, as well. The maximum is found at \texttt{x = 0}.%
{\small }
\end{quotation}

\begin{verbatim}
{\small    // The function, e^(-x^2).
}
{\small    double gaussian(double vars[]) {
}
{\small        return exp(-pow(vars[0], 2));
}
{\small    }
}
 
{\small    // The gradient function, -2x * e^(-x^2).
}
{\small    // Note that the solution is placed in the array.
}
{\small    void gradFunc(double vars[], double sol[]) {
}
{\small      sol[0] = -2 * vars[0] * func(vars);
}
{\small    }
}
 
{\small    // The hessian function, e^(-x^2)(4x^2 - 2).
}
{\small    // Note that the solution is placed in a matrix.
}
{\small    void hessFunc(double vars[], double *sol[]) {
}
{\small      sol[0][0] = func(vars) * (4 * pow(vars[0], 2) - 2);
}
{\small    }
}
 
{\small    int main(int arg, char** argv) {
}
{\small     // Create the NewtonsMethod object
}
{\small 
    NewtonsMethod *nm = new NewtonsMethod(gaussian, gradFunc, hessFunc, 1);
}
{\small     // Build the array that holds the initial value.
}
{\small     double initial[] = {0.1};
}
{\small     // Build the array that will hold the final solution.
}
{\small     double soln[1];
}
{\small     // Run the maximize function
}
{\small     nm->maximize(initial, soln);
}
 
{\small     // Display the results
}
{\small     printf("\nSolution: %.10f\n", soln[0]);
}
 
{\small     return 0;
}
{\small    }
}
{\small   
}
\end{verbatim}

\subsection*{Limitations}

\begin{quotation}
There are limitations with the \texttt{NewtonsMethod} class with regards to
the approximation of the gradient and hessian. In both cases, a delta value
for the quotients is hard-coded as 10$^{-5}$. This could lead to accuracy
issues when the point where the derivative is being calculated is less than
this value. It can also be too accurate at times and lead to unnecessarily
slowing down the procedure. One solution could be to provide a function
where a delta value is set by the user.
\end{quotation}

\subsection*{Revisions}

\begin{itemize}
\item subversion 876, 7/16/09: Added a NewtonsMethod class for general
maximizing.
\end{itemize}

\subsection*{Future Work}

\begin{itemize}
\item 7/16 - Provide greater flexibility to the user for approximating the
gradient and hessian.
\end{itemize}

% html: End of file: `clean.html'

%%%%% END OF DOCUMENT BODY %%%%%
% In the future, we might want to put some additional data here, such
% as when the documentation was converted from wiki to TeX.
%
}}%
%BeginExpansion
%TCIDATA{Version=5.00.0.2606}
%TCIDATA{LaTeXparent=0,0,classes.tex}
                      

%%%%% BEGINNING OF DOCUMENT BODY %%%%%
% html: Beginning of file: `clean.html'
% DOCTYPE HTML PUBLIC "-//W3C//DTD HTML 4.01//EN"
%  This is a (PRE) block.  Make sure it's left aligned or your toc title will be off. 

\section*{\texttt{NewtonsMethod}}

\label{f0}

\subsection*{Key Words}

\begin{quotation}
gradient, hessian, extrema
\end{quotation}

\subsection*{Authors}

\begin{itemize}
\item Alex Henniges

\item Dan Champion
\end{itemize}

\subsection*{Introduction}

\begin{quotation}
The \texttt{NewtonsMethod} class is used to find an extremum of a given
functional. In addition to the functional given, the user can provide the
gradient or hessian function. If not, these are approximated during run-time.
\end{quotation}

\subsection*{Subsidiaries}

\begin{quotation}
Functions:
\end{quotation}

\begin{itemize}
\item NewtonsMethod::maximize

\item NewtonsMethod::step

\item NewtonsMethod::setPrintFunc

\item NewtonsMethod::printInfo
\end{itemize}

\subsection*{Description}

\begin{quotation}
As a class, \texttt{NewtonsMethod} is used as a general way to perform
Newton's method on a function in order to find its extrema. Newton's method
will find an extrema much faster than Euler's method, but is also more
complicated. In order for Newton's method to work, it requires the first and
second-order partial derivatives in addition to the original function. If
the user knows these explicitly, they can be passed in the \mbox{$[$}%
\#Constructor constructor\mbox{$]$} and should lead to more accurate and
possible quicker calculations. If the first and secord-partial derivatives
are not provided, then they will be approximated using quotients.

For our purposes within the \texttt{NewtonsMethod} class, the original
function is defined to take as a parameter an array of \texttt{double}s that
represent the values of each variable. The function should return a double.
The gradient function is defined to take an array of doubles that also
represent the point at which the partial derivatives should be calculated.
In addition, it takes another array of doubles for the partial derivatives
to be placed in. Lastly, a hessian function is also defined to take an array
of doubles for the point at which the calculation is being done. It also
takes a two-dimensional array of double for the second-order partial
derivatives to be placed in. Both the gradient and hessian function do not
return a value.
\end{quotation}

\subsection*{Constructor}

\begin{quotation}
There are three constructors for the \texttt{NewtonsMethod} class, allowing
for a combination of potential functions that can be given explicitly. In
addition, every constructor must be given an integer that indicates the
number of variables given to the functional.{\small }
\end{quotation}

\begin{verbatim}
{\small     typedef double (*orig_function)(double vars[])
}
{\small     typedef void   (*gradient)(double vars[], double sol[])
}
{\small     typedef void   (*hessian)(double vars[], double *sol[])
}
 
{\small     NewtonsMethod(orig_function func, int numVars)
}
{\small     NewtonsMethod(orig_function func, gradient df, int numVars)
}
{\small 
    NewtonsMethod(orig_function func, gradient df, hessian d2f, int numVars)
}
{\small   
}
\end{verbatim}

\subsection*{Practicum}

\begin{quotation}
Below is a full example of how to use the \texttt{NewtonsMethod} class to
find the minimum of an ellipse. In this case, the gradient and hessian are
not given. The maximum found is at (0,0).{\small }
\end{quotation}

\begin{verbatim}
{\small    // This function takes two variables.
}
{\small    // f(x, y) = (1 - x^/4 - y^2/9) ^(1/2)
}
{\small    double ellipse(double vars[]) {
}
{\small        double val = 1 - pow(vars[0], 2) / 4 - pow(vars[1], 2) / 9;
}
{\small        return sqrt(val);
}
{\small    }
}
 
{\small    int main(int arg, char** argv) {
}
{\small     // Create the NewtonsMethod object, 2 variables
}
{\small     NewtonsMethod *nm = new NewtonsMethod(ellipse, 2);
}
{\small     // Build the array that holds the initial values.
}
{\small     double initial[] = {0.1, 2.5};
}
{\small     // Build the array that will hold the final solution.
}
{\small     double soln[2];
}
{\small     // Run the maximize function
}
{\small     nm->maximize(initial, soln);
}
 
{\small     // Display the results
}
{\small     printf("\nSolution: %.10f, %.10f\n", soln[0], soln[1]);
}
 
{\small     return 0;
}
{\small    }
}
{\small   
}
\end{verbatim}

\begin{quotation}
Using the same ellipse, one can use the \texttt{step} function instead of 
\texttt{maximize} to gain greater flexibility over the procedure. In this
case, we also print out useful information after each step.{\small }
\end{quotation}

\begin{verbatim}
{\small      // This function takes two variables.
}
{\small    double ellipse(double vars[]) {
}
{\small        double val = 1 - pow(vars[0], 2) / 4 - pow(vars[1], 2) / 9;
}
{\small        return sqrt(val);
}
{\small    }
}
 
{\small    int main(int arg, char** argv) {
}
{\small     NewtonsMethod *nm = new NewtonsMethod(ellipse, 2);
}
{\small     double x_n[] = {1, 1};
}
{\small     int i = 1;
}
{\small     fprintf(stdout, "Initial\n-----------------\n");
}
{\small     for(int j = 0; j < 2; j++) {
}
{\small       fprintf(stdout, "x_n_%d[%d] = %f\n", i, j, x_n[j]);
}
{\small     }
}
{\small 
    // Continue with the procedure until the length of the gradient is 
}
{\small     // less than 0.000001.
}
{\small     while(nm->step(x_n) > 0.000001) {
}
{\small       fprintf(stdout, "\n***** Step %d *****\n", i++);
}
{\small       nm->printInfo(stdout);
}
{\small       for(int j = 0; j < 2; j++) {
}
{\small         fprintf(stdout, "x_n_%d[%d] = %f\n", i, j, x_n[j]);
}
{\small       }
}
{\small     }
}
{\small     printf("\nSolution: %.10f\n", x_n[0]);
}
 
{\small     return 0;
}
{\small    }
}
{\small   
}
\end{verbatim}

\begin{quotation}
In this example, we use a one variable Gaussian function, but provide a
gradient and hessian, as well. The maximum is found at \texttt{x = 0}.%
{\small }
\end{quotation}

\begin{verbatim}
{\small    // The function, e^(-x^2).
}
{\small    double gaussian(double vars[]) {
}
{\small        return exp(-pow(vars[0], 2));
}
{\small    }
}
 
{\small    // The gradient function, -2x * e^(-x^2).
}
{\small    // Note that the solution is placed in the array.
}
{\small    void gradFunc(double vars[], double sol[]) {
}
{\small      sol[0] = -2 * vars[0] * func(vars);
}
{\small    }
}
 
{\small    // The hessian function, e^(-x^2)(4x^2 - 2).
}
{\small    // Note that the solution is placed in a matrix.
}
{\small    void hessFunc(double vars[], double *sol[]) {
}
{\small      sol[0][0] = func(vars) * (4 * pow(vars[0], 2) - 2);
}
{\small    }
}
 
{\small    int main(int arg, char** argv) {
}
{\small     // Create the NewtonsMethod object
}
{\small 
    NewtonsMethod *nm = new NewtonsMethod(gaussian, gradFunc, hessFunc, 1);
}
{\small     // Build the array that holds the initial value.
}
{\small     double initial[] = {0.1};
}
{\small     // Build the array that will hold the final solution.
}
{\small     double soln[1];
}
{\small     // Run the maximize function
}
{\small     nm->maximize(initial, soln);
}
 
{\small     // Display the results
}
{\small     printf("\nSolution: %.10f\n", soln[0]);
}
 
{\small     return 0;
}
{\small    }
}
{\small   
}
\end{verbatim}

\subsection*{Limitations}

\begin{quotation}
There are limitations with the \texttt{NewtonsMethod} class with regards to
the approximation of the gradient and hessian. In both cases, a delta value
for the quotients is hard-coded as 10$^{-5}$. This could lead to accuracy
issues when the point where the derivative is being calculated is less than
this value. It can also be too accurate at times and lead to unnecessarily
slowing down the procedure. One solution could be to provide a function
where a delta value is set by the user.
\end{quotation}

\subsection*{Revisions}

\begin{itemize}
\item subversion 876, 7/16/09: Added a NewtonsMethod class for general
maximizing.
\end{itemize}

\subsection*{Future Work}

\begin{itemize}
\item 7/16 - Provide greater flexibility to the user for approximating the
gradient and hessian.
\end{itemize}

% html: End of file: `clean.html'

%%%%% END OF DOCUMENT BODY %%%%%
% In the future, we might want to put some additional data here, such
% as when the documentation was converted from wiki to TeX.
%
%
%EndExpansion
}}%
%BeginExpansion
%TCIDATA{Version=5.00.0.2606}
%TCIDATA{LaTeXparent=0,0,geocam.tex}
                      

\chapter{Classes}

%TCIMACRO{\QSubDoc{Include Approximator}{%TCIDATA{Version=5.00.0.2606}
%TCIDATA{LaTeXparent=0,0,classes.tex}
                      

%%%%% BEGINNING OF DOCUMENT BODY %%%%%
% html: Beginning of file: `clean.html'
% DOCTYPE HTML PUBLIC "-//W3C//DTD HTML 4.01//EN"
%  This is a (PRE) block.  Make sure it's left aligned or your toc title will be off. 

\section*{\texttt{Approximator}}

\label{f0}

\subsection*{Key Words}

\begin{quotation}
curvature flow, differential equations, Euler's method, Runga Kutta
\end{quotation}

\subsection*{Authors}

\begin{itemize}
\item Joe Thomas

\item Alex Henniges
\end{itemize}

\subsection*{Introduction}

\begin{quotation}
The Approximator class runs a curvature flow using one of several methods.
The class itself is abstract and the method is chosen by the instantiating
object.
\end{quotation}

\subsection*{Subsidiaries}

\begin{quotation}
Functions:
\end{quotation}

\begin{itemize}
\item \mbox{$[$}Functions\#Approximator::run run\mbox{$]$}
\end{itemize}

\begin{quotation}
Sub-classes:
\end{quotation}

\begin{itemize}
\item EulerApprox

\item RungaApprox
\end{itemize}

\begin{quotation}
Public Variables:
\end{quotation}

\begin{itemize}
\item radiiHistory

\item curvHistory

\item areaHistory

\item volumeHistory
\end{itemize}

\subsection*{Description}

\begin{quotation}
The \texttt{Approximator} class is the shell for running a curvature flow.
The class provides the functionality to determine what system of
differential equations to use, how to perform a step, and even which values
to record for later use. The system is provided by the user at run-time and
is defined to be a function that takes in an empty array of \texttt{double}s
and fills the array with the values calculated in the system of equations.
The \texttt{Approximator} class is abstract with an abstract method \texttt{%
step}. A class that extends \texttt{Approximator} implements \texttt{step}
with the method of approximation to use (i.e Euler's method). Lastly, the 
\texttt{Approximator} stores values after each \texttt{step} of a flow
according to which values were requested at construction. Values include
radii, curvatures, areas, and volumes. These histories can then be accessed
directly from the \texttt{Approximator} object.
\end{quotation}

\subsection*{Constructor}

\begin{quotation}
The constructor takes in a function that defines a system of differential
equations and a string of characters representing what histories to record.
For the function to be a \texttt{sysdiffeq} it must not return a value and
its only parameter is an array of doubles that will be filled in with values
after the function completes. The history string must be nul-terminated and
consisting of only valid characters. The valid characters currently are:
\end{quotation}

\begin{itemize}
\item r - Record radii

\item 2 - Record two-dimensional curvatures

\item 3 - Record three-dimensional curvatures

\item a - Record areas

\item v - Record volumes
\end{itemize}

\begin{quotation}
One cannot list both two- and three-dimensional curvatures.{\small }
\end{quotation}

\begin{verbatim}
{\small   typedef void (*sysdiffeq)(double derivs[]);
}
{\small   Approximator(sysdiffeq funct, char* histories);
}
{\small   
}
\end{verbatim}

\subsection*{Practicum}

\begin{quotation}
This example will show how to run a Yamabe curvature flow on the pentachoron
using precision and accuracy bounds (see \mbox{$[$}Functions\#run run%
\mbox{$]$}) while recording radii, curvatures, and volumes. It will show how
to initialize the system and also how to print out results at the end.%
{\small }
\end{quotation}

\begin{verbatim}
{\small int main(int argc, char** argv) {
}
{\small    map<int, Vertex>::iterator vit;
}
{\small    map<int, Edge>::iterator eit;
}
{\small    map<int, Face>::iterator fit;
}
{\small    map<int, Tetra>::iterator tit;
}
{\small      
}
{\small    vector<int> edges;
}
{\small    vector<int> faces;
}
{\small    vector<int> tetras;
}
{\small     
}
{\small     
}
{\small    time_t start, end;
}
{\small    
}
{\small    // File to read in triangulation from.
}
{\small 
   char from[] = "./Triangulation Files/3D Manifolds/Lutz Format/pentachoron.txt";
}
{\small    // File to convert to proper format.
}
{\small    char to[] = "./Triangulation Files/manifold converted.txt";
}
{\small    // Convert, then read in triangulation.
}
{\small    make3DTriangulationFile(from, to);
}
{\small    read3DTriangulationFile(to);
}
 
{\small    int vertSize = Triangulation::vertexTable.size();
}
{\small    int edgeSize = Triangulation::edgeTable.size();
}
{\small    
}
{\small    // Set the radii
}
{\small    for(int i = 1; i <= vertSize; i++) {
}
{\small 
      Radius::At(Triangulation::vertexTable[i])->setValue(1 + (0.5 - i/5.0) );        
}
{\small    }
}
{\small    // Set the etas
}
{\small    for(int i = 1; i <= edgeSize; i++) {
}
{\small        Eta::At(Triangulation::edgeTable[i])->setValue(1.0);
}
{\small    }
}
 
{\small 
   // Construct an Approximator object that uses the Euler method and Yamabe flow while
}
{\small    // recording radii, 3D curvatures, and volumes.
}
{\small    Approximator *app = new EulerApprox((sysdiffeq) Yamabe, "r3v");
}
 
{\small 
   // Run the Yamabe flow with precision and accuracy bounds of 0.0001 and stepsize of 0.01
}
{\small    app->run(0.0001, 0.0001, 0.01);
}
 
{\small    // Print out radii, curvatures and volumes
}
{\small 
   printResultsStep("./Triangulation Files/ODE Result.txt", &(app->radiiHistory), &(app->curvHistory));
}
{\small 
   printResultsVolumes("./Triangulation Files/Volumes.txt", &(app->volumeHistory));
}
 
{\small    return 0;
}
{\small }
}
\end{verbatim}

% html: End of file: `clean.html'

%%%%% END OF DOCUMENT BODY %%%%%
% In the future, we might want to put some additional data here, such
% as when the documentation was converted from wiki to TeX.
%
}}%
%BeginExpansion
%TCIDATA{Version=5.00.0.2606}
%TCIDATA{LaTeXparent=0,0,classes.tex}
                      

%%%%% BEGINNING OF DOCUMENT BODY %%%%%
% html: Beginning of file: `clean.html'
% DOCTYPE HTML PUBLIC "-//W3C//DTD HTML 4.01//EN"
%  This is a (PRE) block.  Make sure it's left aligned or your toc title will be off. 

\section*{\texttt{Approximator}}

\label{f0}

\subsection*{Key Words}

\begin{quotation}
curvature flow, differential equations, Euler's method, Runga Kutta
\end{quotation}

\subsection*{Authors}

\begin{itemize}
\item Joe Thomas

\item Alex Henniges
\end{itemize}

\subsection*{Introduction}

\begin{quotation}
The Approximator class runs a curvature flow using one of several methods.
The class itself is abstract and the method is chosen by the instantiating
object.
\end{quotation}

\subsection*{Subsidiaries}

\begin{quotation}
Functions:
\end{quotation}

\begin{itemize}
\item \mbox{$[$}Functions\#Approximator::run run\mbox{$]$}
\end{itemize}

\begin{quotation}
Sub-classes:
\end{quotation}

\begin{itemize}
\item EulerApprox

\item RungaApprox
\end{itemize}

\begin{quotation}
Public Variables:
\end{quotation}

\begin{itemize}
\item radiiHistory

\item curvHistory

\item areaHistory

\item volumeHistory
\end{itemize}

\subsection*{Description}

\begin{quotation}
The \texttt{Approximator} class is the shell for running a curvature flow.
The class provides the functionality to determine what system of
differential equations to use, how to perform a step, and even which values
to record for later use. The system is provided by the user at run-time and
is defined to be a function that takes in an empty array of \texttt{double}s
and fills the array with the values calculated in the system of equations.
The \texttt{Approximator} class is abstract with an abstract method \texttt{%
step}. A class that extends \texttt{Approximator} implements \texttt{step}
with the method of approximation to use (i.e Euler's method). Lastly, the 
\texttt{Approximator} stores values after each \texttt{step} of a flow
according to which values were requested at construction. Values include
radii, curvatures, areas, and volumes. These histories can then be accessed
directly from the \texttt{Approximator} object.
\end{quotation}

\subsection*{Constructor}

\begin{quotation}
The constructor takes in a function that defines a system of differential
equations and a string of characters representing what histories to record.
For the function to be a \texttt{sysdiffeq} it must not return a value and
its only parameter is an array of doubles that will be filled in with values
after the function completes. The history string must be nul-terminated and
consisting of only valid characters. The valid characters currently are:
\end{quotation}

\begin{itemize}
\item r - Record radii

\item 2 - Record two-dimensional curvatures

\item 3 - Record three-dimensional curvatures

\item a - Record areas

\item v - Record volumes
\end{itemize}

\begin{quotation}
One cannot list both two- and three-dimensional curvatures.{\small }
\end{quotation}

\begin{verbatim}
{\small   typedef void (*sysdiffeq)(double derivs[]);
}
{\small   Approximator(sysdiffeq funct, char* histories);
}
{\small   
}
\end{verbatim}

\subsection*{Practicum}

\begin{quotation}
This example will show how to run a Yamabe curvature flow on the pentachoron
using precision and accuracy bounds (see \mbox{$[$}Functions\#run run%
\mbox{$]$}) while recording radii, curvatures, and volumes. It will show how
to initialize the system and also how to print out results at the end.%
{\small }
\end{quotation}

\begin{verbatim}
{\small int main(int argc, char** argv) {
}
{\small    map<int, Vertex>::iterator vit;
}
{\small    map<int, Edge>::iterator eit;
}
{\small    map<int, Face>::iterator fit;
}
{\small    map<int, Tetra>::iterator tit;
}
{\small      
}
{\small    vector<int> edges;
}
{\small    vector<int> faces;
}
{\small    vector<int> tetras;
}
{\small     
}
{\small     
}
{\small    time_t start, end;
}
{\small    
}
{\small    // File to read in triangulation from.
}
{\small 
   char from[] = "./Triangulation Files/3D Manifolds/Lutz Format/pentachoron.txt";
}
{\small    // File to convert to proper format.
}
{\small    char to[] = "./Triangulation Files/manifold converted.txt";
}
{\small    // Convert, then read in triangulation.
}
{\small    make3DTriangulationFile(from, to);
}
{\small    read3DTriangulationFile(to);
}
 
{\small    int vertSize = Triangulation::vertexTable.size();
}
{\small    int edgeSize = Triangulation::edgeTable.size();
}
{\small    
}
{\small    // Set the radii
}
{\small    for(int i = 1; i <= vertSize; i++) {
}
{\small 
      Radius::At(Triangulation::vertexTable[i])->setValue(1 + (0.5 - i/5.0) );        
}
{\small    }
}
{\small    // Set the etas
}
{\small    for(int i = 1; i <= edgeSize; i++) {
}
{\small        Eta::At(Triangulation::edgeTable[i])->setValue(1.0);
}
{\small    }
}
 
{\small 
   // Construct an Approximator object that uses the Euler method and Yamabe flow while
}
{\small    // recording radii, 3D curvatures, and volumes.
}
{\small    Approximator *app = new EulerApprox((sysdiffeq) Yamabe, "r3v");
}
 
{\small 
   // Run the Yamabe flow with precision and accuracy bounds of 0.0001 and stepsize of 0.01
}
{\small    app->run(0.0001, 0.0001, 0.01);
}
 
{\small    // Print out radii, curvatures and volumes
}
{\small 
   printResultsStep("./Triangulation Files/ODE Result.txt", &(app->radiiHistory), &(app->curvHistory));
}
{\small 
   printResultsVolumes("./Triangulation Files/Volumes.txt", &(app->volumeHistory));
}
 
{\small    return 0;
}
{\small }
}
\end{verbatim}

% html: End of file: `clean.html'

%%%%% END OF DOCUMENT BODY %%%%%
% In the future, we might want to put some additional data here, such
% as when the documentation was converted from wiki to TeX.
%
%
%EndExpansion

\bigskip 

\bigskip 

%TCIMACRO{\QSubDoc{Include NewtonsMethod}{%TCIDATA{Version=5.00.0.2606}
%TCIDATA{LaTeXparent=0,0,classes.tex}
                      

%%%%% BEGINNING OF DOCUMENT BODY %%%%%
% html: Beginning of file: `clean.html'
% DOCTYPE HTML PUBLIC "-//W3C//DTD HTML 4.01//EN"
%  This is a (PRE) block.  Make sure it's left aligned or your toc title will be off. 

\section*{\texttt{NewtonsMethod}}

\label{f0}

\subsection*{Key Words}

\begin{quotation}
gradient, hessian, extrema
\end{quotation}

\subsection*{Authors}

\begin{itemize}
\item Alex Henniges

\item Dan Champion
\end{itemize}

\subsection*{Introduction}

\begin{quotation}
The \texttt{NewtonsMethod} class is used to find an extremum of a given
functional. In addition to the functional given, the user can provide the
gradient or hessian function. If not, these are approximated during run-time.
\end{quotation}

\subsection*{Subsidiaries}

\begin{quotation}
Functions:
\end{quotation}

\begin{itemize}
\item NewtonsMethod::maximize

\item NewtonsMethod::step

\item NewtonsMethod::setPrintFunc

\item NewtonsMethod::printInfo
\end{itemize}

\subsection*{Description}

\begin{quotation}
As a class, \texttt{NewtonsMethod} is used as a general way to perform
Newton's method on a function in order to find its extrema. Newton's method
will find an extrema much faster than Euler's method, but is also more
complicated. In order for Newton's method to work, it requires the first and
second-order partial derivatives in addition to the original function. If
the user knows these explicitly, they can be passed in the \mbox{$[$}%
\#Constructor constructor\mbox{$]$} and should lead to more accurate and
possible quicker calculations. If the first and secord-partial derivatives
are not provided, then they will be approximated using quotients.

For our purposes within the \texttt{NewtonsMethod} class, the original
function is defined to take as a parameter an array of \texttt{double}s that
represent the values of each variable. The function should return a double.
The gradient function is defined to take an array of doubles that also
represent the point at which the partial derivatives should be calculated.
In addition, it takes another array of doubles for the partial derivatives
to be placed in. Lastly, a hessian function is also defined to take an array
of doubles for the point at which the calculation is being done. It also
takes a two-dimensional array of double for the second-order partial
derivatives to be placed in. Both the gradient and hessian function do not
return a value.
\end{quotation}

\subsection*{Constructor}

\begin{quotation}
There are three constructors for the \texttt{NewtonsMethod} class, allowing
for a combination of potential functions that can be given explicitly. In
addition, every constructor must be given an integer that indicates the
number of variables given to the functional.{\small }
\end{quotation}

\begin{verbatim}
{\small     typedef double (*orig_function)(double vars[])
}
{\small     typedef void   (*gradient)(double vars[], double sol[])
}
{\small     typedef void   (*hessian)(double vars[], double *sol[])
}
 
{\small     NewtonsMethod(orig_function func, int numVars)
}
{\small     NewtonsMethod(orig_function func, gradient df, int numVars)
}
{\small 
    NewtonsMethod(orig_function func, gradient df, hessian d2f, int numVars)
}
{\small   
}
\end{verbatim}

\subsection*{Practicum}

\begin{quotation}
Below is a full example of how to use the \texttt{NewtonsMethod} class to
find the minimum of an ellipse. In this case, the gradient and hessian are
not given. The maximum found is at (0,0).{\small }
\end{quotation}

\begin{verbatim}
{\small    // This function takes two variables.
}
{\small    // f(x, y) = (1 - x^/4 - y^2/9) ^(1/2)
}
{\small    double ellipse(double vars[]) {
}
{\small        double val = 1 - pow(vars[0], 2) / 4 - pow(vars[1], 2) / 9;
}
{\small        return sqrt(val);
}
{\small    }
}
 
{\small    int main(int arg, char** argv) {
}
{\small     // Create the NewtonsMethod object, 2 variables
}
{\small     NewtonsMethod *nm = new NewtonsMethod(ellipse, 2);
}
{\small     // Build the array that holds the initial values.
}
{\small     double initial[] = {0.1, 2.5};
}
{\small     // Build the array that will hold the final solution.
}
{\small     double soln[2];
}
{\small     // Run the maximize function
}
{\small     nm->maximize(initial, soln);
}
 
{\small     // Display the results
}
{\small     printf("\nSolution: %.10f, %.10f\n", soln[0], soln[1]);
}
 
{\small     return 0;
}
{\small    }
}
{\small   
}
\end{verbatim}

\begin{quotation}
Using the same ellipse, one can use the \texttt{step} function instead of 
\texttt{maximize} to gain greater flexibility over the procedure. In this
case, we also print out useful information after each step.{\small }
\end{quotation}

\begin{verbatim}
{\small      // This function takes two variables.
}
{\small    double ellipse(double vars[]) {
}
{\small        double val = 1 - pow(vars[0], 2) / 4 - pow(vars[1], 2) / 9;
}
{\small        return sqrt(val);
}
{\small    }
}
 
{\small    int main(int arg, char** argv) {
}
{\small     NewtonsMethod *nm = new NewtonsMethod(ellipse, 2);
}
{\small     double x_n[] = {1, 1};
}
{\small     int i = 1;
}
{\small     fprintf(stdout, "Initial\n-----------------\n");
}
{\small     for(int j = 0; j < 2; j++) {
}
{\small       fprintf(stdout, "x_n_%d[%d] = %f\n", i, j, x_n[j]);
}
{\small     }
}
{\small 
    // Continue with the procedure until the length of the gradient is 
}
{\small     // less than 0.000001.
}
{\small     while(nm->step(x_n) > 0.000001) {
}
{\small       fprintf(stdout, "\n***** Step %d *****\n", i++);
}
{\small       nm->printInfo(stdout);
}
{\small       for(int j = 0; j < 2; j++) {
}
{\small         fprintf(stdout, "x_n_%d[%d] = %f\n", i, j, x_n[j]);
}
{\small       }
}
{\small     }
}
{\small     printf("\nSolution: %.10f\n", x_n[0]);
}
 
{\small     return 0;
}
{\small    }
}
{\small   
}
\end{verbatim}

\begin{quotation}
In this example, we use a one variable Gaussian function, but provide a
gradient and hessian, as well. The maximum is found at \texttt{x = 0}.%
{\small }
\end{quotation}

\begin{verbatim}
{\small    // The function, e^(-x^2).
}
{\small    double gaussian(double vars[]) {
}
{\small        return exp(-pow(vars[0], 2));
}
{\small    }
}
 
{\small    // The gradient function, -2x * e^(-x^2).
}
{\small    // Note that the solution is placed in the array.
}
{\small    void gradFunc(double vars[], double sol[]) {
}
{\small      sol[0] = -2 * vars[0] * func(vars);
}
{\small    }
}
 
{\small    // The hessian function, e^(-x^2)(4x^2 - 2).
}
{\small    // Note that the solution is placed in a matrix.
}
{\small    void hessFunc(double vars[], double *sol[]) {
}
{\small      sol[0][0] = func(vars) * (4 * pow(vars[0], 2) - 2);
}
{\small    }
}
 
{\small    int main(int arg, char** argv) {
}
{\small     // Create the NewtonsMethod object
}
{\small 
    NewtonsMethod *nm = new NewtonsMethod(gaussian, gradFunc, hessFunc, 1);
}
{\small     // Build the array that holds the initial value.
}
{\small     double initial[] = {0.1};
}
{\small     // Build the array that will hold the final solution.
}
{\small     double soln[1];
}
{\small     // Run the maximize function
}
{\small     nm->maximize(initial, soln);
}
 
{\small     // Display the results
}
{\small     printf("\nSolution: %.10f\n", soln[0]);
}
 
{\small     return 0;
}
{\small    }
}
{\small   
}
\end{verbatim}

\subsection*{Limitations}

\begin{quotation}
There are limitations with the \texttt{NewtonsMethod} class with regards to
the approximation of the gradient and hessian. In both cases, a delta value
for the quotients is hard-coded as 10$^{-5}$. This could lead to accuracy
issues when the point where the derivative is being calculated is less than
this value. It can also be too accurate at times and lead to unnecessarily
slowing down the procedure. One solution could be to provide a function
where a delta value is set by the user.
\end{quotation}

\subsection*{Revisions}

\begin{itemize}
\item subversion 876, 7/16/09: Added a NewtonsMethod class for general
maximizing.
\end{itemize}

\subsection*{Future Work}

\begin{itemize}
\item 7/16 - Provide greater flexibility to the user for approximating the
gradient and hessian.
\end{itemize}

% html: End of file: `clean.html'

%%%%% END OF DOCUMENT BODY %%%%%
% In the future, we might want to put some additional data here, such
% as when the documentation was converted from wiki to TeX.
%
}}%
%BeginExpansion
%TCIDATA{Version=5.00.0.2606}
%TCIDATA{LaTeXparent=0,0,classes.tex}
                      

%%%%% BEGINNING OF DOCUMENT BODY %%%%%
% html: Beginning of file: `clean.html'
% DOCTYPE HTML PUBLIC "-//W3C//DTD HTML 4.01//EN"
%  This is a (PRE) block.  Make sure it's left aligned or your toc title will be off. 

\section*{\texttt{NewtonsMethod}}

\label{f0}

\subsection*{Key Words}

\begin{quotation}
gradient, hessian, extrema
\end{quotation}

\subsection*{Authors}

\begin{itemize}
\item Alex Henniges

\item Dan Champion
\end{itemize}

\subsection*{Introduction}

\begin{quotation}
The \texttt{NewtonsMethod} class is used to find an extremum of a given
functional. In addition to the functional given, the user can provide the
gradient or hessian function. If not, these are approximated during run-time.
\end{quotation}

\subsection*{Subsidiaries}

\begin{quotation}
Functions:
\end{quotation}

\begin{itemize}
\item NewtonsMethod::maximize

\item NewtonsMethod::step

\item NewtonsMethod::setPrintFunc

\item NewtonsMethod::printInfo
\end{itemize}

\subsection*{Description}

\begin{quotation}
As a class, \texttt{NewtonsMethod} is used as a general way to perform
Newton's method on a function in order to find its extrema. Newton's method
will find an extrema much faster than Euler's method, but is also more
complicated. In order for Newton's method to work, it requires the first and
second-order partial derivatives in addition to the original function. If
the user knows these explicitly, they can be passed in the \mbox{$[$}%
\#Constructor constructor\mbox{$]$} and should lead to more accurate and
possible quicker calculations. If the first and secord-partial derivatives
are not provided, then they will be approximated using quotients.

For our purposes within the \texttt{NewtonsMethod} class, the original
function is defined to take as a parameter an array of \texttt{double}s that
represent the values of each variable. The function should return a double.
The gradient function is defined to take an array of doubles that also
represent the point at which the partial derivatives should be calculated.
In addition, it takes another array of doubles for the partial derivatives
to be placed in. Lastly, a hessian function is also defined to take an array
of doubles for the point at which the calculation is being done. It also
takes a two-dimensional array of double for the second-order partial
derivatives to be placed in. Both the gradient and hessian function do not
return a value.
\end{quotation}

\subsection*{Constructor}

\begin{quotation}
There are three constructors for the \texttt{NewtonsMethod} class, allowing
for a combination of potential functions that can be given explicitly. In
addition, every constructor must be given an integer that indicates the
number of variables given to the functional.{\small }
\end{quotation}

\begin{verbatim}
{\small     typedef double (*orig_function)(double vars[])
}
{\small     typedef void   (*gradient)(double vars[], double sol[])
}
{\small     typedef void   (*hessian)(double vars[], double *sol[])
}
 
{\small     NewtonsMethod(orig_function func, int numVars)
}
{\small     NewtonsMethod(orig_function func, gradient df, int numVars)
}
{\small 
    NewtonsMethod(orig_function func, gradient df, hessian d2f, int numVars)
}
{\small   
}
\end{verbatim}

\subsection*{Practicum}

\begin{quotation}
Below is a full example of how to use the \texttt{NewtonsMethod} class to
find the minimum of an ellipse. In this case, the gradient and hessian are
not given. The maximum found is at (0,0).{\small }
\end{quotation}

\begin{verbatim}
{\small    // This function takes two variables.
}
{\small    // f(x, y) = (1 - x^/4 - y^2/9) ^(1/2)
}
{\small    double ellipse(double vars[]) {
}
{\small        double val = 1 - pow(vars[0], 2) / 4 - pow(vars[1], 2) / 9;
}
{\small        return sqrt(val);
}
{\small    }
}
 
{\small    int main(int arg, char** argv) {
}
{\small     // Create the NewtonsMethod object, 2 variables
}
{\small     NewtonsMethod *nm = new NewtonsMethod(ellipse, 2);
}
{\small     // Build the array that holds the initial values.
}
{\small     double initial[] = {0.1, 2.5};
}
{\small     // Build the array that will hold the final solution.
}
{\small     double soln[2];
}
{\small     // Run the maximize function
}
{\small     nm->maximize(initial, soln);
}
 
{\small     // Display the results
}
{\small     printf("\nSolution: %.10f, %.10f\n", soln[0], soln[1]);
}
 
{\small     return 0;
}
{\small    }
}
{\small   
}
\end{verbatim}

\begin{quotation}
Using the same ellipse, one can use the \texttt{step} function instead of 
\texttt{maximize} to gain greater flexibility over the procedure. In this
case, we also print out useful information after each step.{\small }
\end{quotation}

\begin{verbatim}
{\small      // This function takes two variables.
}
{\small    double ellipse(double vars[]) {
}
{\small        double val = 1 - pow(vars[0], 2) / 4 - pow(vars[1], 2) / 9;
}
{\small        return sqrt(val);
}
{\small    }
}
 
{\small    int main(int arg, char** argv) {
}
{\small     NewtonsMethod *nm = new NewtonsMethod(ellipse, 2);
}
{\small     double x_n[] = {1, 1};
}
{\small     int i = 1;
}
{\small     fprintf(stdout, "Initial\n-----------------\n");
}
{\small     for(int j = 0; j < 2; j++) {
}
{\small       fprintf(stdout, "x_n_%d[%d] = %f\n", i, j, x_n[j]);
}
{\small     }
}
{\small 
    // Continue with the procedure until the length of the gradient is 
}
{\small     // less than 0.000001.
}
{\small     while(nm->step(x_n) > 0.000001) {
}
{\small       fprintf(stdout, "\n***** Step %d *****\n", i++);
}
{\small       nm->printInfo(stdout);
}
{\small       for(int j = 0; j < 2; j++) {
}
{\small         fprintf(stdout, "x_n_%d[%d] = %f\n", i, j, x_n[j]);
}
{\small       }
}
{\small     }
}
{\small     printf("\nSolution: %.10f\n", x_n[0]);
}
 
{\small     return 0;
}
{\small    }
}
{\small   
}
\end{verbatim}

\begin{quotation}
In this example, we use a one variable Gaussian function, but provide a
gradient and hessian, as well. The maximum is found at \texttt{x = 0}.%
{\small }
\end{quotation}

\begin{verbatim}
{\small    // The function, e^(-x^2).
}
{\small    double gaussian(double vars[]) {
}
{\small        return exp(-pow(vars[0], 2));
}
{\small    }
}
 
{\small    // The gradient function, -2x * e^(-x^2).
}
{\small    // Note that the solution is placed in the array.
}
{\small    void gradFunc(double vars[], double sol[]) {
}
{\small      sol[0] = -2 * vars[0] * func(vars);
}
{\small    }
}
 
{\small    // The hessian function, e^(-x^2)(4x^2 - 2).
}
{\small    // Note that the solution is placed in a matrix.
}
{\small    void hessFunc(double vars[], double *sol[]) {
}
{\small      sol[0][0] = func(vars) * (4 * pow(vars[0], 2) - 2);
}
{\small    }
}
 
{\small    int main(int arg, char** argv) {
}
{\small     // Create the NewtonsMethod object
}
{\small 
    NewtonsMethod *nm = new NewtonsMethod(gaussian, gradFunc, hessFunc, 1);
}
{\small     // Build the array that holds the initial value.
}
{\small     double initial[] = {0.1};
}
{\small     // Build the array that will hold the final solution.
}
{\small     double soln[1];
}
{\small     // Run the maximize function
}
{\small     nm->maximize(initial, soln);
}
 
{\small     // Display the results
}
{\small     printf("\nSolution: %.10f\n", soln[0]);
}
 
{\small     return 0;
}
{\small    }
}
{\small   
}
\end{verbatim}

\subsection*{Limitations}

\begin{quotation}
There are limitations with the \texttt{NewtonsMethod} class with regards to
the approximation of the gradient and hessian. In both cases, a delta value
for the quotients is hard-coded as 10$^{-5}$. This could lead to accuracy
issues when the point where the derivative is being calculated is less than
this value. It can also be too accurate at times and lead to unnecessarily
slowing down the procedure. One solution could be to provide a function
where a delta value is set by the user.
\end{quotation}

\subsection*{Revisions}

\begin{itemize}
\item subversion 876, 7/16/09: Added a NewtonsMethod class for general
maximizing.
\end{itemize}

\subsection*{Future Work}

\begin{itemize}
\item 7/16 - Provide greater flexibility to the user for approximating the
gradient and hessian.
\end{itemize}

% html: End of file: `clean.html'

%%%%% END OF DOCUMENT BODY %%%%%
% In the future, we might want to put some additional data here, such
% as when the documentation was converted from wiki to TeX.
%
%
%EndExpansion
%
%EndExpansion

\part{Theory}

%TCIMACRO{\QSubDoc{Include glossary}{%TCIDATA{Version=5.00.0.2606}
%TCIDATA{LaTeXparent=0,0,geocam.tex}
                      
%TCIDATA{ChildDefaults=chapter:1,page:1}


\chapter{Glossary}

\begin{description}
\item[circle power] Given a circle $C$ and a point $P$, let $L$ be the line
through the point $P$ that passes through the center of the circle. \ Let $A$
and $B$ be the intersection points of $L$ with the circle.\ Define a signed
distance $\left\Vert \overline{PX}\right\Vert $ for a line segment $%
\overline{PX}$ to be negative if the line segment lies entirely within the
circle, and positive otherwise. \ The \textit{circle power} of $P$ relative
to $C$, denoted by $pow_{C}\left( P\right) $, is given by:%
\begin{equation*}
pow_{C}\left( P\right) =\left\Vert \overline{PA}\right\Vert \left\Vert 
\overline{PB}\right\Vert .
\end{equation*}%
Alternatively, if $C$ is defined implicitly by $\left( x-x_{c}\right)
^{2}+\left( y-y_{c}\right) ^{2}=r_{c}^{2}$, then the circle power can be
expressed as:%
\begin{equation*}
pow_{C}\left( P\right) =\left( x-x_{c}\right) ^{2}+\left( y-y_{c}\right)
^{2}-r_{c}^{2}.
\end{equation*}

\item[common power point] The point of the plane (or $%
%TCIMACRO{\U{211d} }%
%BeginExpansion
\mathbb{R}
%EndExpansion
^{3}$) containing a decorated triangle (or tetrahedron) that has the same
circle power with respect to each of the weight circles. \ 

\item[decorated simplex (edge, triangle, tetrahedron,...)] A simplex is
called \textit{decorated} when weights are assigned to the vertices of the
simplex and then actualized by embedding the simplex into Euclidean space
together with spheres centered at the vertices with radii determined by the
weights. \ The orthocircle (if one exists) is sometimes considered part of
the decorated simplex when appropriate.

\item[edge curvature] Given a three-dimensional piecewise flat manifold $%
\left( M,\mathcal{T},d\right) $, the \textit{edge curvature} along an edge $%
\left\{ i,j\right\} $, measures how much that edge differs from Euclidean
space. \ Specifically, the edge curvature $K_{ij}$ is given by 
\begin{equation*}
K_{ij}=\left( 2\pi -\sum\limits_{\substack{ k,l\text{, such that}  \\ %
\left\{ i,j,k,l\right\} \in \mathcal{T}}}\beta _{ij,kl}\right) l_{ij},
\end{equation*}%
where $l_{ij}$ is the edge length, and $\beta _{ij,kl}$ is the dihedral
angle of the edge $\left\{ i,j\right\} $ of the tetrahedron $\left\{
i,j,k,l\right\} $. \ In a triangulation of three-dimensional Euclidean space 
$K_{ij}=0$ for all edges. \ 

\item[Einstein constant] For a 3-dimensional piecewise flat manifold $\left(
M,\mathcal{T},d\right) $, the \textit{Einstein constant} $\lambda $ is given
by%
\begin{equation*}
\lambda =\frac{EHR\left( M,\mathcal{T},d\right) }{3\mathcal{V}},
\end{equation*}%
where $EHR\left( M,\mathcal{T},d\right) $ is the Einstein-Hilbert-Regge
functional and $\mathcal{V}$ is the total volume.

\item[Einstein metric] Given a 3-dimensional piecewise flat manifold $\left(
M,\mathcal{T},d\right) $, we say that $d$ is an \textit{Einstein metric}
provided there exists $\lambda \in \mathbb{R}$ such that for all edges $%
\left\{ i,j\right\} $ in the triangulation we have:%
\begin{equation*}
K_{ij}=\lambda l_{ij}\frac{\partial \mathcal{V}}{\partial l_{ij}},
\end{equation*}%
where $K_{ij}$ is the edge curvature, $l_{ij}$ is the edge length, and $%
\mathcal{V}$ is the total volume. By summing both sides, we see that $%
\lambda $ is the Einstein constant.

\item[hinge] A hinge consists of a pair of triangles that share a single
edge. \ Often, two adjacent triangles in a simplicial surface are identified
as a hinge while studying the edge they share. \ Any hinge can be
isometrically embedding $%
%TCIMACRO{\U{211d} }%
%BeginExpansion
\mathbb{R}
%EndExpansion
^{2}$. \ There is a natural generalization to three (and higher) dimensions
for two tetrahedra sharing a face. \ Similarly, this generalized hinge can
be isometrically embedded in $%
%TCIMACRO{\U{211d} }%
%BeginExpansion
\mathbb{R}
%EndExpansion
^{3}$. \ 

\item[inversive distance] Start with a decorated edge $e_{ij}$, that is, an
edge of length $l_{ij}$ with weight circles of radius $r_{i},r_{j}$ centered
on its vertices. \ The inversive distance $\eta _{ij}$ of the edge $e_{ij}$
can be calculated with the formula:%
\begin{equation*}
\eta _{ij}=\frac{l_{ij}^{2}-r_{i}^{2}-r_{j}^{2}}{2r_{i}r_{j}}.
\end{equation*}%
When the two weight circles intersect with angle $\theta _{ij}$, we have the
simple formula:%
\begin{equation*}
\eta _{ij}=-\cos \left( \theta _{ij}\right) .
\end{equation*}%
The former formula was obtained by using the law of cosines and solving for
the $\cos \left( \theta _{ij}\right) $ term. \ When the weight circles do
not intersect and do not contain one or the other $\eta _{ij}>1$. \ When the
weight circles intersect at some angle then $-1\leq \eta _{ij}\leq 1$. \ If
one weight circle contains the other we have $\eta _{ij}<1$. \ 

\item[manifold] A second countable, Hausdorff topological space $M$ is a 
\textit{manifold} provided there is an integer $n>0$ such that for each $%
x\in M$ there is an open set $U_{x}$ containing $x$ and a homeomorphism $%
h_{x}:U_{x}\rightarrow B\left( 1,0\right) \subset 
%TCIMACRO{\U{211d} }%
%BeginExpansion
\mathbb{R}
%EndExpansion
^{n}$. \ A discrete (or piecewise flat) space is a manifold provided the
sub-simplices satisfy the following conditions:

\item Dimension 2:

\begin{itemize}
\item All edges have exactly two adjacent faces.

\item For a vertex $v$, the faces incident upon $v$ can be arranged
cyclically as $f_{1},f_{2},...,f_{N},f_{1},...$ so that there is an edge
(containing $v$ as an endpoint) between each pair of consecutive faces $%
f_{i},f_{i+1}$, where $N+1$ is understood to be $1$. \ 
\end{itemize}

\item Dimension 3:

\begin{itemize}
\item All faces have exactly two adjacent tetrahedra.

\item For each edge $e$, the tetrahedra incident upon $e$ can be arranged
cyclically as $\sigma _{1},\sigma _{2},...,\sigma _{M},\sigma _{1},...$ so
that there is a face (containing $e$ as an edge) between each pair of
consecutive tetrahedra $\sigma _{i},\sigma _{i+1}$ where $M+1$ is understood
to be $1$. \ 

\item For each vertex $v$, the number of incident edges, faces and
tetrahedra, $E,F,T$, respectively (including multiple occurrences) satisfy:%
\begin{equation*}
E-F+T=2.
\end{equation*}
\end{itemize}

\item More generally, given a simplicial manifold $M$ of dimension $n$, and
a sub-simplex $\sigma $ of dimension $m<n$, the sub-simplices of $M$ of
dimension greater than $m$ have the structure of $S^{n-m-1}$.

\item[orthocircle] Given a decorated triangle, provided the common power
point is outside all of the weight circles, there exists a circle that is
orthogonal to each of the weight circles. \ That is, the \textit{orthocircle}
is the circle that intersects each of the weight circles orthogonally. \ The
orthocircle does not exist when the common power point is on or inside all
three circles, however if the common power point is at infinity, there is a
line that is orthogonal to the three weight circles which will also be
identified as the orthocircle.

\item[piecewise flat manifold] A triple $\left( M,\mathcal{T},d\right) $
where $\left( M,\mathcal{T}\right) $ is a compact triangulated manifold with
triangulation $\mathcal{T}$, $d$ is a metric on $M$ so that the restriction
of $d$ to each simplex of $\mathcal{T}$ is isometric to a Euclidean simplex
of the same dimension. \ 

\item[triangulation] A collection of $n$-dimensional simplices $\mathcal{T}$
together with pairwise identifications for the $\left( n-1\right) $%
-dimensional faces of the simplices. \ More restrictions are needed to
ensure that the resultant space is a manifold. \ Alternatively, given a
space $M$ of dimension $n$, a \textit{triangulation} of $M$ is a subdivision
of $M$ into components $\left\{ \sigma _{i}\right\} $ by (hyper)surfaces (of
dimension $n-1$) so that each component is homeomorphic to an $n$%
-dimensional ball, and each component is combinatorially (as determined by
the subdivisions) equivalent to an $n$-simplex. \ 

\item[pseudo manifold] A discrete \textit{pseudo manifold} is a relaxation
of the manifold conditions for a discrete space. \ For dimensions two and
three, the bullet conditions given in the entry on manifold may no longer
hold. \ However, the manifold condition is still (trivially) satisfied in
the interior of all top dimensional simplices.

\item[weighted triangulation (or hinge)] A triangulation together with a map 
$w:V\rightarrow 
%TCIMACRO{\U{211d} }%
%BeginExpansion
\mathbb{R}
%EndExpansion
$, where $V$ is the set of vertices of the triangulation. \ Each simplex
becomes a decorated simplex. \ 

\item[dimension] 

\item[metric] 

\item[curvature] 

\item[scalar curvature] 

\item[constant scalar curvature] 

\item[geometric flow] 

\item[curvature flow] 

\item[Yamabe flow] 

\item[normalized total scalar curvature functional] 

\item[Einstein-Hilbert functional] 

\item[Einstein-Hilbert-Regge functional] 

\item[conformal class] 

\item[conformal deformation] 

\item[min-max procedure] 

\item[Yamabe constant] 

\item[flip] 

\item[Pachner move] 

\item[flip algorithm] 

\item[convex hinge] 

\item[nonconvex hinge] 

\item[bone] 

\item[Delaunay triangulation (or hinge)] 

\item[weighted triangulation (or hinge)] A triangulation together with a map 
$w:V\rightarrow 
%TCIMACRO{\U{211d} }%
%BeginExpansion
\mathbb{R}
%EndExpansion
$, where $V$ is the set of vertices of the triangulation. \ Each simplex
becomes a decorated simplex. \ 

\item[weighted Delaunay triangulation] 

\item[Voronoi diagram (or cell)] 

\item[weighted Voronoi diagram (or cell)] 

\item[power diagram (or cell)] 

\item[negative triangles] 

\item[dual, Poincare dual] 

\item[dual length] 

\item[dual volume] 
\end{description}
}}%
%BeginExpansion
%TCIDATA{Version=5.00.0.2606}
%TCIDATA{LaTeXparent=0,0,geocam.tex}
                      
%TCIDATA{ChildDefaults=chapter:1,page:1}


\chapter{Glossary}

\begin{description}
\item[circle power] Given a circle $C$ and a point $P$, let $L$ be the line
through the point $P$ that passes through the center of the circle. \ Let $A$
and $B$ be the intersection points of $L$ with the circle.\ Define a signed
distance $\left\Vert \overline{PX}\right\Vert $ for a line segment $%
\overline{PX}$ to be negative if the line segment lies entirely within the
circle, and positive otherwise. \ The \textit{circle power} of $P$ relative
to $C$, denoted by $pow_{C}\left( P\right) $, is given by:%
\begin{equation*}
pow_{C}\left( P\right) =\left\Vert \overline{PA}\right\Vert \left\Vert 
\overline{PB}\right\Vert .
\end{equation*}%
Alternatively, if $C$ is defined implicitly by $\left( x-x_{c}\right)
^{2}+\left( y-y_{c}\right) ^{2}=r_{c}^{2}$, then the circle power can be
expressed as:%
\begin{equation*}
pow_{C}\left( P\right) =\left( x-x_{c}\right) ^{2}+\left( y-y_{c}\right)
^{2}-r_{c}^{2}.
\end{equation*}

\item[common power point] The point of the plane (or $%
%TCIMACRO{\U{211d} }%
%BeginExpansion
\mathbb{R}
%EndExpansion
^{3}$) containing a decorated triangle (or tetrahedron) that has the same
circle power with respect to each of the weight circles. \ 

\item[decorated simplex (edge, triangle, tetrahedron,...)] A simplex is
called \textit{decorated} when weights are assigned to the vertices of the
simplex and then actualized by embedding the simplex into Euclidean space
together with spheres centered at the vertices with radii determined by the
weights. \ The orthocircle (if one exists) is sometimes considered part of
the decorated simplex when appropriate.

\item[edge curvature] Given a three-dimensional piecewise flat manifold $%
\left( M,\mathcal{T},d\right) $, the \textit{edge curvature} along an edge $%
\left\{ i,j\right\} $, measures how much that edge differs from Euclidean
space. \ Specifically, the edge curvature $K_{ij}$ is given by 
\begin{equation*}
K_{ij}=\left( 2\pi -\sum\limits_{\substack{ k,l\text{, such that}  \\ %
\left\{ i,j,k,l\right\} \in \mathcal{T}}}\beta _{ij,kl}\right) l_{ij},
\end{equation*}%
where $l_{ij}$ is the edge length, and $\beta _{ij,kl}$ is the dihedral
angle of the edge $\left\{ i,j\right\} $ of the tetrahedron $\left\{
i,j,k,l\right\} $. \ In a triangulation of three-dimensional Euclidean space 
$K_{ij}=0$ for all edges. \ 

\item[Einstein constant] For a 3-dimensional piecewise flat manifold $\left(
M,\mathcal{T},d\right) $, the \textit{Einstein constant} $\lambda $ is given
by%
\begin{equation*}
\lambda =\frac{EHR\left( M,\mathcal{T},d\right) }{3\mathcal{V}},
\end{equation*}%
where $EHR\left( M,\mathcal{T},d\right) $ is the Einstein-Hilbert-Regge
functional and $\mathcal{V}$ is the total volume.

\item[Einstein metric] Given a 3-dimensional piecewise flat manifold $\left(
M,\mathcal{T},d\right) $, we say that $d$ is an \textit{Einstein metric}
provided there exists $\lambda \in \mathbb{R}$ such that for all edges $%
\left\{ i,j\right\} $ in the triangulation we have:%
\begin{equation*}
K_{ij}=\lambda l_{ij}\frac{\partial \mathcal{V}}{\partial l_{ij}},
\end{equation*}%
where $K_{ij}$ is the edge curvature, $l_{ij}$ is the edge length, and $%
\mathcal{V}$ is the total volume. By summing both sides, we see that $%
\lambda $ is the Einstein constant.

\item[hinge] A hinge consists of a pair of triangles that share a single
edge. \ Often, two adjacent triangles in a simplicial surface are identified
as a hinge while studying the edge they share. \ Any hinge can be
isometrically embedding $%
%TCIMACRO{\U{211d} }%
%BeginExpansion
\mathbb{R}
%EndExpansion
^{2}$. \ There is a natural generalization to three (and higher) dimensions
for two tetrahedra sharing a face. \ Similarly, this generalized hinge can
be isometrically embedded in $%
%TCIMACRO{\U{211d} }%
%BeginExpansion
\mathbb{R}
%EndExpansion
^{3}$. \ 

\item[inversive distance] Start with a decorated edge $e_{ij}$, that is, an
edge of length $l_{ij}$ with weight circles of radius $r_{i},r_{j}$ centered
on its vertices. \ The inversive distance $\eta _{ij}$ of the edge $e_{ij}$
can be calculated with the formula:%
\begin{equation*}
\eta _{ij}=\frac{l_{ij}^{2}-r_{i}^{2}-r_{j}^{2}}{2r_{i}r_{j}}.
\end{equation*}%
When the two weight circles intersect with angle $\theta _{ij}$, we have the
simple formula:%
\begin{equation*}
\eta _{ij}=-\cos \left( \theta _{ij}\right) .
\end{equation*}%
The former formula was obtained by using the law of cosines and solving for
the $\cos \left( \theta _{ij}\right) $ term. \ When the weight circles do
not intersect and do not contain one or the other $\eta _{ij}>1$. \ When the
weight circles intersect at some angle then $-1\leq \eta _{ij}\leq 1$. \ If
one weight circle contains the other we have $\eta _{ij}<1$. \ 

\item[manifold] A second countable, Hausdorff topological space $M$ is a 
\textit{manifold} provided there is an integer $n>0$ such that for each $%
x\in M$ there is an open set $U_{x}$ containing $x$ and a homeomorphism $%
h_{x}:U_{x}\rightarrow B\left( 1,0\right) \subset 
%TCIMACRO{\U{211d} }%
%BeginExpansion
\mathbb{R}
%EndExpansion
^{n}$. \ A discrete (or piecewise flat) space is a manifold provided the
sub-simplices satisfy the following conditions:

\item Dimension 2:

\begin{itemize}
\item All edges have exactly two adjacent faces.

\item For a vertex $v$, the faces incident upon $v$ can be arranged
cyclically as $f_{1},f_{2},...,f_{N},f_{1},...$ so that there is an edge
(containing $v$ as an endpoint) between each pair of consecutive faces $%
f_{i},f_{i+1}$, where $N+1$ is understood to be $1$. \ 
\end{itemize}

\item Dimension 3:

\begin{itemize}
\item All faces have exactly two adjacent tetrahedra.

\item For each edge $e$, the tetrahedra incident upon $e$ can be arranged
cyclically as $\sigma _{1},\sigma _{2},...,\sigma _{M},\sigma _{1},...$ so
that there is a face (containing $e$ as an edge) between each pair of
consecutive tetrahedra $\sigma _{i},\sigma _{i+1}$ where $M+1$ is understood
to be $1$. \ 

\item For each vertex $v$, the number of incident edges, faces and
tetrahedra, $E,F,T$, respectively (including multiple occurrences) satisfy:%
\begin{equation*}
E-F+T=2.
\end{equation*}
\end{itemize}

\item More generally, given a simplicial manifold $M$ of dimension $n$, and
a sub-simplex $\sigma $ of dimension $m<n$, the sub-simplices of $M$ of
dimension greater than $m$ have the structure of $S^{n-m-1}$.

\item[orthocircle] Given a decorated triangle, provided the common power
point is outside all of the weight circles, there exists a circle that is
orthogonal to each of the weight circles. \ That is, the \textit{orthocircle}
is the circle that intersects each of the weight circles orthogonally. \ The
orthocircle does not exist when the common power point is on or inside all
three circles, however if the common power point is at infinity, there is a
line that is orthogonal to the three weight circles which will also be
identified as the orthocircle.

\item[piecewise flat manifold] A triple $\left( M,\mathcal{T},d\right) $
where $\left( M,\mathcal{T}\right) $ is a compact triangulated manifold with
triangulation $\mathcal{T}$, $d$ is a metric on $M$ so that the restriction
of $d$ to each simplex of $\mathcal{T}$ is isometric to a Euclidean simplex
of the same dimension. \ 

\item[triangulation] A collection of $n$-dimensional simplices $\mathcal{T}$
together with pairwise identifications for the $\left( n-1\right) $%
-dimensional faces of the simplices. \ More restrictions are needed to
ensure that the resultant space is a manifold. \ Alternatively, given a
space $M$ of dimension $n$, a \textit{triangulation} of $M$ is a subdivision
of $M$ into components $\left\{ \sigma _{i}\right\} $ by (hyper)surfaces (of
dimension $n-1$) so that each component is homeomorphic to an $n$%
-dimensional ball, and each component is combinatorially (as determined by
the subdivisions) equivalent to an $n$-simplex. \ 

\item[pseudo manifold] A discrete \textit{pseudo manifold} is a relaxation
of the manifold conditions for a discrete space. \ For dimensions two and
three, the bullet conditions given in the entry on manifold may no longer
hold. \ However, the manifold condition is still (trivially) satisfied in
the interior of all top dimensional simplices.

\item[weighted triangulation (or hinge)] A triangulation together with a map 
$w:V\rightarrow 
%TCIMACRO{\U{211d} }%
%BeginExpansion
\mathbb{R}
%EndExpansion
$, where $V$ is the set of vertices of the triangulation. \ Each simplex
becomes a decorated simplex. \ 

\item[dimension] 

\item[metric] 

\item[curvature] 

\item[scalar curvature] 

\item[constant scalar curvature] 

\item[geometric flow] 

\item[curvature flow] 

\item[Yamabe flow] 

\item[normalized total scalar curvature functional] 

\item[Einstein-Hilbert functional] 

\item[Einstein-Hilbert-Regge functional] 

\item[conformal class] 

\item[conformal deformation] 

\item[min-max procedure] 

\item[Yamabe constant] 

\item[flip] 

\item[Pachner move] 

\item[flip algorithm] 

\item[convex hinge] 

\item[nonconvex hinge] 

\item[bone] 

\item[Delaunay triangulation (or hinge)] 

\item[weighted triangulation (or hinge)] A triangulation together with a map 
$w:V\rightarrow 
%TCIMACRO{\U{211d} }%
%BeginExpansion
\mathbb{R}
%EndExpansion
$, where $V$ is the set of vertices of the triangulation. \ Each simplex
becomes a decorated simplex. \ 

\item[weighted Delaunay triangulation] 

\item[Voronoi diagram (or cell)] 

\item[weighted Voronoi diagram (or cell)] 

\item[power diagram (or cell)] 

\item[negative triangles] 

\item[dual, Poincare dual] 

\item[dual length] 

\item[dual volume] 
\end{description}
%
%EndExpansion

\bigskip

\appendix

\chapter{Appendix}

The appendix fragment is used only once. Subsequent appendices can be
created using the Chapter Section/Body Tag.

\backmatter

\chapter{Afterword}

The back matter often includes one or more of an index, an afterword,
acknowledgements, a bibliography, a colophon, or any other similar item. In
the back matter, chapters do not produce a chapter number, but they are
entered in the table of contents. If you are not using anything in the back
matter, you can delete the back matter TeX field and everything that follows
it.

\end{document}
