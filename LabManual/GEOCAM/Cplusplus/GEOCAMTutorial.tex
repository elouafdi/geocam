%\input{tcilatex}


\documentclass{amsart}
%%%%%%%%%%%%%%%%%%%%%%%%%%%%%%%%%%%%%%%%%%%%%%%%%%%%%%%%%%%%%%%%%%%%%%%%%%%%%%%%%%%%%%%%%%%%%%%%%%%%%%%%%%%%%%%%%%%%%%%%%%%%%%%%%%%%%%%%%%%%%%%%%%%%%%%%%%%%%%%%%%%%%%%%%%%%%%%%%%%%%%%%%%%%%%%%%%%%%%%%%%%%%%%%%%%%%%%%%%%%%%%%%%%%%%%%%%%%%%%%%%%%%%%%%%%%
\usepackage{amsfonts}
\usepackage{amsmath}
\usepackage{amssymb}
\usepackage{graphicx}
\usepackage{latexsym}
\usepackage{amstext}
\usepackage{eucal}
\usepackage{verbatim}
\usepackage{hyperref}

\setcounter{MaxMatrixCols}{10}
%TCIDATA{OutputFilter=Latex.dll}
%TCIDATA{Version=5.00.0.2552}
%TCIDATA{<META NAME="SaveForMode" CONTENT="1">}
%TCIDATA{LastRevised=Wednesday, March 03, 2010 13:54:38}
%TCIDATA{<META NAME="GraphicsSave" CONTENT="32">}
%TCIDATA{Language=American English}

\providecommand{\U}[1]{\protect\rule{.1in}{.1in}}
\newtheorem{theorem}{Theorem}
\theoremstyle{plain}
\newtheorem{acknowledgement}{Acknowledgement}
\newtheorem{algorithm}{Algorithm}
\newtheorem{axiom}{Axiom}
\newtheorem{case}{Case}
\newtheorem{claim}{Claim}
\newtheorem{conclusion}{Conclusion}
\newtheorem{condition}{Condition}
\newtheorem{conjecture}{Conjecture}
\newtheorem{corollary}{Corollary}
\newtheorem{criterion}{Criterion}
\newtheorem{definition}{Definition}
\newtheorem{example}{Example}
\newtheorem{exercise}{Exercise}
\newtheorem{lemma}{Lemma}
\newtheorem{notation}{Notation}
\newtheorem{problem}{Problem}
\newtheorem{proposition}{Proposition}
\newtheorem{remark}{Remark}
\newtheorem{solution}{Solution}
\newtheorem{summary}{Summary}
\numberwithin{equation}{section}
%\input{tcilatex}

\begin{document}
\title{GEOCAM Tutorial}
\author{Daniel Champion, Taylor Johnson, Alex Henniges}
\maketitle

Welcome to the GEOCAM tutorial. This document will guide you through
several of the fundamental components of the project's code. The code that
appears in this document can be found in the file tutorialcode.cpp in the
LabManual directory. We recommend that the reader executes the relevant
portions of the code while reading through this document to fully utilize
the tutorial.

A myriad of functions have been created in the project in order to expand
our research. To access these functions (as a developer) it is necessary
to add them to your project as well as provide all necessary "include"
statements in your code. \texttt{tutorialcode.cpp} contains excellent
examples of this practice.

\section*{Inputting a Triangulation}

The first function from the project that we will use allows us to load a
"Lutz" format triangulation. These triangulations are available for
download at \url{http://www.math.tu-berlin.de/diskregeom/stellar/}; these
triangulations specify only the top dimensional simplices in a
triangulation. To begin, a file is created to store the triangulation.%
\newline
\begin{verbatim}
   //Store Lutz format triangulation in a file:
   char filename[] = "Data/3DManifolds/LutzFormat/manifold_3_7_1.txt";
\end{verbatim}

\bigskip

Another file will be created to store the standard format version of the
triangulation. The standard format contains a list of all simplices
(including sub-simplices) along with all "local" information for each
simplex.\newline
\begin{verbatim}
   //Create file to store standard format triangulation:
   char modified[] = "Data/3DManifolds/StandardFormat/manifold_3_7_1.txt";
\end{verbatim}

\bigskip

Once the Lutz triangulation is loaded it is necessary to convert it to
standard format. In the case of 2D triangulations this is done by calling
the function \verb|makeTriangulationFile|, in the case of 3D triangulations 
\verb|make3DTriangulationFile| is used. To learn more about these
functions see the wiki entry on \verb|makeTriangulationFile|.\newline
\begin{verbatim}
   // Convert from Lutz format to standard format:
   makeTriangulationFile("./tetra_lutz.txt", "./tetra_standard.txt");
 
   // Now read in the triangulation from standard format:
   readTriangulationFile("./tetra_standard.txt");
 
\end{verbatim}

\section*{Triangulation Access \& Manipulation}

Now that the standard format triangulation is loaded into the code it is
possible to investigate the combinatorial properties of a triangulation. \
As an example, we will determine and print out the number of vertices, edges
and faces using the Triangulation class.\newline
\begin{verbatim}
   //Determine the number of vertices edges and faces:
   int vertSize = Triangulation::vertexTable.size();
   int edgeSize = Triangulation::edgeTable.size();
   int faceSize = Triangulation::faceTable.size();
 
   printf("V = %d, E = %d, F = %d\n", vertSize, edgeSize, faceSize);
   pause("press ENTER to continue");
\end{verbatim}

\bigskip

Using the number of vertices, edges, and faces it is possible to determine
the Euler Characteristic of this 2D manifold.\newline
\begin{verbatim}
   int EulerCharacteristic = vertSize - edgeSize + faceSize;
   printf("Euler Characteristic = %d\n", EulerCharacteristic);
   pause("press ENTER to continue");
\end{verbatim}

\bigskip

The triangulation we loaded is just the triangulation of a tetrahedron, and
hence is topologically equivalent to a sphere (hence the Euler
characteristic is 2). It is also possible to determine the local vertices,
edges, or faces using commands from the triangulation class using a command
such as:

\texttt{Triangulation::edgeTable[i].getLocalFaces()}

The next example will find the local faces of each edge in the triangulation
by storing them as pointers to vectors. \newline
\begin{verbatim}
   //Vector to store local faces:
   vector <int>* LF:
 
   //For loop to iterate over all edges of triangulation:
   for(int i = 1; i <= edgeSize; i++) {
      LF = Triangulation::edgeTable[i].getLocalFaces();
      printf("Faces sharing edge %d are:\n", i);
      //for loop to print out each local face that shares the edge i:
      for (int n = 0; n < (*(LF)).size(); n++) {
        printf("%d\n", LF -> at(n));
      }
   }
   pause("press ENTER to continue");
\end{verbatim}

\bigskip

During each iteration of the code written above, the two faces that share
edge \verb|i| are stored in the vector \verb|LF| and then printed out for
viewing. The astute reader will note that since we are investigating a
manifold, each edge will have exactly two local faces.

Two important geometric quantities that are given to a triangulation are
radii and etas. For each vertex, it is possible to retrieve and set the
value of the radius. In the next example the radii will be set to 2. The
value of the radii will then be retrieved using \verb|valueAt|. \newline
\begin{verbatim}
   //The for loop iterates over all vertices:
   for(int j = 1; j <= vertSize; j++){
     //Sets value of radius at vertex j:
     Radius::At(Triangulation::vertexTable[j]) -> setValue(2);
     //Use valueAt to retrieve value of a radius:
     double tempRadius = Radius::valueAt(Triangulation::vertexTable[j]);
     printf("Radius at vertex %d is: %f\n", j, tempRadius);
   }
   pause("press ENTER to continue");
\end{verbatim}

\bigskip

The tetrahedron triangulation we have been using has four vertices. \
Although in the above example the vertices are stored as the integers
\{1,2,3,4\} in general it is possible that the vertices are not stored in
consecutive order nor beginning with the number 1. The later case would be a
problem for all of the above "for loops" because they assume a starting
value of \verb|j|=\verb|1|.

In order to work around this problem we will create iterators for the edges,
faces, and vertices. These iterators are maps that iterate over all of the
edges, faces, or vertices of the given triangulation. \newline
\begin{verbatim}
   //Each map iterates over the given triangulation characteristic:
   map<int, Vertex>::iterator vit;
   map<int, Edge>::iterator eit;
   map<int, Face>::iterator fit;
   pause("press ENTER to continue");
\end{verbatim}

\bigskip

In the case of \verb|map<int, Face>::iterator fit;| the first entry of the
map, \verb|int|, stores the index of the face. The second entry of the map, 
\verb|Face|, stores the actual face of the triangulation. As an example of
the use of these iterators, we will provide a "safer" way of accomplishing
the same task of setting the radii as above.\newline
\begin{verbatim}
   //We use an iterator to vary over all of the vertices:
   for(vit = Triangulation::vertexTable.begin();
       vit != Triangulation::vertexTable.end(); vit++){
     //Sets value of radius at vertex j:
     Radius::At(vit->second)-> setValue(2);
     //Use valueAt to retrieve value of a radius:
     double tempRadius = Radius::valueAt(vit->second);
     printf("Radius at vertex %d is: %f\n", j, tempRadius);
   }
   pause("press ENTER to continue");
\end{verbatim}

\bigskip

Notice that we access the vertices using the code: \texttt{vit->%
second} instead of the lengthy code we used earlier: \texttt{%
Triangulation::vertexTable[j]}. If we wanted to use the lengthy version
using the iterator, we would use:

\texttt{Triangulation::vertexTable[vit->first]}, where the
fragment \texttt{vit->first }accesses the correct index for the
vertex. 

Next, we will use iterators again to load the values of the etas for all of
the edges. So as not to bore the reader, the code below shows how we can
vary the values of the etas as we index through with the iterator. \newline
\begin{verbatim}
   int k=1;
   for(eit=Triangulation::edgeTable.begin();
       eit!=Triangulation::edgeTable.end(); eit++, k++){
     //The actual edge of the triangle is used to set the value of Eta:
     Eta::At(eit->second)->setValue(2 * k);
     printf("Eta at edge %d is: %f\n", eit->first, Eta::valueAt(eit->second));
   }
   pause("press ENTER to continue");
\end{verbatim}

\bigskip

After setting the values of the radii and etas, other geometric quantities
can be calculated that depend on them. As an example, we will calculate
the edge lengths using the formula:%
\begin{equation*}
l_{ij}^{2}=r_{i}^{2}+r_{j}^{2}+2r_{i}r_{j}\eta _{ij}.
\end{equation*}

The calculation below will explicitly perform the calculation above. \
However, this calculation has been implemented in the project as a geoquant
and hence we will compare our calculation to the geoquant value. Take note
of how the geoquant is accessed, it should remind you of how we accessed the
radii and eta values. Indeed, radii and etas are also stored as geoquants
in the project, as are all geometric quantities. \newline
\begin{verbatim}
   //Vector to store local vertices:
   Vector <int>* LV;
   for(eit = Triangulation::edgeTable.begin();
       eit!=Triangulation::edgeTable.end(); eit++){
     double Etaij;
     LV = (eit -> second).getLocalVertices();
     Etaij = Eta::valueAt(eit -> second);
     double ri;
     double rj;
     double temp;
     ri = Radius::valueAt(Triangulation::vertexTable[LV -> at(0)]);
     rj = Radius::valueAt(Triangulation::vertexTable[LV -> at(1)]);
     temp = (ri * ri) + (rj *  rj) + (2 * ri * rj * Etaij);
     temp = sqrt(temp);
     printf("Length at edge %d is: %f\n", eit -> first, temp);
     printf("Geoquant length is: %f\n", Length::valueAt(eit -> second));
   }
   pause("press ENTER to continue");
\end{verbatim}

Geoquants are the most commonly used calculations in the project. All
geoquants can be acquired using the code:\newline
\begin{verbatim}
   ExampleGeoquant::valueAt(vertex1, edge1, edge2,...);
\end{verbatim}

\bigskip

The objects: \texttt{vertex1, edge1, edge2,...} are different for each
geoquant, and thus we are not able to provide the exact input in general. \
For example, we input an edge when calculating length, however we would
input a face when calculating area. 

\section*{Performing Flows \& Optimizations}

We are modeling triangulations because we want to learn more about them. One thing we can do is run a combinatorial Ricci flow on a triangulation.
Assuming we have a triangulation already in the code and whose radii and etas are set, we can build an \texttt{Approximator} object to run the flow.\newline

\begin{verbatim}
   Approximator *app = new RungaApprox((sysdiffeq) AdjRicci, "r2");
\end{verbatim}

\bigskip

There are a several important things we are doing with this one line. First, we are specifying that our approximator for our system of differential equations will use the RungaKutta method at each step. Another option would be Euler's method. The first parameter of our constructor is the name of a function that acts as a system of differential equations. We want to run a Ricci flow that is adjusted to prevent the radii from all shrinking to zero. The second parameter is indicating what quantities we would like our approximator to store at each step. This will become useful after our flow is complete so that we can print and view the results. In this case, we will record radii and two-dimensional curvatures.\newline

At this point we are ready to run a Ricci flow on our triangulation.\newline

\begin{verbatim}
   app->run(0.0001, 0.01);
\end{verbatim}

\bigskip

Once again we are specifying several things in one line. The first parameter is the precision bound. This is our ending condition. The flow will not end until the curvature at each vertex is changing by less than $0.0001$ between steps. The second paramter is the stepsize, $dt$.\newline

At this point, the flow has ran, so we want to see the results. Several print functions are available to display the results. For our case, we want to call \newline

\begin{verbatim}
   printResultsStep("./Data/ODE_Result.txt", 
                 &(app->radiiHistory), &(app->curvHistory));
\end{verbatim}

\bigskip

This will print the radii and curvatures of each step to the file ODE\_Result.txt. Recall that the vectors \verb|radiiHistory| and \verb|curvHistory| are populated only because we requested that those values be recorded. \newline

We are also interested in optimizing functions, in particular, functions that are defined on our triangulations. We do so by using Newton's Method to find the zeroes of the derivative. In our example, we will back away from triangulations for a moment and work with functions that are more basic. To run Newton's Method for this purpose, we need both the first and second derivative of the function we wish to optimize. When the function is of several variables, the first derivative is a gradient and the second derivative is a matrix known as a Hessian. \newline

Let us try to find a local maximum for the function $f(x,y) = \sqrt{1-\frac{x^2}{4} - \frac{y^2}{9}}$. To create a \verb|NewtonsMethod| object that will find the maximum, it must have a way to calculate the gradient and Hessian at any point $(x,y)$. We can provide functions that can calculate them explicitly, or the gradient and hessian can be approximated by the \verb|NewtonsMethod| object. Let us start will the latter:\newline

\begin{verbatim}
   double ellipse(double vars[]) {
     double val = 1 - pow(vars[0], 2) / 4 - pow(vars[1], 2) / 9;
     return sqrt(val);
   }

   .......
   .......
   .......
	 
   // Create the NewtonsMethod object, 2 variables
   NewtonsMethod *nm = new NewtonsMethod(ellipse, 2);
   // Build the array that holds the initial values.
   double initial[] = {1, 1};
   // Build the array that will hold the final solution.
   double soln[2];
   // Run the maximize function
   nm->maximize(initial, soln);

   // Display the results
   printf("\nSolution: %.10f, %.10f\n", soln[0], soln[1]);
\end{verbatim}

\bigskip

The function \verb|ellipse| needs to be defined somewhere in the code and this represents our function $f$. Note that it takes as a parameter an array of values. This array should represent the point at which the function is being evaluated. The function returns a \texttt{double}. The \verb|NewtonsMethod| object, \verb|nm|, is constructed by providing the function we defined above and a number that indicates the dimension of our domain. In our case, $f$ is a function on $\mathbb{R}^2$, so we pass the value $2$. The build an array that is the initial point at which we want to begin our optimization. This point can have considerable impact on the results, so choose wisely. We also create an empty array that will contain our optimal point after the function is complete. The function \verb|maximize| takes the initial and solution points and runs Newtons Method to attempt to find a maximum. \newline

During the maximize step, \verb|nm| is approximating the gradient and Hessian using difference quotients. This may be the best we can do for complicated functions, but in our case we can determine the first and second derivative explicitly: \newline

\begin{verbatim}
   double ellipse(double vars[]) {
     double val = 1 - pow(vars[0], 2) / 4 - pow(vars[1], 2) / 9;
     return sqrt(val);
   }
   void gradient(double vars[], double sol[]) {
   	 double val = 1 - pow(vars[0], 2) / 4 - pow(vars[1], 2) / 9;
     sol[0] = (-vars[0]/4)/sqrt(val);
     sol[1] = (-vars[1]/4)/sqrt(val);
   }
   void hessian(double vars[], double *sol[]) {
     double val = 1 - pow(vars[0], 2) / 4 - pow(vars[1], 2) / 9;
   	 sol[0][0] = ((-1.0/4)*val - (vars[0]/16)) / pow(val, 3.0/2);
   	 sol[1][1] = ((-1.0/9)*val - (vars[1]/81)) / pow(val, 3.0/2);
   	 sol[0][1] = (-vars[0]/4)*(-vars[1]/9)*(-1) / pow(val, 3.0/2);
   	 sol[1][0] = sol[0][1];
   }

   .......
   .......
   .......
	 
   NewtonsMethod *nm = new NewtonsMethod(ellipse, gradient, hessian, 2);
   double x_n[] = {1, 1};
   int i = 1;
   // Continue with the procedure until the length of the gradient is 
   // less than 0.000001.
   while(nm->step(x_n) > 0.000001) {
     fprintf(stdout, "\n***** Step %d *****\n", i++);
     nm->printInfo(stdout);
     for(int j = 0; j < 2; j++) {
       fprintf(stdout, "x_n_%d[%d] = %f\n", i, j, x_n[j]);
     }
   }
   printf("\nSolution: %.10f\n", x_n[0]);
\end{verbatim}

\bigskip

We now have two new functions \verb|gradient| and \verb|hessian|. Both take an array that represents the point at which we are taking derivatives. The gradient places its partial derivatives in the array \verb|sol|. For the Hessian, \verb|sol| is now a two-dimensional array representing the four second partial derivatives. We are also optimizing our solution a little differently. The array \verb|x_n| represents both the initial point and the optimal point and every point that occurs in between those two as we step through Newtons Method. The function call \verb|step| performs just one step of Newtons Method, updating the array \verb|x_n| and returning the length of the gradient vector. The stopping condition is an upperbound on the length of the gradient. We also use one more function in the \verb|NewtonsMethod| class, \verb|printInfo|. This simply displays the gradient and Hessian of the most recent step.\\

\section*{Performing Flips on Triangulations of 2D Manifolds}

We would like to be able to experimentally investigate flip algorithms that will transform weighted triangulations into weighted-Delaunay triangulations. To do this we made a function that can take information describing a hinge and perform a flip on that hinge. This involves changing the combinatorial information of all simplices local to that hinge. The function that can do this for 2D hinges is\\

\begin{verbatim}
   Edge flip(Edge e);
\end{verbatim}

\bigskip

This function is located in \verb|hinge_flip.cpp|, to use it we'll need to include \verb|hinge_flip.h|. The function \verb|flip| takes only an \verb|Edge| object, which for 2D manifolds, will fully describe a hinge in our triangulation. Part of performing a flip involves giving the flipped edge a new length so that it fits into the rest of the triangulation. To compute this new edge length we'll need \verb|Length| geoquants associated with the \verb|Edge| objects, and \verb|EuclideanAngle| geoquants associated with \verb|Vertex,Face| pairs. We can achieve this by simply inputing lengths since the \verb|EuclideanAngle| geoquants will compute their values automatically if the \verb|Length|s are valid.

Assigning lengths to edges could be hard coded into your program or collected from the user via a command line prompt one at a time. However hard coding specific geometric information will make it more difficult to switch between triangulations as recompilation of your code will be necessary anytime the triangulation changes. Prompting the user for inputs can become tedious and error prone for triangulations with many edges. The solutions we will use is to modify our input file so that a length value follows an \verb|Edge| entry, for example\\

\begin{verbatim}
  ...
  Edge: 6
  2 5
  1 3 7 8 
  3 
  5
  Edge: 7
  ...
\end{verbatim}

\bigskip

will indicate that \verb|Edge| $6$ has a length of $5$, because the last line of information pertaining to \verb|Edge| $6$ is a $5$. If we specify a length for each edge in our triangulation input file, those lengths will automatically be assigned to \verb|Length| geoquants by the \verb|readTriangulationFile(char *)| function. Another geoquant that is important in our investigations is the \verb|Radius| geoquant which represents the radius of a circle centered at a vertex. We can specify these radius values for vertices in the same way we specify edge lengths, for example\\

\begin{verbatim}
   ...
   Vertex: 5
   2 6 
   6 8 
   3
   Vertex: 6
   ...
\end{verbatim}

If we have all of these geoquant values specified in our input file, we can go ahead and try to use the \verb|flip(Edge e)| function, and examine the changes it makes to a triangulation.

\begin{verbatim}
   ...
   #include "hinge_flip.h"
   ...
   readTriangulationFile("../../Data/flip_test/eight_triangles_redux.txt");

   map<int, Edge>::iterator eit = Triangulation::edgeTable.begin();
   while(eit != Triangulation::edgeTable.end() && (eit->second).isBorder()) {
      eit++;
   }
   Edge e = eit->second;

   writeTriangulationFile("beforeFlip.txt");

   e = flip(e);

   writeTriangulationFile("afterFlip.txt");
   
   cout << e.getIndex();
   
   system("PAUSE");
\end{verbatim}

\bigskip

Notice that we include \verb|hinge_flip.h|. The first step after this new include statements is to read in the triangulation using \verb|readTriangulationFile|. The file we specify here has length and radius information which we can expect to be assigned correctly to the \verb|Edge| and \verb|Vertex| objects. The loop iterates over the \verb|edgeTable| to find an interior \verb|Edge|. We use \verb|isBorder()|, which returns \verb|true| if the \verb|Edge| has only one local \verb|Face|. An \verb|Edge| with one local \verb|Face| cannot be flipped so in this example we avoid these edges to ensure that calling \verb|Flip| has a non-trivial affect on the triangulation. We then write out the triangulation to a file called \verb|"beforeFlip.txt"|. We perform the flip on the edge we found and write out the triangulation again to a different file \verb|"afterFlip.txt"|. We print the index of the flipped \verb|Edge| so we know which \verb|Edge| we flipped. This way we can open up the two files we wrote and verify the combinatorial information has changed.

\section*{Visualizing Triangulations on 2D Manifolds}

Checking each simplex individually to make sure that the combinatorial or geometric information is correct can be a cumbersome task. To simplify this, we created tools to help visualize a triangulation. Currently, we can draw images of 2D triangulations using\\

\begin{verbatim}
   void ShowTriangulation(char * f) 
\end{verbatim}

\bigskip

This function is located in the file \verb|TriangulationDisplay.cpp|, so to use it we must include \verb|TriangulationDisplay.h|. The reader should open one of these files in their favorite text editor and notice that all the functions and variables defined within are inside a set of braces labeled \verb|namespace TriangulationDisplay|. A \verb|namespace| is a way of grouping the names of functions, variables, and classes that should be associated with one another and to prevent accidental modification to global variables defined within the namespace. The documentation for \verb|TriangulationDisplay| gives an argument for the use of a \verb|namespace| instead of a \verb|class| for this particular situation.

Assuming you have read in a triangulation file with lengths assigned such that the triangulation is planar, we can easily pop up a window with a picture of the triangulation.\\

\begin{verbatim}
   #include "TriangulationDisplay.h"
   ...
   TriangulationDisplay::ShowTriangulation();
\end{verbatim}

\bigskip

All we did was include the appropriate header file and call the \verb|ShowTriangulation()| function. Notice that the function name is preceded by \verb|TriangulationDisplay::|. The operator \verb|::| is called the scoping operator in C++, it allows an area of code to access a namespace that is not the same as the current namespace. In this case we are accessing a function in the namespace \verb|TriangulationDisplay|

\verb|ShowTriangulation| is overloaded to handle being called in other ways, such as\\

\begin{verbatim}
ShowTriangulation(char * f)
\end{verbatim}

\bigskip

The first argument \verb|f| is a filename you wish to load. The file specified by \verb|f| will be loaded with \verb|readTriangulationFile|, this is to provide the option of showing a triangulation with out having to load the triangulation manually. Be cautious in using this as it will undo any change you had previously made to the triangulation.

Once we've called some variation of \verb|ShowTriangulation(...)| we will be presented with a window which should be displaying whatever triangulation was asked for. Some useful key commands to use while viewing a triangulation follow.

\begin{align*}
\rightarrow\;\leftarrow\;\uparrow\;\downarrow &- move\;the \;triangulation \;around\\
[\;] &- \;zoom\; in\; and\; out\\
,\;. &- \;select\; next\; or\; previous\; edge\; (typically\; on\; the \; same \; keys \; as\; <\; and\; >)\\
s &- \; toggle \; view \; between\; flat\; and \; stacked\\
spacebar &- \; flip \; currently\; selected\; edge\\
r &- \;toggle\;radii\\
\end{align*}

\end{document}
