%TCIDATA{Version=5.00.0.2606}
%TCIDATA{LaTeXparent=1,1,functions.tex}
                      

\section*{\texttt{VolumeSecondPartial::VolumeSecondPartial}}

\subsection*{Function Prototype}

\texttt{double VolumeSecondPartial(Vertex v\_i, Vertex v\_j, Tetra t)}

\subsection*{Key Words}

Volume, Hessian Matrix, Newton's Method, partial derivative,
Einstein-Hilbert-Regge functional, geoquant.

\subsection*{Authors}

Daniel Champion

\subsection*{Introduction}

\texttt{VolumeSecondPartial} calculates the second order partial derivatives
of the volume of a tetrahedron with respect to log radii for all pairs of
indices (not necessarily distinct) in the vertex table. \ 

\subsection*{Subsidiaries}

\textbf{Functions:}

\texttt{listDifference}

\texttt{listIntersection}

\texttt{Simplex::isAdjVertex}

\textbf{Global Variables:} \ radii, etas.

\textbf{Local Variables:}

\subsection*{Description}

The volume of a tetrahedron only depends on the lengths of its edges as
calculated from the Cayley-Menger determinant. \ Thus for a given
tetrahedron $t$, it's second order partial derivatives with respect to $\log 
$ radii will vanish except for pairs of radii (not necessarily distinct)
corresponding to the vertices of $t$. \ The first step in the implementation
of \texttt{VolumeSecondPartial} is the determination of the following
trichotomy for a pair $\left\{ i,j\right\} $ of indices in the vertex table:%
\begin{equation*}
\begin{array}{l}
\text{A. \ }i=j\text{ and }i\text{ is a vertex of tetrahedron }t \\ 
\text{B. \ }i\neq j\text{ and both }i\text{ and }j\text{ belong to }t \\ 
\text{C. \ at least one of }i\text{ or }j\text{ doesn't belong to }t.%
\end{array}%
\end{equation*}

Each condition of the trichotomy requires a distinct calculation to
determine the desired partial derivative. \ Nevertheless, the next step in
the implementation is to place the tetrahedron in "standard form" relative
to the indices $i$ and $j$ (for conditions A and B only). \ More
specifically, for condition A the radius for vertex $i$ is stored as $r_{1}$%
, and the remaining radii of the tetrahedron $t$ are assigned $r_{2},r_{3},$
and $r_{4}$ in no particular order. \ The eta values $Eta_{12},Eta_{13},...$
are then assigned preserving the preceding assignments. \ In the case of
condition B, the radii at vertices $i$ and $j$ are assigned to $r_{1}$ and $%
r_{2}$ respectively, and $r_{3}$, and $r_{4}$ the remaining radii of $t$. \
The eta values $Eta_{12},Eta_{13},...$ are again assigned preserving the
preceding assignments.

The formulas for the second order partial derivatives in terms of these
standard form variables was calculated in Mathematica using the
Cayley-Menger determinant, that is:%
\begin{equation*}
288V^{2}=\det\left[ 
\begin{array}{ccccc}
0 & 1 & 1 & 1 & 1 \\ 
1 & 0 & L_{12}^{2} & L_{13}^{2} & L_{14}^{2} \\ 
1 & L_{12}^{2} & 0 & L_{23}^{2} & L_{24}^{2} \\ 
1 & L_{13}^{2} & L_{23}^{2} & 0 & L_{34}^{2} \\ 
1 & L_{14}^{2} & L_{24}^{2} & L_{34}^{2} & 0%
\end{array}
\right] ,
\end{equation*}
where the lengths were determined from the radii and eta values using the
formula%
\begin{equation*}
L_{ij}^{2}=r_{i}^{2}+r_{j}^{2}+2r_{i}r_{j}Eta_{ij}.
\end{equation*}

The formula obtained from Mathematica was outputted into the C programming
language using the function CForm.

This function was designed for use in the optimization of the
Einstein-Hilbert-Regge functional using Newton's method. \ In this procedure
the Hessian matrix of the normalized EHR functional is needed, each entry of
which uses the second order partial derivatives of volume. \ See the entry
on \texttt{EHRSecondPartial}.

\subsection*{Practicum}

Usage:

\texttt{VolumeSecondPartial (Vertex v\_i, Vertex v\_j, Tetra t)}

The integers \texttt{i} and \texttt{j} correspond to vertices in the vertex
table and \texttt{t} is a tetrahedron in the triangulation. \ Specifically
the function returns:%
\begin{equation*}
\text{\texttt{VolumeSecondPartial(v\_i, v\_j, t)} }=\frac{\partial ^{2}}{%
\partial \log r_{i}\partial \log r_{j}}Volume(t).
\end{equation*}

\subsection*{Limitations}

\texttt{VolumeSecondPartial} is fully operational with no know limitations.
\ The function will output appropriate values when given indices $i,$ and $j$
in the vertex table, and a tetrahedron $t$. \ 

\subsection*{Revisions}

subversion 757, 7/9/09, \texttt{VolumeSecondPartial} created.

subversion 1055, 3/12/10, \texttt{VolumeSecondPartial} converted to a
geoquant.

\subsection*{Testing}

Several trials were run outputting the values of \texttt{VolumeSecondPartial}
for a variety of vertices and tetrahedra. \ These values were compared with
calculations performed on Mathematica. \ 

\subsection*{Future Work}

None planned.
