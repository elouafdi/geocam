%TCIDATA{Version=5.00.0.2606}
%TCIDATA{LaTeXparent=1,1,functions.tex}
                      

\section*{\texttt{PartialEdge::PartialEdge}}

\subsection*{Function Prototype}

\texttt{double PartialEdge( Vertex vi, Vertex vj)}

\subsection*{Key Words}

Partial length, geoquant.

\subsection*{Authors}

Daniel Champion, ???

\subsection*{Introduction}

The function \texttt{PartialEdge} calculates the distance from a vertex to
the center of an edge (as determined by the center of a decorated triangle).

\subsection*{Subsidiaries}

\textbf{Functions:}

\qquad\texttt{Geometry::length}

\qquad\texttt{listIntersection}

\textbf{Global Variables:} \ radii, etas.

\textbf{Local Variables:} \ Vertex vi, vj.

\subsection*{Description}

The function \texttt{PartialEdge} is calculated with the simple formula:%
\begin{equation*}
\mathtt{PartialEdge}\text{\texttt{(vi,vj)}}=\frac{%
L_{ij}^{2}+r_{i}^{2}-r_{j}^{2}}{2L_{ij}},
\end{equation*}%
where $r_{i},r_{j}$ are the radii at vertices \texttt{vi}, and \texttt{vj}
respectively, and $L_{ij}$ is the length of the edge $\left\{ vi,vj\right\} $%
. \ Notice that this formula is not symmetric in $i$ and $j$. \ 

This function plays an important role in several areas of the project
including curvature, Dirichlet energy, and the optimization of the
Einstein-Hilbert-Regge functional. \ \texttt{PartialEdge} is used in the
calculation of several quantities used in the implementation of the \texttt{%
CurvaturePartial} function, which is used in the optimization of the
normalized Einstein-Hilbert-Regge functional.

\subsection*{Practicum}

An example of the use of this function is in the calculation of the edge
height function \texttt{EdgeHeight}:

\qquad \texttt{double EdgeHeight( Vertex vi, Vertex vj, Vertex vk) \{}

\qquad\qquad\texttt{Face fijk;}

\qquad\qquad\texttt{vector\TEXTsymbol{<}int\TEXTsymbol{>} temp\_ij = }

\qquad\qquad\qquad\texttt{listIntersection(vi.getLocalFaces(),
vj.getLocalFaces());}

\qquad\qquad\texttt{vector\TEXTsymbol{<}int\TEXTsymbol{>} temp = }

\qquad\qquad\qquad\texttt{listIntersection( \&temp\_ij, vk.getLocalFaces());}

\qquad\qquad\texttt{fijk = Triangulation::faceTable[temp[0]];}

\qquad \qquad \texttt{double result = (PartialEdge(vi, vk)-PartialEdge(vi,vj)%
}

\qquad\qquad\qquad\texttt{*cos(Geometry::angle(vi,
fijk)))/sin(Geometry::angle(vi, fijk));}

\qquad\qquad\texttt{return result;}

\qquad\qquad\texttt{\}}

\subsection*{Limitations}

\texttt{PartialEdge} must receive as input two vertices that define an edge
in the triangulation. \ \texttt{PartialEdge} returns distinct values for
each permutation of the input vertices. \ 

\subsection*{Revisions}

subversion 757, 7/13/09, \texttt{PartialEdge} created.

subversion 1055, 3/12/10, \texttt{PartialEdge}\ converted to a geoquant.

\subsection*{Testing}

\texttt{dij} was tested by working out several examples by hand.

\subsection*{Future Work}

This function has been added to the Geometry class geoquants, and thus this
entry needs to be updated.
