%TCIDATA{Version=5.00.0.2606}
%TCIDATA{LaTeXparent=1,1,functions.tex}
                      

\section*{\texttt{TotalVolume::TotalVolume}}

\subsection*{Function Prototype}

\texttt{double TotalVolume()}

\subsection*{Key Words}

Volume, tetrahedron, Cayley-Menger determinant, geoquant.

\subsection*{Authors}

Daniel Champion

\subsection*{Introduction}

The function \texttt{TotalVolume} calculates the total volume of a three
dimensional triangulated manifold.

\subsection*{Subsidiaries}

\textbf{Functions:}

\qquad Geometry::Volume

\textbf{Global Variables:} \ radii, etas.

\textbf{Local Variables:} none.

\subsection*{Description}

\texttt{TotalVolume} is calculated by summing the volumes of each
tetrahedron in a triangulation. \ The volume of a tetrahedron is calculated
with the Cayley-Menger determinant:%
\begin{equation*}
288V^{2}=\det \left[ 
\begin{array}{ccccc}
0 & 1 & 1 & 1 & 1 \\ 
1 & 0 & L_{12}^{2} & L_{13}^{2} & L_{14}^{2} \\ 
1 & L_{12}^{2} & 0 & L_{23}^{2} & L_{24}^{2} \\ 
1 & L_{13}^{2} & L_{23}^{2} & 0 & L_{34}^{2} \\ 
1 & L_{14}^{2} & L_{24}^{2} & L_{34}^{2} & 0%
\end{array}%
\right] 
\end{equation*}%
where the lengths were determined from the radii and eta values using the
formula%
\begin{equation*}
L_{ij}^{2}=r_{i}^{2}+r_{j}^{2}+2r_{i}r_{j}Eta_{ij}.
\end{equation*}

The formula was obtained using calculations within Mathematica and was
output into the C programming language using the function CForm. \ 

The total volume of a triangulation is used in multiple locations within the
project. \ One example of its use is in the calculation of the normalized
Einstein-Hilbert-Regge functional:%
\begin{equation*}
\widetilde{EHR}=\frac{\sum_{i}K_{i}}{\left( \text{\texttt{TotalVolume()}}%
\right) ^{\frac{1}{3}}}
\end{equation*}%
where $K_{i}$ is the curvature at vertex $i$.

\subsection*{Practicum}

An excellent example of the use of this function is in the calculation of
the normalized Einstein-Hilbert-Regge functional. \ 

\qquad\texttt{double EHR () \{}

\qquad\qquad\texttt{double result;}

\qquad \qquad \texttt{result = (TotalCurvature())/pow(TotalVolume (),
1.0/3.0);}

\qquad\qquad\texttt{return result;}

\qquad\qquad\texttt{\}}

\subsection*{Limitations}

Since \texttt{TotalVolume()} relies critically on the Geometry::volume
function, it thus has the same limitations. \ Specifically, if the edge
lengths of any tetrahedron do not satisfy the necessary conditions to
produce a positive volume tetrahedron, \texttt{TotalVolume()} will output an
undefined number. \ The Cayley-Menger determinant can be used to check this
condition on the edge lengths.

\subsection*{Revisions}

subversion 757, 7/11/09, \texttt{TotalVolume} created.

subversion 1055, 3/12/10, \texttt{TotalVolume}\ converted to a geoquant.

\subsection*{Testing}

Using known volumes of several tetrahedra, the total volume was calculated
by hand and compared with \texttt{TotalVolume}.

\subsection*{Future Work}

No planned future work.
