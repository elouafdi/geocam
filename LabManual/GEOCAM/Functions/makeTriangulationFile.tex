%TCIDATA{Version=5.00.0.2606}
%TCIDATA{LaTeXparent=0,0,functions.tex}
                      

%%%%% BEGINNING OF DOCUMENT BODY %%%%%
% html: Beginning of file: `clean.html'
% DOCTYPE HTML PUBLIC "-//W3C//DTD HTML 4.01//EN"
%  This is a (PRE) block.  Make sure it's left aligned or your toc title will be off. 

\section*{\texttt{makeTriangulationFile}}

\label{f0}{\small }
\begin{verbatim}
{\small void makeTriangulationfile(char* fileIN, char* fileOUT)
}
\end{verbatim}

\subsection*{Keywords}

\begin{quotation}
triangulation, Lutz, simplices
\end{quotation}

\subsection*{Authors}

\begin{itemize}
\item Alex Henniges

\item Mitch Wilson
\end{itemize}

\subsection*{Introduction}

\begin{quotation}
The \texttt{makeTriangulationFile} function converts a text file, given by 
\texttt{fileIn}, in the \texttt{Lutz} format to the standard format, printed
to \texttt{fileOUT}. The file in standard format can then be read into the
system to build the triangulation.
\end{quotation}

\subsection*{Subsidiaries}

\begin{quotation}
Functions:
\end{quotation}

\begin{itemize}
\item Pair::positionOf

\item Pair::contains

\item Pair::isInTuple
\end{itemize}

\begin{quotation}
Global Variables:

Local Variables: \texttt{fileIN}, \texttt{fileOUT}
\end{quotation}

\subsection*{Description}

\begin{quotation}
This function is used to convert one format to another format that we
consider to be the standard for reading in a triangulation. We have dubbed
the format we are converting from \texttt{Lutz}. This is based on the source
we retrieve this format from,
http://www.math.tu-berlin.de/diskregeom/stellar/\footnote{%
See URL http://www.math.tu-berlin.de/diskregeom/stellar/} .

The \texttt{Lutz} format provides a simpler interface than our standard
format, and can therefore allow for a user to create a quick triangulation.
The idea is to provide only the index of every vertex on each face of the
triangulation. No information about edges or adjacencies need to be given.
The file should begin with a ``=\"{} followed by \"{}\mbox{$[$}\"{} and \"{}%
\mbox{$]$}'''s to contain the triangulation and each face. An example 
\texttt{Lutz} format for a tetrahedron is given \mbox{$[$}\#Practicum below%
\mbox{$]$}.

Note that the \texttt{makeTriangulationFile} is used only for
two-dimensional triangulations, and that for three-dimensions, one should
use make3DTriangulationFile.
\end{quotation}

\subsection*{Pracicum}

\begin{quotation}
{\small }
\end{quotation}

\begin{verbatim}
{\small 
  // Convert the tetrahedron written in Lutz format to a file in standard format.
}
{\small 
  makeTriangulationFile("./tetra_lutz.txt", "./tetra_standard.txt");
}
{\small   
}
{\small   // Now read in the triangulation from standard format.
}
{\small   readTriangulationFile("./tetra_standard.txt");
}
{\small   
}
\end{verbatim}

\begin{quotation}
The \texttt{Lutz} format may look like:{\small }
\end{quotation}

\begin{verbatim}
{\small =[[1,2,3],[1,2,4],[2,3,4],[1,3,4]]
}
{\small   
}
\end{verbatim}

\begin{quotation}
The \texttt{makeTriangulationFile} would then create a file with:{\small }
\end{quotation}

\begin{verbatim}
{\small Vertex: 1
}
{\small 2 3 4 
}
{\small 1 2 4 
}
{\small 1 2 4 
}
{\small Vertex: 2
}
{\small 1 3 4 
}
{\small 1 3 5 
}
{\small 1 2 3 
}
{\small Vertex: 3
}
{\small 1 2 4 
}
{\small 2 3 6 
}
{\small 1 3 4 
}
{\small Vertex: 4
}
{\small 1 2 3 
}
{\small 4 5 6 
}
{\small 2 3 4 
}
{\small Edge: 1
}
{\small 1 2
}
{\small 2 3 4 5 
}
{\small 1 2 
}
{\small Edge: 2
}
{\small 1 3
}
{\small 1 3 4 6 
}
{\small 1 4 
}
{\small Edge: 3
}
{\small 2 3
}
{\small 1 2 5 6 
}
{\small 1 3 
}
{\small Edge: 4
}
{\small 1 4
}
{\small 1 2 5 6 
}
{\small 2 4 
}
{\small Edge: 5
}
{\small 2 4
}
{\small 1 3 4 6 
}
{\small 2 3 
}
{\small Edge: 6
}
{\small 3 4
}
{\small 2 3 4 5 
}
{\small 3 4 
}
{\small Face: 1
}
{\small 1 2 3 
}
{\small 1 2 3 
}
{\small 2 3 4 
}
{\small Face: 2
}
{\small 1 2 4 
}
{\small 1 4 5 
}
{\small 1 3 4 
}
{\small Face: 3
}
{\small 2 3 4 
}
{\small 3 5 6 
}
{\small 1 2 4 
}
{\small Face: 4
}
{\small 1 3 4 
}
{\small 2 4 6 
}
{\small 1 2 3 
}
{\small   
}
\end{verbatim}

\subsection*{Limitations}

\begin{quotation}
The limitation with the \texttt{Lutz} format that prevents it from being
considered the standard format is that the user cannot create the most
general of triangulations. To be more specific, it is impossible with the 
\texttt{Lutz} format to specify for there to be two edges with the same
vertices.

A limitation of the \texttt{makeTriangulationFile} is that its requirements
are unintuitive. There should be no ``='' required, for example. Another
limitation is that despite collecting enough information to build the
triangulation, the function instead writes this to a file, requiring the
user to subsequently call the function readTriangulationFile.
\end{quotation}

\subsection*{Revisions}

\begin{itemize}
\item subversion 545, 9/29/08: Added the \texttt{makeTriangulationFile}
function.
\end{itemize}

\subsection*{Testing}

\begin{quotation}
This function has been tested through frequent use.
\end{quotation}

\subsection*{Future Work}

\begin{itemize}
\item 7/1 - Improve the format system.

\item 7/1 - Create the triangulation without performing a conversion to
another file.
\end{itemize}

% html: End of file: `clean.html'

%%%%% END OF DOCUMENT BODY %%%%%
% In the future, we might want to put some additional data here, such
% as when the documentation was converted from wiki to TeX.
%
