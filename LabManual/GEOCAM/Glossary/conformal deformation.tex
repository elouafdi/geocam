\documentclass{article}%
\usepackage{amsmath}%
\setcounter{MaxMatrixCols}{30}%
\usepackage{amsfonts}%
\usepackage{amssymb}%
\usepackage{graphicx}
%TCIDATA{OutputFilter=latex2.dll}
%TCIDATA{Version=5.00.0.2606}
%TCIDATA{CSTFile=40 LaTeX article.cst}
%TCIDATA{Created=Wednesday, February 17, 2010 08:14:46}
%TCIDATA{LastRevised=Wednesday, February 17, 2010 08:21:15}
%TCIDATA{<META NAME="GraphicsSave" CONTENT="32">}
%TCIDATA{<META NAME="SaveForMode" CONTENT="1">}
%TCIDATA{BibliographyScheme=Manual}
%TCIDATA{<META NAME="DocumentShell" CONTENT="Standard LaTeX\Blank - Standard LaTeX Article">}
\newtheorem{theorem}{Theorem}
\newtheorem{acknowledgement}[theorem]{Acknowledgement}
\newtheorem{algorithm}[theorem]{Algorithm}
\newtheorem{axiom}[theorem]{Axiom}
\newtheorem{case}[theorem]{Case}
\newtheorem{claim}[theorem]{Claim}
\newtheorem{conclusion}[theorem]{Conclusion}
\newtheorem{condition}[theorem]{Condition}
\newtheorem{conjecture}[theorem]{Conjecture}
\newtheorem{corollary}[theorem]{Corollary}
\newtheorem{criterion}[theorem]{Criterion}
\newtheorem{definition}[theorem]{Definition}
\newtheorem{example}[theorem]{Example}
\newtheorem{exercise}[theorem]{Exercise}
\newtheorem{lemma}[theorem]{Lemma}
\newtheorem{notation}[theorem]{Notation}
\newtheorem{problem}[theorem]{Problem}
\newtheorem{proposition}[theorem]{Proposition}
\newtheorem{remark}[theorem]{Remark}
\newtheorem{solution}[theorem]{Solution}
\newtheorem{summary}[theorem]{Summary}
\newenvironment{proof}[1][Proof]{\noindent\textbf{#1.} }{\ \rule{0.5em}{0.5em}}
\begin{document}
conformal deformation.\qquad In Riemannian geometry, a conformal deformation
of a Riemannian manifold $\left(  M,g\right)  $ is a family of metrics
$g\left(  t\right)  ,$ where $t\in\left(  -\varepsilon,\varepsilon\right)  $
for some $\varepsilon>0,$ such that $g\left(  0\right)  =0$ and $\frac
{\partial}{\partial t}g=fg,$ where $f:M\rightarrow\mathbb{R}_{+}$ is a
(smooth) positive real-valued function. In piecewise Euclidean geometry, there
are a number of different definitions of conformal deformation; we take the
viewpoint that a conformal deformation is a family of functions $f\left(
t\right)  :V\rightarrow\mathbb{R}$ and a description of the lengths in terms
of variables $f_{i}$ at the vertices such that $\ell_{ij}=\ell_{ij}\left(
f_{i},f_{j}\right)  ,$ $\frac{\partial}{\partial f_{i}}\ell_{ij}=d_{ij},$ and
for any simplex $\left\{  i,j,k\right\}  ,$ $d_{ij}^{2}+d_{jk}^{2}+d_{ki}%
^{2}=d_{ji}^{2}+d_{ik}^{2}+d_{kj}^{2}$ (i.e., every simplex has a center, and
$d_{ij}$ is the length of the segment from vertex $i$ to the projection of the
center onto the edge $\left\{  i,j\right\}  $). One example of a conformal
deformation is the Yamabe flow (and discrete Yamabe flow).


\end{document}