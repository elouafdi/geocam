%TCIDATA{Version=5.00.0.2606}
%TCIDATA{LaTeXparent=0,0,geocam.tex}
                      
%TCIDATA{ChildDefaults=chapter:1,page:1}


\chapter{Geoquants}

\section{Introduction}

Geoquants were developed by Alex Henniges, Joeseph Thomas, and Kurtis
Norwood during the spring of 2009 as the primary component in streamlining
the execution of the code during extended optimization routines. The purpose
is to provide easy access to calculations of geometric quantities without
requiring extraneous calculations of those quantities while they remain the
same.\ Geoquants achieve this through the use of an invalidation tree to
prevent uneccesary calcuations.\bigskip

\section{General Properties}

Each geoquant contains a calculation of a quantity, invalidation
information, and input/output functionality. \ Some of these components are
specific to each geoquant while some are common to all geoquants.

\subsubsection{Calculation}

Each geoquant has a unique calculation yielding a geometric quantity. \ In
general there are no commonalities among the calculations except that all
geoquants output a quantity of geometric nature. \ 

\subsubsection{Input/Output}

Geoquants recieve as input topological data (vertex, face, tetrahedron) \
which serves as a unique identifier among all such quantities of the same
type across the triangulation. The topological data is composed of zero or
more of each dimension of simplex. The following functions are available to
all geoquants:

\begin{itemize}
\item ExampleGeoQuant::At(input)

At returns an instance of the ExampleGeoQuant identified by input.

\item ExampleGeoquant::valueAt(input)

valueAt returns the value of the geoquant identified by input as a
double-precision number.

\item getValue()

getValue returns the value of an already obtained geoquant.

\item setValue(value)

\ setValue will set the value of this geoquant to the specified value and
subsequently invalidate all dependent geoquants.
\end{itemize}

The functions getValue and setValue are called as:

\texttt{Q=ExampleGeoquant::At(input)}

\texttt{Q-\TEXTsymbol{>}getValue}

\texttt{Q-\TEXTsymbol{>}setValue}

\subsubsection{Invalidation}

The value of a geoquant is obtained in one of two ways: it is set by the
user or it is calculated from other geoquants. In the situation where a
geoquant B is calculated in part from the value of geoquant A, we refer to A
as the parent and B as the dependent. When the value of a parent changes,
the parent notifies all of its dependents that their own value is no longer
valid. Once a request for the value of an invalid geoquant is made, that
geoquant requests the value of each of its parents in order to perform the
recalculation of its own value.

\section{Geoquant Catalog}

\subsection{Area}

\paragraph{Input}

Area( Face f )

\paragraph{Calculation}

Area returns the area of a triangular face (in a triangulation). \ The
calculation is done using Heron's formula which takes as inputs the lengths
of the sides of the triangle.%
\begin{equation*}
A=\sqrt{s\left( s-a\right) \left( s-b\right) \left( s-c\right) }
\end{equation*}%
where%
\begin{equation*}
s=\frac{a+b+c}{2}.
\end{equation*}

\paragraph{Invalidation}

Parents: \ all lengths of edges in the input triangle.

\subsection{CurvaturePartial}

\paragraph{Input}

CurvaturePartial( Vertex v, Vertex w )

\paragraph{Calculation}

CurvaturePartial(v,w) returns the following partial derivative:%
\begin{equation*}
CurvaturePartial(v,w)=\frac{\partial K_{v}}{\partial \log r_{w}}.
\end{equation*}

The calculation of $CurvaturePartial(v,w)$ differes depending on the
combintorial relationship between $v$ and $w$. \ See ??? for more
information.

\paragraph{Invalidation}

YIKES!!!

\subsection{Curvature2D}

\paragraph{Input}

Curvature2D( Vertex v )

\paragraph{Calculation}

Curvature2D calculates the discrete curvature at a vertex in a two
dimensional manifold. \ The calculation is done by:%
\begin{equation*}
Curvature2D\left( v\right) =2\pi -\sum\limits_{\substack{ all\text{ }%
triangles\text{ }t  \\ incident\text{ }to\text{ }v}}\theta _{v,t}
\end{equation*}%
where $\theta _{v,t}$ is the angle at vertex $v$ in triangle $t$. \ 

\paragraph{Invalidation}

\bigskip Parents: Every angle incident on vertex v.

\subsection{Curvature3D}

\paragraph{Input}

Curvature3D( Vertex v )

\paragraph{Calculation}

Curvature3D calculates the curvature at a vertex in a three dimensional
manifold. \ If we let $K_{ij}$ be the secional curvature at an edge $\left\{
i,j\right\} $ and let vertex $v$ correspond to $i$, then%
\begin{equation*}
Curvature3D\left( v\right) =\sum\limits_{\substack{ j\text{ s.t. }\left\{
i,j\right\}  \\ \text{is an edge}}}K_{ij}d_{ij}.
\end{equation*}

\paragraph{Invalidation}

Parents: For every edge of v, the sectional curvature of that edge and its
partial edge with respect to v.\bigskip

\subsection{DihedralAngle}

\paragraph{Input}

DihedralAngle( Edge e, Tetra t )

\paragraph{Calculation}

The dihedral angle of an edge in a tetrahedron is calculated using the
spherical law of cosines. \ Given a geometrically determined tetrahedron,
the face angles can be computed. \ At a vertex of the tetrahedron, the face
angles correspond to the sides lengths of a spherical triangle and the
dihedral angles correspond to the angles of the spherical triangle. \ If we
denote the dihedral angle at edge $e=\left\{ i,j\right\} $ in tetrahedron $%
t=\left\{ i,j,k,l\right\} $ by $\beta _{ij,kl}$, and denote the face angle
at vertex $v=i$ of triangle $\left\{ i,j,k\right\} $ by $\gamma _{i,jk}$ we
have that:%
\begin{equation*}
\beta _{ij,kl}=\arccos \left( \frac{\cos \left( \gamma _{i,kl}\right) -\cos
\left( \gamma _{i,jk}\right) \cos \left( \gamma _{i,jl}\right) }{\sin \left(
\gamma _{i,jk}\right) \sin \left( \gamma _{i,jl}\right) }\right)
\end{equation*}

\paragraph{Invalidation}

\bigskip Parents: The face angles incident on vertex $v=i$ and with face $%
f\in t$.

\section{DihedralAngleSum}

\paragraph{Input}

\paragraph{Calculation}

\paragraph{Invalidation}

\bigskip

\subsection{DualArea}

\paragraph{Input}

\paragraph{Calculation}

\paragraph{Invalidation}

\bigskip

\subsection{DualAreaSegment}

\paragraph{Input}

\paragraph{Calculation}

\paragraph{Invalidation}

\bigskip

\subsection{EdgeHeight}

\paragraph{Input}

\paragraph{Calculation}

\paragraph{Invalidation}

\bigskip

\subsection{EHRPartial}

\paragraph{Input}

\paragraph{Calculation}

\paragraph{Invalidation}

\bigskip

\subsection{EHRSecondPartial}

\paragraph{Input}

\paragraph{Calculation}

\paragraph{Invalidation}

\bigskip

\subsection{Eta}

\paragraph{Input}

\paragraph{Calculation}

\paragraph{Invalidation}

\bigskip

\subsection{EuclideanAngle}

\paragraph{Input}

\paragraph{Calculation}

\paragraph{Invalidation}

\bigskip

\subsection{FaceHeight}

\paragraph{Input}

\paragraph{Calculation}

\paragraph{Invalidation}

\bigskip

\subsection{Length}

\paragraph{Input}

\paragraph{Calculation}

\paragraph{Invalidation}

\bigskip

\subsection{PartialEdge}

\paragraph{Input}

\paragraph{Calculation}

\paragraph{Invalidation}

\bigskip

\subsection{Radius}

\paragraph{Input}

\paragraph{Calculation}

\paragraph{Invalidation}

\bigskip

\subsection{SectionalCurvature}

\paragraph{Input}

\paragraph{Calculation}

\paragraph{Invalidation}

\bigskip

\subsection{TotalVolumePartial}

\paragraph{Input}

\paragraph{Calculation}

\paragraph{Invalidation}

\bigskip

\subsection{TotalCurvature}

\paragraph{Input}

\paragraph{Calculation}

\paragraph{Invalidation}

\bigskip

\subsection{TotalVolume}

\paragraph{Input}

\paragraph{Calculation}

\paragraph{Invalidation}

\bigskip

\subsection{Volume}

\paragraph{Input}

\paragraph{Calculation}

\paragraph{Invalidation}

\bigskip

\subsection{VolumeLengthPartial}

\paragraph{Input}

\paragraph{Calculation}

\paragraph{Invalidation}

\bigskip

\subsection{VolumeLengthTetraPartial}

\paragraph{Input}

\paragraph{Calculation}

\paragraph{Invalidation}

\bigskip

\subsection{VolumePartial}

\paragraph{Input}

\paragraph{Calculation}

\paragraph{Invalidation}

\bigskip

\subsection{VolumePartialSum}

\paragraph{Input}

\paragraph{Calculation}

\paragraph{Invalidation}

\bigskip

\subsection{VolumeSecondPartial}

\paragraph{Input}

\paragraph{Calculation}

\paragraph{Invalidation}

\bigskip
